
This work shows the implementation simplicity of a hybrid parallel genetic algorithm in functional-concurrent languages. The executions units was map to build-in concurrent concepts of languages (actors and agents) and the procedurals to functions or methods.

Erlang and Clojure are languages that encourage a \emph{zero mutable state}-\emph{all functional} programming style with advantages in the design an correction of the algorithms, Clojure's protocols allow the principles of OO without the complications of inheritance and the concurrent concepts of this language are specialized and flexible at the same time. The Scala language is multiparadigm and hybrid in relation with the computation modes supported. When a shared data structure is needed this language permit a more direct access and thats could be an advantage.

Among the new trends in pGAs are new parallel platforms, the new languages with concurrent abstractions build-in are parallel platforms too, and their use for develop pGAs can be a very good approach for new GA developments. In the pGA model used in this work the chosen GA architecture are concurrent-rich but the implementation remains simple thanks of the high level of abstraction of the implementation technologies.

Our experiments shows Scala's performance like the best and point to Erlang like an very scalable runtime, the recommendations are to enrich the experiments with more complex case of study and to test the libraries in heterogenous hardware in order to check scalability of each language. 