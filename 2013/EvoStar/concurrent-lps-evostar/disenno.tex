
In order to design the architecture of a software in the GAs application domain, we mostly identify the main concepts involved and the relations among them. Then, using the concepts of the paradigms and programming techniques chosen, we define the structure from the highest levels of abstraction indicating the data to be processed and their flow. The quality and extensibility of that structure might determine the succeed or failure of the software.

The main GA’s components identified in that work for designing our proposal are listed in Table \ref{agpComp}.

\begin{table}
  \centering
  \caption{Parallel GA's components.}\label{agpComp}
   \begin{tabular}{|>{\centering}p{3cm}|p{5cm}|p{3cm}|}
   \hline
   \textbf{AG Component} & \textbf{Rol} & \textbf{Description} \\
     \hline
      chromosome & Representing the solution. & binary string \\
     \hline
      evaluated chromosome & Pair \{chromosome, fitness\}. & relation thats indicate the value of a individual\\
     \hline
      population & Set of chromosomes. & list \\
     \hline
     crossover & Relation between two chromosomes producing other two new ones. & crossover's function \\
     \hline
      mutation & A chromosome modification. & chromosome's change function \\
     \hline
     selection & Means of population filtering. & selection's function \\
     \hline
      pool & Shared population among node's calculating units. & population \\
     \hline
      island & Topology's node. &  \\
     \hline
      migration & Random event for chromosome interchange. & message \\
     \hline
      evolution & Execution. & A generation is made \\
     \hline
      evaluation & Execution. & A fitness calculi is made \\
     \hline
   \end{tabular}

\end{table}

On the other hand, to develop an optimal codification of an algorithm is mandatory to know every characteristic of the programming language in use.

In this work we used hybrid pGAs: an island topology with a pool based pGA in each node. We chose the {\em Max-SAT} problem with 100 variable instances \cite{Hoos2000}.
