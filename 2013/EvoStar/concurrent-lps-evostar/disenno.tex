
In order to design the architecture of a software in the GAs application domain we most identify the main concepts involve and the relations among them. Then, using the concepts of the paradigms and programming techniques chosen, we define the structure from the more high levels of abstraction indicating the data to process and the flow of them. The quality and extensibility of that structure might determine the succeed or failure of the software.

The main GA’s components identify in that work for design our proposal are listed in Table \ref{agpComp}.

\begin{table}
  \centering
  \caption{Parallel GA's components.}\label{agpComp}
   \begin{tabular}{|>{\centering}p{3cm}|p{5cm}|p{3cm}|}
   \hline
   \textbf{AG Component} & \textbf{Rol} & \textbf{Description} \\
     \hline
      chromosome & Solution's representation. & binary string \\
     \hline
      evaluated chromosome & Pair \{chromosome, fitness\}. & relation thats indicate the value of a individual\\
     \hline
      population & Set of chromosomes. & list \\
     \hline
     crossover & Relation between two chromosomes producing other two new ones. & crossover's function \\
     \hline
      mutation & A chromosome modification. & chromosome's change function \\
     \hline
     selection & Means of population filtering. & selection's function \\
     \hline
      pool & Shared population among node's calculi units. & population \\
     \hline
      island & Topology's node. &  \\
     \hline
      migration & Random event for chromosome interchange. & message \\
     \hline
      evolution & Execution. & A generation is made \\
     \hline
      evaluation & Execution. & A fitness calculi is made \\
     \hline
   \end{tabular}

\end{table}

In the order hand, to develop an optimal codification of an algorithm it is mandatory to know each characteristic of the programming language in use.

In this work we have use an hybrid pGAs: an island topology with a pool based pGA in each node. We choose the {\em Max-SAT} problem with 100 variable instances \cite{Hoos2000}.
