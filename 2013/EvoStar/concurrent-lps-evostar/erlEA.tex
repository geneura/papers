
We developed a library in Erlang following the previous design concepts and it was tested with the case study. The code is open, under AGPL license, at 
(hidden for double-blind review)
%\url{https://github.com/jalbertcruz/erlEA/tree/evostar2014}. 
The main modules are briefly described in this section. %subsection?

\subsubsection{Reproducer module}

This module selects the subpopulation and parents for reproduction, and then it does the crossover and activates migrations. As actor, it responds to {\em evolve} and {\em emigrateBest} messages, for iteration and migration operations.

\subsubsection{Evaluator module}

This module consults the pool constantly looking for non evaluated individuals. It is compound by the function {\em evaluate/1} (general evaluation function) and the activation message: {\em evaluate}.

\subsubsection{PoolManager module}

This module initializes the pool’s workers (evaluators and reproducers) and decide the message routes among them.

\subsubsection{Auxiliar modules}

The previous modules contain the GA’s logic, nevertheless it is necessary other nonfunctional components. The auxiliary modules used are:
\vspace{.35cm}

\begin{description}

  \item[experimentRun and experiment] -- Experiments initialization and execution.

  \item[problem] -- Experiment’s parameters specification.

  \item[profiler] -- Execution statistics: execution time, number of iterations, etc.

  \item[islandManager] -- Pool coordination, starting and finalization controls.

  \item[manager] -- Multi-experiments control and final report emission.

\end{description} 