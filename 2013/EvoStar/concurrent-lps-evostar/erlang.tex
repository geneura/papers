
Erlang es un lenguaje de programación funcional, concurrente y
distribuido, adecuado para la construcción de sistemas que requieran grandes niveles de distribución \cite{veldstra:_welcom_erlan}, tolerancia a fallos y disponibilidad. Ha sido en más de una ocasión escogido por encima de C/C++ para el desarrollo de sistemas de uso intensivo de recursos \cite{Cesarini2009} dada la eficiencia de su ejecución \cite{erlang:future}. 

Utiliza el modelo {\em actor} para su implementación del paradigma de programación concurrente y posee facilidades para la integración con otros lenguajes tales como C/C++ y Java. Sus procesos son manejados por su máquina virtual (MV), que tiene un planificador por cada núcleo de la CPU, lo cual lo hacen ideal para la actual (y venideras) generación de procesadores multi-núcleos.

El que sea un lenguaje concurrente significa que posee entre sus tipos de datos el de proceso, en vez de ser facilidades proveídas por bibliotecas como en la mayoría de los lenguajes. Siendo en principio tan ligeros que una misma instancia de la MV puede tener millones en ejecución.