
The main concurrent concepts are actor and message; the functional concepts are function and list. It was used actors (the executing units of the language) for the independent process: islands or evaluators/reproducers. The communication among them was made by messages, that’s the concept available in the pattern.

GA’s logic was expressed by functions, the functional means for data transformation and computation expression, and the data model was coded using lists and tuples: the basics data structures in the functional paradigm.


\begin{table}[h!]
  \centering
   \caption{Erlang's constructions.}\label{erlConstructions}
\begin{tabular}{|>{\centering}p{3.4cm}|p{7cm}|}
  \hline
  % after \tabularnewline: \hline or \cline{col1-col2} \cline{col3-col4} ...
  \textbf{Erlang's Concept} & \textbf{Role} \tabularnewline
     \hline
  tuple & Data structure for immutable compound data. \tabularnewline
     \hline
  list & Sequence data structure for variable length compound data. \tabularnewline
     \hline
  function & Data's relations, operations. \tabularnewline
     \hline
  actor & Executing unit, process. \tabularnewline
     \hline
  message & Communication among actors. \tabularnewline
     \hline
  {\em ets} & Set of chromosome shared by the pool. \tabularnewline
     \hline
  {\em random} module& Random number generation. \tabularnewline
  \hline
\end{tabular}

\end{table}

\begin{table}
  \centering
  \caption{Erlang/AG's concepts mapping.}\label{erlAGRelation}
\begin{tabular}{|>{\centering}p{3cm}|p{6cm}|}
  \hline
  \textbf{Erlang concept} & \textbf{AG concept mapping} \tabularnewline
     \hline
  tuple & evaluated chromosome \tabularnewline
     \hline
  list & chromosomes and populations \tabularnewline
     \hline
  function & crossover, mutation and selection \tabularnewline
     \hline
  actor  & island, evaluator and reproducer \tabularnewline
     \hline
  message & migration \tabularnewline
     \hline
  {\em ets}  & pool \tabularnewline
     \hline

\end{tabular}

\end{table}


