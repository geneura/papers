
\noindent Genetic algorithms (GA) \cite{GA_Goldberg89} are currently one of the most used meta-heuristics to solve engineering problems. They can obtain solutions to complex optimizations problems in adequate times \cite{Luque2011}; in particular, parallel genetic algorithms are especially useful for problems with complex fitness functions and some researchers are found \cite{Alba2001} it improves the quality of solutions in terms of the number of evaluations needed to find one. That reason, together with the improvement in evaluation time brought by the simultaneous running in several nodes, have made parallel and distributed evolutionary algorithms a popular methodology.

However, since running evolutionary algorithms in parallel is quite straightforward (islands linked by connections through which {\em migrants} flow) a lot of effort has been devoted to measure performance and get the basic parameters (such as migration rate) right. But the programming paradigm used in the implementation of such algorithms is far from being an object of study. Object oriented or procedural languages like Java and C/C++ are mostly used, and, even when some researchers show that implementation matters \cite{DBLP:conf/iwann/MereloRACML11}, to approach new languages/paradigms it is not normally seen as a land for scientific improvements.

The multicore’s challenge \cite{SutterL05} shows a current need for making parallel even the simplest program. But this way leads us to use and create design patterns for parallel algorithms; the conversion of a pattern into a language feature is a common practice in the programming languages domain, and sometimes that means a language modification, others the creation of a new one.

This work explores the advantages of some non mainstream languages (and by that we mean that they are not included in the ten most popular languages in any ranking) with concurrent and functional features in order to develop GAs in its parallel versions. It is motivated by the lack of community attention on the subject and the belief that using concepts that simplify the modeling and implementation of such algorithms it might promote their use in research (achieving a paradigm shift to create more efficient algorithms) and in practice (for being the implementations of the parallels variants very simple).

This research tries to show some possible areas of improvement on architecture and engineering best practices for functional-concurrent paradigms like it has been on OO \cite{EO:FEA2000}, by focusing on GAs as a domain of application and describing how their principal traits can be modeled by means of functional-concurrent languages constructs. We are continuing the research reported in
%\cite{DBLP:conf/gecco/CruzGGC13,J.Albert-Cruz2013}.
other papers (hidden for double-blind review).

The rest of the paper is organized as follows: Next section presents the state of the art in parallel software platforms (programming languages paradigms) and its potential for evolutionary algorithms. Our adaptation of GAs to the paradigms and a case study results is explained in section \ref{sec:design}. Finally, we draw some conclusions and propose future lines of work in section \ref{sec:conclusions}. 