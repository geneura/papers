
\noindent Genetic algorithms (GA) are currently one of the most used meta-heuristics to solve Computer Science problems. They can obtain solutions to complex optimizations problems in adequate times \cite{Luque2011}. These algorithms consists in evolving sets of individuals (populations), as it is on nature, to find or improve a solution to a problem.

Parallel genetic algorithms (pGAs) are GAs in which the possible solutions are evaluated or the process of evolution is made in parallel. That kind of behavior is specially good for problems with expensive exploration of the search space; some researchers are found also \cite{Alba2001}, with the different search process, it’s improve the quality of solutions.

%La segunda frase del párrafo anterior no tiene sentido, no sé qué se quiere decir.
%Lo mismo con la primera frase del siguiente párrafo, hasta la coma.

The field is feature rich, several models have been created and new kind of problems are have been tried to solve, nevertheless the programming paradigm used in the implementation of such algorithms is far from being an object of study. Technologies like Java and C/C++ are mostly used, and, although some think that implementation matters \cite{DBLP:conf/iwann/MereloRACML11}, is not much the research that the community do approaching to new programming languages/paradigms.

%Del último párrafo, desde "isn't much the..." en la última frase, no tiene sentido.

The multicore’s challenge \cite{SutterL05} is a current need for making parallel even the simplest program. But this way leads us to use and create design patterns for parallel algorithms; the conversion of a pattern into a language feature is a common practice in the programming languages domain, and sometimes that’s means a language modification, others the creation of a new one.

This work explores the advantages of some non mainstream languages with concurrent and functional features in order to develop pGAs. It is motivated by the lack of community attention on the subject and the belief that using concepts that simplify the modeling and implementation of such algorithms it might promote their use in research (achieving a paradigm shift to create more efficient algorithms) and in practice (for being the implementations of the parallels variants very simple). We are continuing the research reported in \cite{DBLP:conf/gecco/CruzGGC13,J.Albert-Cruz2013} and we are trying to find the better options.

%El resto del trabajo se estructura como sigue: introducción y motivación (Sección \ref{sec:intro}), paradigmas y lenguajes multiparadigmas dentro del que se caracterizan los lenguajes funcionales y concurrentes (Sección \ref{sec:paradigmas}) así como algunos lenguajes emergentes (Sección \ref{sec:emergentes}). A continuación se muestra la modelación e implementación de un algoritmo genético usando conceptos de los paradigmas antes expuestos (secciones \ref{sec:design} y \ref{sec:impl}); y finalmente, se presentan los resultados (Sección \ref{sec:results}) y conclusiones (Sección \ref{sec:conclusions}).
