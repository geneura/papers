
\noindent Genetic algorithms (GA) are currently one of the more used meta-heuristics to solve Computer Science’s problems. They can obtain solutions to complex optimizations problems in adequate times \cite{Luque2011}. These algorithms consists in evolve sets of individuals (populations), as it is on nature, to find or improve a solution to a problem.

The parallel genetic algorithms (pGAs) are GAs in which the possible solutions are evaluated or the process of evolution is made in parallel. That kind of behavior is specially good for problems with expensive exploration of the search space; some researchers are found also \cite{Alba2001}, with the different search process, it’s improve the quality of solutions.

The field is feature rich, several models have been created and new kind of problems are try to solve, nevertheless the programming paradigm used in the implementation of such algorithms are far to be an object of study. Technologies like Java and C/C++ are mostly used, and, although some think implementation matters \cite{DBLP:conf/iwann/MereloRACML11} isn’t much the research that the community do approaching to new programming languages/paradigms.

The multicore’s challenge \cite{SutterL05} is a current need for make parallel even the most simple program. But this way leads us to use and create design patterns for parallel algorithms; the conversion of pattern into language feature is a common practice in the programming languages field, sometimes that’s means a language modification, others the creation of a new one.

This work explores the advantages of some no-mainstream languages with concurrent and functional features in order to develop pGAs. It’s motivated by the lack of community attention of the subject and believing that with the use of concepts that simplify the modeling and implementation of such algorithms it’s might promote theirs use in research (achieving a paradigm shift to create more efficient algorithms) and in practice (for been very simple the implementations of the parallels variants). We are continuing the research reported in \cite{DBLP:conf/gecco/CruzGGC13,J.Albert-Cruz2013} and we try to find the better options.

%El resto del trabajo se estructura como sigue: introducción y motivación (Sección \ref{sec:intro}), paradigmas y lenguajes multiparadigmas dentro del que se caracterizan los lenguajes funcionales y concurrentes (Sección \ref{sec:paradigmas}) así como algunos lenguajes emergentes (Sección \ref{sec:emergentes}). A continuación se muestra la modelación e implementación de un algoritmo genético usando conceptos de los paradigmas antes expuestos (secciones \ref{sec:design} y \ref{sec:impl}); y finalmente, se presentan los resultados (Sección \ref{sec:results}) y conclusiones (Sección \ref{sec:conclusions}).
