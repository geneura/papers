
The functional programming paradigm, despite his advantages, doesn’t have many followers. Several years ago was used in Genetic Programming \cite{Briggs:2008:FGP:1375341.1375345,Huelsbergen:1996:TSE:1595536.1595579,walsh:1999:AFSFESIHLP} and recently in neuroevolution \cite{Sher2013} but in GA is practically nonexistent his presence \cite{Hawkins:2001:GFG:872017.872197}.

This paradigm is characterized by the use of functions like first class concepts, and for encourage to don’t use state change. The latter is particularly useful for develop concurrent algorithms in which the communication by state changes is the origin of errors and complexity.