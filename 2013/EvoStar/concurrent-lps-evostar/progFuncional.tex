
Functional programming paradigm, despite its advantages, does not have many followers. Several years ago was used in Genetic Programming \cite{Briggs:2008:FGP:1375341.1375345,Huelsbergen:1996:TSE:1595536.1595579,walsh:1999:AFSFESIHLP} and recently in neuroevolution \cite{Sher2013} but in GA its presence is practically nonexistent \cite{Hawkins:2001:GFG:872017.872197}.

This paradigm is characterized by the use of functions like first class concepts, and for encouraging to do not use state changes. The latter is particularly useful for develop concurrent algorithms in which the communication by state changes is the origin of errors and complexity.