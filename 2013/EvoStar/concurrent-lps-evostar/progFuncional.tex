
This paradigm is characterized by the use of functions like first
class concepts, and for discouraging the use of state changes. The
latter is particularly useful for developing concurrent algorithms in
which the communication by state changes is the origin of errors and
complexity.

The functional programming paradigm, despite its advantages, does not
have many followers in the field of evolutionary algorithms, although
the fact that people ask online about it \cite{haskell-ga} and that
there are several open source libraries available, such as GA in
Haskell (available from \url{http://hackage.haskell.org/package/GA} or
implies that it has raised
some interest. Several years ago it was used in Genetic
Programming
\cite{Briggs:2008:FGP:1375341.1375345,Huelsbergen:1996:TSE:1595536.1595579,walsh:1999:AFSFESIHLP}
and recently in neuroevolution \cite{Sher2013} but in GA its presence
is practically nonexistent, except for generic attempts like the one
made by Hawkins \cite{Hawkins:2001:GFG:872017.872197}. 

