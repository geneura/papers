
Scala is a programming language with the same concurrent programming pattern (actors) than Erlang. In the Scala implementation we follow the same criteria like Erlang but with differences for his object support and JVM dependence.


\begin{table}[h!]
  \centering
   \caption{Scala's concepts.}\label{sclConstructions}
\begin{tabular}{|>{\centering}p{3.4cm}|p{7cm}|}
  \hline
  % after \tabularnewline: \hline or \cline{col1-col2} \cline{col3-col4} ...
  \textbf{Scala's concepts} & \textbf{Role} \tabularnewline
  tuple & Data structure for immutable compound data. \tabularnewline
    \hline
 list & Sequence data structure for variable length compound data.
 \tabularnewline
    \hline
 function & Data's relations, operations. \tabularnewline
     \hline
    Akka's actor & Executing unit, process. \tabularnewline
     \hline
  symbol/message & Communication among actors. \tabularnewline
     \hline
  {\em HashMap} & Set of chromosome shared by the pool. \tabularnewline
     \hline
\end{tabular}

\end{table}

\begin{table}
  \centering
  \caption{Scala/AG's concepts mapping.}\label{sclAGRelation}
\begin{tabular}{|>{\centering}p{3cm}|p{6cm}|}
  \hline
  \textbf{Scala concept} & \textbf{AG concept mapping} \tabularnewline
  \hline
   tuple & evaluated chromosome \tabularnewline
    \hline
 list & chromosomes and populations \tabularnewline
    \hline
 function & crossover, mutation and selection \tabularnewline
    \hline
  Akka's actor & island, evaluator and reproducer \tabularnewline
     \hline
  symbol/message & migration \tabularnewline
     \hline
  {\em HashMap} & pool \tabularnewline
     \hline
\end{tabular}

\end{table}






