
% *** Authors should verify (and, if needed, correct) their LaTeX system  ***
% *** with the testflow diagnostic prior to trusting their LaTeX platform ***
% *** with production work. IEEE's font choices can trigger bugs that do  ***
% *** not appear when using other class files.                            ***
% The testflow support page is at:
% http://www.michaelshell.org/tex/testflow/

\documentclass[conference]{IEEEtran}
\usepackage[latin1]{inputenc}
\usepackage{subfigure}
\usepackage{amsmath}
\usepackage{amssymb}
\usepackage[ruled,lined,linesnumbered,longend]{algorithm2e}
% Add the compsoc option for Computer Society conferences.
%
% If IEEEtran.cls has not been installed into the LaTeX system files,
% manually specify the path to it like:
% \documentclass[conference]{../sty/IEEEtran}

\def\eGA{{\small \textsc{eGA}}}
\def\eGACo{{\small \textsc{eGACo}}}
\def\eGACm{{\small \textsc{eGACm}}}
\def\codemaker{CM}
\def\codebreaker{CB}

% *** CITATION PACKAGES ***
%
\usepackage{cite}

% *** GRAPHICS RELATED PACKAGES ***
%
\ifCLASSINFOpdf
\usepackage[pdftex]{graphicx}
  % declare the path(s) where your graphic files are
\graphicspath{{./pdf/}{./jpeg/}}
  % and their extensions so you won't have to specify these with
  % every instance of \includegraphics
   \DeclareGraphicsExtensions{.pdf,.jpeg,.png}
\else
  % or other class option (dvipsone, dvipdf, if not using dvips). graphicx
  % will default to the driver specified in the system graphics.cfg if no
  % driver is specified.
  \usepackage[dvips]{graphicx}
  % declare the path(s) where your graphic files are
  \graphicspath{{../eps/}}
  % and their extensions so you won't have to specify these with
  % every instance of \includegraphics
  \DeclareGraphicsExtensions{.eps}
\fi


\usepackage{stfloats}
\usepackage{listings}
% *** PDF, URL AND HYPERLINK PACKAGES ***
%
\usepackage{url}

% correct bad hyphenation here
\hyphenation{op-tical net-works semi-conduc-tor}

\newcommand{\comm}[1]{``#1"}

\newcommand{\code}[2]{
    \lstset{language=#1,
        basicstyle=#2
    }
}

\begin{document}
%
% paper title
% can use linebreaks \\ within to get better formatting as desired
\title{Adapting evolutionary algorithms to the concurrent funcional language Erlang}

% author names and affiliations
% use a multiple column layout for up to three different
% affiliations
\author{\IEEEauthorblockN{J. Albert Cruz}
\IEEEauthorblockA{Centro de Arquitecturas Empresariales\\
Universidad de Ciencias Inform\'aticas (La Habana, Cuba)\\
Email: jalbert@uci.cu}
\and
\IEEEauthorblockN{Juan-J. Merelo, Paloma de las Cuevas and Antonio Mora}
\IEEEauthorblockA{Departamento de Arquitectura y Tecnolog\'ia de Computadores \\
University of Granada\\
Email: {jmerelo,amorag,paloma}@geneura.ugr.es}
}

% make the title area
\maketitle

%\begin{abstract}
%New languages and technologies are taking the actual multicore CPU
%space. One of them is Erlang, a functional language used to build
%concurrent and robust software real-time systems in many fields. Evolutionary
%computing provides elegant solutions for many high performance
%computing problems for which such technology excels and have been particulary developed and applied in several
%domains, so that libraries of diverse characteristics has been implemented and
%converted in valuables products.
%
%However, object oriented, or even structured programming paradigm are
%the dominants ways in the field. In this paper, a functional
%approach to the modeling and implementation of genetic algorithms is
%presented. The language Erlang is used to show how simple and
%efficient could be a develop of a concurrent evolutionary algorithm.
%\end{abstract}

\IEEEpeerreviewmaketitle

\section{Introduction and state of the art}

Evolutionary Computation (EC) seems to be an effective method for improving, and therefore optimizing, behaviours that lead to a better use of available resources.
% Antonio - he puesto Evolutionary Computation, ya que Evolution es muy gen�rico
% Eso s�, es un comienzo un tanto raro, a qu� te refieres con los recursos disponibles. La optimizaci�n se refiere a muchos �mbitos, no s�lo a un mejor uso de los recursos... ;)
% JA - El sentido de "available resources" es: lo que tengo, sea el par�metro que sea, y quiero usar de mejor(subjetivo, lo que el usuario entienda) manera
This scientific field is very active producing libraries of diverse quality and applying it to many domains.
% Antonio - falta una frase para conectar esta con la anterior. Pasar de hablar de EC a decir que para eso (para optimizar usando EC) los cient�ficos han hecho librer�as... mejor decir que el campo de la EC es uno de los m�s prol�ficos en la creaci�n de librer�as, etc, etc...
Nevertheless it is concentrated in the use of widespread implementation technologies such as C/C++, Fortran and Java.
Getting out of that mainstream it is not normally seen a land for improvement science.
% Getting out of that mainstream it is not normally seen a land for improvement science.
% Antonio - Esta frase casi no se entiende. La intento reescribir a ver si dice lo que t� quer�as.
One no mainstream language with very much potential is Erlang, it is an implementation technology that support functional and concurrent paradigms and that's starting to been used in the scientific community \cite{Sher2013}.
% Antonio - haz una breve inroducci�na Erlang aqu� y d� si se usa o no en la comunidad cient�fica. Pon alguna cita. ;)

Genetic Algorithms (GA) \cite{GA_Goldberg89}
% Antonio -  Pongo una cita a GAs
are general function optimizers that encode a potential solution to a specific problem on a simple data structure (a chromosome). There are only two components of them that are problem dependent: the solution encoding and the function that evaluate the quality of a solution, i.e., the fitness function. The rest of the algorithm does not depend on the problem and could be implemented following the best architecture and engineering practices.

 Inside the object oriented world there are reported various analysis for modeling GA behaviour, nevertheless that is not the case of the functional side. This work tries to show some possible areas of improvement on that sense.

This article is focused on GAs as a domain of application, and describes their principal concepts and characteristics, showing how can they be modeled by means of Erlang constructs.



% Antonio - cuenta en un par de l�neas qu� es Erlang. ;)

%Other concurrent oriented languages such as Ada \cite{Santos2002} has been reported as a feasible technology for implement this kind of algorithms.
% Antonio - esto no lo digas si no lo vas a extender un poco. Queda como una frase 'hu�rfana'. ;)
% D� por ejemplo qu� desventajas tiene respecto a Erlang...


% Actualizar esto
% Antonio - s�, rev�salo porque ya no hay secci�n de resultados
The rest of the paper is organized as follows: next section
presents considerations around the functional paradigm and its conceptual relation with evolutionary computation; we will explain then the general characteristics of a pool based concurrent evolutionary evolutionary used as use case in this work. Results and conclusions obtained will be presented in Section \ref{sec:conclusions}.




\section{Evolutionary algorithm in a concurrent functional language}
\label{sec:evo}

A variety of programming patterns, i.e., paradigms, exist for implementing the algorithms models. GAs are characterized by an intensive use of strings
% Antonio - he puesto GAs porque es lo qu ehas comentado en la intro. ;)
 (lists of some kind) for encoding genes, the existence of populations that evolve by theirselves, and the variation in selection criterias through time. A programming language whose characteristics fit these needs would be highly appreciated.

\subsection{Erlang description}
% Antonio - �description?

There is a claim in modern software development for programming languages that help with concurrent programming and simplify coding practice.
% Antonio - qu� quiere decir better abstraction? acl�ralo. :D
The functional programming language Erlang would an answer that provides the actor pattern concept for concurrency and the functional paradigm for general modeling, design and coding of solutions.
% Antonio - este texto parece de publicidad de Erlang, jeje.

Actors are concurrent execution units which use asynchronous message passing for communication. They are implemented as processes in the Erlang's virtual machine
% Antonio - no has comentado qu� es la VM. No has dicho que se implementa sobre Java...
 and not like operating system (OS) threads, therefor they are very lightweight in creation and execution.
% Antonio - �qu� quiere decir que sean ligeros?
The use of messages eliminate the sharing of state and
% Antonio - d� qu� son mensajes inmutables
eliminate many of the typical problems of concurrent development, thats support the emulation of the Object Oriented (OO) paradigm with its modeling facilities.

Functional programming is defined by the use of functions in program composition and by using lists.
% Antonio - Esta frase no se entiende bien y est� mal escrita. ;)
Erlang honored the functional linage and include an ultra fast persistent technologies called Dets and Mnesia. Besides, Erlang offers
% Antonio - explica qu� es esto de macro system
the concept of records for group data.



\subsection{Genetic Algorithm mapping to Erlang}

Genetic algorithms, as many computational models, tend to be described in literature on its operational and imperative way. Their implementation in a functional language must follow a different path, structuring the algorithm model in terms less imperative and more declarative. We are going to use a parallel pool based evolutionary strategy\cite{DBLP:conf/3pgcic/GuervosMFEL12} as use case in order to show our mean.

The pool will be an execution entity (an actor acting like a server) that will own the population and also keep a track of the advance in the solution search. The clients, which are concurrent, will do the calculations and will join and leave the system at any time without consequences. Chromosomes will be encoded as lists and the different parts of the GAs algorithms will be implemented as Erlang functions.
% Antonio - functions -> actors?

An Erlang actor is implemented by a sequence of pairs pattern/expression defining each message that it could handle. It is close to the OO parlance and a way to organize the code. In this case we use one message per service that pool must provide, table \ref{poolTable} presents this.

\begin{table}
  \centering
   \begin{tabular}{|p{3cm}|p{4cm}|}
   \hline
   \textbf{Message} & \textbf{Description}\\
     \hline
\begin{verbatim}
{configPool, NIM}->
\end{verbatim}& Initialization, the parameter NIM is the initial configuration. \\
\hline
\begin{verbatim}
{requestWork, Pid,
   Capacity} ->
\end{verbatim} & Client requests for a population to evolve.\\
\hline
\begin{verbatim}
{generationEnd,
NewIndividuals,
OldIndexes, Pid}->
\end{verbatim} & One client report its successfully end of calculation. \\
\hline
   \end{tabular}
  \caption{The messages that a pool accepts.}\label{poolTable}
\end{table}

Clients are modeled by actors. They are the units of evolution, with the main computation responsibilities; the table \ref{clientTable} shows its interface.

\begin{table}
  \centering
   \begin{tabular}{|p{3cm}|p{4cm}|}
   \hline
   \textbf{Message} & \textbf{Description}\\
     \hline
\begin{verbatim}
initEvolution ->
\end{verbatim} & Marks the beginning of the processing. \\
\hline
\begin{verbatim}
{evolve, P,
  NIndexes} ->
\end{verbatim} & When the pool could assign a subpopulation to process. \\
\hline
\end{tabular}
  \caption{The messages that a client is able to respond to.}\label{clientTable}
\end{table}

\subsection{Generality and configuration}
% No hace falta tantas subdivisiones en un art�culo de dos p�ginas, caray... eliminar todas las sub y subsubsecciones - JJ
% Al dejar solamente secciones ha quedado un poco raro, digo yo, nada vean ahora y qu�tenlas si se debe...

The two previous components have the architecture of the algorithm, they are general and could be used for many problems. In order to solve a particular situation, they must be \emph{injected} by several functions and data structures which define: chromosomes, fitness function, mutation operator, selection criteria and replacement policy. All these particularizations must be implemented in an Erlang source file and configured in the \emph{configBuilder} module.

The design made promotes a clear separation between architecture (the general, constant and paradigmatic foundation part) and problem encoding (the representation and criteria of solution finding).

\section{Experiment and conclusions}
\label{sec:conclusions}

In this ongoing project we are testing the efficiency and simplicity of implementations of GAs by functional programming. The parallel models of GA are mapped to actors in the Erlang languages obtaining easily to understand architectures. All the code has been released as open source code at \url{https://github.com/jalbertcruz/erlEA/}.

The library was tested with \emph{MaxOnes} problem. The chromosomes was 128 elements long, with an initial population of 256 individuals. 50 clients was used and they worked with 20 individuals each time a pool assign a generation to them. The solution, an optimum, was reached after 3229 assignments.

With this concept test we are showing how simple is to structure a parallel GA, now we could proceed with more complex GA models, experiments and problems in order to explore the potential of the technology.

\section*{Acknowledgements}
This work is supported by projects TIN2011-28627-C04-02 and TIN2011-28627-C04-01 and -02 (ANYSELF), awarded by the Spanish Ministry of Science and Innovation and P08-TIC-03903 and TIC-6083 (DNEMESIS) awarded by the Andalusian Regional Government. It is supported too by the PhD Programm of Intelligent Systems and Softcomputing hold by the Asociaci�n Universitaria Iberoamericana de Postgrado -AUIP-, the University of Granada, Spain, and the University of Informatics Sciences, Cuba. It has also been supported by project 83, Campus CEI BioTIC http://ceibiotic.ugr.es

\nocite{*}
\bibliographystyle{IEEEtran}
%\IEEEtriggeratref{3}
\bibliography{concurrent}

\end{document}


