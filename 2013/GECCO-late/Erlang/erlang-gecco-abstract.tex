
% *** Authors should verify (and, if needed, correct) their LaTeX system  ***
% *** with the testflow diagnostic prior to trusting their LaTeX platform ***
% *** with production work. IEEE's font choices can trigger bugs that do  ***
% *** not appear when using other class files.                            ***
% The testflow support page is at:
% http://www.michaelshell.org/tex/testflow/

\documentclass[conference]{IEEEtran}
\usepackage[latin1]{inputenc}
\usepackage{subfigure}
\usepackage{amsmath}
\usepackage{amssymb}
\usepackage[ruled,lined,linesnumbered,longend]{algorithm2e}
% Add the compsoc option for Computer Society conferences.
%
% If IEEEtran.cls has not been installed into the LaTeX system files,
% manually specify the path to it like:
% \documentclass[conference]{../sty/IEEEtran}

\def\eGA{{\small \textsc{eGA}}}
\def\eGACo{{\small \textsc{eGACo}}}
\def\eGACm{{\small \textsc{eGACm}}}
\def\codemaker{CM}
\def\codebreaker{CB}

% *** CITATION PACKAGES ***
%
\usepackage{cite}

% *** GRAPHICS RELATED PACKAGES ***
%
\ifCLASSINFOpdf
\usepackage[pdftex]{graphicx}
  % declare the path(s) where your graphic files are
\graphicspath{{./pdf/}{./jpeg/}}
  % and their extensions so you won't have to specify these with
  % every instance of \includegraphics
   \DeclareGraphicsExtensions{.pdf,.jpeg,.png}
\else
  % or other class option (dvipsone, dvipdf, if not using dvips). graphicx
  % will default to the driver specified in the system graphics.cfg if no
  % driver is specified.
  \usepackage[dvips]{graphicx}
  % declare the path(s) where your graphic files are
  \graphicspath{{../eps/}}
  % and their extensions so you won't have to specify these with
  % every instance of \includegraphics
  \DeclareGraphicsExtensions{.eps}
\fi


\usepackage{stfloats}
\usepackage{listings}
% *** PDF, URL AND HYPERLINK PACKAGES ***
%
\usepackage{url}

% correct bad hyphenation here
\hyphenation{op-tical net-works semi-conduc-tor}

\newcommand{\comm}[1]{``#1"}

\newcommand{\code}[2]{
    \lstset{language=#1,
        basicstyle=#2
    }
}

\begin{document}
%
% paper title
% can use linebreaks \\ within to get better formatting as desired
\title{Adapting evolutionary algorithms to the concurrent funcional language Erlang}

% author names and affiliations
% use a multiple column layout for up to three different
% affiliations
\author{\IEEEauthorblockN{J. Albert Cruz}
\IEEEauthorblockA{Centro de Arquitecturas Empresariales\\
Universidad de Ciencias Inform\'aticas (La Habana, Cuba)\\
Email: jalbert@uci.cu}
\and
\IEEEauthorblockN{Juan-J. Merelo, Paloma de las Cuevas and Antonio Mora}
\IEEEauthorblockA{Departamento de Arquitectura y Tecnolog\'ia de Computadores \\
University of Granada\\
Email: {jmerelo,amorag,paloma}@geneura.ugr.es}
}

% make the title area
\maketitle

\begin{abstract}
New languages and technologies are taking the actual multicore CPU
space. One of them is Erlang, a functional language used to build
concurrent and robust software real-time systems in many fields. Evolutionary
computing provides elegant solutions for many high performance
computing problems for which such technology excels and have been particulary developed and applied in several
domains, so that libraries of diverse characteristics has been implemented and
converted in valuables products. 

However, object oriented, or even structured programming paradigm are
the dominants ways in the field. In this paper, a functional
approach to the modeling and implementation of genetic algorithms is
presented. The language Erlang is used to show how simple and
efficient could be a develop of a concurrent evolutionary algorithm. 
\end{abstract}

\IEEEpeerreviewmaketitle

\section{Introduction and state of the art}

Evolution seems to be an effective method for improving, therefore
optimizing, behaviours that lead to better use of available
resources. For doing so the scientific community has produced several
libraries of diverse quality and applied it in many
domains. Nevertheless it is concentrated in the use of widespread
implementation technologies like C/C++, Fortran and Java. Getting out
of that mainstream it is not normally seen a land for improvement
science. 

This article is centered on \emph{Genetic Algorithms} as a domain of
application and will not develop any theory about that. Nevertheless
the principal concepts involved and it's characteristics most be told
in order to show how was modeled by Erlang constructs. 

Genetic algorithms are general function optimizers that encode a
potential solution to a specific problem on a simple data structure (a
chromosome). There are only two components of them that are problem
dependent: the solution encoding and the function that evaluate the
quality of a solution. The rest is problem independent and could and
most be implemented following the best architecture and engineering
best practices. 

On the object oriented world there is reported various analysis to
model GA behaviour, nevertheless thats not the case on the functional
side. This work try to show on that direction posible areas of
improvement.   

Other concurrent oriented languages like Ada \cite{Santos2002} has
been reported like a feasible technology for implemented this kind of
algorithms. 

% Actualizar esto
The rest of the paper is organized as follows: next section
presents considerations around the functional paradigm and its conceptual relation with evolutionary computation; we will
explain then the general characteristics of a pool based concurrent evolutionary evolutionary used as use case in this work. Results obtained will be presented in Section \ref{sec:results}, and we will finish the paper
with the conclusions that derive from them (commented in Section
\ref{sec:conclusions}).




\section{Implementing an evolutionary algorithm in a concurrent functional language}
\label{sec:evo}

A varied arsenal of programming patterns, i.e., paradigms, exist for implement the algorithms models. EAs are characterized by an intensive use of strings (lists of some kind) for encoding genes, the existence of populations that evolve by herself and the variation of criterias for selection trough time. An programming language whose characteristics fits these needs will be highly appreciated.


There is a claim in modern software development for programming languages that help with concurrent programming and with better abstractions. The functional programming language Erlang is an answer that provides the actor pattern concept for concurrency and the functional paradigm for general modeling and design of solutions.

Actors are a concurrent unit of execution using asynchronous messages passing for communication, it is implemented as process in the virtual machine (VM) of the language's runtime and not like OS threads and for that are very lightweight in birth, live and death. The use of immutable messages eliminate many of the typical problems of concurrent development and support the emulation of the Object Oriented (OO) paradigm with his modeling facilities.

Functional programming has been defined by the role of functions in program composition and by the use of the very efficient lists data structure with many high order functions that operate upon it, Erlang honored the functional linage and include an ultra fast persistent technology call it Dets and Mnesia. Besides, Erlang posees a macro system and the concept of records for encapsulate coding patterns and group data by an entity.



\subsection{Genetic Algorithm mapping to Erlang}

Genetics algorithms, like many models of computations tend to be described in literature in is operational, imperative way. It's implementation in a functional language most follow a different path, that's means to structure the algorithm model in terms less imperative and more declarative. We going to use a parallel pool based evolutionary strategy \cite{DBLP:conf/3pgcic/GuervosMFEL12} as use case in order to show our mean.

The pool (a server) will be an execution entity that will own the population and keep track of the advance in the solution search. The clients, concurrents to, do the calculations and could join and leave at any time without any consequences. Chromosomes will be encoding like lists and the differents steps of the EAs resides on Erlang functions.

An Erlang actors is implemented by a sequence of pairs patterns/expressions that define each message that he could handle, It's close at the OO parlance and a way to organize the code. In this case we use one message for each service that pool most provide, table \ref{poolTable} presents that.

\begin{table}
  \centering
   \begin{tabular}{|p{3cm}|p{4cm}|}
   \hline
   \textbf{Message} & \textbf{Description}\\
     \hline
\begin{verbatim}
{configPool, NIM} ->
\end{verbatim}& Initialization, the parameter NIM is the initial configuration. \\
\hline
\begin{verbatim}
{requestWork, Pid, 
   Capacity} ->
\end{verbatim} & Client requests for a population to evolve.\\
\hline
\begin{verbatim}
{generationEnd, 
  NewIndividuals,
  OldIndexes, Pid} ->
\end{verbatim} & One client report his successfully end of calculation. \\
\hline
   \end{tabular}
  \caption{The messages that a pool accepts.}\label{poolTable}
\end{table}

The clients are modeled by actors to, they are the units of evolution, whit the major calculi responsibilities; the table \ref{clientsTable} shows its interface.

\begin{table}
  \centering
   \begin{tabular}{|p{3cm}|p{4cm}|}
   \hline
   \textbf{Message} & \textbf{Description}\\
     \hline
\begin{verbatim}
initEvolution ->
\end{verbatim} & Marks the beginning of the processing. \\
\hline
\begin{verbatim}
{evolve, P, NIndexes} ->
\end{verbatim} & When the pool could assign a subpopulation to process. \\
\hline
\end{tabular}
  \caption{The messages that a client is able to respond.}\label{clientTable}
\end{table}

\subsubsection{Generality and configuration}

The two previous components have the general architecture of the algorithm, they are general and could be use for many problems. In order to solve a particular situation they most be \emph{injected} by severals functions and data structures that defines: the chromosomes, fitness, mutation, selection criteria and replacement politics. All of that particularization most be implemented in some Erlang source file and configurated in the \emph{configBuilder} module.

The design made promote a clear separation between architecture (the general, constant and paradigmatics foundation part) and problem encoding (the representation and criterias of solution finding).

\section{Experiments and results}

In order to test the 

\section{Discussion, conclusions and future work}



\section*{Acknowledgements}
This work is supported by projects TIN2011-28627-C04-02 and TIN2011-28627-C04-01 and -02 (ANYSELF), awarded by the Spanish Ministry of Science and Innovation and P08-TIC-03903 and TIC-6083 (DNEMESIS) awarded by the Andalusian Regional Government.
% Menciona tambi�n la beca de la UCI
% trigger a \newpage just before the given reference
% number - used to balance the columns on the last page
% adjust value as needed - may need to be readjusted if
% the document is modified later
%\IEEEtriggeratref{8}
% The "triggered" command can be changed if desired:
%\IEEEtriggercmd{\enlargethispage{-5in}}

% references section

\bibliographystyle{IEEEtran}
%\IEEEtriggeratref{3}
\bibliography{geneura,concurrent}

\end{document}


