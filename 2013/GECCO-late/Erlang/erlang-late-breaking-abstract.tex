% This is "sig-alternate.tex" V2.0 May 2012
% This file should be compiled with V2.5 of "sig-alternate.cls" May 2012
%
% This example file demonstrates the use of the 'sig-alternate.cls'
% V2.5 LaTeX2e document class file. It is for those submitting
% articles to ACM Conference Proceedings WHO DO NOT WISH TO
% STRICTLY ADHERE TO THE SIGS (PUBS-BOARD-ENDORSED) STYLE.
% The 'sig-alternate.cls' file will produce a similar-looking,
% albeit, 'tighter' paper resulting in, invariably, fewer pages.
%
% ----------------------------------------------------------------------------------------------------------------
% This .tex file (and associated .cls V2.5) produces:
%       1) The Permission Statement
%       2) The Conference (location) Info information
%       3) The Copyright Line with ACM data
%       4) NO page numbers
%
% as against the acm_proc_article-sp.cls file which
% DOES NOT produce 1) thru' 3) above.
%
% Using 'sig-alternate.cls' you have control, however, from within
% the source .tex file, over both the CopyrightYear
% (defaulted to 200X) and the ACM Copyright Data
% (defaulted to X-XXXXX-XX-X/XX/XX).
% e.g.
% \CopyrightYear{2007} will cause 2007 to appear in the copyright line.
% \crdata{0-12345-67-8/90/12} will cause 0-12345-67-8/90/12 to appear in the copyright line.
%
% ---------------------------------------------------------------------------------------------------------------
% This .tex source is an example which *does* use
% the .bib file (from which the .bbl file % is produced).
% REMEMBER HOWEVER: After having produced the .bbl file,
% and prior to final submission, you *NEED* to 'insert'
% your .bbl file into your source .tex file so as to provide
% ONE 'self-contained' source file.
%
% ================= IF YOU HAVE QUESTIONS =======================
% Questions regarding the SIGS styles, SIGS policies and
% procedures, Conferences etc. should be sent to
% Adrienne Griscti (griscti@acm.org)
%
% Technical questions _only_ to
% Gerald Murray (murray@hq.acm.org)
% ===============================================================
%
% For tracking purposes - this is V2.0 - May 2012

\documentclass{sig-alternate}
\usepackage[latin1]{inputenc}
%\usepackage{subfigure}
\usepackage{amsmath}
\usepackage{amssymb}
\usepackage{cite}
\usepackage{stfloats}
%\usepackage{listings}
\usepackage{url}
\usepackage[ruled,lined,linesnumbered,longend]{algorithm2e}

\def\eGA{{\small \textsc{eGA}}}
\def\eGACo{{\small \textsc{eGACo}}}
\def\eGACm{{\small \textsc{eGACm}}}
\def\codemaker{CM}
\def\codebreaker{CB}

% correct bad hyphenation here
\hyphenation{op-tical net-works semi-conduc-tor}

\begin{document}
%
% --- Author Metadata here ---
\conferenceinfo{GECCO'13,} {July 6-10, 2013, Amsterdam, The Netherlands.}
    \CopyrightYear{2013}
    \crdata{TBA}
    \clubpenalty=10000
    \widowpenalty = 10000

\title{Adapting evolutionary algorithms to the concurrent functional language Erlang}

%\subtitle{[Extended Abstract]
%\titlenote{A full version of this paper is available as
%\textit{Author's Guide to Preparing ACM SIG Proceedings Using
%\LaTeX$2_\epsilon$\ and BibTeX} at
%\texttt{www.acm.org/eaddress.htm}}}
%
% You need the command \numberofauthors to handle the 'placement
% and alignment' of the authors beneath the title.
%
% For aesthetic reasons, we recommend 'three authors at a time'
% i.e. three 'name/affiliation blocks' be placed beneath the title.
%
% NOTE: You are NOT restricted in how many 'rows' of
% "name/affiliations" may appear. We just ask that you restrict
% the number of 'columns' to three.
%
% Because of the available 'opening page real-estate'
% we ask you to refrain from putting more than six authors
% (two rows with three columns) beneath the article title.
% More than six makes the first-page appear very cluttered indeed.
%
% Use the \alignauthor commands to handle the names
% and affiliations for an 'aesthetic maximum' of six authors.
% Add names, affiliations, addresses for
% the seventh etc. author(s) as the argument for the
% \additionalauthors command.
% These 'additional authors' will be output/set for you
% without further effort on your part as the last section in
% the body of your article BEFORE References or any Appendices.


\numberofauthors{2}
 \author{
 \alignauthor
 J. Albert Cruz\\
        \affaddr{Centro de Arquitecturas Empresariales}\\
        \affaddr{Universidad de Ciencias Inform\'aticas}\\
        \affaddr{La Habana, Cuba}\\
        \email{jalbert@uci.cu}
 \alignauthor
 Juan-J. Merelo, Antonio M. Mora and Paloma de las Cuevas\\
 \affaddr{Departamento de Arquitectura y Tecnolog\'ia de Computadores}\\
 \affaddr{Universidad de Granada}\\
 \affaddr{GRanada, Spain}\\
 \email{\{jmerelo,amorag,paloma\}@geneura.ugr.es}
 }

%\numberofauthors{2}
% \author{
% \alignauthor
% J.J. Merelo, A.M. Mora, C. M. Fernandes\\
%        \affaddr{University of Granada}\\
%        \affaddr{Department of Computer Architecture and Technology, ETSIIT}\\
%        \affaddr{18071 - Granada}\\
%        \email{jmerelo,amorag,cfernandes@geneura.ugr.es}
% \alignauthor
% Anna I. Esparcia-Alc�zar\\
% \affaddr{S2 Grupo}\\
% \email{aesparcia@s2grupo.es}
% }


%\numberofauthors{4} %  in this sample file, there are a *total*
% of EIGHT authors. SIX appear on the 'first-page' (for formatting
% reasons) and the remaining two appear in the \additionalauthors section.
%

%\author{
% You can go ahead and credit any number of authors here,
% e.g. one 'row of three' or two rows (consisting of one row of three
% and a second row of one, two or three).
%
% The command \alignauthor (no curly braces needed) should
% precede each author name, affiliation/snail-mail address and
% e-mail address. Additionally, tag each line of
% affiliation/address with \affaddr, and tag the
% e-mail address with \email.
%
% 1st. author
%\alignauthor
%Jack\\
%       \affaddr{Lost island}\\
%       \affaddr{unknow}\\
%       \affaddr{Pacific Ocean}\\
%       \email{jack_the_doctor@lost.com}
% 2nd. author
%\alignauthor
%Sawyer\\
%       \affaddr{Lost island}\\
%       \affaddr{unknow}\\
%       \affaddr{Pacific Ocean}\\
%       \email{sawyer_tom@lost.com}
% 3rd. author
%\alignauthor 
%Lock\\
%       \affaddr{Lost island}\\
%       \affaddr{unknow}\\
%       \affaddr{Pacific Ocean}\\
%       \email{lock@lost.com}
% 4rd. author
%\alignauthor 
%Hurley\\
%       \affaddr{Lost island}\\
%       \affaddr{unknow}\\
%       \affaddr{Pacific Ocean}\\
%       \email{hugo@lost.com}
%}

%\and  % use '\and' if you need 'another row' of author names
% 4th. author
%\alignauthor Lawrence P. Leipuner\\
%       \affaddr{Brookhaven Laboratories}\\
%       \affaddr{Brookhaven National Lab}\\
%       \affaddr{P.O. Box 5000}\\
%       \email{lleipuner@researchlabs.org}
% 5th. author
%\alignauthor Sean Fogarty\\
%       \affaddr{NASA Ames Research Center}\\
%       \affaddr{Moffett Field}\\
%       \affaddr{California 94035}\\
%       \email{fogartys@amesres.org}
% 6th. author
%\alignauthor Charles Palmer\\
%       \affaddr{Palmer Research Laboratories}\\
%       \affaddr{8600 Datapoint Drive}\\
%       \affaddr{San Antonio, Texas 78229}\\
%       \email{cpalmer@prl.com}
%}
% There's nothing stopping you putting the seventh, eighth, etc.
% author on the opening page (as the 'third row') but we ask,
% for aesthetic reasons that you place these 'additional authors'
% in the \additional authors block, viz.
%\additionalauthors{Additional authors: John Smith (The Th{\o}rv{\"a}ld Group,
%email: {\texttt{jsmith@affiliation.org}}) and Julius P.~Kumquat
%(The Kumquat Consortium, email: {\texttt{jpkumquat@consortium.net}}).}
%\date{30 July 1999}
% Just remember to make sure that the TOTAL number of authors
% is the number that will appear on the first page PLUS the
% number that will appear in the \additionalauthors section.

\maketitle

%\begin{abstract}
%Pareto-based island model is a multi-colony distribution scheme recently presented for the resolution, by means of ant colony optimization algorithms, of bi-criteria problems. It yielded very promising results, but the model was implemented considering a unique Pareto-front-shaped unidirectional neighborhood migration topology, and a constant migration rate.
%In the present work two additional neighborhood topology schemes, and four different migration rates have been tested, considering the algorithm which obtained the best results in average in the model presentation article: MOACS (Multi-Objective Ant Colony System).
%Several experiments have been conducted, including statistical tests for better support the study. 
%High values for the migration rate and the use of a bidirectional neighborhood migration topology yields the best results.
%\end{abstract}

% A category with the (minimum) three required fields
\category{D.1.3}{Software}{Programming Techniques}[Concurrent Programming]
%A category including the fourth, optional field follows...
\category{G.1.6}{Mathematics of Computing}{Numerical Analysis}[Optimization]

% Antonio - A ver si te mola esta categor�a tambi�n:
%D.2.8 [Software Engineering]: Metrics�complexity mea-
%sures, performance measures
\terms{Algorithms}
% Antonio - no se si se puede a�adir alg�n otro t�rmino (Theory?)

\keywords{Evolutionary Algorithms, Functional Languages, Concurrent Languages, Erlang, Algorithm Implementation}


%
%%%%%%%%%%%%%%%%%%%%%%%%%%%%%%%   INTRODUCTION   %%%%%%%%%%%%%%%%%%%%%%%%%%%%%%%
%
\section{Introduction and state of the art}

Evolutionary Computation (EC) seems to be an effective method for improving, and therefore optimizing, behaviours that lead to a better use of available resources.
% Antonio - he puesto Evolutionary Computation, ya que Evolution es muy gen�rico
% Eso s�, es un comienzo un tanto raro, a qu� te refieres con los recursos disponibles. La optimizaci�n se refiere a muchos �mbitos, no s�lo a un mejor uso de los recursos... ;)
% JA - El sentido de "available resources" es: lo que tengo, sea el par�metro que sea, y quiero usar de mejor(subjetivo, lo que el usuario entienda) manera
This scientific field is very active producing libraries of diverse quality and applying it to many domains.
% Antonio - falta una frase para conectar esta con la anterior. Pasar de hablar de EC a decir que para eso (para optimizar usando EC) los cient�ficos han hecho librer�as... mejor decir que el campo de la EC es uno de los m�s prol�ficos en la creaci�n de librer�as, etc, etc...
Nevertheless it is concentrated in the use of widespread implementation technologies such as C/C++, Fortran and Java.
Getting out of that mainstream it is not normally seen a land for improvement science.
% Getting out of that mainstream it is not normally seen a land for improvement science.
% Antonio - Esta frase casi no se entiende. La intento reescribir a ver si dice lo que t� quer�as.
One no mainstream language with very much potential is Erlang, it is an implementation technology that support functional and concurrent paradigms and that's starting to been used in the scientific community \cite{Sher2013}.
% Antonio - haz una breve inroducci�na Erlang aqu� y d� si se usa o no en la comunidad cient�fica. Pon alguna cita. ;)

Genetic Algorithms (GA) \cite{GA_Goldberg89}
% Antonio -  Pongo una cita a GAs
are general function optimizers that encode a potential solution to a specific problem on a simple data structure (a chromosome). There are only two components of them that are problem dependent: the solution encoding and the function that evaluate the quality of a solution, i.e., the fitness function. The rest of the algorithm does not depend on the problem and could be implemented following the best architecture and engineering practices.

 Inside the object oriented world there are reported various analysis for modeling GA behaviour, nevertheless that is not the case of the functional side. This work tries to show some possible areas of improvement on that sense.

This article is focused on GAs as a domain of application, and describes their principal concepts and characteristics, showing how can they be modeled by means of Erlang constructs.



% Antonio - cuenta en un par de l�neas qu� es Erlang. ;)

%Other concurrent oriented languages such as Ada \cite{Santos2002} has been reported as a feasible technology for implement this kind of algorithms.
% Antonio - esto no lo digas si no lo vas a extender un poco. Queda como una frase 'hu�rfana'. ;)
% D� por ejemplo qu� desventajas tiene respecto a Erlang...


% Actualizar esto
% Antonio - s�, rev�salo porque ya no hay secci�n de resultados
The rest of the paper is organized as follows: next section
presents considerations around the functional paradigm and its conceptual relation with evolutionary computation; we will explain then the general characteristics of a pool based concurrent evolutionary evolutionary used as use case in this work. Results and conclusions obtained will be presented in Section \ref{sec:conclusions}.




\section{Evolutionary algorithm in a concurrent functional language}
\label{sec:evo}

A variety of programming patterns, i.e., paradigms, exist for implementing the algorithms models. GAs are characterized by an intensive use of strings
% Antonio - he puesto GAs porque es lo qu ehas comentado en la intro. ;)
 (lists of some kind) for encoding genes, the existence of populations that evolve by theirselves, and the variation in selection criterias through time. A programming language whose characteristics fit these needs would be highly appreciated.

\subsection{Erlang description}
% Antonio - �description?

There is a claim in modern software development for programming languages that help with concurrent programming and simplify coding practice.
% Antonio - qu� quiere decir better abstraction? acl�ralo. :D
The functional programming language Erlang would an answer that provides the actor pattern concept for concurrency and the functional paradigm for general modeling, design and coding of solutions.
% Antonio - este texto parece de publicidad de Erlang, jeje.

Actors are concurrent execution units which use asynchronous message passing for communication. They are implemented as processes in the Erlang's virtual machine
% Antonio - no has comentado qu� es la VM. No has dicho que se implementa sobre Java...
 and not like operating system (OS) threads, therefor they are very lightweight in creation and execution.
% Antonio - �qu� quiere decir que sean ligeros?
The use of messages eliminate the sharing of state and
% Antonio - d� qu� son mensajes inmutables
eliminate many of the typical problems of concurrent development, thats support the emulation of the Object Oriented (OO) paradigm with its modeling facilities.

Functional programming is defined by the use of functions in program composition and by using lists.
% Antonio - Esta frase no se entiende bien y est� mal escrita. ;)
Erlang honored the functional linage and include an ultra fast persistent technologies called Dets and Mnesia. Besides, Erlang offers
% Antonio - explica qu� es esto de macro system
the concept of records for group data.



\subsection{Genetic Algorithm mapping to Erlang}

Genetic algorithms, as many computational models, tend to be described in literature on its operational and imperative way. Their implementation in a functional language must follow a different path, structuring the algorithm model in terms less imperative and more declarative. We are going to use a parallel pool based evolutionary strategy\cite{DBLP:conf/3pgcic/GuervosMFEL12} as use case in order to show our mean.

The pool will be an execution entity (an actor acting like a server) that will own the population and also keep a track of the advance in the solution search. The clients, which are concurrent, will do the calculations and will join and leave the system at any time without consequences. Chromosomes will be encoded as lists and the different parts of the GAs algorithms will be implemented as Erlang functions.
% Antonio - functions -> actors?

An Erlang actor is implemented by a sequence of pairs pattern/expression defining each message that it could handle. It is close to the OO parlance and a way to organize the code. In this case we use one message per service that pool must provide, table \ref{poolTable} presents this.

\begin{table}
  \centering
   \begin{tabular}{|p{3cm}|p{4cm}|}
   \hline
   \textbf{Message} & \textbf{Description}\\
     \hline
\begin{verbatim}
{configPool, NIM}->
\end{verbatim}& Initialization, the parameter NIM is the initial configuration. \\
\hline
\begin{verbatim}
{requestWork, Pid,
   Capacity} ->
\end{verbatim} & Client requests for a population to evolve.\\
\hline
\begin{verbatim}
{generationEnd,
NewIndividuals,
OldIndexes, Pid}->
\end{verbatim} & One client report its successfully end of calculation. \\
\hline
   \end{tabular}
  \caption{The messages that a pool accepts.}\label{poolTable}
\end{table}

Clients are modeled by actors. They are the units of evolution, with the main computation responsibilities; the table \ref{clientTable} shows its interface.

\begin{table}
  \centering
   \begin{tabular}{|p{3cm}|p{4cm}|}
   \hline
   \textbf{Message} & \textbf{Description}\\
     \hline
\begin{verbatim}
initEvolution ->
\end{verbatim} & Marks the beginning of the processing. \\
\hline
\begin{verbatim}
{evolve, P,
  NIndexes} ->
\end{verbatim} & When the pool could assign a subpopulation to process. \\
\hline
\end{tabular}
  \caption{The messages that a client is able to respond to.}\label{clientTable}
\end{table}

\subsection{Generality and configuration}
% No hace falta tantas subdivisiones en un art�culo de dos p�ginas, caray... eliminar todas las sub y subsubsecciones - JJ
% Al dejar solamente secciones ha quedado un poco raro, digo yo, nada vean ahora y qu�tenlas si se debe...

The two previous components have the architecture of the algorithm, they are general and could be used for many problems. In order to solve a particular situation, they must be \emph{injected} by several functions and data structures which define: chromosomes, fitness function, mutation operator, selection criteria and replacement policy. All these particularizations must be implemented in an Erlang source file and configured in the \emph{configBuilder} module.

The design made promotes a clear separation between architecture (the general, constant and paradigmatic foundation part) and problem encoding (the representation and criteria of solution finding).

\section{Experiment and conclusions}
\label{sec:conclusions}

In this ongoing project we are testing the efficiency and simplicity of implementations of GAs by functional programming. The parallel models of GA are mapped to actors in the Erlang languages obtaining easily to understand architectures. All the code has been released as open source code at \url{https://github.com/jalbertcruz/erlEA/}.

The library was tested with \emph{MaxOnes} problem. The chromosomes was 128 elements long, with an initial population of 256 individuals. 50 clients was used and they worked with 20 individuals each time a pool assign a generation to them. The solution, an optimum, was reached after 3229 assignments.

With this concept test we are showing how simple is to structure a parallel GA, now we could proceed with more complex GA models, experiments and problems in order to explore the potential of the technology.

\section*{Acknowledgements}
This work is supported by projects TIN2011-28627-C04-02 and TIN2011-28627-C04-01 and -02 (ANYSELF), awarded by the Spanish Ministry of Science and Innovation and P08-TIC-03903 and TIC-6083 (DNEMESIS) awarded by the Andalusian Regional Government. It is supported too by the PhD Programm of Intelligent Systems and Softcomputing hold by the Asociaci�n Universitaria Iberoamericana de Postgrado -AUIP-, the University of Granada, Spain, and the University of Informatics Sciences, Cuba. It has also been supported by project 83, Campus CEI BioTIC http://ceibiotic.ugr.es

%
% The following two commands are all you need in the
% initial runs of your .tex file to
% produce the bibliography for the citations in your paper.
\bibliographystyle{abbrv}
\bibliography{concurrent}  % sigproc.bib is the name of the Bibliography in this case
% You must have a proper ".bib" file
%  and remember to run:
% latex bibtex latex latex
% to resolve all references
%
% ACM needs 'a single self-contained file'!
%
%APPENDICES are optional
%\balancecolumns
%\appendix
%Appendix A
%\section{Headings in Appendices}
%The rules about hierarchical headings discussed above for
%the body of the article are different in the appendices.
%In the \textbf{appendix} environment, the command
%\textbf{section} is used to
%indicate the start of each Appendix, with alphabetic order
%designation (i.e. the first is A, the second B, etc.) and
%a title (if you include one).  So, if you need
%hierarchical structure
%\textit{within} an Appendix, start with \textbf{subsection} as the
%highest level. Here is an outline of the body of this
%document in Appendix-appropriate form:
%\subsection{Introduction}
%\subsection{The Body of the Paper}
%\subsubsection{Type Changes and  Special Characters}
%\subsubsection{Math Equations}
%\paragraph{Inline (In-text) Equations}
%\paragraph{Display Equations}
%\subsubsection{Citations}
%\subsubsection{Tables}
%\subsubsection{Figures}
%\subsubsection{Theorem-like Constructs}
%\subsubsection*{A Caveat for the \TeX\ Expert}
%\subsection{Conclusions}
%\subsection{Acknowledgments}
%\subsection{Additional Authors}
%This section is inserted by \LaTeX; you do not insert it.
%You just add the names and information in the
%\texttt{{\char'134}additionalauthors} command at the start
%of the document.
%\subsection{References}
%Generated by bibtex from your ~.bib file.  Run latex,
%then bibtex, then latex twice (to resolve references)
%to create the ~.bbl file.  Insert that ~.bbl file into
%the .tex source file and comment out
%the command \texttt{{\char'134}thebibliography}.
% This next section command marks the start of
% Appendix B, and does not continue the present hierarchy
%\section{More Help for the Hardy}
%The sig-alternate.cls file itself is chock-full of succinct
%and helpful comments.  If you consider yourself a moderately
%experienced to expert user of \LaTeX, you may find reading
%it useful but please remember not to change it.
%\balancecolumns % GM June 2007
% That's all folks!



\end{document}
