% THIS IS SIGPROC-SP.TEX - VERSION 3.1
% WORKS WITH V3.2SP OF ACM_PROC_ARTICLE-SP.CLS
% APRIL 2009
%
% It is an example file showing how to use the 'acm_proc_article-sp.cls' V3.2SP
% LaTeX2e document class file for Conference Proceedings submissions.
% ----------------------------------------------------------------------------------------------------------------
% This .tex file (and associated .cls V3.2SP) *DOES NOT* produce:
%       1) The Permission Statement
%       2) The Conference (location) Info information
%       3) The Copyright Line with ACM data
%       4) Page numbering
% ---------------------------------------------------------------------------------------------------------------
% It is an example which *does* use the .bib file (from which the .bbl file
% is produced).
% REMEMBER HOWEVER: After having produced the .bbl file,
% and prior to final submission,
% you need to 'insert'  your .bbl file into your source .tex file so as to provide
% ONE 'self-contained' source file.
%
% Questions regarding SIGS should be sent to
% Adrienne Griscti ---> griscti@acm.org
%
% Questions/suggestions regarding the guidelines, .tex and .cls files, etc. to
% Gerald Murray ---> murray@hq.acm.org
%
% For tracking purposes - this is V3.1SP - APRIL 2009

\documentclass{acm_proc_article-sp}

\usepackage{url}

\begin{document}

\title{Adapting evolutionary algorithms to the concurrent functional language Erlang}

%\subtitle{[Extended Abstract]
%\titlenote{A full version of this paper is available as
%\textit{Author's Guide to Preparing ACM SIG Proceedings Using
%\LaTeX$2_\epsilon$\ and BibTeX} at
%\texttt{www.acm.org/eaddress.htm}}}
%
% You need the command \numberofauthors to handle the 'placement
% and alignment' of the authors beneath the title.
%
% For aesthetic reasons, we recommend 'three authors at a time'
% i.e. three 'name/affiliation blocks' be placed beneath the title.
%
% NOTE: You are NOT restricted in how many 'rows' of
% "name/affiliations" may appear. We just ask that you restrict
% the number of 'columns' to three.
%
% Because of the available 'opening page real-estate'
% we ask you to refrain from putting more than six authors
% (two rows with three columns) beneath the article title.
% More than six makes the first-page appear very cluttered indeed.
%
% Use the \alignauthor commands to handle the names
% and affiliations for an 'aesthetic maximum' of six authors.
% Add names, affiliations, addresses for
% the seventh etc. author(s) as the argument for the
% \additionalauthors command.
% These 'additional authors' will be output/set for you
% without further effort on your part as the last section in
% the body of your article BEFORE References or any Appendices.

\numberofauthors{3} %  in this sample file, there are a *total*
% of EIGHT authors. SIX appear on the 'first-page' (for formatting
% reasons) and the remaining two appear in the \additionalauthors section.
%
\author{
% You can go ahead and credit any number of authors here,
% e.g. one 'row of three' or two rows (consisting of one row of three
% and a second row of one, two or three).
%
% The command \alignauthor (no curly braces needed) should
% precede each author name, affiliation/snail-mail address and
% e-mail address. Additionally, tag each line of
% affiliation/address with \affaddr, and tag the
% e-mail address with \email.
%
% 1st. author
\alignauthor
J. Albert Cruz\\
       \affaddr{Centro de Arquitecturas Empresariales}\\
       \affaddr{Universidad de Ciencias Inform\'aticas}\\
       \affaddr{La Habana, Cuba}\\
       \email{jalbert@uci.cu}
% 2nd. author
\alignauthor
Juan-J. Merelo\\
       \affaddr{Departamento de Arquitectura y Tecnolog\'ia de Computadores}\\
       \affaddr{University of Granada}\\
       \affaddr{Granada, Spain}\\
       \email{jmerelo@geneura.ugr.es}
% 3rd. author
\alignauthor
Antonio Miguel Mora\\
       \affaddr{Departamento de Arquitectura y Tecnolog\'ia de Computadores}\\
       \affaddr{University of Granada}\\
       \affaddr{Granada, Spain}\\
       \email{amorag@geneura.ugr.es}
  % use '\and' if you need 'another row' of author names
}
\date{19 March 2013}
% Just remember to make sure that the TOTAL number of authors
% is the number that will appear on the first page PLUS the
% number that will appear in the \additionalauthors section.

\maketitle
%\begin{abstract}
%New languages and technologies are taken the actual multicore space. Erlang is a functional language used for build concurrent, robust soft real-time systems of many fields. Evolutionary computing provides elegant solutions for many high performance computing problems for which such technology shines. Genetic algorithms has been particulary developed and applied in several domains, libraries of diverse characteristics has been implemented and converted in valuables products.
%
%Object oriented, or even structured programming paradigm are the dominants ways in the field. In this paper, a functional oriented approach to the modeling and implementation of genetic algorithms is presented. The language Erlang is used to show how simple and efficient could be a develop of a concurrent evolutionary algorithm.
%\end{abstract}

% A category with the (minimum) three required fields
%\category{H.4}{Information Systems Applications}{Miscellaneous}
%A category including the fourth, optional field follows...
%\category{D.2.8}{Software Engineering}{Metrics}[complexity measures, performance measures]

%\terms{Theory}

%\keywords{ACM proceedings, \LaTeX, text tagging} % NOT required for Proceedings


\section{Introduction}

Evolutionary Computation (EC) seems to be an effective method for improving, and therefore optimizing, behaviours that lead to a better use of available resources. 
% Antonio - he puesto Evolutionary Computation, ya que Evolution es muy gen�rico
% Eso s�, es un comienzo un tanto raro, a qu� te refieres con los recursos disponibles. La optimizaci�n se refiere a muchos �mbitos, no s�lo a un mejor uso de los recursos... ;)
For doing so the scientific community has produced several libraries of diverse quality and applied it in many domains. 
% Antonio - falta una frase para conectar esta con la anterior. Pasar de hablar de EC a decir que para eso (para optimizar usando EC) los cient�ficos han hecho librer�as... mejor decir que el campo de la EC es uno de los m�s prol�ficos en la creaci�n de librer�as, etc, etc...
Nevertheless it is concentrated in the use of widespread implementation technologies such as C/C++, Fortran and Java. 
Getting out of that mainstream is not normally considered in the community.
% Getting out of that mainstream it is not normally seen a land for improvement science.
% Antonio - Esta frase casi no se entiende. La intento reescribir a ver si dice lo que t� quer�as.

% Antonio - haz una breve inroducci�na Erlang aqu� y d� si se usa o no en la comunidad cient�fica. Pon alguna cita. ;)

Genetic Algorithms (GA) \cite{GA_Goldberg89} 
% Antonio -  Pongo una cita a GAs
are general function optimizers that encode a potential solution to a specific problem on a simple data structure (a chromosome). There are only two components of them that are problem dependent: the solution encoding and the function that evaluate the quality of a solution, i.e. the fitness function. The rest of the algorithm does not depend on the problem and could (and should) be implemented following the best architecture and engineering practices.

On the object oriented world there are reported various analysis to model GA behaviour, nevertheless that is not the case on the functional side. This work try to show some possible areas of improvement on that sense.

This article is focused on GAs as a domain of application, and describes their principal concepts and characteristics, showing how can they be modeled by means of Erlang constructs.
% Antonio - cuenta en un par de l�neas qu� es Erlang. ;)

Other concurrent oriented languages such as Ada \cite{Santos2002} has been reported as a feasible technology for implement this kind of algorithms.
% Antonio - esto no lo digas si no lo vas a extender un poco. Queda como una frase 'hu�rfana'. ;)
% D� por ejemplo qu� desventajas tiene respecto a Erlang...


% Actualizar esto
% Antonio - s�, rev�salo porque ya no hay secci�n de resultados
The rest of the paper is organized as follows: next section
presents considerations around the functional paradigm and its conceptual relation with evolutionary computation; we will
explain then the general characteristics of a pool based concurrent evolutionary evolutionary used as use case in this work. Results obtained will be presented in Section \ref{sec:results}, and we will finish the paper
with the conclusions that derive from them (commented in Section
\ref{sec:conclusions}).




\section{Implementing an evolutionary algorithm in a concurrent functional language}
\label{sec:evo}

A varied arsenal of programming patterns, i.e., paradigms, exist for implement the algorithms models. GAs are characterized by an intensive use of strings
% Antonio - he puesto GAs porque es lo qu ehas comentado en la intro. ;)
 (lists of some kind) for encoding genes, the existence of populations that evolve by theirselves, and the variation in selection criterias through time. A programming language whose characteristics fit these needs would be highly appreciated.

\subsection{Erlang characterization}
% Antonio - �description?

There is a claim in modern software development for programming languages that help with concurrent programming and with better abstractions. 
% Antonio - qu� quiere decir better abstraction? acl�ralo. :D
The functional programming language Erlang is an answer that provides the actor pattern concept for concurrency and the functional paradigm for general modeling and design of solutions.
% Antonio - este texto parece de publicidad de Erlang, jeje.

Actors are concurrent execution units which use asynchronous messages passing for communication. They are implemented as processes in the virtual machine (VM) 
% Antonio - no has comentado qu� es la VM. No has dicho que se implementa sobre Java...
of the language's runtime and not like O.S. threads. Thus they are very lightweight in birth, live and death. 
% Antonio - �qu� quiere decir que sean ligeros?
The use of immutable messages 
% Antonio - d� qu� son mensajes inmutables
eliminate many of the typical problems of concurrent development, and support the emulation of the Object Oriented (OO) paradigm with its modeling facilities.

Functional programming has been defined by the role of functions in program composition and by the use of the very efficient lists data structure with many high order functions that operate upon them, 
% Antonio - Esta frase no se entiende bien y est� mal escrita. ;)
Erlang honored the functional linage and include an ultra fast persistent technology called Dets and Mnesia. Besides, Erlang offers a macro system, 
% Antonio - explica qu� es esto de macro system
and the concept of records for encapsulate coding patterns and group data by an entity.



\subsection{Genetic Algorithm mapping to Erlang}

Genetics algorithms, as many computational models tend to be described in literature in is operational, and imperative way. Their implementation in a functional language must follow a different path, structuring the algorithm model in terms less imperative and more declarative. We are going to use a parallel pool based evolutionary strategy\cite{DBLP:conf/3pgcic/GuervosMFEL12} as use case in order to show our mean.

The pool (a server) will be an execution entity that will own the population and keep a track of the advance in the solution search. The clients, which are concurrent, do the calculations and could join and leave at any time without consequences. Chromosomes will be encoded as lists and the different components of the GAs are implemented as Erlang functions.
% Antonio - functions -> actors?

An Erlang actor is implemented by a sequence of pairs pattern/expression that define each message that it could handle. It is close to the OO parlance and a way to organize the code. In this case we use one message per service that pool must provide, table \ref{poolTable} presents this.

\begin{table}
  \centering
   \begin{tabular}{|p{3cm}|p{4cm}|}
   \hline
   \textbf{Message} & \textbf{Description}\\
     \hline
\begin{verbatim}
{configPool, NIM}->
\end{verbatim}& Initialization, the parameter NIM is the initial configuration. \\
\hline
\begin{verbatim}
{requestWork, Pid,
   Capacity} ->
\end{verbatim} & Client requests for a population to evolve.\\
\hline
\begin{verbatim}
{generationEnd,
NewIndividuals,
OldIndexes, Pid}->
\end{verbatim} & One client report its successfully end of calculation. \\
\hline
   \end{tabular}
  \caption{The messages that a pool accepts.}\label{poolTable}
\end{table}

Clients are modeled by actors. They are the units of evolution, with the main computation responsibilities; the table \ref{clientTable} shows its interface.

\begin{table}
  \centering
   \begin{tabular}{|p{3cm}|p{4cm}|}
   \hline
   \textbf{Message} & \textbf{Description}\\
     \hline
\begin{verbatim}
initEvolution ->
\end{verbatim} & Marks the beginning of the processing. \\
\hline
\begin{verbatim}
{evolve, P, 
  NIndexes} ->
\end{verbatim} & When the pool could assign a subpopulation to process. \\
\hline
\end{tabular}
  \caption{The messages that a client is able to respond to.}\label{clientTable}
\end{table}

\subsubsection{Generality and configuration}

The two previous components have the general architecture of the algorithm, they are general and could be used for many problems. In order to solve a particular situation they must be \emph{injected} by several functions and data structures which define: the chromosomes, fitness function, mutation operator, selection criteria and replacement policy. All these particularizations must be implemented in some Erlang source file and configured in the \emph{configBuilder} module.

The design made promotes a clear separation between architecture (the general, constant and paradigmatic foundation part) and problem encoding (the representation and criteria of solution finding).

\section{Conclusions}
\label{sec:conclusions}

In this ongoing project we are testing the efficiency and simplicity of implementations of GAs by functional programming. The parallel models of GA are mapped to actors in the Erlang languages obtaining easily to understand architectures. All the code has been released as open source code at \url{https://github.com/jalbertcruz/erlEA/}.

\section*{Acknowledgements}
This work is supported by projects TIN2011-28627-C04-02 and TIN2011-28627-C04-01 and -02 (ANYSELF), awarded by the Spanish Ministry of Science and Innovation and P08-TIC-03903 and TIC-6083 (DNEMESIS) awarded by the Andalusian Regional Government. It is supported too by the PhD Programm of Intelligent Systems and Softcomputing hold by the Asociación Universitaria Iberoamericana de Postgrado -AUIP-, the University of Granada, Spain, and the University of Informatics Sciences, Cuba.

\nocite{*}
\bibliographystyle{abbrv}
\bibliography{concurrent}
\balancecolumns

\end{document}
