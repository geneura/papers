%%%%%%%%%%%%%%%%%%%%%%% file template.tex %%%%%%%%%%%%%%%%%%%%%%%%%
%
% This is a general template file for the LaTeX package SVJour3
% for Springer journals.          Springer Heidelberg 2010/09/16
%
% Copy it to a new file with a new name and use it as the basis
% for your article. Delete % signs as needed.
%
% This template includes a few options for different layouts and
% content for various journals. Please consult a previous issue of
% your journal as needed.
%
%%%%%%%%%%%%%%%%%%%%%%%%%%%%%%%%%%%%%%%%%%%%%%%%%%%%%%%%%%%%%%%%%%%
%
% First comes an example EPS file -- just ignore it and
% proceed on the \documentclass line
% your LaTeX will extract the file if required
\begin{filecontents*}{example.eps}





%!PS-Adobe-3.0 EPSF-3.0
%%BoundingBox: 19 19 221 221
%%CreationDate: Mon Sep 29 1997
%%Creator: programmed by hand (JK)
%%EndComments
gsave
newpath
  20 20 moveto
  20 220 lineto
  220 220 lineto
  220 20 lineto
closepath
2 setlinewidth
gsave
  .4 setgray fill
grestore
stroke
grestore
\end{filecontents*}
%
\RequirePackage{fix-cm}
%
%\documentclass{svjour3}                     % onecolumn (standard format)
%\documentclass[smallcondensed]{svjour3}     % onecolumn (ditto)
%\documentclass[smallextended]{svjour3}       % onecolumn (second format)
\documentclass[twocolumn]{svjour3}          % twocolumn
%
\smartqed  % flush right qed marks, e.g. at end of proof
%
\usepackage{graphicx}
\usepackage{color}
\usepackage{listings}
\usepackage{fancyvrb}
\usepackage{url}
\usepackage{fix2col}
\usepackage{natbib}
\input{highlight.sty}

\lstset{
basicstyle=\ttfamily \scriptsize,
language=java,
frame=single,
stringstyle=\ttfamily,
showstringspaces=false
}

\hyphenation{sche-me}
%
% \usepackage{mathptmx}      % use Times fonts if available on your TeX system
%
% insert here the call for the packages your document requires
%\usepackage{latexsym}
% etc.
%
% please place your own definitions here and don't use \def but
% \newcommand{}{}
%
% Insert the name of "your journal" with
% \journalname{myjournal}
%
\begin{document}

\title{Service Oriented Evolutionary Algorithms%\thanks{Grants or other notes
%about the article that should go on the front page should be
%placed here. General acknowledgments should be placed at the end of the article.}
}
%\subtitle{Do you have a subtitle?\\ If so, write it here}

%\titlerunning{Short form of title}        % if too long for running head

\author{P. Garc\'ia-S\'anchez \and
        J. Gonz\'alez \and
		P. A. Castillo \and
		M. G. Arenas \and
		J. J. Merelo
}

%\authorrunning{Short form of author list} % if too long for running head

\institute{P. Garc\'ia-S\'anchez \at
              Dept. of Computer Architecture and Computer Technology, \\
              E.T.S. Ing. Inform\'atica y Telecomunicaci\'on and CITIC-UGR\\
              University of Granada, Granada, Spain\\
              \email{pgarcia@atc.ugr.es}           %  \\
%             \emph{Present address:} of F. Author  %  if needed	
}

\date{Received: date / Accepted: date}
% The correct dates will be entered by the editor


\maketitle

\begin{abstract}
This work...

\keywords{
Evolutionary Algorithms \and
Service Oriented Architecture \and
Service Oriented Science \and
Web Services \and
Interoperability \and
Distributed computing}
% \PACS{PACS code1 \and PACS code2 \and more}
% \subclass{MSC code1 \and MSC code2 \and more}
\end{abstract}

\section{Introduction}
\label{sec:intro}




\begin{table*}[htp]
\caption{Comparison of tested EA frameworks in time and development.}

\label{tab:resume}
\begin{center}
\begin{tabular}{cccc}
\hline\noalign{\smallskip}

Name    &  Average solution    & Average Time (s)  & LoC \\
\noalign{\smallskip}\hline\noalign{\smallskip}
OSGiLiath               &   612.36 $\pm$ 6.05  & 0.19 $\pm$ 18.21 &  10\\
OSGiLiath (without OSGi)&   613.36  $\pm$ 4.50 & 0.19 $\pm$ 22.74 &  103\\
MALLBA                  &   578.76 $\pm$ 7.48  & 0.16 $\pm$ 0.0003 &  2073\\
ECJ                     &   602.76 $\pm$ 6.08   & 1.40 $\pm$ 0.03 &  5\\
Algorithm::Evolutionary &   617.60 $\pm$ 12.92  & 7.78 $\pm$ 0.29 &  41\\
\noalign{\smallskip}\hline

\end{tabular}
\end{center}
\label{tablatimes}
\end{table*}



\begin{acknowledgements}
This work has been supported in part by FPU research grant AP2009-2942 and projects AmIVital (CENIT2007-1010), EvOrq (TIC-3903), and TIN2011-28627-C04-02.
\end{acknowledgements}

\bibliographystyle{spbasic}
\bibliography{heterogeneous} 




\end{document}
% end of file template.tex

