\documentclass[a4paper]{article}
\usepackage{url}
\usepackage{apalike}

\begin{document}
\title{Enhancing cloud computing with self-adaptive virtualization}

\author{J. J. Merelo, P. García-Sáncnez\\
GeNeura team, \url{http://geneura.ugr.es}\\
Dept. Computer Architecture and Technology, \url{http://atc.ugr.es}\\
University of Granada, \url{http://www.ugr.es}}

\onecolumn \maketitle \normalsize \vfill


\abstract{Nowadays any company can configure to its wishes a virtual machine and run a payload on it. The payload will be, in principle, independent of the VM configuration. However, modern optimization and virtualization techniques will allow, in the near future, an adaptation of VM characteristics, or even architecture, to optimally match the payload in terms of cost, energy consumption or, of course, payload performance.

This project will aim at creating self-adaptive virtualization tools that, given a the payload stated characteristics and run time behavior, will be able to enhance the offer of virtual data centers via optimization of virtual machine characteristics. This optimization will run at different levels, from virtual machine architecture (even possibly microarchitecture) to virtualization technology through virtual machine parameters. These last can be done nowadays and will be the target of the first part of the project; the rest will be devoted to advances on software-defined hardware, Platform as a Service and a software architecture that allows the building of such kind of systems.

This project will advance the state of the art in self-adaptive systems through the creation of algorithms that are able to improve performance using runtime data; it will also advance the state of the art in cloud computing taking virtualization technologies forward to the point of being self-adaptive.

In the spirit that has taken the cloud to the state it is now, the eventual results of this project will be released as open source.}

{\bf Keywords}: Optimization, cloud computing, evolutionary algorithms, self-* properties

\end{document}
