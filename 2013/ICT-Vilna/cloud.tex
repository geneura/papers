\documentclass[a4paper]{article}


\usepackage{url}


\begin{document}
\title{Enhancing cloud computing with self-adaptive virtualization}

\author{J. J. Merelo, P. Garc\'ia-S\'anchez\\
GeNeura team, http://geneura.ugr.es\\
Dept. Computer Architecture and Technology, http://atc.ugr.es\\
University of Granada, http://www.ugr.es}

\onecolumn \maketitle \normalsize 


\abstract{Nowadays any company can configure to its wishes a virtual machine and run a payload on it. The payload will be, in principle, independent of the VM configuration. However, modern optimization and virtualization techniques will allow, in the near future, an adaptation of VM characteristics, or even architecture, to optimally match the payload in terms of cost, energy consumption or, of course, payload performance.

This project will aim at creating self-adaptive virtualization tools that, given a the payload stated characteristics and run time behavior, will be able to enhance the offer of virtual data centers via optimization of virtual machine characteristics. This optimization will run at different levels, from virtual machine architecture (even possibly microarchitecture) to virtualization technology through virtual machine parameters. These last can be done nowadays and will be the target of the first part of the project; the rest will be devoted to advances on software-defined hardware, Platform as a Service and a software architecture that allows the building of such kind of systems.

This project will advance the state of the art in self-adaptive systems through the creation of algorithms that are able to improve performance using runtime data; it will also advance the state of the art in cloud computing taking virtualization technologies forward to the point of being self-adaptive.

In the spirit that has taken the cloud to the state it is now, the eventual results of this project will be released as open source.}

{\bf Keywords}: Optimization, cloud computing, evolutionary
algorithms, self-* properties

\section{Introduction}

Cloud computing is one of the preferred ways of deploying enterprise
applications and they are increasingly used as the back-office of all
kind of applications, from Software as a Service-delivered ones to
mobile applications. Most of them, however, do not take advantage of
the fact that cloud-based virtual machines are software-defined
resources and, in many cases, its features can be changed either when
they are provisioned or during run time.

This project proposes the creation of adaptive cloud middleware that
is able to, in the provisioning phase, match the infraestructure to
the applications they are going to be loading and, during runtime,
adapt that infraestructure (and its underlying bare-metal hardware) to
its quality of service and performance requirements. 

\section{State of the art}

A quick look at the commercial offerings will present just {\em fixed
  geometry} offerings, from next-to-bare metal VPS (Virtual Private
Servers) to IaaS, PaaS or SaaS that are statically provisioned and
launced by the client.

However, there's been some research on the topic in the last few
years. Papakos et al. \cite{Papakos} proposes
Volare, a context-aware adaptive cloud sevice specially geared for
mobile systems, with its special characteristics. In thie project we
would aim at extending this with a self-aware cloud service that would
adapt, with minimal human intervention, to any kind of payload. Iqbal
et al. \cite{Iqbal2011871} use synthetic workloads to provision in
such a way that a certain Service-Level Agreement can be reached. In
our approach, we would target real-life workloads and, specifically,
whatever workload is actually been created by a customer in order to
optimize whatever features is desired by the client.

\section{Our experience}

GeNeura team (\url{http://geneura.wordpress.com}) has an experience with
distributed architectures for a long time, dating back to the early
nineties \cite{parallel90} and going through web services and
distributed evolutionary algorithms until recently
\cite{Jini:FEA2000,agajaj,LNCS44480129,Araujo2010}

We also have experience in performance evaluation and prediction
\cite{castillo:evostar08,hardwareevo}, as well as in the optimization of
software-defined architectures \cite{gecco08:castillo}.

Recently, we have used cloud computing for distributed evolutionary
computation \cite{sofea:naco,mericloud} and web services
\cite{DBLP:journals/soco/Garcia-SanchezGCAG13}; this has been enhanced
lately 
by the ellaboration of a class syllabus related to cloud computing. 


\bibliographystyle{plain}
\bibliography{cloud,geneura}
\end{document}
