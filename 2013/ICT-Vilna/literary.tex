\documentclass[a4paper]{article}
\usepackage{url}

\begin{document}
\title{Literary engineering}

\author{J. J. Merelo, P. García-Sáncnez\\
GeNeura team, \url{http://geneura.ugr.es}\\
Dept. Computer Architecture and Technology, \url{http://atc.ugr.es}\\
University of Granada, \url{http://www.ugr.es}}

\onecolumn \maketitle \normalsize 


\abstract{This project proposes the construction of a methodology and
  set of tools that aim to assess the quality, identify authors and,
  eventually, use those measures for the improvement of written texts
  (including hypertext), bringing the era of computer-aided
  copy-writing and literature in the same way computer-aided drawing
  has helped, and continues to help, art as well as engineering.}

{\bf Keywords}: Optimization, text mining, evolutionary
algorithms.

\section{Introduction}

The editorial industry is in turmoil since the extensive consumption
of ebooks and the creation of self-publishing platforms such as Kindle
Direct Publishing, iTunes and Lulu. These platforms imply that many more actors are accessing the market, but at the same time, due to low profit levels, it is complicated or even impossible to live professionally off it, since hundreds of daily sales are needed for even a minimum wage.

At the same time, the quality of these works is variable. The
publishing industry has many tools: correctors, editors, career
managers and translators that are not available to single, independent
authors, or are at a high cost. These {\em literary tools} are mainly manual and have not been
automated beyond grammar and spelling corrections. Translation tools,
at the same time, have eventually to be reviewed by a human to reach a
good level of quality. 

This project proposes the creation of methodologies, algorithms and
software tools that aid in automation of the whole creation and
copyediting process:
from idea to final typesetting, and even improvement in new editions
based on the automatic processing of literary reviews. 

This project will match natural language processing tools for
analyzing original works as well as reviews, ontologies and thesaurus
to improve the quality of written text based on several metrics: similarity to
text-mined classic works and diversity and {\em style} parameters
assigned automatically to the work, and eventually improvement of the
manuscript using metaheuristics such as evolutionary algorithms or
simulated annealing. In the case of published text, {\em quality}
assessment can include metrics that depend on the perceived fitness of
the text, including reviews and, in the case of description of
products, sales.

The project involves the creation of new algorithms for text
improvement, sentiment analysis to gauge the reaction to a particular
work mining social networks and review sites, and automatic quality
evaluation based on text mining, as well as development of open-source
literary engineering tools that can be used by authors as a help to
creation as well as editorials as decision-support systems.  

Several scenarios are foreseen: an author using the tool to improve a
literary manuscript as well as creating new editions based on social
network reviews; a games creator using the tool to improve backstories
for characters; interactive fiction apps or websites that change story
based on user interaction and reaction and automatic adaptation of
text description of products in an online website depending on the
customer profile or the interaction of customers with it. 

\section{State of the art}

Proactive evolution of text for its optimization has been done so far
in a very controlled environment, such as technical texts
\cite{Rascu06acontrolled,hernandez2004checking}. It has to be preceded by a set of analysis
that include complex networks
\cite{1367-2630-14-4-043029,0295-5075-100-5-58002} or writing-style
features \cite{ASI:ASI20316}. Incorporation of interactive features
has not been done so far, although nowadays is an easily available
metric. Even more so, criticism can also be gauged to measure the
quality of the text it is talking about. Finally, personalization of
text, so far, has not been done to the best of our knowledge.

That is why this project will advance the state of the art in several
areas, be them algorithmic, methodological or purely technical.

\section{Our experience}

GeNeura team (http://geneura.wordpress.com) includes a published
writer that has won literary prices in its cast, which has them
interested in this kind of things. They have been working in the
advertising industry some time in the past
\cite{merelo:ecal97,AISB97}, but also have experience in interactive
art \cite{DBLP:conf/cec/TrujilloVVG13,DBLP:conf/cec/FernandesIBRG11}.


We also have experience in performance evaluation and prediction
\cite{castillo:evostar08,hardwareevo}, as well as in the optimization of
software-defined architectures \cite{gecco08:castillo}.

Our complex systems experience arises from our early interest in
artificial life \cite{ecal93} but have lately applied complex network
analysis to co-authorship in particular areas
\cite{ec-network-2007,merelo2013complex,DBLP:journals/corr/abs-1108-0261}. 


\bibliographystyle{plain}
\bibliography{literary,geneura}
\end{document}
