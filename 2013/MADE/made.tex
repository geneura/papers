%%
%% Copyright 2007, 2008, 2009 Elsevier Ltd
%%
%% This file is part of the 'Elsarticle Bundle'.
%% ---------------------------------------------
%%
%% It may be distributed under the conditions of the LaTeX Project Public
%% License, either version 1.2 of this license or (at your option) any
%% later version.  The latest version of this license is in
%%    http://www.latex-project.org/lppl.txt
%% and version 1.2 or later is part of all distributions of LaTeX
%% version 1999/12/01 or later.
%%
%% The list of all files belonging to the 'Elsarticle Bundle' is
%% given in the file `manifest.txt'.
%%

%% Template article for Elsevier's document class `elsarticle'
%% with numbered style bibliographic references
%% SP 2008/03/01
%%
%%
%%
%% $Id: elsarticle-template-num.tex 4 2009-10-24 08:22:58Z rishi $
%%
%%
%%%%%\documentclass[preprint,12pt]{elsarticle}

%% Use the option review to obtain double line spacing
%%  \documentclass[preprint,review,12pt]{elsarticle}

%% Use the options 1p,twocolumn; 3p; 3p,twocolumn; 5p; or 5p,twocolumn
%% for a journal layout:
 \documentclass[final,1p,times]{elsarticle}
%% \documentclass[final,1p,times,twocolumn]{elsarticle}
%% \documentclass[final,3p,times]{elsarticle}
%% \documentclass[final,3p,times,twocolumn]{elsarticle}
%% \documentclass[final,5p,times]{elsarticle}
%%\documentclass[final,5p,times,twocolumn]{elsarticle}%%DOS COLUMNAS

%% if you use PostScript figures in your article
%% use the graphics package for simple commands
%% \usepackage{graphics}
%% or use the graphicx package for more complicated commands
%% \usepackage{graphicx}
%% or use the epsfig package if you prefer to use the old commands
%% \usepackage{epsfig}

%% The amssymb package provides various useful mathematical symbols
\usepackage{amssymb}
\usepackage{url}
\usepackage{graphicx}
\usepackage{epsfig}
\usepackage{algorithm}
\usepackage{algorithmic}
\usepackage{subfigure}
\usepackage{color}
\usepackage{amsmath}
%% The amsthm package provides extended theorem environments
%% \usepackage{amsthm}

%% The lineno packages adds line numbers. Start line numbering with
%% \begin{linenumbers}, end it with \end{linenumbers}. Or switch it on
%% for the whole article with \linenumbers after \end{frontmatter}.
%% \usepackage{lineno}

%% natbib.sty is loaded by default. However, natbib options can be
%% provided with \biboptions{...} command. Following options are
%% valid:

%%   round  -  round parentheses are used (default)
%%   square -  square brackets are used   [option]
%%   curly  -  curly braces are used      {option}
%%   angle  -  angle brackets are used    <option>
%%   semicolon  -  multiple citations separated by semi-colon
%%   colon  - same as semicolon, an earlier confusion
%%   comma  -  separated by comma
%%   numbers-  selects numerical citations
%%   super  -  numerical citations as superscripts
%%   sort   -  sorts multiple citations according to order in ref. list
%%   sort&compress   -  like sort, but also compresses numerical citations
%%   compress - compresses without sorting
%%
%% \biboptions{comma,round}

% \biboptions{}


\journal{Applied Soft Computing}
\providecommand{\e}[1]{\ensuremath{\times 10^{#1}}}
\begin{document}

\begin{frontmatter}

%% Title, authors and addresses

%% use the tnoteref command within \title for footnotes;
%% use the tnotetext command for the associated footnote;
%% use the fnref command within \author or \address for footnotes;
%% use the fntext command for the associated footnote;
%% use the corref command within \author for corresponding author footnotes;
%% use the cortext command for the associated footnote;
%% use the ead command for the email address,
%% and the form \ead[url] for the home page:
%%
%% \title{Title\tnoteref{label1}}
%% \tnotetext[label1]{}
%% \author{Name\corref{cor1}\fnref{label2}}
%% \ead{email address}
%% \ead[url]{home page}
%% \fntext[label2]{}
%% \cortext[cor1]{}
%% \address{Address\fnref{label3}}
%% \fntext[label3]{}

\title{MADE: Massive Artificial Drama Engine for non player characters}

%% use optional labels to link authors explicitly to addresses:
%% \author[label1,label2]{<author name>}
%% \address[label1]{<address>}
%% \address[label2]{<address>}
\author[fidesol]{Rub\'en H\'ector Garc\'ia Ortega} %pon tu nombre en bonico
\ead{rhgarcia@fidesol.org} %y tu correo
\author[ugr]{Pablo Garc\'ia-S\'anchez}
\ead{pgarcia@atc.ugr.es}
\author[ugr]{Juan Juli\'an Merelo}
\ead{jmerelo@geneura.ugr.es}

\address[fidesol]{Fundaci\'on I+D del Software Libre, Granada, Spain} %Pon lo que quieras aquí
\address[ugr]{Department of Computer Architecture and Computer Technology and CITIC-UGR, University of Granada, Granada, Spain. Tel: +34958241778. Fax: +34958248993}



\begin{abstract}
This paper presents a study on ...

\end{abstract}

\begin{keyword}
%% keywords here, in the form: keyword \sep keyword
%Service Oriented Architecture \sep OSGi \sep Java \sep Context Management \sep e-health
Evolutionary Algorithms \sep Interactive Drama \sep Multi-agent systems%Meter más
%% MSC codes here, in the form: \MSC code \sep code
%% or \MSC[2008] code \sep code (2000 is the default)
\end{keyword}

\end{frontmatter}

\section{Introduction}
\label{sec:intro}


In videogames, NPCs (Non Player Characters) are a type of characters that live in the game world to provide a more inmersive experience. Modern RPGs (Role Playing Games), such as The Witcher\texttrademark or Skyrim\texttrademark count with hundreds of NPC characters. The effort to create a good interactive fiction script is directally proportional to the number of these characters. That is the reason this kind of agents usually counts with limited behaviours, such as wandering in the villages, selling groceries or guarding the cities. Also, they usually offer scripted conversations, for example, for buy and sell objects to the player. In other cases they interact with the player depending of the player's behaviour: for example, if the player steals something a city guard would attack him.  However, these characters do not interact among them, only with the player, and their activities are only guided with this purpose. In a world with such a number of characters, their collective interactions could improve the gaming experience, leading to a richier and more inmersive world. For example, hungry inhabitants could become thieves, guards could pursuit the thieves, villagers could fell in love with others or different war alliances could emerge.

These facts have motivated us to develop a multi-agent system called MADE (Massive Artificial Drama Engine) to model a self-organized virtual world where their elements influences each other, following a cause-effect behaviours in a coherent manner. This system needs to be a suitable environment for the plot of a specific literary work, being also interesting for the player/spectator. A set of probabilities and states are associated to agents' actions, and these probabilities are optimized by means of an Evolutionary Algorithm (EA) to match with a specific literary archetype, defined by the fiction creator. The archetypes are behaviours and patterns universally accepted and present in the collective imaginary \cite{ArchetypesGarry05}, that allows empathize with the characters and immerse yourself in the story (for example, the well-known ``hero'' archetype).

In this work, several experiments have been carried out to give an insight to the following questions:
\begin{itemize}
 \item Is it possible to model a virtual world, inhabited by thousands of characters, with self-generated and interesting plots?
 \item Can different and interesting literary archetypes emerge in this system?
\end{itemize}


The rest of the work is structured as follows: after a presentation of
the state of
the art in  the area of interactive fiction
 we present the MADE system. Then, we present developed algorithms and experimental setting (Section \ref{sec:experiments}). 
Then, the results of the experiments are shown (Section \ref{sec:results}), followed by conclusions and suggestions for future work lines.


%%%%%%%%%%%%%%%%%%%%%%%%%%%%%%  SOA  %%%%%%%%%%%%%%%%%%%%%%%%%%%%%%
%
\section{State of the art}
\label{sec:soa}
%
Por cierto, acabo de ver esto: \cite{StoryTecGobel2008}.

Auto-generated interactive fiction research is mainly focused in methods to create the process of a story generation \cite{nairat2011character}. Story generation can be divided in two areas: interactive and non-interactive. In the first area, and according to \cite{ReviewArinbjarnar09}, an Interactive Drama is defined in a virtual world where the user has freedom to interact with the NPCs and objects in a dramatically interesting experience, different in each execution, and adapted to the interactions of the user.

As opossed to this concept, MADE is focused in Artificial No Interactive Drama, because its aim is the massive generation of plots for secondary characters, to provide a context for the writer and the player to perceive a virtual world as coherent, detailed and enriched. The story generation (that is, the narrative) is not adressed by MADE, but it has been studied in the systems presents in the survey by Arinbjarnar et al. in \cite{ReviewArinbjarnar09}.

The generation of interactive dramas can also be based in script structure \cite{ArchitectureYoung04}, where each possibility in the story must be previously defined, so there is a limited number of possible plot combinations. There exist other techniques, not based in plot structure, such as...

On the other side, in non-interactive plot generation systems the user does not take control as the protagonist. For example, in the system presented by Pizzi et al. \cite{pizzi2007interactive} the user can interact with the characters, changing their emotions, but making the user an spectator, rather as an actor. %ESTO NO ME QUEDA TAMPOCO CLARO, SI NO ES INTERACTIVA COMO MODIFICAMOS LOS PERSONAJES?

Previous works define the plot as an emergence for the behaviour of the agents that follow a set of rules. In MADE, the agents' behaviour is product of its personality and the environment. That is, the agents does not follow the plot, but they generate the plot itself. %%% ESTO NO LO TENGO MUY CLARO

Futhermore, the previous works generate plots in worlds with a limited number of characters. This restriction does not exist in MADE, where the number of characters to create is unlimited.


%%%%%%%%%%%%%%%%%%  MADE  %%%%%%%%%%%%%%%%%%%
\section{MADE}
\label{sec:made}

%%%%%%%%%%%%%%%%%%  Genetic Algorithm  %%%%%%%%%%%%%%%%%%%
\section{Genetic algorithm for the search of emergent archetypes}
\label{sec:ga}

%%%%%%%%%%%%%%%%%%  Experiments  %%%%%%%%%%%%%%%%%%%
\section{Experimental setup}
\label{sec:experiments}

\subsection{Parameter setting for MADE}

\subsection{Parameter setting for Experiment A}

\subsection{Parameter setting for Experiment B}


%%%%%%%%%%%%%%%%%%  Results  %%%%%%%%%%%%%%%%%%%
\section{Results}
\label{sec:results}


%%%%%%%%%%%%%%%%%%  Conclusions  %%%%%%%%%%%%%%%%%%%
\section{Conclusions}


\section*{Acknowledgements}
This work has been supported in part by FPU research grant AP2009-2942 and projects EvOrq (TIC-3903), CANUBE (CEI2013-P-14) and ANYSELF (TIN2011-28627-C04-02).
%METER AQUI PROYECTOS DE LA FUNDACIÓN SI HACEN FALTA


%% The Appendices part is started with the command \appendix;
%% appendix sections are then done as normal sections
%% \appendix

%% \section{}
%% \label{}

%% References
%%
%% Following citation commands can be used in the body text:
%% Usage of \cite is as follows:
%%   \cite{key}         ==>>  [#]
%%   \cite[chap. 2]{key} ==>> [#, chap. 2]
%%

%% References with bibTeX database:

%\bibliographystyle{elsarticle-num}
%\bibliography{AMIVITAL-ESA}

%% Authors are advised to submit their bibtex database files. They are
%% requested to list a bibtex style file in the manuscript if they do
%% not want to use elsarticle-num.bst.

%% References without bibTeX database:

% \begin{thebibliography}{00}

%% \bibitem must have the following form:
%%   \bibitem{key}...
%%

% \bibitem{}

% \end{thebibliography}




%\bibliographystyle{plain}
%\bibliography{heterogeneous}
\section*{References}

\bibliographystyle{plain}
\bibliography{made}

\end{document}

%%
%% End of file `elsarticle-template-num.tex'.
