\documentclass[runningheads]{llncs}
\usepackage{amssymb}
\setcounter{tocdepth}{3}
\usepackage{graphicx,epsfig}
\usepackage{rotating}
\usepackage{subfig}
\usepackage{listings}
%%%%

\usepackage{color}
\usepackage{alltt}
\usepackage{verbatim}
\usepackage{url}
\usepackage[utf8]{inputenc}
\usepackage[spanish,es-noshorthands]{babel}

%%

\usepackage{url}

%\urldef{\mailsa}\path|pgarcia@atc.ugr.es|

\urldef{\mailsa}\path|ANONYMOUS@SECRET.COM|

\newcommand{\keywords}[1]{\par\addvspace\baselineskip
\noindent\keywordname\enspace\ignorespaces#1}

\lstset{
basicstyle=\ttfamily \scriptsize,
language=c++,
frame=single,
stringstyle=\ttfamily,
showstringspaces=false
}

\renewcommand{\textfraction}{0}
\renewcommand{\topfraction}{1}
\renewcommand{\bottomfraction}{1}
\renewcommand{\floatpagefraction}{0.9}

\begin{document}

\mainmatter  % start of an individual contribution



% first the title is needed
\title{Predicción de tr\'afico mediante co-evolución de Redes Neuronales de Funciones de Base Radial y selección de variables de entrada \thanks{
Este trabajo ha sido realizado gracias a la beca FPU AP2009-2942 y los
proyectos EvOrq (P08-TIC-03903), Proyecto 83 (CANUBE) del CEI-BioTIC
UGR, y TIN2011-28627-C04-02 (ANYSELF). Desarrollado gracias a la financiación del proyecto FEDER de la Unión Europea con título "Sistema de Información y Predicción de bajo coste y autónomo para conocer el Estado de las Carreteras en tiempo real mediante dispositivos distribuidos" (SIPEsCa) del Programa Operativo FEDER de Andalucía 2007-2013. Asimismo, queremos mostrar nuestro agradecimiento al personal e investigadores de la Agencia de Obra Pública de la Junta de Andalucía, Consejería de Fomento y Vivienda, por su dedicación y profesionalidad. }
}
%EvOrq (TIC-3903), CEI BioTIC GENIL (CEB09-0010), MICINN CEI Program (PYR-2010-13) and FPU research grant AP2009-2942 .
% También el CANUBE
% No he podido compilarlo porque falta la bibliografía. ¿Podéis poner el PDF?

% a short form should be given in case it is too long for the running head
\titlerunning{Predicción de tr\'afico mediante co-evolución}
% Tampoco es muy informativo este título y ocupa 15 páginas. ¿Cuál es el límite? - JJ
\author{No author given}
%\author{Anonymous\inst{1}, A. E. Eiben\inst{2}, E. Haasdijk\inst{2}, B. Weel\inst{2} and J.J. Merelo\inst{1}}

%

%\authorrunning{P. Garc\'ia-S\'anchez et al.}
\authorrunning{Anonymous.}
% (feature abused for this document to repeat the title also on left hand pages)
% the affiliations are given next; don't give your e-mail address
% unless you accept that it will be published

%\institute{Dept. of Computer Architecture and Technology, University of Granada, Spain \and Dept. of Computer Science, Vrije Universiteit Amsterdam, The Netherlands
%\mailsa}
\institute{No institute given\mailsa}




%\toctitle{BLABLABLA}

%\tocauthor{Authors' Instructions}
\maketitle
\begin{abstract}
Este trabajo presenta un novedoso sistema de detección de flujo de tráfico, y muestra los resultados del algoritmo L-Co-R en la predicción de la serie temporal que el citado flujo de tráfico representa. La predicción de la aserie temporal se ha llevado comparando con algoritmos existente screados para tal efecto y muestra la adecuación delal algoritmo creado, L-Co-R, ante el problema de la predcción de esta serie temporal.
\end{abstract}


\section{Introducci\'on}
\noindent 
El desarrollo de nuevos sistemas que informen de las condificiones del tráfico y del flujo de
veh\'iculos en las carreteras, es una apuesta de la Dirección General de Tráfico (DGT) dentro del
territorio nacional y de las administracciones homólogas fuera del territorio nacional. Ejemplos de
esta apuesta son los proyectos desarrollados para informar lo más exactamente posible
acerca del estado del tráfico, como el que lleva a cabo Google Maps que proporciona desde hace no
mucho tiempo (2011) información del estado del tráfico en las ciudades más importantes de algunos
% cita - JJ
paises, aunque el proyecto inicial empezó sólo en algunas ciudades de EEUU (2009). Otros ejemplos
nacionales son la apuesta de la DGT por la investigación en tráfico financiando proyectos y
promocionando encuentros sobre movilidad y seguridad vial.  
% citas - JJ

Paralelamente a esta circunstancia, la población actual está evolucionando y se está convirtiendo
en una población permanentemente informada, conectada y ubícua, puesto que en España existen
actualmente más de $52$ millones de clientes de telefonía móvil
(\url{http://goo.gl/WTPN3}). Esta característica permite que casi cualquier
conductor utilice a diario esos dispositivos de comunicación y por lo tanto lo lleven siempre con
ellos mismos para comunicarse tanto de forma hablada como escrita, informarse, consultando
cualquier dato que les pueda interesar (gracias a las tarifas de datos), o utilizando cualquier
otro servicio accesible desde el dispositivo. 

En España existían en 2012 165.568  Km de vías (último dato oficial publicado)
\cite{DatosTrafico2012}, lo que da una idea muy clara de la complejidad de una red, puesto que,
aunque sea sólo por la extensión necesita muchísimo trabajo para su coordinación, su mantenimiento y
su gestión en general. Por lo que al organismo que gestiona la red le es de mucha
utilidad tener información lo más precisa posible, por básica que sea sobre los usuarios de la
utilizan, para poder preveer situaciones de saturación, o de evacuación en caso de alguna
emergencia. Datos tan básicos como número de vehículos que pasan por cada via o a qué hora hay más
afluencia para intentar regular la señalización de cada vía podrían facilitar el trabajo de los
operarios.

En este trabajo se presenta un sistema de recogida de información de estado del tráfico y se
analizan los datos recogidos incluyendo una predicción inicial de las series temporales de datos
recogidas. Una serie temporal es un conjunto de datos
ordenados de forma cronológica que sirve de base para realizar una predicción de valores futuros
dentro de esa misma serie temporal. La predicción de series temporales es un campo de investigación
activo dentro de areas como la estadística o la investigación operativa \cite{Fildes2008},
\cite{Gooijer25years}, y son datos presentes tanto en el campo de la ingeniería como en la
biología, economía o las ciencias sociales entre otras.

La captación de datos se basa en la detección de dispositivos \textit{bluetooth} (BT),
en diferentes vias, lo que nos proporciona una descripción de las condiciones de tráfico en tiempo
real y un conjunto de datos válidos para utilizarlos con técnicas de predicción de series
temporales. La captación de datos incluye tanto dispositivos BT empotrados en algún tipo de
vehículo como dispositivos de usuarios particulares que lleven el bluetooth activado. El principal
dato recolectado de cada dispositivo es la dirección MAC que se encripta y se almacena, lo que
permite identificar de forma unívoca el dispositivo, pero no el usuario, por lo que la privacidad de
la persona usuaria del dispositivo está totalmente reservada. Esta identificación única puede
permitir la identificación del dispositivo en varios puntos de la ciudad, lo que puede dar una idea
del tipo de desplazamientos que ese usuario ha realizado. 

Concretamente el dispositivo utilizado para la captación de datos es Intelify (\url{http://www.intelify.net/}) que ofrece una solución compacta para captar
dispositivos BT y transmitir la información a un servidor remoto manteniendo un índice de detección
adecuado ( incluye un $8.5\%$ de error en la captación). Intelify es un dispositivo autónomo que puede ser instalado en el área donde se desean captar
datos. % No se va a instalar en otra - JJ
Dispone de varios sensores y capta la información para enviarla a través de una conexion 3G. 
Los datos recopilados que se incluyen en este trabajo han sido captados por 5 puntos diferentes
(nodos 1-6) durante un total de 60 días. 

En este primer trabajo se han aplicado varios algoritmos de predicción de series temporales ya
validados en la bibliografía y se presentan los resultados iniciales para predicción a corto plazo
(1 día) sin modificar, tratar o adecuar la serie de datos inicialmente recopilada por el dispositivo
de captación. 

El objetivo final de este proceso de investigación inicial es predecir el flujo de tráfico que
tendrán las vías monitorizadas con este tipo de dispositivo o cualquier otro que proporcione una
información simple y fiable a cerca del paso de vehículos por un determinado punto. Esta predicción
permitirá, por ejemplo, poder tomar decisiones
acerca del mejor itinerario a elegir para un determinado desplazamiento, tanto para los usuarios
de las vias como para las administraciones a la hora de planificar señalización, e intervención en
las vías monitorizadas.

% ¿Sección en mayúscula? JJ
El artículo a partir de este punto se organiza en secciones. La sección \ref{sec:soa} repasa el
estado del arte tanto de los algoritmos de de predicción de series temporales como de los
dispositivos de captación de datos existentes actualmente. Por su parte, la sección
\ref{sec:experiments} presenta los resultados obtenidos en la
predicción a corto plazo del flujo del tráfico. Por último en la última sección se presentan
unas conclusiones iniciales y los pasos que se pondrán en práctica como líneas
futuras de trabajo.

%%%%%%%%%%%%%%%%%%%%%%%%%%%%%% BACKGROUND and SOTA %%%%%%%%%%%%%%%%%%%%%%%%%%%%%

\section{Estado del arte}
\label{sec:soa}

En esta sección se repasarán de forma resumida los dispositivos de captación de tráfico existentes
y posteriormente introduciremos un pequeño estado del arte sobre los algoritmos de predicción de
series temporales que están vigentes. 


Las tecnologías que permiten detectar el flujo de tráfico se clasifican en dos grupos:
intrusivas y no intrusivas.  Las intrusivas implican una modificación física de la via que
monitoriza y esta característica se convierte en su principal inconveniente, puesto que la via debe
ser cerrada para su instalación además de un posterior manteninimiento.  En cualquier caso estas
técnicas sólo cuentan vehículos que atraviesan el sistema de monitorización, sin identificarlos de
ninguna forma. En este grupo se incluyen por ejemplo los sensores piezoelectricos o los tubos
neumáticos. Las tecnologías no intrusivas no producen interrupción del tráfico, puesto que su
colocación no está físicamente en el pavimento, sino en zonas adyacentes de la via como paneles
informativos, carteles de señalización, puestos de emergencias, etc.  En este último grupo se
incluyen los radares, la señalización lumínica o las cámaras de monitorización. 

 
El inconveniente principal de este sistema de monitorización es que no permiten identificar los
vehículos de forma unívoca para poder construir patrones de desplazamiento o matrices de frecuencia
de destinos para un conjunto de orígenes dados. 

% ------------------------------------------------------------------------------
Respecto al estado del arte de los algoritmos de predicción, existen muchos trabajos relacionados
con el tema y que han sido ampliamente utilizados con conjuntos de datos públicos con los que han
sido testeados por sus desarrolladores. Todos los algoritmos de predicción intentan modelar un
sistema numérico que represente el comportamiento de la serie y con el pueda predecirla. Estos
modelos pueden ser lineales o no lineales. 
% Ya vamos con la enumeración. El estado del arte se debe contar, no enumerar - JJ
Dentro del grupo de los lineales hay varios, pero el más conocido es el llamado método ARIMA
\cite{BoxJenk}. ARIMA es un modelo simple y cuyo funcionamiento está ampliamente demostrado, pero
no funciona muy bien con muchas aplicaciones a series temporales reales, puesto que su ajuste no es
fácil y lo debe de realizar operarios especializados. Esta es la principal razón de que surgieran
los métodos no lineales. 

Los modelos no lineales se ajustan mejor a aplicaciones reales aunque la facilidad de uso sigue
siendo escasa \cite{Clements2004}, de ahí que existan propuestas donde se recomienda que se le debe
dar prioridad a la facilidad de uso frente a otros aspectos más técnicos \cite{Gooijer25years}. % ¿qué resultados obtienen? ¿Qué problemas? 

La predicción de series temporales también se ha abordado desde otros campos, como técnicas de
Soft-Computing, \cite{Samanta2011}, \cite{Zhu2011}; Técnicas Fuzzy \cite{Qiu2011}, \cite{Wang2011};
redes neuronales, \cite{Yu2010}; Regresión \cite{Kavaklioglu2011}; y sistemas expertos
\cite{Dash1995}.

En nuestro caso nos hemos centrado en métodos no lineales basados en la utilización de Redes
Nuronales, puesto que tal y como afirma Tang \cite{Tang1991} son métodos que ofrecen una mayor
flexibilidad y efectividad a la hora de predecir una serie temporal
real.  %Esto podías haberlo dicho cuando mencionas las redes
       %neuronales para que no parezca que lo has metido a posteriori
       %- JJ 

Existen muchos trabajos que aplican las redes neuronales a la predicción de series temporales
\cite{Rivas04}.  %Y seguro que hay algunos que no son de este autor - JJ
En concreto las llamadas RBFNs que se proponen en este trabajo se han utilizado en
trabajos recientes como \cite{Hippert10,Lee09}, % ¿Recientes? - JJ
 \cite{PerezGodoy2010} donde se aplican en un
algoritmo evolutivo híbrido para la predicción a corto, medio plazo
del precio del aceite de oliva. % ¿Por qué no lo pones también antes
                                % cuando hablas en general de los
                                % métocos? Cuando llega aquí el lector
                                % ya se ha olvidado que estaba
                                % hablando de tu método. - JJ 

% ¿Esto no debería ir antes? ¿Está relacionado con tu método? ¿Es el
% estado del arte? - JJ
Para realizar la predicción se utilizan periodos de tiempo de la serie de tamaño
finito llamados ``lags''. Los lags deben representar perfectamente el comportamiento de la serie y
su relación y según el teorema de Takens \cite{Takens1980} se establece que cogiendo adecuadamente
estos lags, el comportamiento del espacio multidimensional construido a partir de dicho lag, puede
representar de forma correcta el comportamiento de la serie temporal real. 

Existen métodos automáticos que intentan determinar el lag más adecuado como pueden ser el
algoritmo QIEHI \cite{Araujo2010a}, \cite{Araujo2010b}. E incluso
algunos autores han ido más allá, % Ponía "alla", pása el corrector
                                % ortográfico - JJ
buscando no sólo el lag, sino las características externas que afectan a la serie y que con un
preprocesamiento previo de la serie temporal, puede desembocar en la elección de un lag  más
adecuado para la predicción, \cite{garcia2008}. También se ha realizado este estudio después de la
predicción,  \cite{Maus2011}, incluyendo un post procesamiento de los datos que modifica el lag de
salida del método. 

Tal y como se ha ido detallando, los procedimientos empleados en la literatura son variados y
pueden incluir pre o post procesamiento de los datos. % ¿ein? - JJ

En la propuesta que se presenta en este artículo, se utiliza para la predicción un método no
lineal que utiliza redes de base radial (RBFs) para determinar de forma cooperativa el lag y la
predicción de la serie temporal a la que se aplique el algoritmo, si necesidad de ningún pre o pro
procesamiento. % ¿Y por qué no pones eso en el título? - JJ


%%%%%%%%%%%%%%%%%%%%%%%%%%%% EXPERIMENTS AND RESULTS %%%%%%%%%%%%%%%%%%%%%%%%%%%

\section{Experimentos y resultados}
\label{sec:experiments}

En esta sección se analizan los datos recolectados y se obtienen los resultados estadísticos
iniciales a cerca de la predicción realizada. 

Debido al gran número de métodos existentes para realizar predicción, hemos seleccionado sólo 6 de
ellos para poder realizar comparaciones entre los diferentes resultados. En esta selección se han
incluido métodos lineales y no lineales. En concreto la selección Es
ARIMA, Croston, Theta, % ¿por qué? - JJ
Spline, L-Co-R y como método de control hemos seleccionado la media. Todos los métodos excepto
L-Co-R están incluidos dentro del paquete estadístico R. L-Co-R es un algoritmo coevolutivo
desarrollado por parte del equipo que firma el trabajo que ha generado resultados muy buenos dentro
de la predicción a corto plazo. Se trata de un método no lineal basado en evolucionar RBFs y Lags
simultáneamente para adecuar la red al lag elegido y el lag elegido a la red que el método
construye.

Los experimentos se han realizado utilizando las cinco series de datos descritas en la sección
anterior. Los primeros 53 días de la serie se han utilizado para entrenar los diferentes métodos y
poder construir el modelo predictivo, mientras que los 7 días restantes se han utilizado para
testear la predicción generada. La predicción se genera con un horizonte de 1, es decir conocidos
los datos de la serie con $n$ elementos, el algoritmo predice el elemento $n+1$. 

Para cada método y cada serie, se han calculado una serie de medidas de error para poder
comparar la predicción de los diferentes métodos:
\begin{itemize}
  \item \it{Mean Absolute Error} (MAE):
        \begin{equation}\label{eq:MAE}
            MAE = mean(\mid e_t\mid)
        \end{equation}

  \item \it{Mean Absolute Percentage Error} (MAPE):
        \begin{equation}\label{eq:MAPE}
            MAPE = mean(\mid p_t\mid)
        \end{equation}

        \bigskip
  \item \it{Median Absolute Percentage Error} (MdAPE):
        \begin{equation}\label{eq:MDAPE}
            MdAPE = median(\mid p_t\mid)
        \end{equation}

  \item \it{Mean Absolute Scaled Error} (MASE):
        \begin{equation}\label{eq:MASE}
            MASE = mean(\mid q_t\mid)
        \end{equation}

  \item \it{Mean Squared Error} (MSE):
        \begin{equation}\label{eq:MSE}
            MSE = \frac{1}{n}\sum_{i=1}^n {e_t}^2
        \end{equation}

%  \item Root Mean Squared Error (RMSE):
%        \begin{equation}\label{eq:RMSE}
%            RMSE = \sqrt{\frac{1}{n}\sum_{i=1}^n {e_t}^2}
%        \end{equation}

donde  $Y_t$ es el dato observado en tiempo $t = {1,...,n}$; $F_t$ es la predicción de  $Y_t$; $e_t$
es el error de predicción (i.e. $e_t= Y_t - F_t$); $p_t = 100e_t/Y_t$ es el porcentaje de error y 
         $q_t = \displaystyle\frac{e_t}{\displaystyle\frac{1}{n-1} \sum_{i=2}^n \mid Y_i - Y_{i-1} \mid }$
\end{itemize}

Puesto que el algoritmo L-Co-R no es determinista, y para eliminar los posibles beneficios del
comportamiento del generador de números aleatorios que utiliza, se ha ejecutado 30 veces para cada
una de las series, es decir, los datos mostrados como resultado del algoritmo L-Co-R son medias de
error de 30 ejecuciones.

La tabla \ref{tb:ts-results} muestra los diferentes valores de error para cada uno de los métodos y
para cfada serie. Destacados en el texto aparecen el mejor (menor) valor obtenido para cada nodo y
cada medida de error. Tal y como se puede apreciar en la tabla, la tendencia es siempre que el
mejor algoritmo es justo el algoritmo L-Co-R puesto que obtiene los menores errores para las
medidas MAE, MAPE, MdAPE y MASE (excepto para el nodo 2), aunque para otras medidas como MSE no hay
un claro algoritmo ganador. 

\begin{table} \footnotesize
 \begin{center}
 \begin{tabular}{|l|c|c|c|c|c|c|}


\hline 
 \multicolumn{6}{|c|}{Nodo 1} \\ 
 \hline 
 & \emph{MAE} & \emph{MAPE} & \emph{MASE} & \emph{MdAPE} & \emph{MSE} \\%& \emph{SMAPE} \\ 
\hline

\emph{ARIMA } & $514,83$ & $40,42$ & $1,14$ & $30,76$ & $424860,5$ \\%& $36,83$ \\ 
\emph{CROSTON } & $509,91$ & $39,28$ & $1,13$ & $33,8$ & $387403,9$ \\%& $36,46$ \\ 
\emph{THETA } & $528,59$ & $41,16$ & $1,17$ & $32,78$ & $443898,5$ \\%& $37,87$ \\ 
\emph{SPLINE } & $550,02$ & $43,87$ & $1,21$ & $30,1$ & $488504,9$ \\%& $39,04$ \\ 
\emph{MEAN } & $456,95$ & $38,57$ & $1,01$ & $26,11$ & $318623,5$ \\%& $32,48$ \\ 
\emph{L-Co-R} & $\textbf{372,77}$ & $\textbf{23,66}$ & $\textbf{0,81}$ & $\textbf{18,46}$ &
$\textbf{231731,8}$ \\%& $\textbf{28,87}$ \\ 
\hline 
 \multicolumn{6}{|c|}{Nodo 2} \\ 
 \hline 
 & \emph{MAE} & \emph{MAPE} & \emph{MASE} & \emph{MdAPE} & \emph{MSE} \\%& \emph{SMAPE} \\ 
\hline

\emph{ARIMA } & $433,52$ & $26,64$ & $\textbf{1,03}$ & $22,38$ & $\textbf{310105,3}$ \\%&$25,15$
\emph{CROSTON } & $532,87$ & $27,98$ & $1,26$ & $27,89$ & $398979,9$ \\%& $30,5$ \\ 
\emph{THETA } & $435,58$ & $26,77$ & $1,03$ & $22,55$ & $312496,1$ \\%& $25,28$ \\ 
\emph{SPLINE } & $535,86$ & $34,53$ & $1,27$ & $24,7$ & $447557,4$ \\%& $31,17$ \\ 
\emph{MEAN } & $489,57$ & $24,44$ & $1,16$ & $28,64$ & $361070,5$ \\%& $27,52$ \\ 
\emph{L-Co-R} & $\textbf{429,78}$ & $\textbf{11,41}$ & $1,04$ & $\textbf{11,98}$ & $311561,7$ \\%&
$\textbf{23,5}$ \\ 
\hline 
 \multicolumn{6}{|c|}{Node 3} \\ 
 \hline 
 & \emph{MAE} & \emph{MAPE} & \emph{MASE} & \emph{MdAPE} & \emph{MSE} \\%& \emph{SMAPE} \\ 
\hline

\emph{ARIMA } & $373,17$ & $35,11$ & $1,32$ & $27,54$ & $\textbf{150293,6}$ \\%& $\textbf{33,45}$\\ 
\emph{CROSTON } & $377,42$ & $37,03$ & $1,34$ & $27,89$ & $160970,9$ \\%& $33,78$ \\ 
\emph{THETA } & $394,95$ & $38,11$ & $1,4$ & $29,41$ & $175710,7$ \\%& $35,34$ \\ 
\emph{SPLINE } & $384,68$ & $38,86$ & $1,36$ & $29,36$ & $184807,2$ \\%& $34,36$ \\ 
\emph{MEAN } & $373,18$ & $35,11$ & $1,32$ & $27,53$ & $150307,8$ \\%& $33,45$ \\ 
\emph{L-Co-R} & $\textbf{363,3}$ & $\textbf{10,56}$ & $\textbf{0,84}$ & $\textbf{17,01}$ &
$182469,9$ \\%& $35,32$ \\ 
\hline 
 \multicolumn{6}{|c|}{Nodo 4} \\ 
 \hline 
 & \emph{MAE} & \emph{MAPE} & \emph{MASE} & \emph{MdAPE} & \emph{MSE} \\%& \emph{SMAPE} \\ 
\hline

\emph{ARIMA } & $3168,36$ & $40,62$ & $1,04$ & $18,99$ & $21792093$ \\%& $37,44$ \\ 
\emph{CROSTON } & $3789,4$ & $56,83$ & $1,25$ & $54,33$ & $25037291$ \\%& $49,12$ \\ 
\emph{THETA } & $3085,3$ & $38,14$ & $1,01$ & $25,42$ & $21510757$ \\%& $37,48$ \\ 
\emph{SPLINE } & $3758,06$ & $39,45$ & $1,24$ & $43,53$ & $28745309$ \\%& $48,79$ \\ 
\emph{MEAN } & $4545,17$ & $92,73$ & $1,5$ & $40,07$ & $26612429$ \\%& $55,57$ \\ 
\emph{L-Co-R} & $\textbf{540,59}$ & $\textbf{6,26}$ & $\textbf{0,07}$ & $\textbf{4,14}$ &
$\textbf{1120317,2}$ \\%& $\textbf{8,17}$ \\ 
\hline 
 \multicolumn{6}{|c|}{Nodo 5} \\ 
 \hline 
 & \emph{MAE} & \emph{MAPE} & \emph{MASE} & \emph{MdAPE} & \emph{MSE} \\%& \emph{SMAPE} \\ 
\hline

\emph{ARIMA } & $5012,95$ & $43,8$ & $1,13$ & $20,19$ & $\textbf{29974781}$ \\%& $\textbf{33,85}$\\ 
\emph{CROSTON } & $5583,14$ & $46,32$ & $1,26$ & $26,12$ & $35082013$ \\%& $37,51$ \\ 
\emph{THETA } & $5779,37$ & $45,78$ & $1,3$ & $28,25$ & $35859271$ \\%& $38,94$ \\ 
\emph{SPLINE } & $5878,35$ & $46,26$ & $1,32$ & $28,69$ & $37288518$ \\%& $39,64$ \\ 
\emph{MEAN } & $5013,48$ & $43,81$ & $1,13$ & $20,19$ & $29980578$ \\%& $33,85$ \\ 
\emph{L-Co-R} & $\textbf{4978,08}$ & $\textbf{24,59}$ & $\textbf{0,87}$ & $\textbf{9,37}$ &
$35875329$ \\%& $35,8$ \\ 


\hline

 \end{tabular}
 \end{center}
 \caption{Valores obtenidos para las cinco medidas de error (MAE,
MAPE, MASE, MdAPE y MSE) para los métodos de predicción seleccionados (ARIMA, Croston, Theta,
Spline, Mean and L-Co-R) para todos los nodos. En negrita se resalta el mejor resultado para cada
caso.
 \label{tb:ts-results}}
 \end{table}

Los resultados de cada método se han comparado utilizando dos medidas no paramétricas: El test de
Friedman y el test de Iman-Davenport mediante las cuales podemos determinar si existen diferencias
significativas entre los métodos o no, estableciendo así un ranking de métodos de mejor a peor en
el proceso de predicción. Los resultados revelan que existen diferencias significativas cuando el
valor $P-value$ es mejor que $0.05$. 



\begin{table} \footnotesize
 \begin{center}
 \begin{tabular}{|l|c|c|}
\hline
\emph{Error} & \emph{Friedman P-value} & \emph{Iman-Davenport P-value} \\
\hline
\emph{MAE} & $0.0052$ & $0.0003$ \\
\emph{MAPE} & $0.0080$ & $0.0008$ \\
\emph{MASE} & $0.0141$ & $0.0030$ \\
\emph{MdAPE} & $0.0063$ & $0.0005$ \\
\emph{MSE} & $0.0213$ & $0.0066$ \\
%\emph{SMAPE} & $0.0440$ & $0.0233$ \\
\hline

 \end{tabular}
 \end{center}
 \caption{Resultado del valor \emph{P-value} para los test de Friedman y Iman-Davenport
para cada medida de error. Un valor de \emph{P-value} menor que $0.05$ (todos en este caso)
indica que existen diferencias significativas entre los errores calculados para cada
método.
 \label{tb:friedman-iman-tests}}
 \end{table}

El ranking formado por los valores resultados de los test no-paramétricos, determinan cual es el
mejor método. Una vez que este método se ha encontrado, se puede comparar de nuevo con el método de
Holm \cite{Holm1979}. En la tabla \ref{tb:holm-all} se resumen los resultados así como la
existencia de diferencias significativas entre los métodos. En este caso, los resultados muestran
que el algoritmo L-Co-R es el mejor para cuatro de las medidas de error utilizadas (MAE, MAPE,
MASE, y MdAPE), aunque no se puede realizar esta afirmación si lo comparamos con el método ARIMA,
donde no se encuentran para estas series, diferencias significativas en los resultados de
predicción.  


\begin{table}[!htp]\footnotesize
\begin{center}
\begin{tabular}{|cccccc|}
\hline
\emph{Error}&\emph{Método de Control}&\emph{Ranking}&
\emph{Método}&$p$& \emph{Hipótesis nula}\\
\hline
&&1&ARIMA&0.176&Aceptada\\
&&2&MEAN&0.028& \textbf{Rechazada}\\
MAE&L-Co-R&3&THETA&0.007& \textbf{Rechazada}\\
&&4&CROSTON&0.007& \textbf{Rechazada}\\
&&5&SPLINE&2.003E-4& \textbf{Rechazada}\\
\hline
&&1&ARIMA&0.090& Aceptada\\
&&2&MEAN&0.0.063& Aceptado \\
MAPE&L-Co-R&3&THETA&0.011& \textbf{Rechazada}\\
&&4&CROSTON&0.002& \textbf{Rechazada}\\
&&5&SPLINE&3.857E-4& \textbf{Rechazada}\\
\hline
&&1&ARIMA&0.398& Aceptada\\
&&2&MEAN&0.063& Aceptado \\
MASE&L-Co-R&3&THETA&0.028& \textbf{Rechazada}\\
&&4&CROSTON&0.018& \textbf{Rechazada}\\
&&5&SPLINE&7.232E-4& \textbf{Rechazada}\\
\hline
&&1&ARIMA&0.176&Aceptada\\
&&2&MEAN&0.042& \textbf{Rechazada}\\
MdAPE&L-Co-R&3&THETA&0.004& \textbf{Rechazada}\\
&&4&SPLINE&0.002& \textbf{Rechazada}\\
&&5&CROSTON&7.232E-4& \textbf{Rechazada}\\
\hline
&&1&L-Co-R&0.500&Aceptada\\
&&2&MEAN&0.398& Aceptado \\
MSE&ARIMA&3&CROSTON&0.176& Aceptado \\
&&4&THETA&0.176& Aceptado \\
&&5&SPLINE&7.232E-4& \textbf{Rechazada}\\
\hline
&&1&L-Co-R&0.866& Aceptada\\
&&2&MEAN&0.310& Aceptado \\
SMAPE&ARIMA&3&CROSTON&0.091 & Aceptado \\
&&4&THETA&0.043& \textbf{Rechazada}\\
&&5&SPLINE&0.006& \textbf{Rechazada}\\
\hline
\end{tabular}
\end{center}
\caption{Resultados del test de Holm para cada medida de error. Valores de $p$ menores o
iguales a $0.05$ indican que los resultados de error calculados por el algoritmo de control y los
demás métodos, se pueden considerar significativamente diferentes.
\label{tb:holm-all}}

\end{table}








%%%%%%%%%%%%%%%%%%%%%%%%% CONCLUSIONS AND FUTURE WORK %%%%%%%%%%%%%%%%%%%%%%%%%%

\section{Conclusiones y trabajo futuro}
\label{sec:conclusions}

En este trabajo se presenta la utilización de un dispositivo de captación de emisiones BT para
recopilar series temporales relacionadas con el flujo del tráfico en cualquier tipo de vía.
Concretamente se han monitorizado 6 vías obteniendo las correspondientes series para cada uno de
los  dispositivos (Nodo 1-5). 

Utilizando estas series temporales reales que incluyen el número de dispositivos captados por día,
se ha realizado un estudio de las series para predecir a corto plazo (1 día) cual será el flujo de
tráfico en cada uno de los nodos. Se han seleccionado 5 métodos de predicción con los que se han
calculado 5 medidas de error. 

Los resultados obtenidos demuestran que la predicción con el algoritmo L-Co-R a corto plazo es
significativamente mejor que las predicciones obtenidas con el resto de los métodos para todas las
medidas de error seleccionadas menos una. 

Este trabajo abre un abanico de posibilidades de análisis de la gran cantidad de datos captados y
la posibilidad de desarrollar métodos de predicción no sólo a corto plazo, sino a medio y largo
plazo, lo que facilitará la tarea de previsión del flujo de tráfico en las vías monitorizadas y por
consiguiente una mejor gestión del tráfico en dichas vías por parte de las autoridades competentes.


\bibliographystyle{splncs}

% me parecen excesivas citas, y además no me gusta nada que estén
% aquí... - JJ
\begin{thebibliography}{x}

\bibitem{DatosTrafico2012}
Dirección General de Tráfico. Evolución 1970-2012. Ministerio de Fomento, Mayo 2013
\url{
http://www.fomento.gob.es/MFOM/LANG_CASTELLANO/DIRECCIONES_GENERALES/CARRETERAS/CATYEVO_RED_CARRETE
RAS/}

\bibitem{BoxJenk}
G.~Box, G.~Jenkins, Time series analysis: forecasting and control, San
  Francisco: Holden Day, 1976.

\bibitem{Gooijer25years}
J.~{De Gooijer}, R.~Hyndman, 25 years of time series forecasting, International
  Journal of Forecasting 22~(3) (2006) 443--473.

\bibitem{Fildes2008}
R.~Fildes, K.~Nikolopoulos, S.~Crone, A.~Syntetos, Forecasting and operational
  research: a review, Journal of the Operational Research Society 59 (2008)
  1150--1172.

%\bibitem{Brown1959}
%#R.~Brown, Statistical forecasting for inventory control, McGraw-Hill, 1959.

%\bibitem{Winters1960}
%P.~Winters, Forecasting sales by exponentially weighted moving averages,
%  Management Science 6~(3) (1960) 324--342.

%\bibitem{Snyder1985}
%R.~Snyder, Recursive estimation of dynamic linear models, Journal of the Royal
%  Statistical Society. Series B (Methodological) 47~(2) (1985) 272--276.

%\bibitem{Tong1978}
%H.~Tong, On a threshold model, Pattern recognition and signal processing, NATO
%  ASI Series E: Applied Sc. 29 (1978) 575--586.

%\bibitem{Tong1983}
%H.~Tong, Threshold models in non-linear time series analysis, Springer-Verlag,
%  1983.

%T\bibitem{Chan1986}
%K.~Chan, H.~Tong, On estimating thresholds in autoregressive models, Journal of
%  Time Series Analysis 7~(3) (1986) 179--190.

%\bibitem{Brockwell1992157}
%P.~Brockwell, R.~Hyndman, On continuous-time threshold autoregression,
%  International Journal of Forecasting 8~(2) (1992) 157--173.

\bibitem{Clements2004}
M.~Clements, P.~Franses, N.~Swanson, Forecasting economic and financial
  time-series with non-linear models, International Journal of Forecasting
  20~(2) (2004) 169--183.

\bibitem{Samanta2011}
B.~Samanta, Prediction of chaotic time series using computational intelligence,
  Expert Systems with Applications 38~(9) (2011) 11406--11411.

\bibitem{Zhu2011}
S.~Zhu, J.~Wang, W.~Zhao, J.~Wang, A seasonal hybrid procedure for electricity
  demand forecasting in china, Applied Energy 88~(11) (2011) 3807--3815.

\bibitem{Qiu2011}
W.~Qiu, X.~Liu, H.~Li, A generalized method for forecasting based on fuzzy time
  series, Expert Systems with Applications 38~(8) (2011) 10446--10453.

\bibitem{Wang2011}
C.~Wang, A comparison study between fuzzy time series model and arima model for
  forecasting taiwan export, Expert Systems with Applications 38~(8) (2011)
  9296--9304.

\bibitem{Yu2010}
T.~Yu, K.~Huarng, A neural network-based fuzzy time series model to improve
  forecasting, Expert Systems with Applications 37~(4) (2010) 3366--3372.

\bibitem{Kavaklioglu2011}
K.~Kavaklioglu, Modeling and prediction of turkey's electricity consumption
  using support vector regression, Applied Energy 88~(1) (2011) 368--375.

\bibitem{Dash1995}
P.~K. Dash, A.~C. Liew, S.~Rahman, G.~Ramakrishna, Building a fuzzy expert
  system for electric load forecasting using a hybrid neural network, Expert
  Systems with Applications 9~(3) (1995) 407--421.

\bibitem{Tang1991}
Z.~Tang, C.~de~Almeida, P.~Fishwick, Time series forecasting using neural
  networks vs. box-jenkins methodology, Simulation 57~(5) (1991) 303--310.

%\bibitem{Zhang1998b}
%G.~Zhang, B.~Patuwo, M.~Hu, Forecasting with artificial neural networks: The
%  state of the art, International Journal of Forecasting 14~(1) (1998) 35--62.

%\bibitem{Jain2007}
%A.~Jain, A.~Kumar, Hybrid neural network models for hydrologic time series
%  forecasting, Applied Soft Computing 7~(2) (2007) 585--592.

%\bibitem{Arizmendi1993}
%C.~M. Arizmendi, J.~Sanchez, N.~E. Ramos, G.~I. Ramos, Time series predictions
%  with neural nets: Application to airborne pollen forecasting, International
%  Journal of Biometeorology 37~(3) (1993) 139--144.

\bibitem{Rivas04}
V.~Rivas, J.~Merelo, P.~Castillo, M.~Arenas, J.~Castellano, Evolving rbf neural
  networks for time-series forecasting with evrbf, Information Sciences
  165~(3-4) (2004) 207 -- 220.

%\bibitem{Bezerianos1999}
%A.~Bezerianos, S.~Papadimitriou, D.~Alexopoulos, Radial basis function neural
%  networks for the characterization of heart rate variability dynamics,
%  Artificial Intelligence in Medicine 15~(3) (1999) 215--234.

%\bibitem{Broomhead88}
%D.~Broomhead, D.~Lowe, Multivariable functional interpolation and adaptive
%  networks, Complex Systems 2 (1988) 321--355.

%\bibitem{Carse1996}
%B.~Carse, T.~Fogarty, Fast evolutionary learning of minimal radial basis
%  function neural networks using a genetic algorithm, in: Proceedings of
%  Evolutionary Computing, Vol. 1143 of Lecture Notes in Computer Science,
%  Springer Berlin/Heidelberg, 1996, pp. 1--22.

%\bibitem{Whitehead96}
%B.~Whitehead, T.~Choate, Cooperative-competitive genetic evolution of radial
%  basis function centers and widths for time series prediction, IEEE
%  Transactions on Neural Networks 7~(4) (1996) 869--880.

%\bibitem{Harpham06}
%C.~Harpham, C.~Dawson, The effect of different basis functions on a radial
%  basis function network for time series prediction: A comparative study,
%  Neurocomputing 69~(16-18) (2006) 2161--2170.

%\bibitem{Du2008}
%H.~Du, N.~Zhang, Time series prediction using evolving radial basis function
%  networks with new encoding scheme, Neurocomputing 71~(7-9) (2008) 1388--1400.

%\bibitem{Chatterjee06}
%A.~Chatterjee, P.~Siarry, Nonlinear inertia weight variation for dynamic
%  adaptation in particle swarm optimization, Computers \& Operations Research
%  33~(3) (2006) 859--871.

\bibitem{Hippert10}
H.~Hippert, J.~Taylor, An evaluation of bayesian techniques for controlling
  model complexity and selecting inputs in a neural network for short-term load
  forecasting, Neural Networks 23~(3) (2010) 386--395.

\bibitem{Lee09}
C.~Lee, C.~Ko, Time series prediction using rbf neural networks with a
  nonlinear time-varying evolution pso algorithm, Neurocomputing 73~(1-3)
  (2009) 449--460.

\bibitem{PerezGodoy2010}
M.~Perez-Godoy, P.~P\'erez-Recuerda, M.~Fr\'{i}as, A.~Rivera, C.~Carmona,
  M.~Parras, Co$^2$rbfn for short and medium term forecasting of the
  extra-virgin olive oil price, in: J.~Gonz\'alez, D.~Pelta, C.~Cruz,
  G.~Terrazas, N.~Krasnogor (Eds.), Proceedings of Nature Inspired Cooperative
  Strategies for Optimization, Vol. 284, Springer Berlin/Heidelberg, 2010, pp.
  113--125.

\bibitem{Takens1980}
F.~Takens, Dynamical Systems and Turbulence, Lecture Notes In Mathematics, Vol.
  898, Springer, New York, NY, 1980, Ch. Detecting strange attractor in
  turbulence, pp. 366--381.

\bibitem{Araujo2010a}
R.~Ara\'ujo, A quantum-inspired evolutionary hybrid intelligent apporach fo
  stock market prediction, International Jorunal of Intelligent Computing and
  Cybernetics 3~(10) (2010) 24--54.

\bibitem{Araujo2010b}
R.~Ara\'ujo, Hybrid intelligent methodology to design translation invariant
  morphological operators for brazilian stock market prediction, Neural
  Networks 23~(10) (2010) 1238--1251.

\bibitem{garcia2008}
R.~Garc{\'\i}a-Pajares, J.~Benitez, G.~Sainz~Palmero, Feature selection form
  time series forecasting: a case study, in: Proceedings of 8th International
  Conference on Hybrid Intelligent Systems, 2008, pp. 555--560.

\bibitem{Maus2011}
A.~Maus, J.~C. Sprott, Neural network method for determining embedding
  dimension of a time series, Communications in Nonlinear Science and Numerical
  Simulation 16~(8) (2011) 3294--3302.


\bibitem{Fildes1992}
R.~Fildes, The evaluation of extrapolative forecasting methods, International
  Journal of Forecasting 8~(1) (1992) 81--98.



\bibitem{Holm1979}
S.~Holm, A simple sequentially rejective multiple test procedure, Scandinavian
  Journal of Statistics 6~(2) (1979) 65--70. 


\end{thebibliography}

%\bibliography{sipesca}

\end{document}

