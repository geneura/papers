  


<!DOCTYPE html>
<html>
  <head prefix="og: http://ogp.me/ns# fb: http://ogp.me/ns/fb# githubog: http://ogp.me/ns/fb/githubog#">
    <meta charset='utf-8'>
    <meta http-equiv="X-UA-Compatible" content="IE=edge">
        <title>papers/2013/Short_Term_Traffic_Flow_Prediction/geneura4ShortTermTrafficFlow.tex at master · geneura/papers</title>
    <link rel="search" type="application/opensearchdescription+xml" href="/opensearch.xml" title="GitHub" />
    <link rel="fluid-icon" href="https://github.com/fluidicon.png" title="GitHub" />
    <link rel="apple-touch-icon" sizes="57x57" href="/apple-touch-icon-114.png" />
    <link rel="apple-touch-icon" sizes="114x114" href="/apple-touch-icon-114.png" />
    <link rel="apple-touch-icon" sizes="72x72" href="/apple-touch-icon-144.png" />
    <link rel="apple-touch-icon" sizes="144x144" href="/apple-touch-icon-144.png" />
    <link rel="logo" type="image/svg" href="http://github-media-downloads.s3.amazonaws.com/github-logo.svg" />
    <meta name="msapplication-TileImage" content="/windows-tile.png">
    <meta name="msapplication-TileColor" content="#ffffff">

    
    
    <link rel="icon" type="image/x-icon" href="/favicon.ico" />

    <meta content="authenticity_token" name="csrf-param" />
<meta content="P1iAtqAjrFtXPAAuwZ/tbj8TZUfM7Mmsx9Ih6+6bJ0s=" name="csrf-token" />

    <link href="https://a248.e.akamai.net/assets.github.com/assets/github-869b3a6b0657c3c2ec9f9c691a3a910c288f571f.css" media="screen" rel="stylesheet" type="text/css" />
    <link href="https://a248.e.akamai.net/assets.github.com/assets/github2-5fe05bba981d40bbf5e1c773c913d045716f46db.css" media="screen" rel="stylesheet" type="text/css" />
    


      <script src="https://a248.e.akamai.net/assets.github.com/assets/frameworks-f615f7544ba636b083d742a0b415479b5b674fd4.js" type="text/javascript"></script>
      <script src="https://a248.e.akamai.net/assets.github.com/assets/github-61342ca1c438225514589e9d4ea7edfad3f829b0.js" type="text/javascript"></script>
      

        <link rel='permalink' href='/geneura/papers/blob/59b21017a53e303571b28b8f35a4e52fa787169f/2013/Short_Term_Traffic_Flow_Prediction/geneura4ShortTermTrafficFlow.tex'>
    <meta property="og:title" content="papers"/>
    <meta property="og:type" content="githubog:gitrepository"/>
    <meta property="og:url" content="https://github.com/geneura/papers"/>
    <meta property="og:image" content="https://secure.gravatar.com/avatar/18237ae36b83646bd0388f5b3e1b0069?s=420&amp;d=https://a248.e.akamai.net/assets.github.com%2Fimages%2Fgravatars%2Fgravatar-user-420.png"/>
    <meta property="og:site_name" content="GitHub"/>
    <meta property="og:description" content="Trabajos de geneura para desarrollar en privado. Contribute to papers development by creating an account on GitHub."/>
    <meta property="twitter:card" content="summary"/>
    <meta property="twitter:site" content="@GitHub">
    <meta property="twitter:title" content="geneura/papers"/>

    <meta name="description" content="Trabajos de geneura para desarrollar en privado. Contribute to papers development by creating an account on GitHub." />

  <link href="https://github.com/geneura/papers/commits/master.atom?login=amorag&token=cff448a8685549a5fbc49cb8e252eeb7" rel="alternate" title="Recent Commits to papers:master" type="application/atom+xml" />

  </head>


  <body class="logged_in page-blob windows vis-private env-production  ">
    <div id="wrapper">

      

      

      

      


        <div class="header header-logged-in true">
          <div class="container clearfix">

            <a class="header-logo-blacktocat" href="https://github.com/">
  <span class="mega-icon mega-icon-blacktocat"></span>
</a>

            <div class="divider-vertical"></div>

              <a href="/notifications" class="notification-indicator tooltipped downwards" title="You have no unread notifications">
    <span class="mail-status all-read"></span>
  </a>
  <div class="divider-vertical"></div>


              <div class="topsearch command-bar-activated ">
      <form accept-charset="UTF-8" action="/search" class="command_bar_form" id="top_search_form" method="get">
  <a href="/search/advanced" class="advanced-search-icon tooltipped downwards command-bar-search" id="advanced_search" title="Advanced search"><span class="mini-icon mini-icon-advanced-search "></span></a>

  <input type="text" name="q" id="command-bar" placeholder="Search or type a command" tabindex="1" data-username="amorag" autocapitalize="off">

  <span class="mini-icon help tooltipped downwards" title="Show command bar help">
    <span class="mini-icon mini-icon-help"></span>
  </span>

  <input type="hidden" name="ref" value="commandbar">

  <div class="divider-vertical"></div>
</form>
  <ul class="top-nav">
      <li class="explore"><a href="https://github.com/explore">Explore</a></li>
      <li><a href="https://gist.github.com">Gist</a></li>
      <li><a href="/blog">Blog</a></li>
    <li><a href="http://help.github.com">Help</a></li>
  </ul>
</div>


            

  
    <ul id="user-links">
      <li>
        <a href="https://github.com/amorag" class="name">
          <img height="20" src="https://secure.gravatar.com/avatar/7ab6c86a3ccc0064bf25d47d7612b942?s=140&amp;d=https://a248.e.akamai.net/assets.github.com%2Fimages%2Fgravatars%2Fgravatar-user-420.png" width="20" /> amorag
        </a>
      </li>
      <li>
        <a href="/new" id="new_repo" class="tooltipped downwards" title="Create a new repo">
          <span class="mini-icon mini-icon-create"></span>
        </a>
      </li>
      <li>
        <a href="/settings/profile" id="account_settings"
          class="tooltipped downwards"
          title="Account settings ">
          <span class="mini-icon mini-icon-account-settings"></span>
        </a>
      </li>
      <li>
        <a href="/logout" data-method="post" id="logout" class="tooltipped downwards" title="Sign out">
          <span class="mini-icon mini-icon-logout"></span>
        </a>
      </li>
    </ul>



            
          </div>
        </div>


      

      


            <div class="site hfeed" itemscope itemtype="http://schema.org/WebPage">
      <div class="hentry">
        
        <div class="pagehead repohead instapaper_ignore readability-menu ">
          <div class="container">
            <div class="title-actions-bar">
              


<ul class="pagehead-actions">

    <li class="nspr">
      <a href="/geneura/papers/pull/new/master" class="button minibutton btn-pull-request" icon_class="mini-icon-pull-request"><span class="mini-icon mini-icon-pull-request"></span>Pull Request</a>
    </li>

    <li class="subscription">
      <form accept-charset="UTF-8" action="/notifications/subscribe" data-autosubmit="true" data-remote="true" method="post"><div style="margin:0;padding:0;display:inline"><input name="authenticity_token" type="hidden" value="P1iAtqAjrFtXPAAuwZ/tbj8TZUfM7Mmsx9Ih6+6bJ0s=" /></div>  <input id="repository_id" name="repository_id" type="hidden" value="7190438" />

    <div class="select-menu js-menu-container js-select-menu">
      <span class="minibutton select-menu-button js-menu-target">
        <span class="js-select-button">
          <span class="mini-icon mini-icon-watching"></span>
          Watch
        </span>
      </span>

      <div class="select-menu-modal-holder js-menu-content">
        <div class="select-menu-modal">
          <div class="select-menu-header">
            <span class="select-menu-title">Notification status</span>
            <span class="mini-icon mini-icon-remove-close js-menu-close"></span>
          </div> <!-- /.select-menu-header -->

          <div class="select-menu-list js-navigation-container js-select-menu-pane">

            <div class="select-menu-item js-navigation-item js-navigation-target selected">
              <span class="select-menu-item-icon mini-icon mini-icon-confirm"></span>
              <div class="select-menu-item-text">
                <input checked="checked" id="do_included" name="do" type="radio" value="included" />
                <h4>Not watching</h4>
                <span class="description">You only receive notifications for discussions in which you participate or are @mentioned.</span>
                <span class="js-select-button-text hidden-select-button-text">
                  <span class="mini-icon mini-icon-watching"></span>
                  Watch
                </span>
              </div>
            </div> <!-- /.select-menu-item -->

            <div class="select-menu-item js-navigation-item js-navigation-target ">
              <span class="select-menu-item-icon mini-icon mini-icon-confirm"></span>
              <div class="select-menu-item-text">
                <input id="do_subscribed" name="do" type="radio" value="subscribed" />
                <h4>Watching</h4>
                <span class="description">You receive notifications for all discussions in this repository.</span>
                <span class="js-select-button-text hidden-select-button-text">
                  <span class="mini-icon mini-icon-unwatch"></span>
                  Unwatch
                </span>
              </div>
            </div> <!-- /.select-menu-item -->

            <div class="select-menu-item js-navigation-item js-navigation-target ">
              <span class="select-menu-item-icon mini-icon mini-icon-confirm"></span>
              <div class="select-menu-item-text">
                <input id="do_ignore" name="do" type="radio" value="ignore" />
                <h4>Ignoring</h4>
                <span class="description">You do not receive any notifications for discussions in this repository.</span>
                <span class="js-select-button-text hidden-select-button-text">
                  <span class="mini-icon mini-icon-mute"></span>
                  Stop ignoring
                </span>
              </div>
            </div> <!-- /.select-menu-item -->

          </div> <!-- /.select-menu-list -->

        </div> <!-- /.select-menu-modal -->
      </div> <!-- /.select-menu-modal-holder -->
    </div> <!-- /.select-menu -->

</form>
    </li>

    <li class="js-toggler-container js-social-container starring-container ">
      <a href="/geneura/papers/unstar" class="minibutton js-toggler-target star-button starred upwards" title="Unstar this repo" data-remote="true" data-method="post" rel="nofollow">
        <span class="mini-icon mini-icon-remove-star"></span>
        <span class="text">Unstar</span>
      </a>
      <a href="/geneura/papers/star" class="minibutton js-toggler-target star-button unstarred upwards" title="Star this repo" data-remote="true" data-method="post" rel="nofollow">
        <span class="mini-icon mini-icon-star"></span>
        <span class="text">Star</span>
      </a>
      <a class="social-count js-social-count" href="/geneura/papers/stargazers">0</a>
    </li>

        <li>
          <a href="/geneura/papers/fork" class="minibutton js-toggler-target fork-button lighter upwards" title="Fork this repo" rel="nofollow" data-method="post">
            <span class="mini-icon mini-icon-branch-create"></span>
            <span class="text">Fork</span>
          </a>
          <a href="/geneura/papers/network" class="social-count">0</a>
        </li>


</ul>

              <h1 itemscope itemtype="http://data-vocabulary.org/Breadcrumb" class="entry-title private">
                <span class="repo-label"><span>private</span></span>
                <span class="mega-icon mega-icon-private-repo"></span>
                <span class="author vcard">
                  <a href="/geneura" class="url fn" itemprop="url" rel="author">
                  <span itemprop="title">geneura</span>
                  </a></span> /
                <strong><a href="/geneura/papers" class="js-current-repository">papers</a></strong>
              </h1>
            </div>

            
  <ul class="tabs">
    <li><a href="/geneura/papers" class="selected" highlight="[:repo_source, :repo_downloads, :repo_commits, :repo_tags, :repo_branches]">Code</a></li>
    <li><a href="/geneura/papers/network" highlight="[:repo_network]">Network</a></li>
    <li><a href="/geneura/papers/pulls" highlight="[:repo_pulls]">Pull Requests <span class='counter'>0</span></a></li>

      <li><a href="/geneura/papers/issues" highlight="[:repo_issues]">Issues <span class='counter'>1</span></a></li>

      <li><a href="/geneura/papers/wiki" highlight="[:repo_wiki]">Wiki</a></li>


    <li><a href="/geneura/papers/graphs" highlight="[:repo_graphs, :repo_contributors]">Graphs</a></li>


  </ul>
  
<div class="tabnav">

  <span class="tabnav-right">
    <ul class="tabnav-tabs">
          <li><a href="/geneura/papers/tags" class="tabnav-tab" highlight="repo_tags">Tags <span class="counter blank">0</span></a></li>
    </ul>
    
  <div class="tabnav-widget search repo-search ">
    <form accept-charset="UTF-8" action="/geneura/papers/search" method="get">
      <span class="fieldwrap">
        <input type="text" name="q" value=""
          placeholder="Search source code&hellip;" tabindex="2" autocapitalize="off" /><button type="submit" class="minibutton empty-icon"><span class="mini-icon mini-icon-search-input"></span></button>
      </span>
      <input type="hidden" id="lang-value" name="l" value="" />
      <input type="hidden" id="start-value" name="start" value="" />
</form>  </div>

  </span>

  <div class="tabnav-widget scope">


    <div class="select-menu js-menu-container js-select-menu js-branch-menu">
      <a class="minibutton select-menu-button js-menu-target" data-hotkey="w" data-ref="master">
        <span class="mini-icon mini-icon-branch"></span>
        <i>branch:</i>
        <span class="js-select-button">master</span>
      </a>

      <div class="select-menu-modal-holder js-menu-content js-navigation-container js-select-menu-pane">

        <div class="select-menu-modal js-select-menu-pane">
          <div class="select-menu-header">
            <span class="select-menu-title">Switch branches/tags</span>
            <span class="mini-icon mini-icon-remove-close js-menu-close"></span>
          </div> <!-- /.select-menu-header -->

          <div class="select-menu-filters">
            <div class="select-menu-text-filter">
              <input type="text" id="commitish-filter-field" class="js-select-menu-text-filter js-filterable-field js-navigation-enable" placeholder="Find or create a branch…">
            </div> <!-- /.select-menu-text-filter -->
            <div class="select-menu-tabs">
              <ul>
                <li class="select-menu-tab">
                  <a href="#" data-tab-filter="branches" class="js-select-menu-tab">Branches</a>
                </li>
                <li class="select-menu-tab">
                  <a href="#" data-tab-filter="tags" class="js-select-menu-tab">Tags</a>
                </li>
              </ul>
            </div><!-- /.select-menu-tabs -->
          </div><!-- /.select-menu-filters -->

          <div class="select-menu-list select-menu-tab-bucket js-select-menu-tab-bucket css-truncate" data-tab-filter="branches" data-filterable-for="commitish-filter-field" data-filterable-type="substring">


              <div class="select-menu-item js-navigation-item js-navigation-target selected">
                <span class="select-menu-item-icon mini-icon mini-icon-confirm"></span>
                <a href="/geneura/papers/blob/master/2013/Short_Term_Traffic_Flow_Prediction/geneura4ShortTermTrafficFlow.tex" class="js-navigation-open select-menu-item-text js-select-button-text css-truncate-target" data-name="master" rel="nofollow" title="master">master</a>
              </div> <!-- /.select-menu-item -->

              <form accept-charset="UTF-8" action="/geneura/papers/branches" class="js-create-branch select-menu-item select-menu-new-item-form js-navigation-item js-navigation-target js-new-item-form" method="post"><div style="margin:0;padding:0;display:inline"><input name="authenticity_token" type="hidden" value="P1iAtqAjrFtXPAAuwZ/tbj8TZUfM7Mmsx9Ih6+6bJ0s=" /></div>


                <span class="mini-icon mini-icon-branch-create select-menu-item-icon"></span>
                <div class="select-menu-item-text">
                  <h4>Create branch: <span class="js-new-item-name"></span></h4>
                  <span class="description">from ‘master’</span>
                </div>
                <input type="hidden" name="name" id="name" class="js-new-item-submit" />
                <input type="hidden" name="branch" id="branch" value="master" />
                <input type="hidden" name="path" id="branch" value="2013/Short_Term_Traffic_Flow_Prediction/geneura4ShortTermTrafficFlow.tex" />

              </form> <!-- /.select-menu-footer -->


          </div> <!-- /.select-menu-list -->


          <div class="select-menu-list select-menu-tab-bucket js-select-menu-tab-bucket css-truncate" data-tab-filter="tags" data-filterable-for="commitish-filter-field" data-filterable-type="substring">


            <div class="select-menu-no-results js-not-filterable">Nothing to show</div>

          </div> <!-- /.select-menu-list -->

        </div> <!-- /.select-menu-modal -->
      </div> <!-- /.select-menu-modal-holder -->
    </div> <!-- /.select-menu -->

  </div> <!-- /.scope -->

  <ul class="tabnav-tabs">
    <li><a href="/geneura/papers" class="selected tabnav-tab" highlight="repo_source">Files</a></li>
    <li><a href="/geneura/papers/commits/master" class="tabnav-tab" highlight="repo_commits">Commits</a></li>
    <li><a href="/geneura/papers/branches" class="tabnav-tab" highlight="repo_branches" rel="nofollow">Branches <span class="counter ">1</span></a></li>
  </ul>

</div>

  
  
  


            
          </div>
        </div><!-- /.repohead -->

        <div id="js-repo-pjax-container" class="container context-loader-container" data-pjax-container>
          


<!-- blob contrib key: blob_contributors:v21:1bfc0195f2c9ebe4460b0f85a25e5498 -->
<!-- blob contrib frag key: views10/v8/blob_contributors:v21:1bfc0195f2c9ebe4460b0f85a25e5498 -->


<div id="slider">
    <div class="frame-meta">

      <p title="This is a placeholder element" class="js-history-link-replace hidden"></p>

        <div class="breadcrumb">
          <span class='bold'><span itemscope="" itemtype="http://data-vocabulary.org/Breadcrumb"><a href="/geneura/papers" class="js-slide-to" data-direction="back" itemscope="url"><span itemprop="title">papers</span></a></span></span> / <span itemscope="" itemtype="http://data-vocabulary.org/Breadcrumb"><a href="/geneura/papers/tree/master/2013" class="js-slide-to" data-direction="back" itemscope="url"><span itemprop="title">2013</span></a></span> / <span itemscope="" itemtype="http://data-vocabulary.org/Breadcrumb"><a href="/geneura/papers/tree/master/2013/Short_Term_Traffic_Flow_Prediction" class="js-slide-to" data-direction="back" itemscope="url"><span itemprop="title">Short_Term_Traffic_Flow_Prediction</span></a></span> / <strong class="final-path">geneura4ShortTermTrafficFlow.tex</strong> <span class="js-zeroclipboard zeroclipboard-button" data-clipboard-text="2013/Short_Term_Traffic_Flow_Prediction/geneura4ShortTermTrafficFlow.tex" data-copied-hint="copied!" title="copy to clipboard"><span class="mini-icon mini-icon-clipboard"></span></span>
        </div>

      <a href="/geneura/papers/find/master" class="js-slide-to" data-hotkey="t" style="display:none">Show File Finder</a>


        
  <div class="commit file-history-tease">
    <img class="main-avatar" height="24" src="https://secure.gravatar.com/avatar/a489962558b28ac6fbf5ad65b6a5cf40?s=140&amp;d=https://a248.e.akamai.net/assets.github.com%2Fimages%2Fgravatars%2Fgravatar-user-420.png" width="24" />
    <span class="author"><span rel="author">Víctor Manuel Rivas Santos</span></span>
    <time class="js-relative-date" datetime="2013-02-20T12:44:00-08:00" title="2013-02-20 12:44:00">February 20, 2013</time>
    <div class="commit-title">
        <a href="/geneura/papers/commit/5e09dbbe0529d8e58569f784fa3babb5a89f41a5" class="message">MOdificando estado del arte y empezando a introducir la experimentaci…</a>
    </div>

    <div class="participation">
      <p class="quickstat"><a href="#blob_contributors_box" rel="facebox"><strong>1</strong> contributor</a></p>
      
    </div>
    <div id="blob_contributors_box" style="display:none">
      <h2>Users on GitHub who have contributed to this file</h2>
      <ul class="facebox-user-list">
        <li>
          <img height="24" src="https://secure.gravatar.com/avatar/7ab6c86a3ccc0064bf25d47d7612b942?s=140&amp;d=https://a248.e.akamai.net/assets.github.com%2Fimages%2Fgravatars%2Fgravatar-user-420.png" width="24" />
          <a href="/amorag">amorag</a>
        </li>
      </ul>
    </div>
  </div>


    </div><!-- ./.frame-meta -->

    <div class="frames">
      <div class="frame" data-permalink-url="/geneura/papers/blob/59b21017a53e303571b28b8f35a4e52fa787169f/2013/Short_Term_Traffic_Flow_Prediction/geneura4ShortTermTrafficFlow.tex" data-title="papers/2013/Short_Term_Traffic_Flow_Prediction/geneura4ShortTermTrafficFlow.tex at master · geneura/papers · GitHub" data-type="blob">

        <div id="files" class="bubble">
          <div class="file">
            <div class="meta">
              <div class="info">
                <span class="icon"><b class="mini-icon mini-icon-text-file"></b></span>
                <span class="mode" title="File Mode">file</span>
                  <span>805 lines (625 sloc)</span>
                <span>52.444 kb</span>
              </div>
              <div class="actions">
                <div class="button-group">
                        <a class="minibutton"
                           href="/geneura/papers/edit/master/2013/Short_Term_Traffic_Flow_Prediction/geneura4ShortTermTrafficFlow.tex"
                           data-method="post" rel="nofollow" data-hotkey="e">Edit</a>
                  <a href="/geneura/papers/raw/master/2013/Short_Term_Traffic_Flow_Prediction/geneura4ShortTermTrafficFlow.tex" class="button minibutton " id="raw-url">Raw</a>
                    <a href="/geneura/papers/blame/master/2013/Short_Term_Traffic_Flow_Prediction/geneura4ShortTermTrafficFlow.tex" class="button minibutton ">Blame</a>
                  <a href="/geneura/papers/commits/master/2013/Short_Term_Traffic_Flow_Prediction/geneura4ShortTermTrafficFlow.tex" class="button minibutton " rel="nofollow">History</a>
                </div><!-- /.button-group -->
              </div><!-- /.actions -->

            </div>
                <div class="data type-tex js-blob-data">
      <table cellpadding="0" cellspacing="0" class="lines">
        <tr>
          <td>
            <pre class="line_numbers"><span id="L1" rel="#L1">1</span>
<span id="L2" rel="#L2">2</span>
<span id="L3" rel="#L3">3</span>
<span id="L4" rel="#L4">4</span>
<span id="L5" rel="#L5">5</span>
<span id="L6" rel="#L6">6</span>
<span id="L7" rel="#L7">7</span>
<span id="L8" rel="#L8">8</span>
<span id="L9" rel="#L9">9</span>
<span id="L10" rel="#L10">10</span>
<span id="L11" rel="#L11">11</span>
<span id="L12" rel="#L12">12</span>
<span id="L13" rel="#L13">13</span>
<span id="L14" rel="#L14">14</span>
<span id="L15" rel="#L15">15</span>
<span id="L16" rel="#L16">16</span>
<span id="L17" rel="#L17">17</span>
<span id="L18" rel="#L18">18</span>
<span id="L19" rel="#L19">19</span>
<span id="L20" rel="#L20">20</span>
<span id="L21" rel="#L21">21</span>
<span id="L22" rel="#L22">22</span>
<span id="L23" rel="#L23">23</span>
<span id="L24" rel="#L24">24</span>
<span id="L25" rel="#L25">25</span>
<span id="L26" rel="#L26">26</span>
<span id="L27" rel="#L27">27</span>
<span id="L28" rel="#L28">28</span>
<span id="L29" rel="#L29">29</span>
<span id="L30" rel="#L30">30</span>
<span id="L31" rel="#L31">31</span>
<span id="L32" rel="#L32">32</span>
<span id="L33" rel="#L33">33</span>
<span id="L34" rel="#L34">34</span>
<span id="L35" rel="#L35">35</span>
<span id="L36" rel="#L36">36</span>
<span id="L37" rel="#L37">37</span>
<span id="L38" rel="#L38">38</span>
<span id="L39" rel="#L39">39</span>
<span id="L40" rel="#L40">40</span>
<span id="L41" rel="#L41">41</span>
<span id="L42" rel="#L42">42</span>
<span id="L43" rel="#L43">43</span>
<span id="L44" rel="#L44">44</span>
<span id="L45" rel="#L45">45</span>
<span id="L46" rel="#L46">46</span>
<span id="L47" rel="#L47">47</span>
<span id="L48" rel="#L48">48</span>
<span id="L49" rel="#L49">49</span>
<span id="L50" rel="#L50">50</span>
<span id="L51" rel="#L51">51</span>
<span id="L52" rel="#L52">52</span>
<span id="L53" rel="#L53">53</span>
<span id="L54" rel="#L54">54</span>
<span id="L55" rel="#L55">55</span>
<span id="L56" rel="#L56">56</span>
<span id="L57" rel="#L57">57</span>
<span id="L58" rel="#L58">58</span>
<span id="L59" rel="#L59">59</span>
<span id="L60" rel="#L60">60</span>
<span id="L61" rel="#L61">61</span>
<span id="L62" rel="#L62">62</span>
<span id="L63" rel="#L63">63</span>
<span id="L64" rel="#L64">64</span>
<span id="L65" rel="#L65">65</span>
<span id="L66" rel="#L66">66</span>
<span id="L67" rel="#L67">67</span>
<span id="L68" rel="#L68">68</span>
<span id="L69" rel="#L69">69</span>
<span id="L70" rel="#L70">70</span>
<span id="L71" rel="#L71">71</span>
<span id="L72" rel="#L72">72</span>
<span id="L73" rel="#L73">73</span>
<span id="L74" rel="#L74">74</span>
<span id="L75" rel="#L75">75</span>
<span id="L76" rel="#L76">76</span>
<span id="L77" rel="#L77">77</span>
<span id="L78" rel="#L78">78</span>
<span id="L79" rel="#L79">79</span>
<span id="L80" rel="#L80">80</span>
<span id="L81" rel="#L81">81</span>
<span id="L82" rel="#L82">82</span>
<span id="L83" rel="#L83">83</span>
<span id="L84" rel="#L84">84</span>
<span id="L85" rel="#L85">85</span>
<span id="L86" rel="#L86">86</span>
<span id="L87" rel="#L87">87</span>
<span id="L88" rel="#L88">88</span>
<span id="L89" rel="#L89">89</span>
<span id="L90" rel="#L90">90</span>
<span id="L91" rel="#L91">91</span>
<span id="L92" rel="#L92">92</span>
<span id="L93" rel="#L93">93</span>
<span id="L94" rel="#L94">94</span>
<span id="L95" rel="#L95">95</span>
<span id="L96" rel="#L96">96</span>
<span id="L97" rel="#L97">97</span>
<span id="L98" rel="#L98">98</span>
<span id="L99" rel="#L99">99</span>
<span id="L100" rel="#L100">100</span>
<span id="L101" rel="#L101">101</span>
<span id="L102" rel="#L102">102</span>
<span id="L103" rel="#L103">103</span>
<span id="L104" rel="#L104">104</span>
<span id="L105" rel="#L105">105</span>
<span id="L106" rel="#L106">106</span>
<span id="L107" rel="#L107">107</span>
<span id="L108" rel="#L108">108</span>
<span id="L109" rel="#L109">109</span>
<span id="L110" rel="#L110">110</span>
<span id="L111" rel="#L111">111</span>
<span id="L112" rel="#L112">112</span>
<span id="L113" rel="#L113">113</span>
<span id="L114" rel="#L114">114</span>
<span id="L115" rel="#L115">115</span>
<span id="L116" rel="#L116">116</span>
<span id="L117" rel="#L117">117</span>
<span id="L118" rel="#L118">118</span>
<span id="L119" rel="#L119">119</span>
<span id="L120" rel="#L120">120</span>
<span id="L121" rel="#L121">121</span>
<span id="L122" rel="#L122">122</span>
<span id="L123" rel="#L123">123</span>
<span id="L124" rel="#L124">124</span>
<span id="L125" rel="#L125">125</span>
<span id="L126" rel="#L126">126</span>
<span id="L127" rel="#L127">127</span>
<span id="L128" rel="#L128">128</span>
<span id="L129" rel="#L129">129</span>
<span id="L130" rel="#L130">130</span>
<span id="L131" rel="#L131">131</span>
<span id="L132" rel="#L132">132</span>
<span id="L133" rel="#L133">133</span>
<span id="L134" rel="#L134">134</span>
<span id="L135" rel="#L135">135</span>
<span id="L136" rel="#L136">136</span>
<span id="L137" rel="#L137">137</span>
<span id="L138" rel="#L138">138</span>
<span id="L139" rel="#L139">139</span>
<span id="L140" rel="#L140">140</span>
<span id="L141" rel="#L141">141</span>
<span id="L142" rel="#L142">142</span>
<span id="L143" rel="#L143">143</span>
<span id="L144" rel="#L144">144</span>
<span id="L145" rel="#L145">145</span>
<span id="L146" rel="#L146">146</span>
<span id="L147" rel="#L147">147</span>
<span id="L148" rel="#L148">148</span>
<span id="L149" rel="#L149">149</span>
<span id="L150" rel="#L150">150</span>
<span id="L151" rel="#L151">151</span>
<span id="L152" rel="#L152">152</span>
<span id="L153" rel="#L153">153</span>
<span id="L154" rel="#L154">154</span>
<span id="L155" rel="#L155">155</span>
<span id="L156" rel="#L156">156</span>
<span id="L157" rel="#L157">157</span>
<span id="L158" rel="#L158">158</span>
<span id="L159" rel="#L159">159</span>
<span id="L160" rel="#L160">160</span>
<span id="L161" rel="#L161">161</span>
<span id="L162" rel="#L162">162</span>
<span id="L163" rel="#L163">163</span>
<span id="L164" rel="#L164">164</span>
<span id="L165" rel="#L165">165</span>
<span id="L166" rel="#L166">166</span>
<span id="L167" rel="#L167">167</span>
<span id="L168" rel="#L168">168</span>
<span id="L169" rel="#L169">169</span>
<span id="L170" rel="#L170">170</span>
<span id="L171" rel="#L171">171</span>
<span id="L172" rel="#L172">172</span>
<span id="L173" rel="#L173">173</span>
<span id="L174" rel="#L174">174</span>
<span id="L175" rel="#L175">175</span>
<span id="L176" rel="#L176">176</span>
<span id="L177" rel="#L177">177</span>
<span id="L178" rel="#L178">178</span>
<span id="L179" rel="#L179">179</span>
<span id="L180" rel="#L180">180</span>
<span id="L181" rel="#L181">181</span>
<span id="L182" rel="#L182">182</span>
<span id="L183" rel="#L183">183</span>
<span id="L184" rel="#L184">184</span>
<span id="L185" rel="#L185">185</span>
<span id="L186" rel="#L186">186</span>
<span id="L187" rel="#L187">187</span>
<span id="L188" rel="#L188">188</span>
<span id="L189" rel="#L189">189</span>
<span id="L190" rel="#L190">190</span>
<span id="L191" rel="#L191">191</span>
<span id="L192" rel="#L192">192</span>
<span id="L193" rel="#L193">193</span>
<span id="L194" rel="#L194">194</span>
<span id="L195" rel="#L195">195</span>
<span id="L196" rel="#L196">196</span>
<span id="L197" rel="#L197">197</span>
<span id="L198" rel="#L198">198</span>
<span id="L199" rel="#L199">199</span>
<span id="L200" rel="#L200">200</span>
<span id="L201" rel="#L201">201</span>
<span id="L202" rel="#L202">202</span>
<span id="L203" rel="#L203">203</span>
<span id="L204" rel="#L204">204</span>
<span id="L205" rel="#L205">205</span>
<span id="L206" rel="#L206">206</span>
<span id="L207" rel="#L207">207</span>
<span id="L208" rel="#L208">208</span>
<span id="L209" rel="#L209">209</span>
<span id="L210" rel="#L210">210</span>
<span id="L211" rel="#L211">211</span>
<span id="L212" rel="#L212">212</span>
<span id="L213" rel="#L213">213</span>
<span id="L214" rel="#L214">214</span>
<span id="L215" rel="#L215">215</span>
<span id="L216" rel="#L216">216</span>
<span id="L217" rel="#L217">217</span>
<span id="L218" rel="#L218">218</span>
<span id="L219" rel="#L219">219</span>
<span id="L220" rel="#L220">220</span>
<span id="L221" rel="#L221">221</span>
<span id="L222" rel="#L222">222</span>
<span id="L223" rel="#L223">223</span>
<span id="L224" rel="#L224">224</span>
<span id="L225" rel="#L225">225</span>
<span id="L226" rel="#L226">226</span>
<span id="L227" rel="#L227">227</span>
<span id="L228" rel="#L228">228</span>
<span id="L229" rel="#L229">229</span>
<span id="L230" rel="#L230">230</span>
<span id="L231" rel="#L231">231</span>
<span id="L232" rel="#L232">232</span>
<span id="L233" rel="#L233">233</span>
<span id="L234" rel="#L234">234</span>
<span id="L235" rel="#L235">235</span>
<span id="L236" rel="#L236">236</span>
<span id="L237" rel="#L237">237</span>
<span id="L238" rel="#L238">238</span>
<span id="L239" rel="#L239">239</span>
<span id="L240" rel="#L240">240</span>
<span id="L241" rel="#L241">241</span>
<span id="L242" rel="#L242">242</span>
<span id="L243" rel="#L243">243</span>
<span id="L244" rel="#L244">244</span>
<span id="L245" rel="#L245">245</span>
<span id="L246" rel="#L246">246</span>
<span id="L247" rel="#L247">247</span>
<span id="L248" rel="#L248">248</span>
<span id="L249" rel="#L249">249</span>
<span id="L250" rel="#L250">250</span>
<span id="L251" rel="#L251">251</span>
<span id="L252" rel="#L252">252</span>
<span id="L253" rel="#L253">253</span>
<span id="L254" rel="#L254">254</span>
<span id="L255" rel="#L255">255</span>
<span id="L256" rel="#L256">256</span>
<span id="L257" rel="#L257">257</span>
<span id="L258" rel="#L258">258</span>
<span id="L259" rel="#L259">259</span>
<span id="L260" rel="#L260">260</span>
<span id="L261" rel="#L261">261</span>
<span id="L262" rel="#L262">262</span>
<span id="L263" rel="#L263">263</span>
<span id="L264" rel="#L264">264</span>
<span id="L265" rel="#L265">265</span>
<span id="L266" rel="#L266">266</span>
<span id="L267" rel="#L267">267</span>
<span id="L268" rel="#L268">268</span>
<span id="L269" rel="#L269">269</span>
<span id="L270" rel="#L270">270</span>
<span id="L271" rel="#L271">271</span>
<span id="L272" rel="#L272">272</span>
<span id="L273" rel="#L273">273</span>
<span id="L274" rel="#L274">274</span>
<span id="L275" rel="#L275">275</span>
<span id="L276" rel="#L276">276</span>
<span id="L277" rel="#L277">277</span>
<span id="L278" rel="#L278">278</span>
<span id="L279" rel="#L279">279</span>
<span id="L280" rel="#L280">280</span>
<span id="L281" rel="#L281">281</span>
<span id="L282" rel="#L282">282</span>
<span id="L283" rel="#L283">283</span>
<span id="L284" rel="#L284">284</span>
<span id="L285" rel="#L285">285</span>
<span id="L286" rel="#L286">286</span>
<span id="L287" rel="#L287">287</span>
<span id="L288" rel="#L288">288</span>
<span id="L289" rel="#L289">289</span>
<span id="L290" rel="#L290">290</span>
<span id="L291" rel="#L291">291</span>
<span id="L292" rel="#L292">292</span>
<span id="L293" rel="#L293">293</span>
<span id="L294" rel="#L294">294</span>
<span id="L295" rel="#L295">295</span>
<span id="L296" rel="#L296">296</span>
<span id="L297" rel="#L297">297</span>
<span id="L298" rel="#L298">298</span>
<span id="L299" rel="#L299">299</span>
<span id="L300" rel="#L300">300</span>
<span id="L301" rel="#L301">301</span>
<span id="L302" rel="#L302">302</span>
<span id="L303" rel="#L303">303</span>
<span id="L304" rel="#L304">304</span>
<span id="L305" rel="#L305">305</span>
<span id="L306" rel="#L306">306</span>
<span id="L307" rel="#L307">307</span>
<span id="L308" rel="#L308">308</span>
<span id="L309" rel="#L309">309</span>
<span id="L310" rel="#L310">310</span>
<span id="L311" rel="#L311">311</span>
<span id="L312" rel="#L312">312</span>
<span id="L313" rel="#L313">313</span>
<span id="L314" rel="#L314">314</span>
<span id="L315" rel="#L315">315</span>
<span id="L316" rel="#L316">316</span>
<span id="L317" rel="#L317">317</span>
<span id="L318" rel="#L318">318</span>
<span id="L319" rel="#L319">319</span>
<span id="L320" rel="#L320">320</span>
<span id="L321" rel="#L321">321</span>
<span id="L322" rel="#L322">322</span>
<span id="L323" rel="#L323">323</span>
<span id="L324" rel="#L324">324</span>
<span id="L325" rel="#L325">325</span>
<span id="L326" rel="#L326">326</span>
<span id="L327" rel="#L327">327</span>
<span id="L328" rel="#L328">328</span>
<span id="L329" rel="#L329">329</span>
<span id="L330" rel="#L330">330</span>
<span id="L331" rel="#L331">331</span>
<span id="L332" rel="#L332">332</span>
<span id="L333" rel="#L333">333</span>
<span id="L334" rel="#L334">334</span>
<span id="L335" rel="#L335">335</span>
<span id="L336" rel="#L336">336</span>
<span id="L337" rel="#L337">337</span>
<span id="L338" rel="#L338">338</span>
<span id="L339" rel="#L339">339</span>
<span id="L340" rel="#L340">340</span>
<span id="L341" rel="#L341">341</span>
<span id="L342" rel="#L342">342</span>
<span id="L343" rel="#L343">343</span>
<span id="L344" rel="#L344">344</span>
<span id="L345" rel="#L345">345</span>
<span id="L346" rel="#L346">346</span>
<span id="L347" rel="#L347">347</span>
<span id="L348" rel="#L348">348</span>
<span id="L349" rel="#L349">349</span>
<span id="L350" rel="#L350">350</span>
<span id="L351" rel="#L351">351</span>
<span id="L352" rel="#L352">352</span>
<span id="L353" rel="#L353">353</span>
<span id="L354" rel="#L354">354</span>
<span id="L355" rel="#L355">355</span>
<span id="L356" rel="#L356">356</span>
<span id="L357" rel="#L357">357</span>
<span id="L358" rel="#L358">358</span>
<span id="L359" rel="#L359">359</span>
<span id="L360" rel="#L360">360</span>
<span id="L361" rel="#L361">361</span>
<span id="L362" rel="#L362">362</span>
<span id="L363" rel="#L363">363</span>
<span id="L364" rel="#L364">364</span>
<span id="L365" rel="#L365">365</span>
<span id="L366" rel="#L366">366</span>
<span id="L367" rel="#L367">367</span>
<span id="L368" rel="#L368">368</span>
<span id="L369" rel="#L369">369</span>
<span id="L370" rel="#L370">370</span>
<span id="L371" rel="#L371">371</span>
<span id="L372" rel="#L372">372</span>
<span id="L373" rel="#L373">373</span>
<span id="L374" rel="#L374">374</span>
<span id="L375" rel="#L375">375</span>
<span id="L376" rel="#L376">376</span>
<span id="L377" rel="#L377">377</span>
<span id="L378" rel="#L378">378</span>
<span id="L379" rel="#L379">379</span>
<span id="L380" rel="#L380">380</span>
<span id="L381" rel="#L381">381</span>
<span id="L382" rel="#L382">382</span>
<span id="L383" rel="#L383">383</span>
<span id="L384" rel="#L384">384</span>
<span id="L385" rel="#L385">385</span>
<span id="L386" rel="#L386">386</span>
<span id="L387" rel="#L387">387</span>
<span id="L388" rel="#L388">388</span>
<span id="L389" rel="#L389">389</span>
<span id="L390" rel="#L390">390</span>
<span id="L391" rel="#L391">391</span>
<span id="L392" rel="#L392">392</span>
<span id="L393" rel="#L393">393</span>
<span id="L394" rel="#L394">394</span>
<span id="L395" rel="#L395">395</span>
<span id="L396" rel="#L396">396</span>
<span id="L397" rel="#L397">397</span>
<span id="L398" rel="#L398">398</span>
<span id="L399" rel="#L399">399</span>
<span id="L400" rel="#L400">400</span>
<span id="L401" rel="#L401">401</span>
<span id="L402" rel="#L402">402</span>
<span id="L403" rel="#L403">403</span>
<span id="L404" rel="#L404">404</span>
<span id="L405" rel="#L405">405</span>
<span id="L406" rel="#L406">406</span>
<span id="L407" rel="#L407">407</span>
<span id="L408" rel="#L408">408</span>
<span id="L409" rel="#L409">409</span>
<span id="L410" rel="#L410">410</span>
<span id="L411" rel="#L411">411</span>
<span id="L412" rel="#L412">412</span>
<span id="L413" rel="#L413">413</span>
<span id="L414" rel="#L414">414</span>
<span id="L415" rel="#L415">415</span>
<span id="L416" rel="#L416">416</span>
<span id="L417" rel="#L417">417</span>
<span id="L418" rel="#L418">418</span>
<span id="L419" rel="#L419">419</span>
<span id="L420" rel="#L420">420</span>
<span id="L421" rel="#L421">421</span>
<span id="L422" rel="#L422">422</span>
<span id="L423" rel="#L423">423</span>
<span id="L424" rel="#L424">424</span>
<span id="L425" rel="#L425">425</span>
<span id="L426" rel="#L426">426</span>
<span id="L427" rel="#L427">427</span>
<span id="L428" rel="#L428">428</span>
<span id="L429" rel="#L429">429</span>
<span id="L430" rel="#L430">430</span>
<span id="L431" rel="#L431">431</span>
<span id="L432" rel="#L432">432</span>
<span id="L433" rel="#L433">433</span>
<span id="L434" rel="#L434">434</span>
<span id="L435" rel="#L435">435</span>
<span id="L436" rel="#L436">436</span>
<span id="L437" rel="#L437">437</span>
<span id="L438" rel="#L438">438</span>
<span id="L439" rel="#L439">439</span>
<span id="L440" rel="#L440">440</span>
<span id="L441" rel="#L441">441</span>
<span id="L442" rel="#L442">442</span>
<span id="L443" rel="#L443">443</span>
<span id="L444" rel="#L444">444</span>
<span id="L445" rel="#L445">445</span>
<span id="L446" rel="#L446">446</span>
<span id="L447" rel="#L447">447</span>
<span id="L448" rel="#L448">448</span>
<span id="L449" rel="#L449">449</span>
<span id="L450" rel="#L450">450</span>
<span id="L451" rel="#L451">451</span>
<span id="L452" rel="#L452">452</span>
<span id="L453" rel="#L453">453</span>
<span id="L454" rel="#L454">454</span>
<span id="L455" rel="#L455">455</span>
<span id="L456" rel="#L456">456</span>
<span id="L457" rel="#L457">457</span>
<span id="L458" rel="#L458">458</span>
<span id="L459" rel="#L459">459</span>
<span id="L460" rel="#L460">460</span>
<span id="L461" rel="#L461">461</span>
<span id="L462" rel="#L462">462</span>
<span id="L463" rel="#L463">463</span>
<span id="L464" rel="#L464">464</span>
<span id="L465" rel="#L465">465</span>
<span id="L466" rel="#L466">466</span>
<span id="L467" rel="#L467">467</span>
<span id="L468" rel="#L468">468</span>
<span id="L469" rel="#L469">469</span>
<span id="L470" rel="#L470">470</span>
<span id="L471" rel="#L471">471</span>
<span id="L472" rel="#L472">472</span>
<span id="L473" rel="#L473">473</span>
<span id="L474" rel="#L474">474</span>
<span id="L475" rel="#L475">475</span>
<span id="L476" rel="#L476">476</span>
<span id="L477" rel="#L477">477</span>
<span id="L478" rel="#L478">478</span>
<span id="L479" rel="#L479">479</span>
<span id="L480" rel="#L480">480</span>
<span id="L481" rel="#L481">481</span>
<span id="L482" rel="#L482">482</span>
<span id="L483" rel="#L483">483</span>
<span id="L484" rel="#L484">484</span>
<span id="L485" rel="#L485">485</span>
<span id="L486" rel="#L486">486</span>
<span id="L487" rel="#L487">487</span>
<span id="L488" rel="#L488">488</span>
<span id="L489" rel="#L489">489</span>
<span id="L490" rel="#L490">490</span>
<span id="L491" rel="#L491">491</span>
<span id="L492" rel="#L492">492</span>
<span id="L493" rel="#L493">493</span>
<span id="L494" rel="#L494">494</span>
<span id="L495" rel="#L495">495</span>
<span id="L496" rel="#L496">496</span>
<span id="L497" rel="#L497">497</span>
<span id="L498" rel="#L498">498</span>
<span id="L499" rel="#L499">499</span>
<span id="L500" rel="#L500">500</span>
<span id="L501" rel="#L501">501</span>
<span id="L502" rel="#L502">502</span>
<span id="L503" rel="#L503">503</span>
<span id="L504" rel="#L504">504</span>
<span id="L505" rel="#L505">505</span>
<span id="L506" rel="#L506">506</span>
<span id="L507" rel="#L507">507</span>
<span id="L508" rel="#L508">508</span>
<span id="L509" rel="#L509">509</span>
<span id="L510" rel="#L510">510</span>
<span id="L511" rel="#L511">511</span>
<span id="L512" rel="#L512">512</span>
<span id="L513" rel="#L513">513</span>
<span id="L514" rel="#L514">514</span>
<span id="L515" rel="#L515">515</span>
<span id="L516" rel="#L516">516</span>
<span id="L517" rel="#L517">517</span>
<span id="L518" rel="#L518">518</span>
<span id="L519" rel="#L519">519</span>
<span id="L520" rel="#L520">520</span>
<span id="L521" rel="#L521">521</span>
<span id="L522" rel="#L522">522</span>
<span id="L523" rel="#L523">523</span>
<span id="L524" rel="#L524">524</span>
<span id="L525" rel="#L525">525</span>
<span id="L526" rel="#L526">526</span>
<span id="L527" rel="#L527">527</span>
<span id="L528" rel="#L528">528</span>
<span id="L529" rel="#L529">529</span>
<span id="L530" rel="#L530">530</span>
<span id="L531" rel="#L531">531</span>
<span id="L532" rel="#L532">532</span>
<span id="L533" rel="#L533">533</span>
<span id="L534" rel="#L534">534</span>
<span id="L535" rel="#L535">535</span>
<span id="L536" rel="#L536">536</span>
<span id="L537" rel="#L537">537</span>
<span id="L538" rel="#L538">538</span>
<span id="L539" rel="#L539">539</span>
<span id="L540" rel="#L540">540</span>
<span id="L541" rel="#L541">541</span>
<span id="L542" rel="#L542">542</span>
<span id="L543" rel="#L543">543</span>
<span id="L544" rel="#L544">544</span>
<span id="L545" rel="#L545">545</span>
<span id="L546" rel="#L546">546</span>
<span id="L547" rel="#L547">547</span>
<span id="L548" rel="#L548">548</span>
<span id="L549" rel="#L549">549</span>
<span id="L550" rel="#L550">550</span>
<span id="L551" rel="#L551">551</span>
<span id="L552" rel="#L552">552</span>
<span id="L553" rel="#L553">553</span>
<span id="L554" rel="#L554">554</span>
<span id="L555" rel="#L555">555</span>
<span id="L556" rel="#L556">556</span>
<span id="L557" rel="#L557">557</span>
<span id="L558" rel="#L558">558</span>
<span id="L559" rel="#L559">559</span>
<span id="L560" rel="#L560">560</span>
<span id="L561" rel="#L561">561</span>
<span id="L562" rel="#L562">562</span>
<span id="L563" rel="#L563">563</span>
<span id="L564" rel="#L564">564</span>
<span id="L565" rel="#L565">565</span>
<span id="L566" rel="#L566">566</span>
<span id="L567" rel="#L567">567</span>
<span id="L568" rel="#L568">568</span>
<span id="L569" rel="#L569">569</span>
<span id="L570" rel="#L570">570</span>
<span id="L571" rel="#L571">571</span>
<span id="L572" rel="#L572">572</span>
<span id="L573" rel="#L573">573</span>
<span id="L574" rel="#L574">574</span>
<span id="L575" rel="#L575">575</span>
<span id="L576" rel="#L576">576</span>
<span id="L577" rel="#L577">577</span>
<span id="L578" rel="#L578">578</span>
<span id="L579" rel="#L579">579</span>
<span id="L580" rel="#L580">580</span>
<span id="L581" rel="#L581">581</span>
<span id="L582" rel="#L582">582</span>
<span id="L583" rel="#L583">583</span>
<span id="L584" rel="#L584">584</span>
<span id="L585" rel="#L585">585</span>
<span id="L586" rel="#L586">586</span>
<span id="L587" rel="#L587">587</span>
<span id="L588" rel="#L588">588</span>
<span id="L589" rel="#L589">589</span>
<span id="L590" rel="#L590">590</span>
<span id="L591" rel="#L591">591</span>
<span id="L592" rel="#L592">592</span>
<span id="L593" rel="#L593">593</span>
<span id="L594" rel="#L594">594</span>
<span id="L595" rel="#L595">595</span>
<span id="L596" rel="#L596">596</span>
<span id="L597" rel="#L597">597</span>
<span id="L598" rel="#L598">598</span>
<span id="L599" rel="#L599">599</span>
<span id="L600" rel="#L600">600</span>
<span id="L601" rel="#L601">601</span>
<span id="L602" rel="#L602">602</span>
<span id="L603" rel="#L603">603</span>
<span id="L604" rel="#L604">604</span>
<span id="L605" rel="#L605">605</span>
<span id="L606" rel="#L606">606</span>
<span id="L607" rel="#L607">607</span>
<span id="L608" rel="#L608">608</span>
<span id="L609" rel="#L609">609</span>
<span id="L610" rel="#L610">610</span>
<span id="L611" rel="#L611">611</span>
<span id="L612" rel="#L612">612</span>
<span id="L613" rel="#L613">613</span>
<span id="L614" rel="#L614">614</span>
<span id="L615" rel="#L615">615</span>
<span id="L616" rel="#L616">616</span>
<span id="L617" rel="#L617">617</span>
<span id="L618" rel="#L618">618</span>
<span id="L619" rel="#L619">619</span>
<span id="L620" rel="#L620">620</span>
<span id="L621" rel="#L621">621</span>
<span id="L622" rel="#L622">622</span>
<span id="L623" rel="#L623">623</span>
<span id="L624" rel="#L624">624</span>
<span id="L625" rel="#L625">625</span>
<span id="L626" rel="#L626">626</span>
<span id="L627" rel="#L627">627</span>
<span id="L628" rel="#L628">628</span>
<span id="L629" rel="#L629">629</span>
<span id="L630" rel="#L630">630</span>
<span id="L631" rel="#L631">631</span>
<span id="L632" rel="#L632">632</span>
<span id="L633" rel="#L633">633</span>
<span id="L634" rel="#L634">634</span>
<span id="L635" rel="#L635">635</span>
<span id="L636" rel="#L636">636</span>
<span id="L637" rel="#L637">637</span>
<span id="L638" rel="#L638">638</span>
<span id="L639" rel="#L639">639</span>
<span id="L640" rel="#L640">640</span>
<span id="L641" rel="#L641">641</span>
<span id="L642" rel="#L642">642</span>
<span id="L643" rel="#L643">643</span>
<span id="L644" rel="#L644">644</span>
<span id="L645" rel="#L645">645</span>
<span id="L646" rel="#L646">646</span>
<span id="L647" rel="#L647">647</span>
<span id="L648" rel="#L648">648</span>
<span id="L649" rel="#L649">649</span>
<span id="L650" rel="#L650">650</span>
<span id="L651" rel="#L651">651</span>
<span id="L652" rel="#L652">652</span>
<span id="L653" rel="#L653">653</span>
<span id="L654" rel="#L654">654</span>
<span id="L655" rel="#L655">655</span>
<span id="L656" rel="#L656">656</span>
<span id="L657" rel="#L657">657</span>
<span id="L658" rel="#L658">658</span>
<span id="L659" rel="#L659">659</span>
<span id="L660" rel="#L660">660</span>
<span id="L661" rel="#L661">661</span>
<span id="L662" rel="#L662">662</span>
<span id="L663" rel="#L663">663</span>
<span id="L664" rel="#L664">664</span>
<span id="L665" rel="#L665">665</span>
<span id="L666" rel="#L666">666</span>
<span id="L667" rel="#L667">667</span>
<span id="L668" rel="#L668">668</span>
<span id="L669" rel="#L669">669</span>
<span id="L670" rel="#L670">670</span>
<span id="L671" rel="#L671">671</span>
<span id="L672" rel="#L672">672</span>
<span id="L673" rel="#L673">673</span>
<span id="L674" rel="#L674">674</span>
<span id="L675" rel="#L675">675</span>
<span id="L676" rel="#L676">676</span>
<span id="L677" rel="#L677">677</span>
<span id="L678" rel="#L678">678</span>
<span id="L679" rel="#L679">679</span>
<span id="L680" rel="#L680">680</span>
<span id="L681" rel="#L681">681</span>
<span id="L682" rel="#L682">682</span>
<span id="L683" rel="#L683">683</span>
<span id="L684" rel="#L684">684</span>
<span id="L685" rel="#L685">685</span>
<span id="L686" rel="#L686">686</span>
<span id="L687" rel="#L687">687</span>
<span id="L688" rel="#L688">688</span>
<span id="L689" rel="#L689">689</span>
<span id="L690" rel="#L690">690</span>
<span id="L691" rel="#L691">691</span>
<span id="L692" rel="#L692">692</span>
<span id="L693" rel="#L693">693</span>
<span id="L694" rel="#L694">694</span>
<span id="L695" rel="#L695">695</span>
<span id="L696" rel="#L696">696</span>
<span id="L697" rel="#L697">697</span>
<span id="L698" rel="#L698">698</span>
<span id="L699" rel="#L699">699</span>
<span id="L700" rel="#L700">700</span>
<span id="L701" rel="#L701">701</span>
<span id="L702" rel="#L702">702</span>
<span id="L703" rel="#L703">703</span>
<span id="L704" rel="#L704">704</span>
<span id="L705" rel="#L705">705</span>
<span id="L706" rel="#L706">706</span>
<span id="L707" rel="#L707">707</span>
<span id="L708" rel="#L708">708</span>
<span id="L709" rel="#L709">709</span>
<span id="L710" rel="#L710">710</span>
<span id="L711" rel="#L711">711</span>
<span id="L712" rel="#L712">712</span>
<span id="L713" rel="#L713">713</span>
<span id="L714" rel="#L714">714</span>
<span id="L715" rel="#L715">715</span>
<span id="L716" rel="#L716">716</span>
<span id="L717" rel="#L717">717</span>
<span id="L718" rel="#L718">718</span>
<span id="L719" rel="#L719">719</span>
<span id="L720" rel="#L720">720</span>
<span id="L721" rel="#L721">721</span>
<span id="L722" rel="#L722">722</span>
<span id="L723" rel="#L723">723</span>
<span id="L724" rel="#L724">724</span>
<span id="L725" rel="#L725">725</span>
<span id="L726" rel="#L726">726</span>
<span id="L727" rel="#L727">727</span>
<span id="L728" rel="#L728">728</span>
<span id="L729" rel="#L729">729</span>
<span id="L730" rel="#L730">730</span>
<span id="L731" rel="#L731">731</span>
<span id="L732" rel="#L732">732</span>
<span id="L733" rel="#L733">733</span>
<span id="L734" rel="#L734">734</span>
<span id="L735" rel="#L735">735</span>
<span id="L736" rel="#L736">736</span>
<span id="L737" rel="#L737">737</span>
<span id="L738" rel="#L738">738</span>
<span id="L739" rel="#L739">739</span>
<span id="L740" rel="#L740">740</span>
<span id="L741" rel="#L741">741</span>
<span id="L742" rel="#L742">742</span>
<span id="L743" rel="#L743">743</span>
<span id="L744" rel="#L744">744</span>
<span id="L745" rel="#L745">745</span>
<span id="L746" rel="#L746">746</span>
<span id="L747" rel="#L747">747</span>
<span id="L748" rel="#L748">748</span>
<span id="L749" rel="#L749">749</span>
<span id="L750" rel="#L750">750</span>
<span id="L751" rel="#L751">751</span>
<span id="L752" rel="#L752">752</span>
<span id="L753" rel="#L753">753</span>
<span id="L754" rel="#L754">754</span>
<span id="L755" rel="#L755">755</span>
<span id="L756" rel="#L756">756</span>
<span id="L757" rel="#L757">757</span>
<span id="L758" rel="#L758">758</span>
<span id="L759" rel="#L759">759</span>
<span id="L760" rel="#L760">760</span>
<span id="L761" rel="#L761">761</span>
<span id="L762" rel="#L762">762</span>
<span id="L763" rel="#L763">763</span>
<span id="L764" rel="#L764">764</span>
<span id="L765" rel="#L765">765</span>
<span id="L766" rel="#L766">766</span>
<span id="L767" rel="#L767">767</span>
<span id="L768" rel="#L768">768</span>
<span id="L769" rel="#L769">769</span>
<span id="L770" rel="#L770">770</span>
<span id="L771" rel="#L771">771</span>
<span id="L772" rel="#L772">772</span>
<span id="L773" rel="#L773">773</span>
<span id="L774" rel="#L774">774</span>
<span id="L775" rel="#L775">775</span>
<span id="L776" rel="#L776">776</span>
<span id="L777" rel="#L777">777</span>
<span id="L778" rel="#L778">778</span>
<span id="L779" rel="#L779">779</span>
<span id="L780" rel="#L780">780</span>
<span id="L781" rel="#L781">781</span>
<span id="L782" rel="#L782">782</span>
<span id="L783" rel="#L783">783</span>
<span id="L784" rel="#L784">784</span>
<span id="L785" rel="#L785">785</span>
<span id="L786" rel="#L786">786</span>
<span id="L787" rel="#L787">787</span>
<span id="L788" rel="#L788">788</span>
<span id="L789" rel="#L789">789</span>
<span id="L790" rel="#L790">790</span>
<span id="L791" rel="#L791">791</span>
<span id="L792" rel="#L792">792</span>
<span id="L793" rel="#L793">793</span>
<span id="L794" rel="#L794">794</span>
<span id="L795" rel="#L795">795</span>
<span id="L796" rel="#L796">796</span>
<span id="L797" rel="#L797">797</span>
<span id="L798" rel="#L798">798</span>
<span id="L799" rel="#L799">799</span>
<span id="L800" rel="#L800">800</span>
<span id="L801" rel="#L801">801</span>
<span id="L802" rel="#L802">802</span>
<span id="L803" rel="#L803">803</span>
<span id="L804" rel="#L804">804</span>
</pre>
          </td>
          <td width="100%">
                  <div class="highlight"><pre><div class='line' id='LC1'><span class="c">%%</span></div><div class='line' id='LC2'><span class="c">%% Copyright 2007, 2008, 2009 Elsevier Ltd</span></div><div class='line' id='LC3'><span class="c">%%</span></div><div class='line' id='LC4'><span class="c">%% This file is part of the &#39;Elsarticle Bundle&#39;.</span></div><div class='line' id='LC5'><span class="c">%% ---------------------------------------------</span></div><div class='line' id='LC6'><span class="c">%%</span></div><div class='line' id='LC7'><span class="c">%% It may be distributed under the conditions of the LaTeX Project Public</span></div><div class='line' id='LC8'><span class="c">%% License, either version 1.2 of this license or (at your option) any</span></div><div class='line' id='LC9'><span class="c">%% later version.  The latest version of this license is in</span></div><div class='line' id='LC10'><span class="c">%%    http://www.latex-project.org/lppl.txt</span></div><div class='line' id='LC11'><span class="c">%% and version 1.2 or later is part of all distributions of LaTeX</span></div><div class='line' id='LC12'><span class="c">%% version 1999/12/01 or later.</span></div><div class='line' id='LC13'><span class="c">%%</span></div><div class='line' id='LC14'><span class="c">%% The list of all files belonging to the &#39;Elsarticle Bundle&#39; is</span></div><div class='line' id='LC15'><span class="c">%% given in the file `manifest.txt&#39;.</span></div><div class='line' id='LC16'><span class="c">%%</span></div><div class='line' id='LC17'><br/></div><div class='line' id='LC18'><span class="c">%% Template article for Elsevier&#39;s document class `elsarticle&#39;</span></div><div class='line' id='LC19'><span class="c">%% with harvard style bibliographic references</span></div><div class='line' id='LC20'><span class="c">%% SP 2008/03/01</span></div><div class='line' id='LC21'><span class="c">%%</span></div><div class='line' id='LC22'><span class="c">%%</span></div><div class='line' id='LC23'><span class="c">%%</span></div><div class='line' id='LC24'><span class="c">%% $Id: elsarticle-template-harv.tex 4 2009-10-24 08:22:58Z rishi $</span></div><div class='line' id='LC25'><span class="c">%%</span></div><div class='line' id='LC26'><span class="c">%%</span></div><div class='line' id='LC27'><span class="k">\documentclass</span><span class="na">[preprint,authoryear,12pt]</span><span class="nb">{</span>elsarticle<span class="nb">}</span></div><div class='line' id='LC28'><br/></div><div class='line' id='LC29'><span class="c">%% Use the option review to obtain double line spacing</span></div><div class='line' id='LC30'><span class="c">%% \documentclass[authoryear,preprint,review,12pt]{elsarticle}</span></div><div class='line' id='LC31'><br/></div><div class='line' id='LC32'><span class="c">%% Use the options 1p,twocolumn; 3p; 3p,twocolumn; 5p; or 5p,twocolumn</span></div><div class='line' id='LC33'><span class="c">%% for a journal layout:</span></div><div class='line' id='LC34'><span class="c">%% \documentclass[final,authoryear,1p,times]{elsarticle}</span></div><div class='line' id='LC35'><span class="c">%% \documentclass[final,authoryear,1p,times,twocolumn]{elsarticle}</span></div><div class='line' id='LC36'><span class="c">%% \documentclass[final,authoryear,3p,times]{elsarticle}</span></div><div class='line' id='LC37'><span class="c">%% \documentclass[final,authoryear,3p,times,twocolumn]{elsarticle}</span></div><div class='line' id='LC38'><span class="c">%% \documentclass[final,authoryear,5p,times]{elsarticle}</span></div><div class='line' id='LC39'><span class="c">%% \documentclass[final,authoryear,5p,times,twocolumn]{elsarticle}</span></div><div class='line' id='LC40'><br/></div><div class='line' id='LC41'><span class="c">%% if you use PostScript figures in your article</span></div><div class='line' id='LC42'><span class="c">%% use the graphics package for simple commands</span></div><div class='line' id='LC43'><span class="c">%% \usepackage{graphics}</span></div><div class='line' id='LC44'><span class="c">%% or use the graphicx package for more complicated commands</span></div><div class='line' id='LC45'><span class="c">%% \usepackage{graphicx}</span></div><div class='line' id='LC46'><span class="c">%% or use the epsfig package if you prefer to use the old commands</span></div><div class='line' id='LC47'><span class="c">%% \usepackage{epsfig}</span></div><div class='line' id='LC48'><br/></div><div class='line' id='LC49'><span class="c">%% The amssymb package provides various useful mathematical symbols</span></div><div class='line' id='LC50'><span class="k">\usepackage</span><span class="nb">{</span>amssymb<span class="nb">}</span></div><div class='line' id='LC51'><span class="c">%% The amsthm package provides extended theorem environments</span></div><div class='line' id='LC52'><span class="c">%% \usepackage{amsthm}</span></div><div class='line' id='LC53'><br/></div><div class='line' id='LC54'><span class="c">%% The lineno packages adds line numbers. Start line numbering with</span></div><div class='line' id='LC55'><span class="c">%% \begin{linenumbers}, end it with \end{linenumbers}. Or switch it on</span></div><div class='line' id='LC56'><span class="c">%% for the whole article with \linenumbers after \end{frontmatter}.</span></div><div class='line' id='LC57'><span class="c">%% \usepackage{lineno}</span></div><div class='line' id='LC58'><br/></div><div class='line' id='LC59'><span class="c">%% natbib.sty is loaded by default. However, natbib options can be</span></div><div class='line' id='LC60'><span class="c">%% provided with \biboptions{...} command. Following options are</span></div><div class='line' id='LC61'><span class="c">%% valid:</span></div><div class='line' id='LC62'><br/></div><div class='line' id='LC63'><span class="c">%%   round  -  round parentheses are used (default)</span></div><div class='line' id='LC64'><span class="c">%%   square -  square brackets are used   [option]</span></div><div class='line' id='LC65'><span class="c">%%   curly  -  curly braces are used      {option}</span></div><div class='line' id='LC66'><span class="c">%%   angle  -  angle brackets are used    &lt;option&gt;</span></div><div class='line' id='LC67'><span class="c">%%   semicolon  -  multiple citations separated by semi-colon (default)</span></div><div class='line' id='LC68'><span class="c">%%   colon  - same as semicolon, an earlier confusion</span></div><div class='line' id='LC69'><span class="c">%%   comma  -  separated by comma</span></div><div class='line' id='LC70'><span class="c">%%   authoryear - selects author-year citations (default)</span></div><div class='line' id='LC71'><span class="c">%%   numbers-  selects numerical citations</span></div><div class='line' id='LC72'><span class="c">%%   super  -  numerical citations as superscripts</span></div><div class='line' id='LC73'><span class="c">%%   sort   -  sorts multiple citations according to order in ref. list</span></div><div class='line' id='LC74'><span class="c">%%   sort&amp;compress   -  like sort, but also compresses numerical citations</span></div><div class='line' id='LC75'><span class="c">%%   compress - compresses without sorting</span></div><div class='line' id='LC76'><span class="c">%%   longnamesfirst  -  makes first citation full author list</span></div><div class='line' id='LC77'><span class="c">%%</span></div><div class='line' id='LC78'><span class="c">%% \biboptions{longnamesfirst,comma}</span></div><div class='line' id='LC79'><br/></div><div class='line' id='LC80'><span class="c">% \biboptions{}</span></div><div class='line' id='LC81'><br/></div><div class='line' id='LC82'><span class="k">\journal</span><span class="nb">{</span>Transportation Research: Part C<span class="nb">}</span></div><div class='line' id='LC83'><br/></div><div class='line' id='LC84'><span class="k">\begin</span><span class="nb">{</span>document<span class="nb">}</span></div><div class='line' id='LC85'><br/></div><div class='line' id='LC86'><span class="k">\begin</span><span class="nb">{</span>frontmatter<span class="nb">}</span></div><div class='line' id='LC87'><br/></div><div class='line' id='LC88'><span class="c">%% Title, authors and addresses</span></div><div class='line' id='LC89'><br/></div><div class='line' id='LC90'><span class="c">%% use the tnoteref command within \title for footnotes;</span></div><div class='line' id='LC91'><span class="c">%% use the tnotetext command for the associated footnote;</span></div><div class='line' id='LC92'><span class="c">%% use the fnref command within \author or \address for footnotes;</span></div><div class='line' id='LC93'><span class="c">%% use the fntext command for the associated footnote;</span></div><div class='line' id='LC94'><span class="c">%% use the corref command within \author for corresponding author footnotes;</span></div><div class='line' id='LC95'><span class="c">%% use the cortext command for the associated footnote;</span></div><div class='line' id='LC96'><span class="c">%% use the ead command for the email address,</span></div><div class='line' id='LC97'><span class="c">%% and the form \ead[url] for the home page:</span></div><div class='line' id='LC98'><span class="c">%%</span></div><div class='line' id='LC99'><span class="c">%% \title{Title\tnoteref{label1}}</span></div><div class='line' id='LC100'><span class="c">%% \tnotetext[label1]{}</span></div><div class='line' id='LC101'><span class="c">%% \author{Name\corref{cor1}\fnref{label2}}</span></div><div class='line' id='LC102'><span class="c">%% \ead{email address}</span></div><div class='line' id='LC103'><span class="c">%% \ead[url]{home page}</span></div><div class='line' id='LC104'><span class="c">%% \fntext[label2]{}</span></div><div class='line' id='LC105'><span class="c">%% \cortext[cor1]{}</span></div><div class='line' id='LC106'><span class="c">%% \address{Address\fnref{label3}}</span></div><div class='line' id='LC107'><span class="c">%% \fntext[label3]{}</span></div><div class='line' id='LC108'><br/></div><div class='line' id='LC109'><span class="k">\title</span><span class="nb">{</span>Articulo bonito que queremos que publiquen<span class="nb">}</span></div><div class='line' id='LC110'><br/></div><div class='line' id='LC111'><span class="c">%% use optional labels to link authors explicitly to addresses:</span></div><div class='line' id='LC112'><span class="c">%% \author[label1,label2]{&lt;author name&gt;}</span></div><div class='line' id='LC113'><span class="c">%% \address[label1]{&lt;address&gt;}</span></div><div class='line' id='LC114'><span class="c">%% \address[label2]{&lt;address&gt;}</span></div><div class='line' id='LC115'><span class="k">\author</span><span class="nb">{</span>Muchos y bonitos autores<span class="nb">}</span></div><div class='line' id='LC116'><span class="k">\author</span><span class="na">[uja]</span><span class="nb">{</span>V.M. Rivas<span class="k">\corref</span><span class="nb">{</span>vrivas<span class="nb">}}</span></div><div class='line' id='LC117'><span class="k">\ead</span><span class="nb">{</span>vrivas@ujaen.es<span class="nb">}</span></div><div class='line' id='LC118'><span class="k">\ead</span><span class="na">[url]</span><span class="nb">{</span>http://geneura.wordpress.com<span class="nb">}</span></div><div class='line' id='LC119'><span class="k">\author</span><span class="na">[ugr]</span><span class="nb">{</span>M.G. Arenas<span class="nb">}</span></div><div class='line' id='LC120'><span class="k">\author</span><span class="na">[ugr]</span><span class="nb">{</span>P.A. Castillo-Valdivieso<span class="nb">}</span></div><div class='line' id='LC121'><span class="k">\address</span><span class="na">[uja]</span><span class="nb">{</span>Department of Computer Sciences, University of Jaen (Spain)<span class="nb">}</span></div><div class='line' id='LC122'><span class="k">\address</span><span class="na">[ugr]</span><span class="nb">{</span>Department of Architecture and Computer Technology. CITIC. University of Granada (Spain)<span class="nb">}</span></div><div class='line' id='LC123'><span class="k">\cortext</span><span class="na">[vrivas]</span><span class="nb">{</span>Campus Las Lagunillas S/N; E23071, Jaen (Spain)<span class="nb">}</span></div><div class='line' id='LC124'><br/></div><div class='line' id='LC125'><br/></div><div class='line' id='LC126'><span class="k">\address</span><span class="nb">{}</span></div><div class='line' id='LC127'><br/></div><div class='line' id='LC128'><span class="k">\begin</span><span class="nb">{</span>abstract<span class="nb">}</span></div><div class='line' id='LC129'><span class="c">%% Text of abstract</span></div><div class='line' id='LC130'>Blah, blah, blah...</div><div class='line' id='LC131'><span class="k">\end</span><span class="nb">{</span>abstract<span class="nb">}</span></div><div class='line' id='LC132'><br/></div><div class='line' id='LC133'><span class="k">\begin</span><span class="nb">{</span>keyword<span class="nb">}</span></div><div class='line' id='LC134'><span class="c">%% keywords here, in the form: keyword \sep keyword</span></div><div class='line' id='LC135'><br/></div><div class='line' id='LC136'><span class="c">%% MSC codes here, in the form: \MSC code \sep code</span></div><div class='line' id='LC137'><span class="c">%% or \MSC[2008] code \sep code (2000 is the default)</span></div><div class='line' id='LC138'><br/></div><div class='line' id='LC139'><span class="k">\end</span><span class="nb">{</span>keyword<span class="nb">}</span></div><div class='line' id='LC140'><br/></div><div class='line' id='LC141'><span class="k">\end</span><span class="nb">{</span>frontmatter<span class="nb">}</span></div><div class='line' id='LC142'><br/></div><div class='line' id='LC143'><span class="c">% \linenumbers</span></div><div class='line' id='LC144'><br/></div><div class='line' id='LC145'><span class="c">%% main text</span></div><div class='line' id='LC146'><span class="k">\section</span><span class="nb">{</span>Introduction<span class="nb">}</span></div><div class='line' id='LC147'><span class="k">\label</span><span class="nb">{</span>sec:introduction<span class="nb">}</span></div><div class='line' id='LC148'>The development of new systems of information on traffic conditions and use of roads made by vehicles seems a key point in the current context. </div><div class='line' id='LC149'>With a increasingly informed population, provided with ubiquitous communication devices, which are commonly used by about <span class="s">$</span><span class="nb"> </span><span class="m">90</span><span class="nb"> </span><span class="nv">\%</span><span class="nb"> </span><span class="s">$</span> of the population, obtaining information about the traffic in any of the more than 165,000 kilometres of Spanish roads would mean to optimally manage a communications network vital for a high percentage of users.</div><div class='line' id='LC150'><br/></div><div class='line' id='LC151'>In this work, a system based on bluetooth (BT) device discovery is proposed, so that an information system on the traffic status is obtained, as well as a valuable set of data suitable to be used in a time-series forecasting process.</div><div class='line' id='LC152'><br/></div><div class='line' id='LC153'>Both, the data-collecting and the time-series forecasting processes are necessary steps to achieve an ultimate goal: to have information about traffic flows that occur or will occur in a certain area, allowing to optimally manage motion decisions by citizens.</div><div class='line' id='LC154'><br/></div><div class='line' id='LC155'>Therefore, from the viewpoint of the transport management various needs have been found:</div><div class='line' id='LC156'><br/></div><div class='line' id='LC157'><span class="k">\begin</span><span class="nb">{</span>itemize<span class="nb">}</span></div><div class='line' id='LC158'>&nbsp;&nbsp;<span class="k">\item</span> A versatile, autonomous data collector, and monitoring device</div><div class='line' id='LC159'>&nbsp;&nbsp;<span class="k">\item</span> Traffic data-collecting in real time.</div><div class='line' id='LC160'>&nbsp;&nbsp;<span class="k">\item</span> The adequate processing of the data being collected,</div><div class='line' id='LC161'>&nbsp;&nbsp;<span class="k">\item</span> And finally, a system that allows sharing data and information with those who make decisions about mobility.</div><div class='line' id='LC162'><span class="k">\end</span><span class="nb">{</span>itemize<span class="nb">}</span></div><div class='line' id='LC163'><br/></div><div class='line' id='LC164'>In order to fulfil these requirements, the BT device discovery used for this work is able to catch waves emitted by different technological components. The components can be the ones embedded in vehicles (hands-free or GPS), accessories that the users incorporate to their vehicles, as well as mobile phones, tablets or laptops.</div><div class='line' id='LC165'>The main data being collected is the MAC address of the device BT card.</div><div class='line' id='LC166'>This is an unique identifier for each device, so that passing vehicles can be identified.</div><div class='line' id='LC167'>The intrusiveness is minimal since one-way encryption algorithms are used to hide the real MAC addresses stored. And, it is also safe from the point of view of data privacy given that this piece of information can not be attached to any given person.</div><div class='line' id='LC168'><br/></div><div class='line' id='LC169'>The rest of the paper shows how the large amount of data collected, related to passing BT devices, is considered as the starting point that allows to calculate statistics, to study several indicators about the use of vehicles, and to perform time-series forecasting by the monitored area population.</div><div class='line' id='LC170'>Fot this reason, the paper is organized as follows:</div><div class='line' id='LC171'>???????In Section <span class="k">\ref</span><span class="nb">{</span>soa<span class="nb">}</span> current technologies to monitor the traffic that passes through a certain area is summarized.</div><div class='line' id='LC172'>Section <span class="k">\ref</span><span class="nb">{</span>obj<span class="nb">}</span> details the goals of this paper.</div><div class='line' id='LC173'>In Section <span class="k">\ref</span><span class="nb">{</span>hw<span class="nb">}</span>, the Intelify device is presented.</div><div class='line' id='LC174'>In Section <span class="k">\ref</span><span class="nb">{</span>analisis<span class="nb">}</span> several analysis and statistics are reported from the data obtained.</div><div class='line' id='LC175'>Finally, we present some conclusions and future work (Section 5).</div><div class='line' id='LC176'><br/></div><div class='line' id='LC177'><br/></div><div class='line' id='LC178'><span class="k">\section</span><span class="nb">{</span>Preliminaries<span class="nb">}</span></div><div class='line' id='LC179'><span class="k">\label</span><span class="nb">{</span>sec:soa<span class="nb">}</span></div><div class='line' id='LC180'>This section introduces the two techniques used in this work. In the one hand, the methods developed to detect traffic flows; in the other hand, algorithms and techniques suitable to be used for time-series prediction tasks.</div><div class='line' id='LC181'><br/></div><div class='line' id='LC182'><span class="k">\subsection</span><span class="nb">{</span>Traffic detection technologies<span class="nb">}</span></div><div class='line' id='LC183'><span class="k">\label</span><span class="nb">{</span>subsec:traffic<span class="nb">}</span></div><div class='line' id='LC184'><br/></div><div class='line' id='LC185'>Traffic detection technologies can be generally classified into two groups: intrusive and nonintrusive.</div><div class='line' id='LC186'><br/></div><div class='line' id='LC187'>Intrusive detection technologies are installed on or within the roadway, requiring lane closures. Using this type of technology is inherently more hazardous and is generally more time consuming, especially for temporary traffic data collection. This technology has a number of drawbacks:</div><div class='line' id='LC188'><span class="k">\begin</span><span class="nb">{</span>itemize<span class="nb">}</span></div><div class='line' id='LC189'>&nbsp;&nbsp;<span class="k">\item</span> Installation requires pavement cut.</div><div class='line' id='LC190'>&nbsp;&nbsp;<span class="k">\item</span> Improper installation decreases pavement life.</div><div class='line' id='LC191'>&nbsp;&nbsp;<span class="k">\item</span> Installation and maintenance require lane closure.</div><div class='line' id='LC192'>&nbsp;&nbsp;<span class="k">\item</span> Detection accuracy may decrease when design requires detection of a large variety of vehicle classes.</div><div class='line' id='LC193'>&nbsp;&nbsp;<span class="k">\item</span> Poor pavement condition can dramatically shorten the life span of intrusive sensors.</div><div class='line' id='LC194'><span class="k">\end</span><span class="nb">{</span>itemize<span class="nb">}</span></div><div class='line' id='LC195'><br/></div><div class='line' id='LC196'>Non-intrusive technologies are traffic detection sensors that cause minimal disruption to normal traffic operations during installation, operation and maintenance compared to conventional detection methods. They can also be deployed more safely than conventional detection methods, since they are located adjacent to the roadway and require minimal interaction with traffic flow. Non-intrusive technologies can be classified in two big groups: active technologies (microwave radar, ultrasonic and laser radar), and passive technologies (infrared, acoustic and video image processing). </div><div class='line' id='LC197'><br/></div><div class='line' id='LC198'>Figure <span class="k">\ref</span><span class="nb">{</span>tipossensores<span class="nb">}</span> shows a classification of information systems according to the intrusiveness of the technology.</div><div class='line' id='LC199'><br/></div><div class='line' id='LC200'><span class="k">\begin</span><span class="nb">{</span>figure<span class="nb">}</span>[htpb] </div><div class='line' id='LC201'><span class="k">\begin</span><span class="nb">{</span>center<span class="nb">}</span> </div><div class='line' id='LC202'><span class="k">\includegraphics</span><span class="na">[scale=0.55]</span><span class="nb">{</span>intrusivos.jpeg<span class="nb">}</span>  <span class="c">%tipossensores.jpg}</span></div><div class='line' id='LC203'><span class="k">\end</span><span class="nb">{</span>center<span class="nb">}</span> </div><div class='line' id='LC204'><span class="k">\caption</span><span class="nb">{</span>Information systems classification, according to the intrusiveness of the technology.<span class="nb">}</span> </div><div class='line' id='LC205'><span class="k">\label</span><span class="nb">{</span>tipossensores<span class="nb">}</span> </div><div class='line' id='LC206'><span class="k">\end</span><span class="nb">{</span>figure<span class="nb">}</span></div><div class='line' id='LC207'><br/></div><div class='line' id='LC208'>Current technologies most frequently used in traffic monitoring include pneumatic tubes, loop detectors, floating vehicles, and automatic recognition systems, among others.</div><div class='line' id='LC209'><br/></div><div class='line' id='LC210'><br/></div><div class='line' id='LC211'><span class="k">\emph</span><span class="nb">{</span>Manual counts<span class="nb">}</span> is the most traditional method. In this case trained observers gather traffic data that cannot be efficiently obtained through automated counts; e.g., vehicle occupancy rate, pedestrians and vehicle classifications. The most common equipments used are tally sheet, mechanical count boards, and electronic count board systems.</div><div class='line' id='LC212'><br/></div><div class='line' id='LC213'><span class="k">\emph</span><span class="nb">{</span>Passive and active infra-red<span class="nb">}</span> sensors are based on detecting the presence, speed, and type of vehicles using the infrared energy radiating from the detection area. The main drawbacks are the performance during bad weather, as well as limited lane coverage.</div><div class='line' id='LC214'><br/></div><div class='line' id='LC215'><span class="k">\emph</span><span class="nb">{</span>Microwave radar<span class="nb">}</span> can detect moving vehicles and their speed (Doppler radar). It records count data, speed, and simple vehicle classification. Its behaviour is not affected by weather conditions.</div><div class='line' id='LC216'><br/></div><div class='line' id='LC217'><span class="k">\emph</span><span class="nb">{</span>Ultrasonic sensors<span class="nb">}</span> devices emit sound waves to detect vehicles by measuring the time for the signal to return to the device. They can be affected by temperature or bad weather. </div><div class='line' id='LC218'><br/></div><div class='line' id='LC219'><span class="k">\emph</span><span class="nb">{</span>Pneumatic road tubes<span class="nb">}</span> are placed across the road lanes to detect the pressure changes produced when a vehicle tyre passes over them. The pulse of air that is created is recorded and processed by a counter located on the side of the road. The main drawback of this technology is that it has limited lane coverage and its efficiency is subject to weather, temperature, and traffic conditions. This system may also not be efficient in measuring low speed flows.</div><div class='line' id='LC220'><span class="k">\emph</span><span class="nb">{</span>Piezoelectric sensors<span class="nb">}</span> are very similar to pneumatic road tubes, although the principle is to convert mechanical energy into electrical energy. Indeed, mechanical deformation of the piezoelectric material modifies its surface charge density leading to a potential difference that appears between the electrodes. The amplitude and frequency of the signal is directly proportional to the degree of deformation. This system can be used to measure weight and speed.</div><div class='line' id='LC221'><br/></div><div class='line' id='LC222'><span class="k">\emph</span><span class="nb">{</span>Magnetic loops<span class="nb">}</span> (inductive, magnetic, or video processing based) may be used temporarily or permanently, the latter being the more usual. </div><div class='line' id='LC223'>It is the most conventional technology used to collect traffic data. The loops are embedded in roadways in a square formation that generates a magnetic field. The information is then transmitted to a counting device placed on the side of the road. This has a generally short life expectancy because it can be damaged by heavy vehicles, but is not affected by bad weather conditions. This technology has been widely deployed over the last decades. However, the implementation and maintenance costs can be expensive.</div><div class='line' id='LC224'><br/></div><div class='line' id='LC225'>The use of so-called <span class="k">\emph</span><span class="nb">{</span>floating vehicles<span class="nb">}</span> consists on a vehicle provided with sensors to collect information while driving on a predefined route.</div><div class='line' id='LC226'>This active device data collection is one of the most popular among operators of roads, used especially for the collection of travel time and for loop detector calibration.</div><div class='line' id='LC227'>Depending on the level of automation in the data collection, the cost can vary.</div><div class='line' id='LC228'><br/></div><div class='line' id='LC229'>In some areas, such as electronic toll or transit systems, <span class="k">\emph</span><span class="nb">{</span>automatic vehicle identification systems<span class="nb">}</span> (AVI) are also widely used. </div><div class='line' id='LC230'>These sensors are non-exhaustive data sources to identify tags located in vehicles, such as in payment-systems without stopping. </div><div class='line' id='LC231'>The system detects the pass, and the data is sent to the server where it is processed in order to carry out an event (pay toll, opening in the fence, etc).</div><div class='line' id='LC232'><br/></div><div class='line' id='LC233'>The main disadvantage of these systems is that they are unable to identify vehicles detected, in order to obtain origin/destination matrixes.</div><div class='line' id='LC234'>Just the number of vehicles and their type can be calculated, but does not allow to obtain moves flow, nor to determine whether a certain vehicle passes repeatedly.</div><div class='line' id='LC235'>In addition, its high cost makes it unprofitable covering secondary roads with them, so they are often located on major roads.</div><div class='line' id='LC236'><br/></div><div class='line' id='LC237'>Finally, the <span class="k">\emph</span><span class="nb">{</span>automatic recognition<span class="nb">}</span> technology has experienced an increase in recent years due to its ability to detect individual vehicles without relying on in-vehicle systems. </div><div class='line' id='LC238'>Video image detection is a good example: video cameras record vehicle numbers, type and speed by means of different video techniques, e.g. trip line and tracking. </div><div class='line' id='LC239'>Furthermore, they are used for automatic detection of incidents on the road. </div><div class='line' id='LC240'>That is the main advantage over previous information systems.</div><div class='line' id='LC241'>However, system reliability might not be the best, as the system can be sensitive to meteorological conditions.</div><div class='line' id='LC242'>Moreover, these systems are very costly compared to the previous ones.</div><div class='line' id='LC243'>Finally, from a privacy point of view, the Spanish Data Protection Agency<span class="k">\footnote</span><span class="nb">{</span>Agencia de Protección de Datos<span class="nb">}</span> considers the car license plate as a personal data, so that it would require the user consent.</div><div class='line' id='LC244'><br/></div><div class='line' id='LC245'><br/></div><div class='line' id='LC246'><br/></div><div class='line' id='LC247'><br/></div><div class='line' id='LC248'><span class="k">\subsubsection</span><span class="nb">{</span>Commercial products<span class="nb">}</span></div><div class='line' id='LC249'>There are different companies working in the traffic information area using approaches similar to the presented in this work.</div><div class='line' id='LC250'><br/></div><div class='line' id='LC251'><span class="k">\begin</span><span class="nb">{</span>itemize<span class="nb">}</span></div><div class='line' id='LC252'><br/></div><div class='line' id='LC253'><span class="k">\item</span> <span class="k">\textbf</span><span class="nb">{</span>Bit Carrier<span class="nb">}</span> <span class="k">\cite</span><span class="nb">{</span>patenteBC<span class="nb">}</span> <span class="k">\cite</span><span class="nb">{</span>BitCarrier<span class="nb">}</span>: It offers a traffic management system based in BT to count people and commercial routes (pathsolver). Its technology was implanted in highways managed by Abertis for traffic control and monitoring. Actually it has a 150 devices network in Catalonia, so it allows count the traffic times of 200.000 persons each day.</div><div class='line' id='LC254'>&nbsp;</div><div class='line' id='LC255'><span class="k">\item</span> <span class="k">\textbf</span><span class="nb">{</span>Trafficnow<span class="nb">}</span> <span class="k">\cite</span><span class="nb">{</span>Trafficnow<span class="nb">}</span>: Another BT system product. A pilot experience has been implanted in Vigo.</div><div class='line' id='LC256'><br/></div><div class='line' id='LC257'>&nbsp;<span class="k">\item</span> <span class="k">\textbf</span><span class="nb">{</span>Traffax Inc<span class="nb">}</span> <span class="k">\cite</span><span class="nb">{</span>TraffaxInc<span class="nb">}</span>: It is a company that also has used BT for calculating origin-destination and transport time matrixes.</div><div class='line' id='LC258'><br/></div><div class='line' id='LC259'>&nbsp;<span class="k">\item</span> <span class="k">\textbf</span><span class="nb">{</span>Savari Networks<span class="nb">}</span> <span class="k">\cite</span><span class="nb">{</span>SavariNetworks<span class="nb">}</span>: It offers the commercial product StreetWAVE for traffic monitoring to know in real time the traffic status.</div><div class='line' id='LC260'><br/></div><div class='line' id='LC261'><br/></div><div class='line' id='LC262'>&nbsp;<span class="k">\item</span> <span class="k">\textbf</span><span class="nb">{</span>TrafficCast<span class="nb">}</span> <span class="k">\cite</span><span class="nb">{</span>TrafficCast<span class="nb">}</span>: They have developed prediction models in different cities based on different technologies, such as cameras, BT and RFID included in the vehicles.</div><div class='line' id='LC263'><br/></div><div class='line' id='LC264'><span class="k">\end</span><span class="nb">{</span>itemize<span class="nb">}</span></div><div class='line' id='LC265'><br/></div><div class='line' id='LC266'>The proposal presented in this work have some common features with the previous approaches, offering similar functionalities but with reduced cost.</div><div class='line' id='LC267'><br/></div><div class='line' id='LC268'><span class="k">\subsection</span><span class="nb">{</span>Time series forecasting<span class="nb">}</span></div><div class='line' id='LC269'><span class="k">\label</span><span class="nb">{</span>subsec:time-series<span class="nb">}</span></div><div class='line' id='LC270'>Briefly described, a time series is a set of chronologically collected data. Time series forecasting is the task of predicting values of a given series using its own past and present values; the values of any related exogeneous variable can be also used, when available.</div><div class='line' id='LC271'>Time series forecasting is a major field of research, mainly in the areas of statistics <span class="k">\cite</span><span class="nb">{</span>Gooijer25years<span class="nb">}</span> and operational research <span class="k">\cite</span><span class="nb">{</span>Fildes2008<span class="nb">}</span>, as time series can be found in fields like Engineering, Biology, Economy, or Social Sciences, among many others.</div><div class='line' id='LC272'><br/></div><div class='line' id='LC273'>Time series forecasting is usually tackled trying to find out an underlying model that describes the series behaviour. For this reason, there exist a wide variety of methods to perform forecast using both linear and nonlinear models. The group of linear methods comprises the exponential smoothing methods <span class="k">\cite</span><span class="nb">{</span>Brown1959,Winters1960<span class="nb">}</span>, simple exponential smoothing (SES), or State space models <span class="k">\cite</span><span class="nb">{</span>Snyder1985<span class="nb">}</span>, among many others. Nevertheless, the most well-known, widely used linear methods are the ARIMA ones <span class="k">\cite</span><span class="nb">{</span>BoxJenk<span class="nb">}</span>. ARIMA methods can be summarized as iterative cycles in which: a) the time series is classified as belonging to a pre-established class; b) according to that class, a set of parameters is estimated; and c) the obtained model is verified in other to accept it, or search for another one returning to first step. The models provide by ARIMA integrate in their equations autoregressive (AR) and moving average (MA) components to have into account pass values as well as previous forecasting errors.</div><div class='line' id='LC274'><br/></div><div class='line' id='LC275'>Despite their ease of use, linear models have shown no to be accurate for many real applications, being this the main reason why new (but also more complex and difficult to be used) nonlinear methods have been developed. Nonlinear models include regime-switching models, which comprise the wide variety of existing threshold autoregressive (TAR) models <span class="k">\cite</span><span class="nb">{</span>Tong1978<span class="nb">}</span>. Clements <span class="k">\cite</span><span class="nb">{</span>Clements2004<span class="nb">}</span> exposed main drawbacks of nonlinear methods, mainly the excessively complex models they provide, the lack of robust performance, and, worst of all, the difficulty to use. De Gooijer <span class="k">\cite</span><span class="nb">{</span>Gooijer25years<span class="nb">}</span> also concludes that future research on nonlinear models should include, among others, the search for easy to use software.</div><div class='line' id='LC276'><br/></div><div class='line' id='LC277'><span class="c">%\subsection{Soft computing methods for time series forecasting}</span></div><div class='line' id='LC278'><br/></div><div class='line' id='LC279'>Forecasting values for time series has been also faced with soft computing approaches; for instance, the ones by Samanta <span class="k">\cite</span><span class="nb">{</span>Samanta2011<span class="nb">}</span> and Zhu <span class="k">\cite</span><span class="nb">{</span>Zhu2011<span class="nb">}</span>, who developed methods based on cooperative particle swarm optimization. In <span class="k">\cite</span><span class="nb">{</span>Qiu2011<span class="nb">}</span> and <span class="k">\cite</span><span class="nb">{</span>Wang2011<span class="nb">}</span> fuzzy time series models are proposed in order to predict new values.Similarly, Yu and Huarng <span class="k">\cite</span><span class="nb">{</span>Yu2010<span class="nb">}</span> applied ANNs for training and forecasting in their fuzzy time series model. Models such as support vector regression <span class="k">\cite</span><span class="nb">{</span>Kavaklioglu2011<span class="nb">}</span> and fuzzy expert system <span class="k">\cite</span><span class="nb">{</span>Dash1995<span class="nb">}</span> were proposed for the electricity demand forecasting, among others.</div><div class='line' id='LC280'><br/></div><div class='line' id='LC281'>As can be seen, ANNs have been applied to time series and they are currently recognized as an important tool for forecasting. </div><div class='line' id='LC282'><br/></div><div class='line' id='LC283'>Tang <span class="k">\cite</span><span class="nb">{</span>Tang1991<span class="nb">}</span> concluded that ANNs could provide better long-term forecasting; moreover, ANNs can perform better than ARIMA models with short series of input data. Furthermore, contrary to the traditional linear and nonlinear time series models, ANNs are nonlinear data-driven approaches with more flexibility and effectiveness in modeling for forecasting <span class="k">\cite</span><span class="nb">{</span>Zhang1998b<span class="nb">}</span>. Jain and Kumar determined in their work <span class="k">\cite</span><span class="nb">{</span>Jain2007<span class="nb">}</span> that the ANN models were able to produce more accurate forecasts than traditional models because they do not presuppose any functional form of the model to be developed and they do not depend on the assumptions of linearity.</div><div class='line' id='LC284'><br/></div><div class='line' id='LC285'>There exist numerous works of different application areas where ANNs are used to forecast time series. The work by Arizmendi <span class="k">\cite</span><span class="nb">{</span>Arizmendi1993<span class="nb">}</span> obtained accurate predictions of the airborne pollen concentrations using ANNs. Zhang and Hu <span class="k">\cite</span><span class="nb">{</span>Zhang1998b<span class="nb">}</span> employed ANNs, and Rivas et al. <span class="k">\cite</span><span class="nb">{</span>Rivas04<span class="nb">}</span> RBFNs, for forecasting British pound and US dollar exchange rates. Bezerianos et al. <span class="k">\cite</span><span class="nb">{</span>Bezerianos1999<span class="nb">}</span> employed RBFNs for the assessment and prediction of the heart rate variability. </div><div class='line' id='LC286'><br/></div><div class='line' id='LC287'><span class="c">%\subsection{Radial Basic Function Networks}</span></div><div class='line' id='LC288'><br/></div><div class='line' id='LC289'>Specifically inside the ANNs, the use of RBFs as activation functions for them and its application to time series forecasting were firstly considered by Broomhead and Lowe in 1988 <span class="k">\cite</span><span class="nb">{</span>Broomhead88<span class="nb">}</span>. Afterwards, new works by Carse and Fogarty <span class="k">\cite</span><span class="nb">{</span>Carse1996<span class="nb">}</span>, and Whitehead and Choate <span class="k">\cite</span><span class="nb">{</span>Whitehead96<span class="nb">}</span> focused on the prediction of time series.</div><div class='line' id='LC290'><br/></div><div class='line' id='LC291'>In later works, Harpham and Dawson <span class="k">\cite</span><span class="nb">{</span>Harpham06<span class="nb">}</span> studied the effect of different basis functions on an RBFN for time series prediction. Moreover, Du <span class="k">\cite</span><span class="nb">{</span>Du2008<span class="nb">}</span> used time series with an encoding scheme for training RBFNs by GAs. Both the architecture (numbers and selections of nodes and inputs) and the parameters (centers and widths) of the RBFNs were represented in one chromosome and evolved simultaneously by GAs so that the selection of nodes and inputs could be automatically achieved.</div><div class='line' id='LC292'><br/></div><div class='line' id='LC293'>Previous works found in literature can also be classified according to the prediction horizon. Thus, forecasting can be divided into short-term, medium-term, and long-term. Generally, forecasting is trended to short-term prediction such as one-step ahead prediction, since longer period prediction (medium-term or long-term) is more difficult, and sometimes may not be reliant because of the error propagation <span class="k">\cite</span><span class="nb">{</span>Chatterjee06<span class="nb">}</span>. Thus, neural network models have been traditionally applied in short-term forecasting <span class="k">\cite</span><span class="nb">{</span>Hippert10,Lee09<span class="nb">}</span>. For instance, the work by Perez-Godoy <span class="k">\cite</span><span class="nb">{</span>PerezGodoy2010<span class="nb">}</span> applied a hybrid evolutionary cooperative-competitive algorithm for the design of RBFNs to the short-term and even medium-term forecasting of the extra-virgin olive oil price.</div><div class='line' id='LC294'><br/></div><div class='line' id='LC295'><span class="c">%\subsubsection{Lags selection}</span></div><div class='line' id='LC296'><br/></div><div class='line' id='LC297'>As mentioned, another problem that emerges working with time series is the correct choice of the lags considered for representing the series. The relationship involving time series historical data defines a <span class="s">$</span><span class="nb">d</span><span class="s">$</span>-dimensional space where <span class="s">$</span><span class="nb">d</span><span class="s">$</span> is the minimum dimension capable of representing such a relationship. Takens&#39; theorem <span class="k">\cite</span><span class="nb">{</span>Takens1980<span class="nb">}</span> establishes that if <span class="s">$</span><span class="nb">d</span><span class="s">$</span> is sufficiently large is possible to build a state space using the correct time lags and if this space is correctly rebuilt also guarantees that the dynamics of this space is topologically identical to the dynamics of the real systems state space. <span class="c">%Many methods can be found in the literature for the correct definition of the variable $d$, that is, the correct choice of the important time lags of the system dynamics, sometimes called as active dimension of the dynamics generating a time series from the observed series \cite{Tanaka2001}.</span></div><div class='line' id='LC298'><br/></div><div class='line' id='LC299'>In order to tackle the lags selection problem, an evolutionary method that performs a search for the minimum number of dimensions, Time-delay Added Evolutionary Forecasting (TAEF), is presented in <span class="k">\cite</span><span class="nb">{</span>Ferreira2008<span class="nb">}</span>. The methodology is inspired in Takens&#39; theorem [***CITA***] and consists of an iterative hybrid model composed of an ANN combined with a genetic algorithm (GA). In <span class="k">\cite</span><span class="nb">{</span>Luko2010<span class="nb">}</span> the evolutionary selection of lags is divided into two stages: first, the optimal dimension of the reconstructed phase space is determined by the false nearest neighbors algorithm and then a near-optimal set of time lags is found with a genetic algorithm for a fuzzy inference system.</div><div class='line' id='LC300'><br/></div><div class='line' id='LC301'><span class="c">%In general, these proposals are based on the primary dependences among the variables, do not consider any possible induced dependences, and discard any possible correlation that can exist among the time series parameters, even higher order correlations.</span></div><div class='line' id='LC302'>There are some methods that carry out an automatic search of the relevant lags. QIEHI algorithm <span class="k">\cite</span><span class="nb">{</span>Araujo2010a<span class="nb">}</span>, for instance, is an evolutionary hybrid intelligent method which is composed of an ANN and a modified evolutionary algorithm to search the minimum dimension to determine the characteristic phase for time series. <span class="c">%The model is built in two stages as in \cite{Ferreira2008}.</span></div><div class='line' id='LC303'>Another hybrid methodology composed of a modular morphological ANN with an evolutionary algorithm that searches for the best time lags is described in <span class="k">\cite</span><span class="nb">{</span>Araujo2010b<span class="nb">}</span>. <span class="c">%With the same modular morphological neural network, the time lags are obtained by means of a particle swarm optimizer in \cite{Araujo2010c} and by means of a modified GA in \cite{Araujo2011}.</span></div><div class='line' id='LC304'>In <span class="k">\cite</span><span class="nb">{</span>garcia2008<span class="nb">}</span> a study on the selection not only of the lags but also of the exogenous features with classical feature selection algorithms as pre-processing stage is performed. <span class="c">%The authors show the utility of a feature selection pre-processing stage for time series forecasting with different models.</span></div><div class='line' id='LC305'>The lag selection is performed as a post-processing stage in <span class="k">\cite</span><span class="nb">{</span>Maus2011<span class="nb">}</span> with a sensitivity computation of the output to each time lag. <span class="c">%The initial stage trains a single-layer, feed-forward ANN based on $d$ time lags, with $d$ chosen large enough to capture the relevant dynamics of the time series. %Another methodology which considers the lag selection as a post-processing stage is</span></div><div class='line' id='LC306'><span class="c">%TDSEP \cite{Sun2006} uses a GA for the optimal selection of time lags for a previously obtained and diagonalized second-order correlation matrices.</span></div><div class='line' id='LC307'><br/></div><div class='line' id='LC308'>As can be observed, the approaches in the literature consider the lags selection as a pre- or post-processing or as a part of the learning process but, instead of together, in hybrid processes with two or three stages. On the contrary, our goal is to address the selection of the lags which represent the series (with any type of correlation) jointly with the design process.</div><div class='line' id='LC309'><br/></div><div class='line' id='LC310'><span class="c">%\subsection{Radial Basic Function Networks}</span></div><div class='line' id='LC311'><br/></div><div class='line' id='LC312'><br/></div><div class='line' id='LC313'><br/></div><div class='line' id='LC314'><span class="k">\section</span><span class="nb">{</span>Data-collecting<span class="nb">}</span></div><div class='line' id='LC315'><span class="k">\label</span><span class="nb">{</span>sec:data<span class="nb">}</span></div><div class='line' id='LC316'><br/></div><div class='line' id='LC317'>Identifying the MAC address of BT devices passing by the roads has being achieved using Intelify (see Figure <span class="k">\ref</span><span class="nb">{</span>intelify<span class="nb">}</span>), a hardware solution with a low power consumption and a high detection range.</div><div class='line' id='LC318'><br/></div><div class='line' id='LC319'><span class="k">\begin</span><span class="nb">{</span>figure<span class="nb">}</span>[htpb] </div><div class='line' id='LC320'><span class="k">\begin</span><span class="nb">{</span>center<span class="nb">}</span> </div><div class='line' id='LC321'><span class="k">\includegraphics</span><span class="na">[scale=0.5]</span><span class="nb">{</span>intelifychisme1.png<span class="nb">}</span></div><div class='line' id='LC322'><span class="k">\end</span><span class="nb">{</span>center<span class="nb">}</span> </div><div class='line' id='LC323'><span class="k">\caption</span><span class="nb">{</span>Intelify device with a connected USB 3G dongle.<span class="nb">}</span> </div><div class='line' id='LC324'><span class="k">\label</span><span class="nb">{</span>intelify<span class="nb">}</span> </div><div class='line' id='LC325'><span class="k">\end</span><span class="nb">{</span>figure<span class="nb">}</span></div><div class='line' id='LC326'><br/></div><div class='line' id='LC327'>Intelify is a small autonomous computer that can be installed in any area to be monitored. It is an autonomous unit, provided by several sensors so that it can discover what is happening in its surroundings (like the flow of people and vehicles). At the same time the environment is been scanned, Intelify is able to send the information to a central server for further processing and interpretation. Table <span class="k">\ref</span><span class="nb">{</span>caracteristicas<span class="nb">}</span> shows main features of the device.</div><div class='line' id='LC328'><br/></div><div class='line' id='LC329'><span class="k">\begin</span><span class="nb">{</span>table<span class="nb">}</span>[htpb]</div><div class='line' id='LC330'><span class="k">\begin</span><span class="nb">{</span>center<span class="nb">}</span></div><div class='line' id='LC331'><span class="k">\begin</span><span class="nb">{</span>tabular<span class="nb">}{</span>|c|c|<span class="nb">}</span></div><div class='line' id='LC332'><span class="k">\hline</span></div><div class='line' id='LC333'>Dimensions <span class="nb">&amp;</span> 113x163x30mm <span class="k">\\</span></div><div class='line' id='LC334'><span class="k">\hline</span></div><div class='line' id='LC335'>&nbsp;&nbsp;&nbsp;&nbsp;&nbsp;<span class="nb">&amp;</span> Power             <span class="k">\\</span></div><div class='line' id='LC336'>LEDs <span class="nb">&amp;</span> 3G activity       <span class="k">\\</span></div><div class='line' id='LC337'>&nbsp;&nbsp;&nbsp;&nbsp;&nbsp;<span class="nb">&amp;</span> Ethernet activity <span class="k">\\</span></div><div class='line' id='LC338'><span class="k">\hline</span></div><div class='line' id='LC339'>&nbsp;&nbsp;&nbsp;&nbsp;&nbsp;&nbsp;&nbsp;&nbsp;&nbsp;&nbsp;&nbsp;<span class="nb">&amp;</span> Ethernet  <span class="k">\\</span></div><div class='line' id='LC340'>Networking <span class="nb">&amp;</span> Wireless  <span class="k">\\</span></div><div class='line' id='LC341'>&nbsp;&nbsp;&nbsp;&nbsp;&nbsp;&nbsp;&nbsp;&nbsp;&nbsp;&nbsp;&nbsp;<span class="nb">&amp;</span> Bluetooth <span class="k">\\</span>  </div><div class='line' id='LC342'>&nbsp;&nbsp;&nbsp;&nbsp;&nbsp;&nbsp;&nbsp;&nbsp;&nbsp;&nbsp;&nbsp;<span class="nb">&amp;</span> 3G        <span class="k">\\</span></div><div class='line' id='LC343'><span class="k">\hline</span></div><div class='line' id='LC344'>USB ports <span class="nb">&amp;</span> City Analytics Antenna <span class="k">\\</span></div><div class='line' id='LC345'>		  <span class="nb">&amp;</span> 3G dongle              <span class="k">\\</span></div><div class='line' id='LC346'><span class="k">\hline</span></div><div class='line' id='LC347'>Other ports <span class="nb">&amp;</span> RS-232 <span class="k">\\</span></div><div class='line' id='LC348'>			<span class="nb">&amp;</span> VGA    <span class="k">\\</span></div><div class='line' id='LC349'><span class="k">\hline</span></div><div class='line' id='LC350'>&nbsp;	  <span class="nb">&amp;</span> 18v - 1200mA          <span class="k">\\</span></div><div class='line' id='LC351'>Power <span class="nb">&amp;</span> external jack - 5.5mm <span class="k">\\</span></div><div class='line' id='LC352'>&nbsp;&nbsp;&nbsp;&nbsp;&nbsp;&nbsp;<span class="nb">&amp;</span> internal jack - 2.1mm <span class="k">\\</span></div><div class='line' id='LC353'><span class="k">\hline</span></div><div class='line' id='LC354'>Network     <span class="nb">&amp;</span> Ethernet RJ45 <span class="k">\\</span></div><div class='line' id='LC355'>connections <span class="nb">&amp;</span> 3G USB Modem  <span class="k">\\</span></div><div class='line' id='LC356'><span class="k">\hline</span></div><div class='line' id='LC357'>Antennas <span class="nb">&amp;</span> City Analytics USB antenna <span class="k">\\</span></div><div class='line' id='LC358'>		 <span class="nb">&amp;</span> wireless antenna <span class="k">\\</span></div><div class='line' id='LC359'><span class="k">\hline</span></div><div class='line' id='LC360'>Microphone <span class="nb">&amp;</span> noise sensor <span class="k">\\</span></div><div class='line' id='LC361'><span class="k">\hline</span></div><div class='line' id='LC362'>Temperature <span class="nb">&amp;</span> main board temperature sensor with extrapolation <span class="k">\\</span></div><div class='line' id='LC363'><span class="k">\hline</span></div><div class='line' id='LC364'>Box <span class="nb">&amp;</span> 1.5mm aluminum box <span class="k">\\</span></div><div class='line' id='LC365'>&nbsp;&nbsp;&nbsp;&nbsp;<span class="nb">&amp;</span> external use possible  <span class="k">\\</span></div><div class='line' id='LC366'><span class="k">\hline</span></div><div class='line' id='LC367'>Operating System <span class="nb">&amp;</span> Debian 6.0 Squeeze  <span class="k">\\</span></div><div class='line' id='LC368'><span class="k">\hline</span></div><div class='line' id='LC369'><span class="k">\end</span><span class="nb">{</span>tabular<span class="nb">}</span></div><div class='line' id='LC370'><span class="k">\end</span><span class="nb">{</span>center<span class="nb">}</span></div><div class='line' id='LC371'><span class="k">\caption</span><span class="nb">{</span>Main features of the Intelify device.<span class="nb">}</span></div><div class='line' id='LC372'><span class="k">\label</span><span class="nb">{</span>caracteristicas<span class="nb">}</span></div><div class='line' id='LC373'><span class="k">\end</span><span class="nb">{</span>table<span class="nb">}</span></div><div class='line' id='LC374'><br/></div><div class='line' id='LC375'>This hardware device is based on technology developed by Ciudad 2020 <span class="k">\cite</span><span class="nb">{</span>cityanalytics,Blobject<span class="nb">}</span>, and the services it offers are based on a net of Intelify devices. Using such an structure, it has the capacity to discover information about the physical environment and help with decision making to any kind of organization based on people flow and behavior.</div><div class='line' id='LC376'><br/></div><div class='line' id='LC377'>Valuable information about tourism, trade and mobility can be gathered through the deployment of autonomous devices around a city. A specific example is the service offered in <span class="k">\cite</span><span class="nb">{</span>numerodepersonas<span class="nb">}</span>. It offers information about foot traffic through Cordoba city center.</div><div class='line' id='LC378'><br/></div><div class='line' id='LC379'>The cost of this solution is 1000 euros per device, including maintenance of remote computer, communications using a 3G telephony service and storage and data management.</div><div class='line' id='LC380'><br/></div><div class='line' id='LC381'>Data accuracy is very representative, compared against other technologies.</div><div class='line' id='LC382'>In <span class="k">\cite</span><span class="nb">{</span>estudioprecision<span class="nb">}</span> it was obtained an a priori error estimation of <span class="s">$</span><span class="m">8</span><span class="nb">.</span><span class="m">5</span><span class="nv">\%</span><span class="s">$</span> of detections.</div><div class='line' id='LC383'><span class="c">%This study \cite{estudioprecision} shows that the average error found in the detection of people is 8.5\%.</span></div><div class='line' id='LC384'><br/></div><div class='line' id='LC385'><br/></div><div class='line' id='LC386'>??????Descripción de los datos recopilados</div><div class='line' id='LC387'><br/></div><div class='line' id='LC388'><span class="k">\section</span><span class="nb">{</span>Experiments and results<span class="nb">}</span></div><div class='line' id='LC389'><span class="k">\label</span><span class="nb">{</span>sec:experiments<span class="nb">}</span></div><div class='line' id='LC390'><br/></div><div class='line' id='LC391'>????? TODO ESTO HAY QUE CAMBIARLO PARA LOS DATOS REALES CON LSO QUE AL FINAL HEMOS TRABAJADO</div><div class='line' id='LC392'><br/></div><div class='line' id='LC393'>In this section the analysis of collected data during the monitoring period (November 8 to December 9) to obtain statistics and so study the use of vehicles is carried out.</div><div class='line' id='LC394'><br/></div><div class='line' id='LC395'>&nbsp;<span class="c">% Concretamente, en las siguientes subsecciones, reportaremos información acerca del número total de vehículos detectados por cada nodo, en días laborables o festivos, sobre la densidad del tráfico por rango horario, acerca de los desplazamientos individuales, y de la velocidad media en un tramo delimitado por dos nodos consecutivos.</span></div><div class='line' id='LC396'>Specifically, the following subsections report information about the total number of vehicles detected by each node, on weekdays or holidays, information on traffic density by time range on individual movements, and the average speed on a section delimited by two consecutive nodes.</div><div class='line' id='LC397'><br/></div><div class='line' id='LC398'>&nbsp;<span class="c">% Por último, puesto que el día 14 de noviembre de 2012 se celebró la huelga general, estudiaremos cómo afectó la huelga al tráfico en el área metropolitana de Granada, comparando los totales de detección de dispositivos entre el día de la huelga (14 de noviembre), y el miércoles siguiente (21 de noviembre).</span></div><div class='line' id='LC399'>Finally, since November 14 2012 a general strike was held, we will study how the strike affected traffic in the metropolitan area of Granada (Spain), by comparing total number of devices detected that day (November 14), and the following day (November 15).</div><div class='line' id='LC400'><br/></div><div class='line' id='LC401'><br/></div><div class='line' id='LC402'><span class="k">\subsection</span><span class="nb">{</span>Total number of vehicles detected (weekdays and holidays)<span class="nb">}</span></div><div class='line' id='LC403'>&nbsp;<span class="c">% VehiculosTotales.txt tiene los vehículos totales detectados en cada nodo.</span></div><div class='line' id='LC404'><br/></div><div class='line' id='LC405'><span class="c">% La primera parte del análisis ha consistido en calcular el número de dispositivos detectados por cada uno de los nodos instalados.</span></div><div class='line' id='LC406'>The first analysis consisted in calculating the number of devices detected by each node.</div><div class='line' id='LC407'><br/></div><div class='line' id='LC408'>&nbsp;<span class="k">\begin</span><span class="nb">{</span>table<span class="nb">}</span></div><div class='line' id='LC409'>&nbsp;<span class="k">\begin</span><span class="nb">{</span>center<span class="nb">}</span></div><div class='line' id='LC410'>&nbsp;<span class="k">\begin</span><span class="nb">{</span>tabular<span class="nb">}{</span>|c|c|<span class="nb">}</span></div><div class='line' id='LC411'>&nbsp;<span class="k">\hline</span></div><div class='line' id='LC412'>Node Id.  <span class="nb">&amp;</span>  N. of devices detected  <span class="k">\\</span></div><div class='line' id='LC413'>&nbsp;<span class="k">\hline</span></div><div class='line' id='LC414'>&nbsp;&nbsp;&nbsp;&nbsp;1     <span class="nb">&amp;</span>    31408  <span class="k">\\</span></div><div class='line' id='LC415'>&nbsp;<span class="k">\hline</span></div><div class='line' id='LC416'>&nbsp;&nbsp;&nbsp;&nbsp;2     <span class="nb">&amp;</span>    45032  <span class="k">\\</span></div><div class='line' id='LC417'>&nbsp;<span class="k">\hline</span></div><div class='line' id='LC418'>&nbsp;&nbsp;&nbsp;&nbsp;3     <span class="nb">&amp;</span>    33165  <span class="k">\\</span></div><div class='line' id='LC419'>&nbsp;<span class="k">\hline</span></div><div class='line' id='LC420'>&nbsp;&nbsp;&nbsp;&nbsp;4     <span class="nb">&amp;</span>    358494  <span class="k">\\</span></div><div class='line' id='LC421'>&nbsp;<span class="k">\hline</span></div><div class='line' id='LC422'>&nbsp;&nbsp;&nbsp;&nbsp;5     <span class="nb">&amp;</span>    297874  <span class="k">\\</span></div><div class='line' id='LC423'>&nbsp;<span class="k">\hline</span></div><div class='line' id='LC424'>&nbsp;&nbsp;&nbsp;&nbsp;6     <span class="nb">&amp;</span>    7872  <span class="k">\\</span></div><div class='line' id='LC425'>&nbsp;<span class="k">\hline</span></div><div class='line' id='LC426'>&nbsp;<span class="k">\end</span><span class="nb">{</span>tabular<span class="nb">}</span></div><div class='line' id='LC427'>&nbsp;<span class="k">\end</span><span class="nb">{</span>center<span class="nb">}</span></div><div class='line' id='LC428'>&nbsp;<span class="k">\caption</span><span class="nb">{</span>Number of BT devices detected by each node.</div><div class='line' id='LC429'>&nbsp;<span class="k">\label</span><span class="nb">{</span>VehiculosTotales<span class="nb">}}</span></div><div class='line' id='LC430'>&nbsp;<span class="k">\end</span><span class="nb">{</span>table<span class="nb">}</span></div><div class='line' id='LC431'><br/></div><div class='line' id='LC432'>&nbsp;<span class="c">% En total se han detectado 773845 dispositivos BT al paso por los seis nodos. </span></div><div class='line' id='LC433'>&nbsp;<span class="c">% Como se observa en la Tabla \ref{VehiculosTotales}, los dos nodos situados en la autovía de Sierra Nevada (A44, Circunvalación de Granada, nodos 4 y 5) son los que más datos han recogido, mientras que el situado en una calle secundaria (nodo 6) ha sido el que menos dispositivos ha recogido.</span></div><div class='line' id='LC434'><br/></div><div class='line' id='LC435'>In total, 773,845 BT devices have been detected by the six nodes.</div><div class='line' id='LC436'>As shown in Table <span class="k">\ref</span><span class="nb">{</span>VehiculosTotales<span class="nb">}</span>, nodes located in the Sierra Nevada Highway (A44, nodes 4 and 5) have collected a higher number of data, while the node located in a side street (node 6) has detected the smallest number of devices.</div><div class='line' id='LC437'><br/></div><div class='line' id='LC438'><br/></div><div class='line' id='LC439'><span class="k">\subsection</span><span class="nb">{</span>Total vehicles detected on non-working days<span class="nb">}</span></div><div class='line' id='LC440'>&nbsp;<span class="c">% VehiculosFestivos.txt tiene los vehículos detectados en día festivo en cada nodo.</span></div><div class='line' id='LC441'><br/></div><div class='line' id='LC442'>&nbsp;<span class="c">% A continuación, y para comparar la intensidad circulatoria entre días laborables y no laborables, se ha calculado el número de pasos en días festivos y no laborables.</span></div><div class='line' id='LC443'>To compare the traffic intensity between working and non-working days, the number of pass on holidays and non-working days have been obtained.</div><div class='line' id='LC444'><br/></div><div class='line' id='LC445'>&nbsp;<span class="k">\begin</span><span class="nb">{</span>table<span class="nb">}</span></div><div class='line' id='LC446'>&nbsp;<span class="k">\begin</span><span class="nb">{</span>center<span class="nb">}</span></div><div class='line' id='LC447'>&nbsp;<span class="k">\begin</span><span class="nb">{</span>tabular<span class="nb">}{</span>|c|c|<span class="nb">}</span></div><div class='line' id='LC448'>&nbsp;<span class="k">\hline</span></div><div class='line' id='LC449'>Node Id.  <span class="nb">&amp;</span>  N. of devices detected  <span class="k">\\</span></div><div class='line' id='LC450'>&nbsp;<span class="k">\hline</span></div><div class='line' id='LC451'>&nbsp;&nbsp;&nbsp;&nbsp;1     <span class="nb">&amp;</span>    2149  <span class="k">\\</span></div><div class='line' id='LC452'>&nbsp;<span class="k">\hline</span></div><div class='line' id='LC453'>&nbsp;&nbsp;&nbsp;&nbsp;2     <span class="nb">&amp;</span>    2804  <span class="k">\\</span></div><div class='line' id='LC454'>&nbsp;<span class="k">\hline</span></div><div class='line' id='LC455'>&nbsp;&nbsp;&nbsp;&nbsp;3     <span class="nb">&amp;</span>    2832  <span class="k">\\</span></div><div class='line' id='LC456'>&nbsp;<span class="k">\hline</span></div><div class='line' id='LC457'>&nbsp;&nbsp;&nbsp;&nbsp;4     <span class="nb">&amp;</span>    32182  <span class="k">\\</span></div><div class='line' id='LC458'>&nbsp;<span class="k">\hline</span></div><div class='line' id='LC459'>&nbsp;&nbsp;&nbsp;&nbsp;5     <span class="nb">&amp;</span>    24166  <span class="k">\\</span></div><div class='line' id='LC460'>&nbsp;<span class="k">\hline</span></div><div class='line' id='LC461'>&nbsp;&nbsp;&nbsp;&nbsp;6     <span class="nb">&amp;</span>    1269  <span class="k">\\</span></div><div class='line' id='LC462'>&nbsp;<span class="k">\hline</span></div><div class='line' id='LC463'>&nbsp;<span class="k">\end</span><span class="nb">{</span>tabular<span class="nb">}</span></div><div class='line' id='LC464'>&nbsp;<span class="k">\end</span><span class="nb">{</span>center<span class="nb">}</span></div><div class='line' id='LC465'>&nbsp;<span class="k">\caption</span><span class="nb">{</span>Total number of BT devices detected by each node (only on non-working days).</div><div class='line' id='LC466'>&nbsp;<span class="k">\label</span><span class="nb">{</span>VehiculosFestivos<span class="nb">}}</span></div><div class='line' id='LC467'>&nbsp;<span class="k">\end</span><span class="nb">{</span>table<span class="nb">}</span></div><div class='line' id='LC468'><br/></div><div class='line' id='LC469'>&nbsp;<span class="c">% La Tabla \ref{VehiculosFestivos} muestra un descenso en el número de dispositivos detectados por todos los nodos en días no laborables, frente al número de detecciones en días laborables. </span></div><div class='line' id='LC470'>&nbsp;<span class="c">% Aún así, los nodos situados en la autovía de Sierra Nevada siguen recogiendo muchos más datos que el resto, debido al tráfico denso que soporta esta vía en días no laborables.</span></div><div class='line' id='LC471'><br/></div><div class='line' id='LC472'>Table <span class="k">\ref</span><span class="nb">{</span>VehiculosFestivos<span class="nb">}</span> shows how the number of detected devices lowers by all nodes on non-working days, compared to the number of detections on weekdays.</div><div class='line' id='LC473'>Nodes located in the Sierra Nevada Highway still collected much more data than the remainder, due to the traffic this road supports on holidays.</div><div class='line' id='LC474'><br/></div><div class='line' id='LC475'><br/></div><div class='line' id='LC476'><span class="k">\subsection</span><span class="nb">{</span>Traffic density on the road by time range<span class="nb">}</span></div><div class='line' id='LC477'>&nbsp;<span class="c">% VehiculosDiferentesPorHoras.txt contiene para cada uno de los nodos, los dispositivos diferentes detectados en cada rango horario (final de la línea)</span></div><div class='line' id='LC478'><br/></div><div class='line' id='LC479'>&nbsp;<span class="c">% La densidad circulatoria por horas podemos calcularla a partir del total de dispositivos diferentes detectados en cada rango horario.</span></div><div class='line' id='LC480'>Traffic density can be calculated taking into account the total number of detected devices by time range.</div><div class='line' id='LC481'><br/></div><div class='line' id='LC482'>&nbsp;<span class="k">\begin</span><span class="nb">{</span>figure<span class="nb">}</span>[htb]</div><div class='line' id='LC483'>&nbsp;<span class="k">\begin</span><span class="nb">{</span>center<span class="nb">}</span></div><div class='line' id='LC484'>&nbsp;<span class="k">\includegraphics</span><span class="na">[scale=0.4]</span><span class="nb">{</span>VehiculosDiferentesPorHoras.jpg<span class="nb">}</span></div><div class='line' id='LC485'>&nbsp;<span class="k">\caption</span><span class="nb">{</span>For each node, the total number of different detected devices by time range is shown. Figure is shown in logarithmic scale.</div><div class='line' id='LC486'>&nbsp;<span class="c">% Para cada uno de los nodos se muestra el total de dispositivos diferentes detectados en cada rango horario. Para cada rango de horas se muestra el total detectado en cada uno de los seis nodos</span></div><div class='line' id='LC487'>&nbsp;<span class="k">\label</span><span class="nb">{</span>VehiculosDiferentesPorHoras<span class="nb">}}</span></div><div class='line' id='LC488'>&nbsp;<span class="k">\end</span><span class="nb">{</span>center<span class="nb">}</span></div><div class='line' id='LC489'>&nbsp;<span class="k">\end</span><span class="nb">{</span>figure<span class="nb">}</span></div><div class='line' id='LC490'>&nbsp;<span class="c">% Gráfica (histograma) de la densidad de la vía por horas: desde las 00-7h, de 7-10h, de 10-13h, de 13-16h, de 16-20h, y de 20-24h.</span></div><div class='line' id='LC491'>&nbsp;</div><div class='line' id='LC492'>&nbsp;<span class="c">% La Figura \ref{VehiculosDiferentesPorHoras} muestra mayor densidad, en todos los nodos, a las horas punta de entrada o salida del trabajo y colegios.</span></div><div class='line' id='LC493'>Figure <span class="k">\ref</span><span class="nb">{</span>VehiculosDiferentesPorHoras<span class="nb">}</span> shows higher density on all nodes, at peak times or out of work and school.</div><div class='line' id='LC494'><br/></div><div class='line' id='LC495'><br/></div><div class='line' id='LC496'><span class="k">\subsection</span><span class="nb">{</span>Total detections by time range<span class="nb">}</span></div><div class='line' id='LC497'>&nbsp;<span class="c">% VehiculosPorHoras.txt indica para cada nodo, el número de dispositivos detectados en el rango horario, sin diferenciar si el dispositivo es siempre el mismo o no. </span></div><div class='line' id='LC498'><br/></div><div class='line' id='LC499'>&nbsp;<span class="c">% Adicionalmente podemos calcular para cada nodo, el número de dispositivos detectados en cada rango de horas, sin diferenciar si el dispositivo es siempre el mismo o no. Así pues, se incluirán pasos repetidos del mismo vehículo.</span></div><div class='line' id='LC500'>Additionally we can calculate for each node, the number of detected devices by time range, without differentiating whether the device is the same or not (repeated passes). Thus, repeated passes of the same vehicle are counted.</div><div class='line' id='LC501'><br/></div><div class='line' id='LC502'>&nbsp;<span class="k">\begin</span><span class="nb">{</span>figure<span class="nb">}</span>[htb]</div><div class='line' id='LC503'>&nbsp;<span class="k">\begin</span><span class="nb">{</span>center<span class="nb">}</span></div><div class='line' id='LC504'>&nbsp;<span class="k">\includegraphics</span><span class="na">[scale=0.4]</span><span class="nb">{</span>VehiculosPorHoras.jpg<span class="nb">}</span></div><div class='line' id='LC505'>&nbsp;<span class="k">\caption</span><span class="nb">{</span>For each of the six nodes, the total number of detected devices by time range is shown. Figure is shown in logarithmic scale.</div><div class='line' id='LC506'>&nbsp;<span class="c">% Para cada uno de los nodos se muestra el total de dispositivos detectados en cada rango horario. Para cada rango de horas se muestra el total detectado en cada uno de los seis nodos</span></div><div class='line' id='LC507'>&nbsp;<span class="k">\label</span><span class="nb">{</span>VehiculosPorHoras<span class="nb">}}</span></div><div class='line' id='LC508'>&nbsp;<span class="k">\end</span><span class="nb">{</span>center<span class="nb">}</span></div><div class='line' id='LC509'>&nbsp;<span class="k">\end</span><span class="nb">{</span>figure<span class="nb">}</span></div><div class='line' id='LC510'>&nbsp;</div><div class='line' id='LC511'>&nbsp;<span class="c">% Al igual que en el caso anterior, observamos una mayor densidad circulatoria en todos los nodos, a las horas punta de entrada o salida del trabajo y colegios (ver la Figura \ref{VehiculosPorHoras}).</span></div><div class='line' id='LC512'>As in the previous case, a greater traffic density can be observed on all nodes, at peak times or out of work and school. (see Figure <span class="k">\ref</span><span class="nb">{</span>VehiculosPorHoras<span class="nb">}</span>).</div><div class='line' id='LC513'><br/></div><div class='line' id='LC514'><br/></div><div class='line' id='LC515'><span class="k">\subsection</span><span class="nb">{</span>Number of individual vehicles detections<span class="nb">}</span></div><div class='line' id='LC516'>&nbsp;<span class="c">% Intervalos.txt contine el número de vehículos detectados entre 0 y 5 veces, entre 5 y 10 veces y así sucesivamente hasta más de 25 veces detallado en cada nodo.</span></div><div class='line' id='LC517'><br/></div><div class='line' id='LC518'>&nbsp;<span class="c">% A continuación podemos sacar partido de la capacidad del sistema propuesto para individualizar los dispositivos BT, pudiendo detectar si esos mismos vehículos repiten paso por diferentes nodos.</span></div><div class='line' id='LC519'>We can take advantage of the proposed system&#39;s ability to identify BT devices. Thus, it can be detected whether vehicles pass by different nodes.</div><div class='line' id='LC520'><br/></div><div class='line' id='LC521'>&nbsp;<span class="k">\begin</span><span class="nb">{</span>figure<span class="nb">}</span>[htb]</div><div class='line' id='LC522'>&nbsp;<span class="k">\begin</span><span class="nb">{</span>center<span class="nb">}</span></div><div class='line' id='LC523'>&nbsp;<span class="k">\includegraphics</span><span class="na">[scale=0.4]</span><span class="nb">{</span>Intervalos.jpg<span class="nb">}</span></div><div class='line' id='LC524'>&nbsp;<span class="k">\caption</span><span class="nb">{</span>For each of the six nodes, the total number of detected devices N times (repeated occurrences of the same device) are shown. Figure is shown in logarithmic scale.</div><div class='line' id='LC525'>&nbsp;<span class="c">% Para cada uno de los nodos se muestra el total de dispositivos detectados cierto número de veces (repetidas apariciones del mismo dispositivo)</span></div><div class='line' id='LC526'>&nbsp;<span class="k">\label</span><span class="nb">{</span>Intervalos<span class="nb">}}</span></div><div class='line' id='LC527'>&nbsp;<span class="k">\end</span><span class="nb">{</span>center<span class="nb">}</span></div><div class='line' id='LC528'>&nbsp;<span class="k">\end</span><span class="nb">{</span>figure<span class="nb">}</span> </div><div class='line' id='LC529'>&nbsp;<span class="c">% Histograma que muestre cuántos coches se detectan entre una y 5 veces, entre 6 y 10, entre 10 y 20, y más de 20 veces.</span></div><div class='line' id='LC530'><br/></div><div class='line' id='LC531'>&nbsp;<span class="c">% En la Figura \ref{Intervalos} podemos ver que hay una gran cantidad de vehículos que repiten su paso por ciertos nodos (principalmente los situados en la A44) hasta 10 veces. </span></div><div class='line' id='LC532'>&nbsp;<span class="c">% Incluso se puede ver que en los nodos 4 y 5 hay alrededor de 1000 vehículos que repiten su paso más de 25 veces. En el resto de nodos, más de 25 repeticiones del mismo dispositivo se han detectado alrededor de 120 veces solamente.</span></div><div class='line' id='LC533'><br/></div><div class='line' id='LC534'>Figure <span class="k">\ref</span><span class="nb">{</span>Intervalos<span class="nb">}</span> shows a large number of vehicles that pass repeated times (up to 10 times) by some of the nodes (mainly those located in the A44).</div><div class='line' id='LC535'>Even it can be seen that nodes 4 and 5 detect about 1,000 vehicles passing more than 25 times repeated. </div><div class='line' id='LC536'>On the other nodes, over 25 repetitions of the same device have been detected only around 120 times.</div><div class='line' id='LC537'><br/></div><div class='line' id='LC538'><br/></div><div class='line' id='LC539'><span class="k">\subsection</span><span class="nb">{</span>Complexity of displacement<span class="nb">}</span></div><div class='line' id='LC540'>&nbsp;<span class="c">% NodosPorDondePasan.txt contiene el número de vehículos que han pasado por 2 nodos, por 3 nodos y así hasta por los 6 nodos y el número medio de veces que han pasado 2, 3, 4, 5 o 6 veces por cada nodo.</span></div><div class='line' id='LC541'><br/></div><div class='line' id='LC542'>&nbsp;<span class="c">% En el estudio de la complejidad de los desplazamientos se han calculado el número de vehículos que han pasado por 2 nodos, por 3 nodos y hasta 6 nodos. En la Tabla \ref{tNodosPorDondePasan} se muestra además el número medio de veces que han pasado un vehículo por 2, 3, 4, 5 o 6 nodos.</span></div><div class='line' id='LC543'>To study the complexity of displacements, the number of vehicles that have passed through two nodes, 3 nodes and up to 6 nodes  were calculated. </div><div class='line' id='LC544'>Table 4 also shows the average number of times that vehicles have passed through 2, 3, 4, 5 or 6 nodes.</div><div class='line' id='LC545'><br/></div><div class='line' id='LC546'>&nbsp;<span class="k">\begin</span><span class="nb">{</span>table<span class="nb">}</span></div><div class='line' id='LC547'>&nbsp;<span class="k">\begin</span><span class="nb">{</span>center<span class="nb">}</span></div><div class='line' id='LC548'>&nbsp;<span class="k">\begin</span><span class="nb">{</span>tabular<span class="nb">}{</span>|c|c|c|c|<span class="nb">}</span></div><div class='line' id='LC549'>&nbsp;<span class="k">\hline</span></div><div class='line' id='LC550'>&nbsp;No. of nodes <span class="nb">&amp;</span> 	No. of devices <span class="nb">&amp;</span> 	Total number of passes <span class="nb">&amp;</span> 	Mean <span class="s">$</span><span class="nv">\pm</span><span class="s">$</span> std. dev.  <span class="k">\\</span></div><div class='line' id='LC551'>&nbsp;<span class="k">\hline</span></div><div class='line' id='LC552'>1 <span class="nb">&amp;</span> 	72989 <span class="nb">&amp;</span> 	 165033 <span class="nb">&amp;</span> 	<span class="s">$</span><span class="m">2</span><span class="nb">.</span><span class="m">26</span><span class="nb"> </span><span class="nv">\pm</span><span class="nb"> </span><span class="m">31</span><span class="nb">.</span><span class="m">16</span><span class="s">$</span>  <span class="k">\\</span></div><div class='line' id='LC553'>&nbsp;<span class="k">\hline</span></div><div class='line' id='LC554'>2 <span class="nb">&amp;</span> 	53947 <span class="nb">&amp;</span> 	 425667 <span class="nb">&amp;</span> 	<span class="s">$</span><span class="m">7</span><span class="nb">.</span><span class="m">89</span><span class="nb"> </span><span class="nv">\pm</span><span class="nb"> </span><span class="m">11</span><span class="nb">.</span><span class="m">48</span><span class="s">$</span>  <span class="k">\\</span></div><div class='line' id='LC555'>&nbsp;<span class="k">\hline</span></div><div class='line' id='LC556'>3 <span class="nb">&amp;</span> 	8125 <span class="nb">&amp;</span> 	 131570 <span class="nb">&amp;</span> 	<span class="s">$</span><span class="m">16</span><span class="nb">.</span><span class="m">19</span><span class="nb"> </span><span class="nv">\pm</span><span class="nb"> </span><span class="m">24</span><span class="nb">.</span><span class="m">71</span><span class="s">$</span>  <span class="k">\\</span></div><div class='line' id='LC557'>&nbsp;<span class="k">\hline</span></div><div class='line' id='LC558'>4 <span class="nb">&amp;</span> 	1359 <span class="nb">&amp;</span> 	 39241 <span class="nb">&amp;</span> 	<span class="s">$</span><span class="m">28</span><span class="nb">.</span><span class="m">88</span><span class="nb"> </span><span class="nv">\pm</span><span class="nb"> </span><span class="m">140</span><span class="nb">.</span><span class="m">82</span><span class="s">$</span>  <span class="k">\\</span></div><div class='line' id='LC559'>&nbsp;<span class="k">\hline</span></div><div class='line' id='LC560'>5 <span class="nb">&amp;</span> 	254 <span class="nb">&amp;</span> 	 8603 <span class="nb">&amp;</span> 	<span class="s">$</span><span class="m">33</span><span class="nb">.</span><span class="m">87</span><span class="nb"> </span><span class="nv">\pm</span><span class="nb"> </span><span class="m">59</span><span class="nb">.</span><span class="m">51</span><span class="s">$</span>  <span class="k">\\</span></div><div class='line' id='LC561'>&nbsp;<span class="k">\hline</span></div><div class='line' id='LC562'>6 <span class="nb">&amp;</span> 	61 <span class="nb">&amp;</span> 	 3731 <span class="nb">&amp;</span> 	<span class="s">$</span><span class="m">61</span><span class="nb">.</span><span class="m">16</span><span class="nb"> </span><span class="nv">\pm</span><span class="nb"> </span><span class="m">94</span><span class="nb">.</span><span class="m">78</span><span class="s">$</span>  <span class="k">\\</span></div><div class='line' id='LC563'>&nbsp;<span class="k">\hline</span></div><div class='line' id='LC564'>&nbsp;<span class="k">\end</span><span class="nb">{</span>tabular<span class="nb">}</span></div><div class='line' id='LC565'>&nbsp;<span class="k">\end</span><span class="nb">{</span>center<span class="nb">}</span></div><div class='line' id='LC566'>&nbsp;<span class="k">\caption</span><span class="nb">{</span>Total number of vehicles that have passed through two nodes, 3 nodes and up to 6 nodes, and average number of times that vehicles have passed through 2, 3, 4, 5 or 6 nodes. In some cases the deviations are high because some devices have a very high number of occurrences for some nodes.</div><div class='line' id='LC567'>&nbsp;<span class="c">% Número de vehículos que han pasado por 2 nodos, por 3 nodos y hasta 6 nodos, y el número medio de veces que han pasado los vehículos por 2, 3, 4, 5 o 6 nodos. En algunos casos las desviaciones son altas debido a que algunos dispositivos tienen un número muy alto de apariciones por algunos de los nodos</span></div><div class='line' id='LC568'>&nbsp;<span class="k">\label</span><span class="nb">{</span>tNodosPorDondePasan<span class="nb">}}</span></div><div class='line' id='LC569'>&nbsp;<span class="k">\end</span><span class="nb">{</span>table<span class="nb">}</span></div><div class='line' id='LC570'><br/></div><div class='line' id='LC571'>&nbsp;<span class="c">% La información anterior se complementa visualmente con la Figura \ref{fNodosPorDondePasan}, en la que se muestra (en escala logarítmica) cuántos coches pasan sólo por un nodo, por dos nodos, por tres nodos, etc.</span></div><div class='line' id='LC572'>The above information is complemented with Figure <span class="k">\ref</span><span class="nb">{</span>fNodosPorDondePasan<span class="nb">}</span>, that shows how many cars pass by only one node, two nodes, three nodes, etc.</div><div class='line' id='LC573'><br/></div><div class='line' id='LC574'>&nbsp;<span class="k">\begin</span><span class="nb">{</span>figure<span class="nb">}</span>[htb]</div><div class='line' id='LC575'>&nbsp;<span class="k">\begin</span><span class="nb">{</span>center<span class="nb">}</span></div><div class='line' id='LC576'>&nbsp;<span class="k">\includegraphics</span><span class="na">[scale=0.4]</span><span class="nb">{</span>NodosPorDondePasan.jpg<span class="nb">}</span></div><div class='line' id='LC577'>&nbsp;<span class="k">\caption</span><span class="nb">{</span>Figure shows how many cars pass by only one node, two nodes, three nodes, etc. Figure is shown in logarithmic scale.</div><div class='line' id='LC578'>&nbsp;<span class="c">% La figura muestra cuántos coches pasan sólo por un nodo, por dos nodos, por tres nodos, etc. Para mejorar la visualización se ha utilizado escala logarítmica</span></div><div class='line' id='LC579'>&nbsp;<span class="k">\label</span><span class="nb">{</span>fNodosPorDondePasan<span class="nb">}}</span></div><div class='line' id='LC580'>&nbsp;<span class="k">\end</span><span class="nb">{</span>center<span class="nb">}</span></div><div class='line' id='LC581'>&nbsp;<span class="k">\end</span><span class="nb">{</span>figure<span class="nb">}</span></div><div class='line' id='LC582'>&nbsp;<span class="c">% Histograma que muestre cuántos coches pasan sólo por un nodo, por dos nodos, por tres nodos, etc.</span></div><div class='line' id='LC583'><br/></div><div class='line' id='LC584'>&nbsp;<span class="c">% Como se esperaba, gran parte de los dispositivos BT pasan pocas veces por casi todos los nodos, mientras que la mayoría pasa sólo por uno o dos de ellos (sus desplazamientos se centran en una parte de la zona pequeña monitorizada).</span></div><div class='line' id='LC585'>As expected, most of the BT devices rarely passed by all nodes, while most of devices pass only by one or two of nodes (their displacements are focused on a small part of the monitored area).</div><div class='line' id='LC586'><br/></div><div class='line' id='LC587'><br/></div><div class='line' id='LC588'><span class="k">\subsection</span><span class="nb">{</span>Effect of Nov-14 strike on zone traffic<span class="nb">}</span></div><div class='line' id='LC589'>&nbsp;<span class="c">% Tabla en la que mostremos cómo afectó el día 14-N con la huelga.</span></div><div class='line' id='LC590'><br/></div><div class='line' id='LC591'>&nbsp;<span class="c">% Justo en mitad del periodo de monitorización, en España se celebró un día de huelga general (14 de noviembre de 2012), lo que ha quedado reflejado en la cantidad de coches detectados en los nodos.</span></div><div class='line' id='LC592'><br/></div><div class='line' id='LC593'>&nbsp;<span class="c">% Se ha analizado el efecto de la huelga general en el tráfico de la zona monitorizada, mostrando en la Tabla \ref{huelga} el número de pasos detectados el día 14 de noviembre y justo el día siguiente.</span></div><div class='line' id='LC594'><br/></div><div class='line' id='LC595'>Right in the middle of the monitoring period in Spain was held a day of general strike (November 14, 2012), which has been reflected in the number of detected devices (cars) on the nodes.</div><div class='line' id='LC596'><br/></div><div class='line' id='LC597'>The effect of the general strike in the traffic of the monitored area has been analyzed and shown in Table <span class="k">\ref</span><span class="nb">{</span>huelga<span class="nb">}</span> as the number of detected devices on November 14 and the very next day.</div><div class='line' id='LC598'><br/></div><div class='line' id='LC599'>&nbsp;<span class="k">\begin</span><span class="nb">{</span>table<span class="nb">}</span></div><div class='line' id='LC600'>&nbsp;<span class="k">\begin</span><span class="nb">{</span>center<span class="nb">}</span></div><div class='line' id='LC601'>&nbsp;<span class="k">\begin</span><span class="nb">{</span>tabular<span class="nb">}{</span>|c|c|c|<span class="nb">}</span></div><div class='line' id='LC602'>&nbsp;<span class="k">\hline</span></div><div class='line' id='LC603'>&nbsp;Node  <span class="nb">&amp;</span>  Total number of passes (nov-14) <span class="nb">&amp;</span> Total number of passes (nov-15) <span class="k">\\</span></div><div class='line' id='LC604'>&nbsp;<span class="k">\hline</span></div><div class='line' id='LC605'>&nbsp;2	<span class="nb">&amp;</span> 1841  <span class="nb">&amp;</span> 2722  <span class="k">\\</span></div><div class='line' id='LC606'>&nbsp;<span class="k">\hline</span></div><div class='line' id='LC607'>&nbsp;3	<span class="nb">&amp;</span> 891   <span class="nb">&amp;</span> 1169 <span class="k">\\</span></div><div class='line' id='LC608'>&nbsp;<span class="k">\hline</span></div><div class='line' id='LC609'>&nbsp;4	<span class="nb">&amp;</span> 10807	<span class="nb">&amp;</span> 16942 <span class="k">\\</span></div><div class='line' id='LC610'>&nbsp;<span class="k">\hline</span></div><div class='line' id='LC611'>&nbsp;5	<span class="nb">&amp;</span> 831  	<span class="nb">&amp;</span> 4017<span class="k">\\</span></div><div class='line' id='LC612'>&nbsp;<span class="k">\hline</span></div><div class='line' id='LC613'>&nbsp;6	<span class="nb">&amp;</span> 946   <span class="nb">&amp;</span> 1419 <span class="k">\\</span></div><div class='line' id='LC614'>&nbsp;<span class="k">\hline</span></div><div class='line' id='LC615'>&nbsp;<span class="k">\end</span><span class="nb">{</span>tabular<span class="nb">}</span></div><div class='line' id='LC616'>&nbsp;<span class="k">\end</span><span class="nb">{</span>center<span class="nb">}</span></div><div class='line' id='LC617'>&nbsp;<span class="k">\caption</span><span class="nb">{</span>Comparison of the number of passes for each node between the general strike day (November 14) and the very next day. Node 1 results are not reported because the hardware device suffered a power supply problem for a couple of days at that time.</div><div class='line' id='LC618'>&nbsp;<span class="c">% Comparación del número de pasos por cada nodo entre el día de la huelga general (14 de noviembre) y el día siguiente. No aparece recogido el nodo 1 debido a que el dispositivo hardware sufrió un problema de alimentación durante un par de días en esas fechas</span></div><div class='line' id='LC619'>&nbsp;<span class="k">\label</span><span class="nb">{</span>huelga<span class="nb">}}</span></div><div class='line' id='LC620'>&nbsp;<span class="k">\end</span><span class="nb">{</span>table<span class="nb">}</span></div><div class='line' id='LC621'><br/></div><div class='line' id='LC622'>&nbsp;<span class="c">% En la Tabla \ref{huelga} se aprecia un número de vehículos detectados menor el día de la huelga respecto al día siguiente (laborable en el que la actividad debió ser normal en la zona).</span></div><div class='line' id='LC623'>Table 8 shows a lowest number of detected vehicles the day of the strike that on the following day (working day in which the activity should be normal in the area).</div><div class='line' id='LC624'><br/></div><div class='line' id='LC625'><br/></div><div class='line' id='LC626'><span class="k">\subsection</span><span class="nb">{</span>Analysis of vehicles speed between two consecutive nodes<span class="nb">}</span></div><div class='line' id='LC627'>&nbsp;<span class="c">% tabla de la velocidad media de los vehículos en el tramo (en laborables y en fin de semana).</span></div><div class='line' id='LC628'>&nbsp;<span class="c">% El &quot;Granada 5&quot; está en el km 123,250 ; el &quot;Granada 4&quot; está en el km 119,550 ; hay 3700 metros entre ambos. </span></div><div class='line' id='LC629'><br/></div><div class='line' id='LC630'>&nbsp;<span class="c">% Por último, a partir de los dos nodos consecutivos localizados en la autovía A44, podemos calcular velocidades medias en el tramo delimitado por los nodos 4 (situado en el km 119,550) y 5 (situado en el km 123,250) de un total de 3700 metros. </span></div><div class='line' id='LC631'>&nbsp;<span class="c">% En realidad podemos calcular la velocidad media en el tramo a nivel global, no la velocidad a la que ha ido cada vehículo en cada instante dentro del tramo.</span></div><div class='line' id='LC632'><br/></div><div class='line' id='LC633'>Finally, taking two consecutive nodes, located on the A44 highway, average speeds in the section bounded by nodes 4 (located at km 119.550) and 5 (located at km 123.250) can be calculated.</div><div class='line' id='LC634'>This highway section where the study takes place is of 3700 meters long.</div><div class='line' id='LC635'>Actually, we can calculate the average speed in the global section, not the speed at which each vehicle has at each instant within the section.</div><div class='line' id='LC636'><br/></div><div class='line' id='LC637'>&nbsp;<span class="k">\begin</span><span class="nb">{</span>table<span class="nb">}</span></div><div class='line' id='LC638'>&nbsp;<span class="k">\begin</span><span class="nb">{</span>center<span class="nb">}</span></div><div class='line' id='LC639'>&nbsp;<span class="k">\begin</span><span class="nb">{</span>tabular<span class="nb">}{</span>|c|c|<span class="nb">}</span></div><div class='line' id='LC640'>&nbsp;<span class="k">\hline</span></div><div class='line' id='LC641'>Speed range (km/h) <span class="nb">&amp;</span>  No. of passes  <span class="k">\\</span></div><div class='line' id='LC642'>&nbsp;<span class="k">\hline</span></div><div class='line' id='LC643'>v<span class="s">$</span><span class="nv">\leq</span><span class="s">$</span>60.0	<span class="nb">&amp;</span> 1495  <span class="k">\\</span></div><div class='line' id='LC644'>&nbsp;<span class="k">\hline</span></div><div class='line' id='LC645'>60.0<span class="s">$</span><span class="nv">\leq</span><span class="s">$</span>v<span class="s">$</span><span class="nv">\leq</span><span class="s">$</span>70.0 <span class="nb">&amp;</span> 2585  <span class="k">\\</span></div><div class='line' id='LC646'>&nbsp;<span class="k">\hline</span></div><div class='line' id='LC647'>70.0<span class="s">$</span><span class="nv">\leq</span><span class="s">$</span>v<span class="s">$</span><span class="nv">\leq</span><span class="s">$</span>80.0 <span class="nb">&amp;</span> 7421  <span class="k">\\</span></div><div class='line' id='LC648'>&nbsp;<span class="k">\hline</span></div><div class='line' id='LC649'>80.0<span class="s">$</span><span class="nv">\leq</span><span class="s">$</span>v<span class="s">$</span><span class="nv">\leq</span><span class="s">$</span>90.0 <span class="nb">&amp;</span> 16339  <span class="k">\\</span></div><div class='line' id='LC650'>&nbsp;<span class="k">\hline</span></div><div class='line' id='LC651'>90.0<span class="s">$</span><span class="nv">\leq</span><span class="s">$</span>v<span class="s">$</span><span class="nv">\leq</span><span class="s">$</span>100.0 <span class="nb">&amp;</span> 20144  <span class="k">\\</span></div><div class='line' id='LC652'>&nbsp;<span class="k">\hline</span></div><div class='line' id='LC653'>100.0<span class="s">$</span><span class="nv">\leq</span><span class="s">$</span>v<span class="s">$</span><span class="nv">\leq</span><span class="s">$</span>120.0 <span class="nb">&amp;</span> 14384  <span class="k">\\</span></div><div class='line' id='LC654'>&nbsp;<span class="k">\hline</span></div><div class='line' id='LC655'>120.0<span class="s">$</span><span class="nv">\leq</span><span class="s">$</span>v<span class="s">$</span><span class="nv">\leq</span><span class="s">$</span>140.0 <span class="nb">&amp;</span> 5434  <span class="k">\\</span></div><div class='line' id='LC656'>&nbsp;<span class="k">\hline</span></div><div class='line' id='LC657'>v<span class="s">$</span><span class="nv">\geq</span><span class="s">$</span>140.0 <span class="nb">&amp;</span> 1326  <span class="k">\\</span></div><div class='line' id='LC658'>&nbsp;<span class="k">\hline</span></div><div class='line' id='LC659'>&nbsp;<span class="k">\end</span><span class="nb">{</span>tabular<span class="nb">}</span></div><div class='line' id='LC660'>&nbsp;<span class="k">\end</span><span class="nb">{</span>center<span class="nb">}</span></div><div class='line' id='LC661'>&nbsp;<span class="k">\caption</span><span class="nb">{</span>Average speeds (globally) in the section bounded by nodes 4 and 5.</div><div class='line' id='LC662'>&nbsp;&nbsp;<span class="c">% Velocidades medias (a nivel global) en el tramo delimitado por los nodos 4 y 5</span></div><div class='line' id='LC663'>&nbsp;<span class="k">\label</span><span class="nb">{</span>velocidad<span class="nb">}}</span></div><div class='line' id='LC664'>&nbsp;<span class="k">\end</span><span class="nb">{</span>table<span class="nb">}</span></div><div class='line' id='LC665'><br/></div><div class='line' id='LC666'><br/></div><div class='line' id='LC667'>&nbsp;<span class="c">% En dicho tramo, la velocidad está limitada a 100 km/h. Sin embargo, y aunque la gran mayoría respetan el límite, vemos que una gran cantidad de coches superan dicha limitación.</span></div><div class='line' id='LC668'>In that section, the speed is limited to 100 km/h. However, although most of the vehicles respect this limit, a lot of cars exceed this limitation.</div><div class='line' id='LC669'><br/></div><div class='line' id='LC670'><br/></div><div class='line' id='LC671'><span class="k">\subsection</span><span class="nb">{</span>Flow traffic prediction by means of time-series forecasting<span class="nb">}</span></div><div class='line' id='LC672'>As far as data is chronologically collected, methods coming from the area of time-series forecasting can be used in order to predict future behaviour of traffic flow.</div><div class='line' id='LC673'><br/></div><div class='line' id='LC674'>From the big variety of existing mtethods, seven different ones have been selected in order to compare the validity of their predictions. The methods being used are the followings: </div><div class='line' id='LC675'>...</div><div class='line' id='LC676'>In order to compare results, a statistical study is included allowing us to draw some conclusions.</div><div class='line' id='LC677'><br/></div><div class='line' id='LC678'>The experimentation has been carried out using the data collected for nodes 1 to 5 over ??N??? days. <span class="s">$</span><span class="m">75</span><span class="nv">\%</span><span class="s">$</span> of the data collected has been used to train the methods, and obtain their associated models, and the remaining <span class="s">$</span><span class="m">25</span><span class="nv">\%</span><span class="s">$</span> has been used to test their generalization capabilities. </div><div class='line' id='LC679'><br/></div><div class='line' id='LC680'>For each method and each node, a set of measures have been computed in order to determine the accuracy of the prediction. As most textbooks recommend, one of the chosen measures is  the Mean Absolute Percentage Error (MAPE) <span class="k">\cite</span><span class="nb">{</span>Bowerman2004<span class="nb">}</span>, which was used in the first M-competition <span class="k">\cite</span><span class="nb">{</span>Makridakis1982<span class="nb">}</span>. Median Absolute Percentage Error (MdAPE) <span class="k">\cite</span><span class="nb">{</span>Armstrong,Fildes1992<span class="nb">}</span> is also used, as a normalized version of the previous one. More recently, description of many others measures can be found in  <span class="k">\cite</span><span class="nb">{</span>Gooijer25years<span class="nb">}</span> and <span class="k">\cite</span><span class="nb">{</span>Hyndman2006<span class="nb">}</span>. Among all of them, in this work we use the followings:</div><div class='line' id='LC681'><br/></div><div class='line' id='LC682'><span class="k">\begin</span><span class="nb">{</span>itemize<span class="nb">}</span></div><div class='line' id='LC683'>&nbsp;&nbsp;<span class="k">\item</span> Mean Absolute Error (MAE):</div><div class='line' id='LC684'>&nbsp;&nbsp;&nbsp;&nbsp;&nbsp;&nbsp;&nbsp;&nbsp;<span class="k">\begin</span><span class="nb">{</span>equation<span class="nb">}</span><span class="k">\label</span><span class="nb">{</span>eq:MAE<span class="nb">}</span></div><div class='line' id='LC685'>&nbsp;&nbsp;&nbsp;&nbsp;&nbsp;&nbsp;&nbsp;&nbsp;&nbsp;&nbsp;&nbsp;&nbsp;MAE = mean(<span class="k">\mid</span> e<span class="nb">_</span>t<span class="k">\mid</span>)</div><div class='line' id='LC686'>&nbsp;&nbsp;&nbsp;&nbsp;&nbsp;&nbsp;&nbsp;&nbsp;<span class="k">\end</span><span class="nb">{</span>equation<span class="nb">}</span></div><div class='line' id='LC687'><br/></div><div class='line' id='LC688'>&nbsp;&nbsp;<span class="k">\item</span> Mean Absolute Percentage Error (MAPE):</div><div class='line' id='LC689'>&nbsp;&nbsp;&nbsp;&nbsp;&nbsp;&nbsp;&nbsp;&nbsp;<span class="k">\begin</span><span class="nb">{</span>equation<span class="nb">}</span><span class="k">\label</span><span class="nb">{</span>eq:MAPE<span class="nb">}</span></div><div class='line' id='LC690'>&nbsp;&nbsp;&nbsp;&nbsp;&nbsp;&nbsp;&nbsp;&nbsp;&nbsp;&nbsp;&nbsp;&nbsp;MAPE = mean(<span class="k">\mid</span> p<span class="nb">_</span>t<span class="k">\mid</span>)</div><div class='line' id='LC691'>&nbsp;&nbsp;&nbsp;&nbsp;&nbsp;&nbsp;&nbsp;&nbsp;<span class="k">\end</span><span class="nb">{</span>equation<span class="nb">}</span></div><div class='line' id='LC692'><br/></div><div class='line' id='LC693'>&nbsp;&nbsp;&nbsp;&nbsp;&nbsp;&nbsp;&nbsp;&nbsp;<span class="k">\bigskip</span></div><div class='line' id='LC694'>&nbsp;&nbsp;<span class="k">\item</span> Median Absolute Percentage Error (MdAPE):</div><div class='line' id='LC695'>&nbsp;&nbsp;&nbsp;&nbsp;&nbsp;&nbsp;&nbsp;&nbsp;<span class="k">\begin</span><span class="nb">{</span>equation<span class="nb">}</span><span class="k">\label</span><span class="nb">{</span>eq:MDAPE<span class="nb">}</span></div><div class='line' id='LC696'>&nbsp;&nbsp;&nbsp;&nbsp;&nbsp;&nbsp;&nbsp;&nbsp;&nbsp;&nbsp;&nbsp;&nbsp;MdAPE = median(<span class="k">\mid</span> p<span class="nb">_</span>t<span class="k">\mid</span>)</div><div class='line' id='LC697'>&nbsp;&nbsp;&nbsp;&nbsp;&nbsp;&nbsp;&nbsp;&nbsp;<span class="k">\end</span><span class="nb">{</span>equation<span class="nb">}</span></div><div class='line' id='LC698'><br/></div><div class='line' id='LC699'>&nbsp;&nbsp;<span class="k">\item</span> Symmetric Median Absolute Percentage Error (sMdAPE):</div><div class='line' id='LC700'>&nbsp;&nbsp;&nbsp;&nbsp;&nbsp;&nbsp;&nbsp;&nbsp;<span class="k">\begin</span><span class="nb">{</span>equation<span class="nb">}</span><span class="k">\label</span><span class="nb">{</span>eq:SMDAPE<span class="nb">}</span></div><div class='line' id='LC701'>&nbsp;&nbsp;&nbsp;&nbsp;&nbsp;&nbsp;&nbsp;&nbsp;&nbsp;&nbsp;&nbsp;&nbsp;sMdAPE = median(200<span class="k">\mid</span> Y<span class="nb">_</span>t - F<span class="nb">_</span>t<span class="k">\mid</span> (Y<span class="nb">_</span>t + F<span class="nb">_</span>t))</div><div class='line' id='LC702'>&nbsp;&nbsp;&nbsp;&nbsp;&nbsp;&nbsp;&nbsp;&nbsp;<span class="k">\end</span><span class="nb">{</span>equation<span class="nb">}</span></div><div class='line' id='LC703'><br/></div><div class='line' id='LC704'>&nbsp;&nbsp;<span class="k">\item</span> Mean Absolute Scaled Error (MASE):</div><div class='line' id='LC705'>&nbsp;&nbsp;&nbsp;&nbsp;&nbsp;&nbsp;&nbsp;&nbsp;<span class="k">\begin</span><span class="nb">{</span>equation<span class="nb">}</span><span class="k">\label</span><span class="nb">{</span>eq:MASE<span class="nb">}</span></div><div class='line' id='LC706'>&nbsp;&nbsp;&nbsp;&nbsp;&nbsp;&nbsp;&nbsp;&nbsp;&nbsp;&nbsp;&nbsp;&nbsp;MASE = mean(<span class="k">\mid</span> q<span class="nb">_</span>t<span class="k">\mid</span>)</div><div class='line' id='LC707'>&nbsp;&nbsp;&nbsp;&nbsp;&nbsp;&nbsp;&nbsp;&nbsp;<span class="k">\end</span><span class="nb">{</span>equation<span class="nb">}</span></div><div class='line' id='LC708'><br/></div><div class='line' id='LC709'>where <span class="s">$</span><span class="nb">Y_t</span><span class="s">$</span> is the observation at time <span class="s">$</span><span class="nb">t </span><span class="o">=</span><span class="nb"> {</span><span class="m">1</span><span class="nb">,...,n}</span><span class="s">$</span>; <span class="s">$</span><span class="nb">F_t</span><span class="s">$</span> is the forecast of <span class="s">$</span><span class="nb">Y_t</span><span class="s">$</span>; <span class="s">$</span><span class="nb">e_t</span><span class="s">$</span> is the forecast error (i.e. <span class="s">$</span><span class="nb">e_t</span><span class="o">=</span><span class="nb"> Y_t </span><span class="o">-</span><span class="nb"> F_t</span><span class="s">$</span>); <span class="s">$</span><span class="nb">p_t </span><span class="o">=</span><span class="nb"> </span><span class="m">100</span><span class="nb">e_t</span><span class="o">/</span><span class="nb">Y_t</span><span class="s">$</span> is the percentage error, and</div><div class='line' id='LC710'>&nbsp;&nbsp;&nbsp;&nbsp;&nbsp;&nbsp;&nbsp;&nbsp;&nbsp;<span class="s">$</span><span class="nb">q_t </span><span class="o">=</span><span class="nb"> </span><span class="nv">\displaystyle\frac</span><span class="nb">{e_t}{</span><span class="nv">\displaystyle\frac</span><span class="nb">{</span><span class="m">1</span><span class="nb">}{n</span><span class="o">-</span><span class="m">1</span><span class="nb">} </span><span class="nv">\sum</span><span class="nb">_{i</span><span class="o">=</span><span class="m">2</span><span class="nb">}^n </span><span class="nv">\mid</span><span class="nb"> Y_i </span><span class="o">-</span><span class="nb"> Y_{i</span><span class="o">-</span><span class="m">1</span><span class="nb">} </span><span class="nv">\mid</span><span class="nb"> }</span><span class="s">$</span></div><div class='line' id='LC711'><span class="k">\end</span><span class="nb">{</span>itemize<span class="nb">}</span></div><div class='line' id='LC712'><br/></div><div class='line' id='LC713'>Methods ??? have been executed using the R software, including the &quot;forecast&quot; package (cita). In the other hand, L-co-R is a co-evolutionary algorithm based on radial basis function neural networks. The sthocastic nature of the later lead us to run it 30 times per every node, so that average errors be computed.</div><div class='line' id='LC714'><br/></div><div class='line' id='LC715'>Table ??? shows the result of any of the methods over each of the nodes; bold faces outline the best results. As can be seen...</div><div class='line' id='LC716'><br/></div><div class='line' id='LC717'>Testing whether there exist significant differences between the methods has been carried out using non-parametrical test.<span class="k">\section</span><span class="nb">{</span>Conclusions<span class="nb">}</span></div><div class='line' id='LC718'><span class="k">\label</span><span class="nb">{</span>sec:conclusions<span class="nb">}</span></div><div class='line' id='LC719'><br/></div><div class='line' id='LC720'><br/></div><div class='line' id='LC721'><span class="c">%% The Appendices part is started with the command \appendix;</span></div><div class='line' id='LC722'><span class="c">%% appendix sections are then done as normal sections</span></div><div class='line' id='LC723'><span class="c">%% \appendix</span></div><div class='line' id='LC724'><br/></div><div class='line' id='LC725'><span class="c">%% \section{}</span></div><div class='line' id='LC726'><span class="c">%% \label{}</span></div><div class='line' id='LC727'><br/></div><div class='line' id='LC728'><span class="c">%% References</span></div><div class='line' id='LC729'><span class="c">%%</span></div><div class='line' id='LC730'><span class="c">%% Following citation commands can be used in the body text:</span></div><div class='line' id='LC731'><span class="c">%%</span></div><div class='line' id='LC732'><span class="c">%%  \citet{key}  ==&gt;&gt;  Jones et al. (1990)</span></div><div class='line' id='LC733'><span class="c">%%  \citep{key}  ==&gt;&gt;  (Jones et al., 1990)</span></div><div class='line' id='LC734'><span class="c">%%</span></div><div class='line' id='LC735'><span class="c">%% Multiple citations as normal:</span></div><div class='line' id='LC736'><span class="c">%% \citep{key1,key2}         ==&gt;&gt; (Jones et al., 1990; Smith, 1989)</span></div><div class='line' id='LC737'><span class="c">%%                            or  (Jones et al., 1990, 1991)</span></div><div class='line' id='LC738'><span class="c">%%                            or  (Jones et al., 1990a,b)</span></div><div class='line' id='LC739'><span class="c">%% \cite{key} is the equivalent of \citet{key} in author-year mode</span></div><div class='line' id='LC740'><span class="c">%%</span></div><div class='line' id='LC741'><span class="c">%% Full author lists may be forced with \citet* or \citep*, e.g.</span></div><div class='line' id='LC742'><span class="c">%%   \citep*{key}            ==&gt;&gt; (Jones, Baker, and Williams, 1990)</span></div><div class='line' id='LC743'><span class="c">%%</span></div><div class='line' id='LC744'><span class="c">%% Optional notes as:</span></div><div class='line' id='LC745'><span class="c">%%   \citep[chap. 2]{key}    ==&gt;&gt; (Jones et al., 1990, chap. 2)</span></div><div class='line' id='LC746'><span class="c">%%   \citep[e.g.,][]{key}    ==&gt;&gt; (e.g., Jones et al., 1990)</span></div><div class='line' id='LC747'><span class="c">%%   \citep[see][pg. 34]{key}==&gt;&gt; (see Jones et al., 1990, pg. 34)</span></div><div class='line' id='LC748'><span class="c">%%  (Note: in standard LaTeX, only one note is allowed, after the ref.</span></div><div class='line' id='LC749'><span class="c">%%   Here, one note is like the standard, two make pre- and post-notes.)</span></div><div class='line' id='LC750'><span class="c">%%</span></div><div class='line' id='LC751'><span class="c">%%   \citealt{key}          ==&gt;&gt; Jones et al. 1990</span></div><div class='line' id='LC752'><span class="c">%%   \citealt*{key}         ==&gt;&gt; Jones, Baker, and Williams 1990</span></div><div class='line' id='LC753'><span class="c">%%   \citealp{key}          ==&gt;&gt; Jones et al., 1990</span></div><div class='line' id='LC754'><span class="c">%%   \citealp*{key}         ==&gt;&gt; Jones, Baker, and Williams, 1990</span></div><div class='line' id='LC755'><span class="c">%%</span></div><div class='line' id='LC756'><span class="c">%% Additional citation possibilities</span></div><div class='line' id='LC757'><span class="c">%%   \citeauthor{key}       ==&gt;&gt; Jones et al.</span></div><div class='line' id='LC758'><span class="c">%%   \citeauthor*{key}      ==&gt;&gt; Jones, Baker, and Williams</span></div><div class='line' id='LC759'><span class="c">%%   \citeyear{key}         ==&gt;&gt; 1990</span></div><div class='line' id='LC760'><span class="c">%%   \citeyearpar{key}      ==&gt;&gt; (1990)</span></div><div class='line' id='LC761'><span class="c">%%   \citetext{priv. comm.} ==&gt;&gt; (priv. comm.)</span></div><div class='line' id='LC762'><span class="c">%%   \citenum{key}          ==&gt;&gt; 11 [non-superscripted]</span></div><div class='line' id='LC763'><span class="c">%% Note: full author lists depends on whether the bib style supports them;</span></div><div class='line' id='LC764'><span class="c">%%       if not, the abbreviated list is printed even when full requested.</span></div><div class='line' id='LC765'><span class="c">%%</span></div><div class='line' id='LC766'><span class="c">%% For names like della Robbia at the start of a sentence, use</span></div><div class='line' id='LC767'><span class="c">%%   \Citet{dRob98}         ==&gt;&gt; Della Robbia (1998)</span></div><div class='line' id='LC768'><span class="c">%%   \Citep{dRob98}         ==&gt;&gt; (Della Robbia, 1998)</span></div><div class='line' id='LC769'><span class="c">%%   \Citeauthor{dRob98}    ==&gt;&gt; Della Robbia</span></div><div class='line' id='LC770'><br/></div><div class='line' id='LC771'><br/></div><div class='line' id='LC772'><span class="c">%% References with bibTeX database:</span></div><div class='line' id='LC773'><br/></div><div class='line' id='LC774'><span class="k">\bibliographystyle</span><span class="nb">{</span>elsarticle-harv<span class="nb">}</span></div><div class='line' id='LC775'><span class="k">\bibliography</span><span class="nb">{</span>&lt;your-bib-database&gt;<span class="nb">}</span></div><div class='line' id='LC776'><br/></div><div class='line' id='LC777'><span class="c">%% Authors are advised to submit their bibtex database files. They are</span></div><div class='line' id='LC778'><span class="c">%% requested to list a bibtex style file in the manuscript if they do</span></div><div class='line' id='LC779'><span class="c">%% not want to use elsarticle-harv.bst.</span></div><div class='line' id='LC780'><br/></div><div class='line' id='LC781'><span class="c">%% References without bibTeX database:</span></div><div class='line' id='LC782'><br/></div><div class='line' id='LC783'><span class="c">% \begin{thebibliography}{00}</span></div><div class='line' id='LC784'><br/></div><div class='line' id='LC785'><span class="c">%% \bibitem must have one of the following forms:</span></div><div class='line' id='LC786'><span class="c">%%   \bibitem[Jones et al.(1990)]{key}...</span></div><div class='line' id='LC787'><span class="c">%%   \bibitem[Jones et al.(1990)Jones, Baker, and Williams]{key}...</span></div><div class='line' id='LC788'><span class="c">%%   \bibitem[Jones et al., 1990]{key}...</span></div><div class='line' id='LC789'><span class="c">%%   \bibitem[\protect\citeauthoryear{Jones, Baker, and Williams}{Jones</span></div><div class='line' id='LC790'><span class="c">%%       et al.}{1990}]{key}...</span></div><div class='line' id='LC791'><span class="c">%%   \bibitem[\protect\citeauthoryear{Jones et al.}{1990}]{key}...</span></div><div class='line' id='LC792'><span class="c">%%   \bibitem[\protect\astroncite{Jones et al.}{1990}]{key}...</span></div><div class='line' id='LC793'><span class="c">%%   \bibitem[\protect\citename{Jones et al., }1990]{key}...</span></div><div class='line' id='LC794'><span class="c">%%   \harvarditem[Jones et al.]{Jones, Baker, and Williams}{1990}{key}...</span></div><div class='line' id='LC795'><span class="c">%%</span></div><div class='line' id='LC796'><br/></div><div class='line' id='LC797'><span class="c">% \bibitem[ ()]{}</span></div><div class='line' id='LC798'><br/></div><div class='line' id='LC799'><span class="c">% \end{thebibliography}</span></div><div class='line' id='LC800'><br/></div><div class='line' id='LC801'><span class="k">\end</span><span class="nb">{</span>document<span class="nb">}</span></div><div class='line' id='LC802'><br/></div><div class='line' id='LC803'><span class="c">%%</span></div><div class='line' id='LC804'><span class="c">%% End of file `elsarticle-template-harv.tex&#39;.</span></div></pre></div>
          </td>
        </tr>
      </table>
  </div>

          </div>
        </div>

        <a href="#jump-to-line" rel="facebox" data-hotkey="l" class="js-jump-to-line" style="display:none">Jump to Line</a>
        <div id="jump-to-line" style="display:none">
          <h2>Jump to Line</h2>
          <form accept-charset="UTF-8" class="js-jump-to-line-form">
            <input class="textfield js-jump-to-line-field" type="text">
            <div class="full-button">
              <button type="submit" class="button">Go</button>
            </div>
          </form>
        </div>

      </div>
    </div>
</div>

<div id="js-frame-loading-template" class="frame frame-loading large-loading-area" style="display:none;">
  <img class="js-frame-loading-spinner" src="https://a248.e.akamai.net/assets.github.com/images/spinners/octocat-spinner-128.gif?1347543529" height="64" width="64">
</div>


        </div>
      </div>
      <div class="context-overlay"></div>
    </div>

      <div id="footer-push"></div><!-- hack for sticky footer -->
    </div><!-- end of wrapper - hack for sticky footer -->

      <!-- footer -->
      <div id="footer">
  <div class="container clearfix">

      <dl class="footer_nav">
        <dt>GitHub</dt>
        <dd><a href="https://github.com/about">About us</a></dd>
        <dd><a href="https://github.com/blog">Blog</a></dd>
        <dd><a href="https://github.com/contact">Contact &amp; support</a></dd>
        <dd><a href="http://enterprise.github.com/">GitHub Enterprise</a></dd>
        <dd><a href="http://status.github.com/">Site status</a></dd>
      </dl>

      <dl class="footer_nav">
        <dt>Applications</dt>
        <dd><a href="http://mac.github.com/">GitHub for Mac</a></dd>
        <dd><a href="http://windows.github.com/">GitHub for Windows</a></dd>
        <dd><a href="http://eclipse.github.com/">GitHub for Eclipse</a></dd>
        <dd><a href="http://mobile.github.com/">GitHub mobile apps</a></dd>
      </dl>

      <dl class="footer_nav">
        <dt>Services</dt>
        <dd><a href="http://get.gaug.es/">Gauges: Web analytics</a></dd>
        <dd><a href="http://speakerdeck.com">Speaker Deck: Presentations</a></dd>
        <dd><a href="https://gist.github.com">Gist: Code snippets</a></dd>
        <dd><a href="http://jobs.github.com/">Job board</a></dd>
      </dl>

      <dl class="footer_nav">
        <dt>Documentation</dt>
        <dd><a href="http://help.github.com/">GitHub Help</a></dd>
        <dd><a href="http://developer.github.com/">Developer API</a></dd>
        <dd><a href="http://github.github.com/github-flavored-markdown/">GitHub Flavored Markdown</a></dd>
        <dd><a href="http://pages.github.com/">GitHub Pages</a></dd>
      </dl>

      <dl class="footer_nav">
        <dt>More</dt>
        <dd><a href="http://training.github.com/">Training</a></dd>
        <dd><a href="https://github.com/edu">Students &amp; teachers</a></dd>
        <dd><a href="http://shop.github.com">The Shop</a></dd>
        <dd><a href="/plans">Plans &amp; pricing</a></dd>
        <dd><a href="http://octodex.github.com/">The Octodex</a></dd>
      </dl>

      <hr class="footer-divider">


    <p class="right">&copy; 2013 <span title="0.21112s from fe13.rs.github.com">GitHub</span>, Inc. All rights reserved.</p>
    <a class="left" href="https://github.com/">
      <span class="mega-icon mega-icon-invertocat"></span>
    </a>
    <ul id="legal">
        <li><a href="https://github.com/site/terms">Terms of Service</a></li>
        <li><a href="https://github.com/site/privacy">Privacy</a></li>
        <li><a href="https://github.com/security">Security</a></li>
    </ul>

  </div><!-- /.container -->

</div><!-- /.#footer -->


    <div class="fullscreen-overlay js-fullscreen-overlay" id="fullscreen_overlay">
  <div class="fullscreen-container js-fullscreen-container">
    <div class="textarea-wrap">
      <textarea name="fullscreen-contents" id="fullscreen-contents" class="js-fullscreen-contents" placeholder="" data-suggester="fullscreen_suggester"></textarea>
          <div class="suggester-container">
              <div class="suggester fullscreen-suggester js-navigation-container" id="fullscreen_suggester"
                 data-url="/geneura/papers/suggestions/commit">
              </div>
          </div>
    </div>
  </div>
  <div class="fullscreen-sidebar">
    <a href="#" class="exit-fullscreen js-exit-fullscreen tooltipped leftwards" title="Exit Zen Mode">
      <span class="mega-icon mega-icon-normalscreen"></span>
    </a>
    <a href="#" class="theme-switcher js-theme-switcher tooltipped leftwards"
      title="Switch themes">
      <span class="mini-icon mini-icon-brightness"></span>
    </a>
  </div>
</div>



    <div id="ajax-error-message" class="flash flash-error">
      <span class="mini-icon mini-icon-exclamation"></span>
      Something went wrong with that request. Please try again.
      <a href="#" class="mini-icon mini-icon-remove-close ajax-error-dismiss"></a>
    </div>

    
    
    <span id='server_response_time' data-time='0.21168' data-host='fe13'></span>
    
  </body>
</html>

