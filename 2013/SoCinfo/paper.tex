\documentclass{llncs}
\usepackage[T1]{fontenc}
\usepackage[utf8]{inputenc}
\usepackage[spanish]{babel}
\usepackage{graphicx}
\usepackage{graphics}
\DeclareGraphicsExtensions{.eps,.ps,.pdf}
\usepackage{url}
\usepackage{lscape}
\usepackage{eurosym}
 
\def\CC{{C\hspace{-.05em}\raisebox{.4ex}{\tiny\bf ++}}~}
\addtolength{\textfloatsep}{-0.5cm}
\addtolength{\intextsep}{-0.5cm}

%%%%%%%%%%%%%%%%%%%%%%%%%%%%%%%%%%%%%%%%%%%%%%%%%%%%%%%%%%%%%%%%%%%%%%%%%
\title{Studying individualised transit indicators using a new low-cost information system}  
%Obtención de indicadores de exposición mediante un nuevo sistema de información de bajo coste
%Estudio de los indicadores de exposición en el área metropolitana de Granada mediante un nuevo sistema de información de bajo coste

\author {
P.A. Castillo, P. García-Fernández, P. García-Sánchez, \\ M.G. Arenas, A.M. Mora, G. Romero, J.J. Merelo
}
\institute{
Department of Architecture and Computer Technology. CITIC \\
University of Granada (Spain). {\tt pacv@ugr.es}
}


\date{}
\begin{document}
\renewcommand{\tablename}{Table }
\renewcommand{\figurename}{Figure }
\maketitle

\begin{abstract}

A new low-cost information system to monitor the traffic in real-time is proposed.
%Current information systems used for data collection and to generate information on the state of the roads have two drawbacks: the first is that they have no ability to identify and target the vehicles detected. The second is their high cost, which makes them expensive to cover the secondary road network, so they are usually located just on main routes.
This system is based on scanning bluetooth devices that are near the detection node. 
A large amounts of data from passes of bluetooth devices by different nodes (movements or displacements) have been collected. 
Thus we can determine the frequency of appearance, calculate speeds between nodes, or calculate the number of devices that pass certain site each day (on both working or non-working days).
The analysis of collected data has given statistics and indicators about the use of vehicles by the population of the monitored area.
Specifically, we have obtained information about the total number of vehicles detected by each node, on weekdays or holidays, information on traffic density by time range on individual movements, and the average speed on a section delimited by two consecutive nodes.

{\bf Keywords}: traffic, new technologies, bluetooth, monitoring

\end{abstract}


%\newpage


%%%%%%%%%%%%%%%%%%%%%%%%%%%%%%%%%%%%%%%%%%%%%%%%%%%%%%%%%%%%%%%%%%%%%%%%%
\section{Introduction}

Having a system of information on traffic conditions and the use of roads by vehicles seems key in the current context. 
With a population increasingly informed, provided with communication devices ubiquitous commonly used about $ 90 \% $ of the population, obtaining information about the traffic would mean to optimally manage a communications network vital for a high percentage of users.

Current technologies used in traffic monitoring include pneumatic tubes, loop detectors, floating vehicles or automatic recognition systems, among others. The main disadvantage of these systems is that they are unable to identify vehicles detected, in order to obtain origin/destination matrixes. Just the number of vehicles and their type can be calculated, but does not allow to obtain moves flow, nor to determine whether a certain vehicle passes repeatedly. In addition, its high cost makes it unprofitable covering secundary roads with them, so they are often located on major roads. Moreover, technologies based on video image detection are very costly compared to the previous and can be sensitive to meteorological conditions.

This work presents a new low-cost information system to monitor traffic on different road types and in real time.

Our ultimate goal is to have information about traffic flows that occur in a certain area, allowing to optimally manage motion decisions by citizens.
Therefore, various needs from the viewpoint of the transport management have been found:

\begin{itemize}
  \item A versatile and autonomous data collection and monitoring device is needed.
  \item It is also necessary to collect traffic data in real time.
  \item Once the data has been collected, it has to be processed properly.
  \item And finally, a system that allows sharing data and information with those who make decisions about mobility is needed, both from the institutional and personal points of view.
\end{itemize}


Proposed system is based on bluetooth (BT) device discovery. 
Specifically, it catches waves emitted by different technological components incorporated on vehicles (handsfree, {GPS}), accessories that the users incorporate to their vehicles, as well as their mobile phones.
The main data that is collected is the MAC address of the device BT card.
This is an unique identifier for each device, allowing us to identify passing vehicles.
From the point of view of data privacy, it is noteworthy that the data collected is not associated to any user because there is no information that enables the identification of the information we collect with a specific person. 
Encryption technology unidirectional with nonstandard characters that preclude identifying the MAC of the wireless device is used. Thus, intrusiveness is minimal.
A large amount of data related to passing BT devices will be collected, to calculate statistics and to study several indicators about the use of vehicles by the monitored area population.


The main objective is building a low-cost system, with a fast implantation and highly reliable. 
It provides real-time information about the traffic status, not only to the official organisms and agencies in charge of the traffic controlling, but also to any person who requests it (available as web services).

Our aim is getting exposure indicators using a new system based on the BT devices detection using several collecting nodes. Thus, we are able to monitor the traffic density and car journeys, identifying the vehicles when they move from one nodo to another inside the monitored zone.

Proposed system is part of a future prediction system that helps to make decisions, and able to apply knowledge in applications related to mobility. It is expected that the development and deployment of these systems will offer a set of information services with added value that are not achieved with current technologies.



%%%%%%%%%%%%%%%%%%%%%%%%%%%%%%%%%%%%%%%%%%%%%%%%%%%%%%%%%%%%%%%%%%%%%%%%%
\section{References}

\begin{itemize}

 \item P.A. Castillo, P. García-Sánchez, A.M. Mora, M.G. Arenas, G. Romero, J.J. Merelo, P. García-Fernández, A. Romeo. Sistema de información autónomo y de bajo coste para conocer el estado de las carreteras en tiempo real. Actas de las III Jornadas de Computación Empotrada (JCE), pp.39-44, ISBN: 978-84-695-4424-2, J. Barba et al. Editores. Elche (Alicante, Spain). 2012
 
 \item P. García-Fernández, P.A. Castillo, P. García-Sánchez, M.G. Arenas, A.M. Mora, G. Romero, J.J. Merelo y V. M. Rivas. Estudio de los indicadores de exposición en el área metropolitana de Granada mediante un nuevo sistema de información de bajo coste. Congreso Internacional de Seguridad Vial, pp. 87. Santander, Spain, 2013

 \item Pedro A. Castillo, P. García-Fernández, P. García-Sánchez, M.G. Arenas, A.M. Mora, G. Romero, J.J. Merelo y V. M. Rivas. Estudio de los indicadores de exposición al riesgo mediante un sistema de monitorización del tráfico basado en la tecnología bluetooth. To appear in Actas de las IV Jornadas de Computación Empotrada (JCE - CEDI 2013). Madrid, Spain, 2013

\end{itemize}

%%%%%%%%%%%%%%%%%%%%%%%%%%%%%%%%%%%%%%%%%%%%%%%%%%%%%%%%%%%%%%%%%%%%%%%%%
\section{Contact author information}

Pedro A. Castillo

Department of Architecture and Computer Technology. CITIC

University of Granada (Spain)

email: {\tt pacv@ugr.es}



\end{document}
