\begin{table}
  \centering
   \caption{Construcciones de Clojure.}\label{cljComp}
\begin{tabular}{|p{3.4cm}|p{7cm}|}
  \hline
  % after \\: \hline or \cline{col1-col2} \cline{col3-col4} ...
  \textbf{Concepto Clojure} & \textbf{Papel} \\
     \hline
   vector & Tipo de datos para representar entes compuestos cuyas componentes sean de diferentes tipos y no varíen en el tiempo. \\
     \hline
  lista & Tipo de datos para representar entes compuestos cuyas componentes sean de igual tipo y varíen en el tiempo.\\
     \hline
  función & Relaciones entre datos, operaciones y comunicación entre agentes. \\
     \hline
  ref/agent & Unidad de ejecución, proceso. \\
     \hline
  función miembro de protocolos & Comunicación entre agentes.  \\
     \hline
  hash-map & Listado de cromosomas compartidos mediante el pool. \\
     \hline
\end{tabular}

\end{table}

\begin{table}
  \centering
  \caption{Mapeo entre conceptos de Clojure y de AGs.}\label{cljAGRelation}
\begin{tabular}{|p{3cm}|p{6cm}|}
  \hline
  % after \\: \hline or \cline{col1-col2} \cline{col3-col4} ...
  \textbf{Concepto Clojure} & \textbf{Concepto AG en el que se emplea} \\
     \hline
  vector & cromosoma evaluado \\
     \hline
  lista & cromosoma y población \\
     \hline
  función & cruzamiento, mutación y selección \\
     \hline
  agent  & isla, evaluador y reproductor \\
     \hline
  función miembro de protocolos & migración \\
     \hline
 hash-map & pool \\
     \hline
\end{tabular}

\end{table}

Los principales conceptos concurrentes y funcionales utilizados son: agente, ref, función y lista. Su uso en la implementación de la arquitectura híbrida utilizada se basó en la flexibilidad y rapidez de implementación que permitiera. 

Para los procesos independientes existentes, ya fueran las islas o los evaluadores y reproductores se emplearon agentes; para la comunicación entre ellos se usaron funciones miembro de protocolos. La lógica de funcionamiento del AG se expresó en funciones (medio funcional para la transformación de los datos) y las unidades de datos (cromosomas, poblaciones y configuraciones) se codificaron en listas y tuplas que son las estructuras de datos básicas del paradigma funcional.


