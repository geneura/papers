\begin{table}
  \centering
   \caption{Construcciones de Erlang.}\label{erlComp}
\begin{tabular}{|p{3.4cm}|p{7cm}|}
  \hline
  % after \\: \hline or \cline{col1-col2} \cline{col3-col4} ...
  \textbf{Concepto Erlang} & \textbf{Papel} \\
     \hline
  tupla & Tipo de datos para representar entes compuestos cuyas componentes sean de diferentes tipos y no varíen en el tiempo. \\
     \hline
  lista & Tipo de datos para representar entes compuestos cuyas componentes sean de igual tipo y varíen en el tiempo.\\
     \hline
  función & Relaciones entre datos, operaciones. \\
     \hline
  actor & Unidad de ejecución, proceso. \\
     \hline
  mensaje & Comunicación entre actores. \\
     \hline
  ets & Listado de cromosomas compartidos mediante el pool. \\
     \hline
  módulo random & Generación de números aleatorios.\\
  \hline
\end{tabular}

\end{table}

\begin{table}
  \centering
  \caption{Mapeo entre conceptos de Erlang y de AGs.}\label{erlAGRelation}
\begin{tabular}{|p{3cm}|p{6cm}|}
  \hline
  % after \\: \hline or \cline{col1-col2} \cline{col3-col4} ...
  \textbf{Concepto Erlang} & \textbf{Concepto AG en el que se emplea} \\
     \hline
  tupla & cromosoma evaluado \\
     \hline
  lista & cromosoma y población \\
     \hline
  función & cruzamiento, mutación y selección \\
     \hline
  actor  & isla, evaluador y reproductor \\
     \hline
  mensaje & migración \\
     \hline
  ets & pool \\
     \hline
  módulo random & naturaleza estocástica del AG \\
  \hline
\end{tabular}

\end{table}

Los principales conceptos concurrentes y funcionales utilizados son: actor, mensaje, función y lista. Su uso en la implementación de la arquitectura híbrida utilizada se basó en la flexibilidad y rapidez de implementación que permitiera. Para los procesos independientes existentes, ya fueran las islas o los evaluadores y reproductores se emplearon actores; para la comunicación entre ellos se usaron mensajes al ser estos el concepto indicado para comunicar actores. La lógica de funcionamiento del AG se expresó en funciones (medio funcional para la transformación de los datos) y las unidades de datos (cromosomas, poblaciones y configuraciones) se codificaron en listas y tuplas que son las estructuras de datos básicas del paradigma funcional.


