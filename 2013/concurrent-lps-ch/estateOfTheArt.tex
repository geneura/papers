\label{sec:fprog2ae}

Independientemente a la variedad de problemas a los que es aplicable la CE, en particular los Algoritmos Genéticos (AG), los conceptos en los que se basan son los mismos y así lo reflejan de una manera u otra las diferentes implementaciones realizadas. Lógicamente esto ha culminado en la creación de bibliotecas que encapsulan los elementos comunes según la variante arquitectónica que se necesite usar. A semejanza del resto del espectro de proyectos software, el paradigma dominante para la creación de tales bibliotecas es el Orientado a Objetos (OO).

\subsection{Herramientas existentes}

Son numerosas las herramientas desarrolladas para el trabajo con
algoritmos evolutivos; en dependencia de su objetivo soportan
diferentes plataformas, modelos de AE, integración con otros entornos
y tamaños de los problemas. Entre las más representativas se
encuentran: ECKit\footnote{http://cs.gmu.edu/~eclab/tools.html},
JDEAL\footnote{http://laseeb.isr.ist.utl.pt/sw/jdeal/home.html},
ECJ\footnote{http://cs.gmu.edu/~eclab/projects/ecj/},
ParadisEO\footnote{http://paradiseo.gforge.inria.fr/},
EASEA\footnote{https://lsiit.u-strasbg.fr/easea/index.php/EASEA\_platform}
y MALLBA\footnote{http://neo.lcc.uma.es/software/mallba/index.php}. De
estas, la más popular (al menos en número de usuarios y mención en la
literatura científica) es ECJ, escrita en Java. Como se ve, en todos
los casos se trata de lenguajes {\em tradicionales}, procedurales o,
en su mayor parte, orientados a objetos. Una más completa revisión de estos entornos se puede consultar en \cite{J.A.Parejo2011}.


\subsubsection{ECJ}

ECJ consiste en un conjunto de herramientas para el trabajo con poblaciones, posee facilidades para la optimización paralela y la optimización multiobjetivo \cite{Luke2010}. Entre otras formas de representación admite la programación genética. Está implementado en Java y soporta múltiples modelos de AG paralelos: modelo de islas y celular asíncrono, maestro-esclavo y distribución coevolutiva.

\subsubsection{ParadisEO}

Marco de trabajo orientado a objetos escrito en C++ para el diseño de metaheurísticas paralelas y distribuidas, basado en Evolving Objects \cite{Keijzer2001}, incluye soporte para algoritmos evolutivos, búsquedas locales con soporte para los patrones paralelos y distribuidos más comunes \cite{PARADISEO}. Posee soporte solamente para los modelos clásicos de AG paralelos.

% ParadisEO viene de EO, que precisamente lo iniciamos entre otro y
% yo, aunque hace tiempo que no hacemos nada.

\subsubsection{EASEA}

\emph{EAsy Specification of Evolutionary Algorithms} es una plataforma de evolución artificial masivamente paralela desarrollada por SONIC (Stochastic Optimisation and Nature Inspired Computing). Posee un lenguaje propio en el que se se describe el algoritmo: estructura de los individuos, operador de inicialización, función de aptitud y los operadores de entrecruzamiento y mutación; a partir de esto la plataforma traduce a un conjunto de ficheros C++ que pueden además ser compilados para que usen las GPUs nVidia. Solo soporta el modelo de islas de pGA.

\subsubsection{MALLBA}

MALLBA consiste en una biblioteca integrada de plantillas para optimización combinatoria: técnicas exactas, heurísticas e híbridas. Soporta ejecución tanto en ambientes secuenciales como paralelos, teniendo en cuenta el uso tanto de redes LANs como WANs \cite{MALLBA}. Está implementada en C++ e incluye las versiones distribuidas y celulares de AG así como las técnicas CHC y $\mu$CHC. 