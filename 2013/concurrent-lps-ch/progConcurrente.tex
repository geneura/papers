
El paradigma de programación concurrente, o como lo acuñara Joe Armstrong en \cite{Armstrong2003}: la {\em programación orientada a concurrencia} se caracteriza, en los lenguajes que la soportan, por la presencia de construcciones para el procesamiento de procesos (unidades de ejecución de código) que les dan el mismo tratamiento que a cualquier otro dato.

A la hora de desarrollar aplicaciones concurrentes el problema mayor a resolver es el de la comunicación entre procesos. Entre
los esfuerzos por formalizar y simplificar dicha actividad se encuentra el {\em Communicating Sequential Processes} de Hoare \cite{Hoare:1978:CSP:359576.359585}, dicho lenguaje de descripción de patrones de interacción ha servido de base para la creación de nuevos lenguajes de programación o librerías.


To be continued...