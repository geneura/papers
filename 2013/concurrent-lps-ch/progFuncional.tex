El paradigma de la programación funcional por su parte, aún cuando ofrece varias ventajas, no ha sido muy usado. Un tiempo atrás se exploraron en el campo de la Programación Genética \cite{Briggs:2008:FGP:1375341.1375345,Huelsbergen:1996:TSE:1595536.1595579,walsh:1999:AFSFESIHLP}, más recientemente en la neuroevolución  \cite{Sher2013}; sin embargo dentro de los AG ha sido poca su presencia \cite{Hawkins:2001:GFG:872017.872197}.

La programación funcional se caracteriza por el uso de las funciones como datos (pasándolas por parámetros y devolviéndolas como resultados), en particular de las funciones puras: aquellas cuyo resultado solo depende de los parámetros de entrada, excluyendo los cambios de estado. Esto la hace particularmente adecuada para el desarrollo de algoritmos concurrentes pues estos tienen la primera fuente de errores y complejidad en la comunicación entre procesos, a través de cambios de estado.

El uso de listas, con implementaciones muy eficientes, es omnipresente en la programación funcional; en los AGs por su parte es una de las estructuras de datos más utilizadas, lo cual facilita el proceso de implementación de los diversos modelos de algoritmos evolutivos.

