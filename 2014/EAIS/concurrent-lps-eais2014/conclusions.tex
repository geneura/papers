This work shows the simplicity of the modeling and implementation of a hybrid parallel genetic algorithm in three different concurrent-functional languages. Most of the developed code is open, under AGPL license, at \url{https://github.com/jalbertcruz/}. In particular we described a canonical island-based GA implementation in Scala to show its simplicity using the built-in concurrent concepts.

Erlang and Clojure are languages that encourage a \emph{non mutable state}-\emph{all functional} programming style with advantages in the design and correction of the algorithms. The protocols of Clojure allow the principles of OO without the complications of inheritance; its concurrent concepts  are specialized and flexible at the same time. The Scala language is multi-paradigm and hybrid in relation with the computation models supported. When a shared data structure is needed this language allows a more direct access and that could be an advantage, although this has not been shown in our experiments through the scaling capability.

Among the new trends in pGAs are new parallel platforms, the new languages with built-in concurrent abstractions are parallel platforms too, and their use for developing pGAs can be a very good approach for new GA developments. The functional side, which is present in all of them, is a key component to compose software components and simplifying the communication strategies among concurrent activities. In the pGA model used in this work the chosen GA architecture is concurrent-rich but the implementation remains simple thanks of the high level of abstraction of the implementation technologies.

Our experiments show that the performance of Scala is the best and point to Erlang as a very scalable runtime.

In order to complete the methodology we plan to study others concurrent oriented languages such Go, Haskell, and F\# as well as going deeper in other concurrent features of the already studied languages.

We are also planning to enrich the experiments with more complex cases of study and to test the libraries in heterogeneous hardware in order to check the scalability of each language.

