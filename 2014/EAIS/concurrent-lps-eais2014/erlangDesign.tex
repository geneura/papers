
The main concurrent concepts are \textbf{actor} and \textbf{message}, while the functional ones are \textbf{function} and \textbf{list}.

We propose to use \textbf{actors} (the execution units of the language) for independent processes: islands or evaluators/reproducers; for the communication among the actors we use \textbf{messages}, which is the concept available in the pattern for that aim.

To express the pGA’s logic we propose to use the functions (functional features for data transformation and computation expression). For the data model we propose use lists and tuples (the basic data structures in the functional paradigm).


\begin{table}[h!]
  \centering
   \caption{Erlang constructions.}\label{erlConstructions}
\begin{tabular}{|>{\centering}p{2.6cm}|p{5cm}|}
  \hline
  % after \tabularnewline: \hline or \cline{col1-col2} \cline{col3-col4} ...
  \textbf{Erlang Concept} & \textbf{Role} \tabularnewline
     \hline
  tuple & Data structure for immutable compound data. \tabularnewline
     \hline
  list & Sequence data structure for variable length compound data. \tabularnewline
     \hline
  function & Data relations, operations. \tabularnewline
     \hline
  actor & Execution unit, process. \tabularnewline
     \hline
  message & Communication among actors. \tabularnewline
     \hline
  {\em ets} & Set of chromosome shared by the pool. \tabularnewline
     \hline
  {\em random} module& Random number generation. \tabularnewline
  \hline
\end{tabular}

\end{table}

\begin{table}
  \centering
  \caption{Erlang/AG concepts mapping.}\label{erlAGRelation}
\begin{tabular}{|>{\centering}p{2.6cm}|p{5cm}|}
  \hline
  \textbf{Erlang concept} & \textbf{AG concept mapping} \tabularnewline
     \hline
  tuple & evaluated chromosome \tabularnewline
     \hline
  list & chromosomes and populations \tabularnewline
     \hline
  function & crossover, mutation and selection \tabularnewline
     \hline
  actor  & island, evaluator and reproducer \tabularnewline
     \hline
  message & migration \tabularnewline
     \hline
  {\em ets}  & pool \tabularnewline
     \hline

\end{tabular}

\end{table}


