
\IEEEPARstart{G}{enetic} algorithms (GA) \cite{GA_Goldberg89} are currently one of the most used meta-heuristics to solve engineering problems. Furthermore, parallel genetic algorithms (pGAs) are useful to find  solutions of complex optimizations problems in adequate times \cite{Luque2011}; in particular, problems with complex fitness. Some authors \cite{Alba2001} state that using pGAs improves the quality of solutions in terms of the number of evaluations needed to find one. This reason, together with the improvement in evaluation time brought by the simultaneous running in several nodes, have made parallel and distributed evolutionary algorithms a popular methodology.

Running evolutionary algorithms in parallel is quite straightforward, but programming paradigms used for the implementation of such algorithms is far from being an object of study. Object oriented or procedural languages like Java and C/C++ are mostly used. Even when some researchers show that implementation matters \cite{DBLP:conf/iwann/MereloRACML11}, parallels approaches in new languages/paradigms is not normally seen as a land for scientific improvements.

New parallel platforms have been identified as new trends in pGAs \cite{Luque2011}, however only hardware is considered. Software platforms, specifically programming languages, remain poorly explored; only Ada \cite{Santos2002} and Erlang \cite{A.Bienz2011,Kerdprasop2013} were slightly tested.

%The multicore’s challenge \cite{SutterL05} shows a current need for making parallel even the simplest program. But this way leads us to use and create design patterns for parallel algorithms; the conversion of a pattern into a language feature is a common practice in the programming languages domain, and sometimes that means a language modification, others the creation of a new one.

This work explores the advantages of some non mainstream languages (not included in the top ten of any most popular languages ranking) with concurrent and functional features in order to develop GAs in its parallel versions. It is motivated by the lack of community attention on the subject and the belief that using concepts that simplify the modeling and implementation of such algorithms might promote their use in research  and in practice.

This research is intended to show some possible areas of improvement on architecture and engineering best practices for concurrent-functional paradigms, as was made for Object Oriented Programming languages \cite{EO:FEA2000}, by focusing on pGAs as a domain of application and describing how their principal traits can be modeled by means of concurrent-functional languages constructs. We are continuing the research reported in \cite{DBLP:conf/gecco/CruzGGC13,J.Albert-Cruz2013}.

The rest of the paper is organized as follows. Next section presents the state of the art in concurrent and functional programming language paradigms and its potential use for implementing pGAs. Our proposal to adapt pGAs to the paradigms using a study case is explained in section \ref{sec:impl} as well as the experimental results in section \ref{sec:results}. In section \ref{sec:sample} we show a sample program of canonical island/GA implemented in Scala.

Finally, we draw the conclusions and future lines of work in section \ref{sec:conclusions}.

