In order to highlight the simplicity of the code based in the built in concurrent concepts of the selected languages. We will describe a canonical island-based GA implementation in Scala. We take and island-based GA implementation because is simpler parallel GA model than  hybrid pGAs we use for time evaluations.

Scala inherit keywords from several languages to express programming concepts: from Java take {\em extends} for express inheritance and from Python/Ruby take {\em def} to define methods. In order to help the type inference they have Pascal-like syntax for declaring types.

The listing \ref{island} shows a class declaration: an actor class.

\lstset{language=scala,caption={Actor declaration.},label=island, basicstyle=\ttfamily\footnotesize}
\begin{lstlisting}[frame=none]
class Island extends Actor {
  // Set of actors (workers)
  var workers: Set[ActorRef] = _
  def receive = {
     case 'start =>
      // All executing units to work!
      workers.forEach(_ ! 'start)
  }
}
\end{lstlisting}

{\em Island} is an {\em actor}, it have a set of others actors (the workers reproducers and evaluators). All actors classes should have a {\em receive} method, this block of code is compound of a list of case instructions: one for each kind of message the actor is able to respond. In this method there is only one case: for send a {\em 'start} message to each worker. All actors have a method of name {\em !} to send messages to them.

\lstset{language=scala,caption={Functional processing of data.},label=reproducer}
\begin{lstlisting}[frame=none]
// One of the worker classes
class Reproducer extends Actor {
  def receive = {
    case ('evolve, pool: HashMap, n: Int) =>
      val pop = pool.filter((a: (List,
               (Int, Int))) => a._2._2 == 2).
               keys.toList.map(i =>
                            (i, pool(i)._1))
      val (res, resultData) =
        Reproducer.evolve(
          Reproducer.extractSubpop(pop, n),
          parentsCount = n / 2 })
        // Continue the iteration with
        // res and resultData
 }
}
\end{lstlisting}

The listing \ref{reproducer} shows the class for the reproducers, in this case the message processed is composed of a tuple of 3 elements. The first statement apply a filter and a transformation (method {\em map}) over the pool of individuals.

\lstset{language=scala,caption={Main code.},label=main}
\begin{lstlisting}[frame=none]
// Creating 4 islands
val islands = for(_ <- 1 to 4)
              yield sys.actorOf(Props[Island])
		
// Puting the migrants destination & start
// each island
for(i <- 0 to 3){
	islands(i) ! ('migrantsDest,
	              islands((i+1)%4))
	islands(i) ! 'start
}
\end{lstlisting}

Finally the listing \ref{main} create islands and start the evolutions.  