
The concurrent programming paradigm (or concurrent oriented programming \cite{Armstrong2003}) is characterized by the presence of programming constructs for managing processes like first class objects. That is, with operators for acting upon them and the possibility of using them like parameters or function's result values. This simplifies the coding of concurrent algorithms due to the direct mapping between patterns of communications and processes with language expressions.

Concurrent programming is hard for many reasons, the communication/synchronization between processes is key in the design of such algorithms. One of the best efforts to formalize and simplify that is the Hoare’s {\em Communicating Sequential Processes} \cite{Hoare:1978:CSP:359576.359585}, this interaction description language is the theoretical support for many libraries and new programming languages.

When a concurrent programming language is used normally it has a particular way of handling units of execution, being independent of the operation system has several advantages: one program in those languages will work the same way on different operating systems. Also they can efficiently manage a lot of processes even on a mono-processor machine.
