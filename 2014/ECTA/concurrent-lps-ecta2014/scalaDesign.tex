
Scala is a programming language with the same concurrent programming pattern (actors) as Erlang. In the Scala implementation we followed the same criteria utilized in Erlang but with differences for its object support and JVM dependence.

\begin{table*}
  \centering
   \caption{Scala concepts.}\label{sclConstructions}
%\begin{tabular}{|>{\centering}p{2.6cm}|p{5cm}|}
\begin{tabular}{|>{\centering}p{3.4cm}|p{7cm}|}
  \hline
  % after \tabularnewline: \hline or \cline{col1-col2} \cline{col3-col4} ...
  \textbf{Scala concepts} & \textbf{Role} \tabularnewline
  tuple & Data structure for immutable compound data. \tabularnewline
    \hline
 list & Sequence data structure for variable length compound data.
 \tabularnewline
    \hline
 function & Data relations, operations. \tabularnewline
     \hline
    Akka's actor & Execution unit, process. \tabularnewline
     \hline
  symbol/message & Communication among actors. \tabularnewline
     \hline
  {\em HashMap} & Set of chromosome shared by the pool. \tabularnewline
     \hline
\end{tabular}

\end{table*}

\begin{table*}
  \centering
  \caption{Scala/AG concepts mapping.}\label{sclAGRelation}
%\begin{tabular}{|>{\centering}p{2.6cm}|p{5cm}|}
\begin{tabular}{|>{\centering}p{3cm}|p{6cm}|}
  \hline
  \textbf{Scala concept} & \textbf{AG concept mapping} \tabularnewline
  \hline
   tuple & evaluated chromosome \tabularnewline
    \hline
 list & chromosomes and populations \tabularnewline
    \hline
 function & crossover, mutation and selection \tabularnewline
    \hline
  Akka's actor & island, evaluator and reproducer \tabularnewline
     \hline
  symbol/message & migration \tabularnewline
     \hline
  {\em HashMap} & pool \tabularnewline
     \hline
\end{tabular}

\end{table*}






