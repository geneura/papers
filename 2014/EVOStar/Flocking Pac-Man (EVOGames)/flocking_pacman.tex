\documentclass[a4paper]{llncs}
\usepackage[latin1]{inputenc}
\usepackage{color}
\usepackage[english]{babel}
\usepackage[pdftex]{graphicx}
\usepackage{epstopdf}
\usepackage{amsfonts}
\usepackage{algorithm,algorithmic}
\addtolength{\intextsep}{-5mm}

\usepackage{url}
\urldef{\mailsa}\path|federico.liberatore@urjc.es|
\urldef{\mailsb}\path|{amorag, pacv, jmerelo}@geneura.ugr.es|
\newcommand{\keywords}[1]{\par\addvspace\baselineskip
\noindent\keywordname\enspace\ignorespaces#1}

\begin{document}
\mainmatter

%
%     %%%%%%%%%%   TITLE   %%%%%%%%%%
%

\title{Evolving Evil: Optimizing Flocking Strategies through Genetic Algorithms for the Ghost Team in the Game of Ms. Pac-Man}
\titlerunning{Flocking Strategies and Genetic Algorithms in the Game of Ms. Pac-Man}

%
%     %%%%%%%%%%   AUTHORS   %%%%%%%%%%
%

\author{F. Liberatore, A.M. Mora, P.A. Castillo, J.J. Merelo}
\authorrunning{F. Liberatore, A.M. Mora, P.A. Castillo, J.J. Merelo}
\institute{
Departamento de Arquitectura y Tecnolog\'ia de Computadores. \\
CITIC-UGR, ETSIIT. \\
University of Granada, Spain \\
\mailsa\\
\mailsb\\
}

\maketitle

%
%     %%%%%%%%%%   ABSTRACT   %%%%%%%%%%
%
\begin{abstract} 
Flocking strategies are sets of behavior rules for the interaction of agents that allow to devise controllers with reduced complexity that generate emerging behavior. In this paper, we present an application of genetic algorithms and flocking strategies to control the Ghost Team in the game Ms. Pac-Man. In particular, we define flocking strategies for the Ghost Team and optimize them for robustness with respect to the stochastic elements of the game and effectivity against different possible opponents by means of genetic algorithm. The performance of the methodology proposed is tested and compared with that of other standard controllers. The results show that flocking strategies are capable of modeling complex behaviors and produce effective and challenging agents.
\keywords{Flocking Strategies, Genetic Algorithms, Artificial Intelligence, Ms. Pac-Man, Videogames, Evolutionary Computation.}
\end{abstract}

%
%     %%%%%%%%%%   INTRODUCTION   %%%%%%%%%%
%
\section{Introduction}
\label{sec:intro}

The game of Ms. Pac-Man was released in 1981 and, albeit similar to the original Pac-Man, it features a female protagonist, new maze designs, and several other gameplay changes over the original game, such as a stochastic event that reverses the direction of movement of the ghosts. Videogames such as Ms. Pac-Man are an ideal testbed for computational intelligence (CI) methods as they allow for the confrontation of multiple intelligent agents in a simple, yet challenging, context. A Ms. Pac-Man vs Ghosts competition has been run since 2009 \cite{Lucas2009}. Participants can submit controllers for either Ms. Pac-Man or the Ghost Team. During the competition, the controllers are ranked according to the results of random matches between two controllers of the same kind (e.g., Ghosts controllers) against two other controllers of the opposite kind (e.g., Ms.Pac-Man controllers). The controllers of each type that get the best score win the match and increase their rank. The competition is won by the controller of each kind having the highest rank.

As the Ghost Team is a group of individuals that perform simple actions (i.e., moving up, down, left, or right), it seems a natural proving ground for algorithms based on the paradigm of Swarm Intelligence (SI). For this reason, in this work an evolved controller for the Ghost Team based on Genetic Algorithms (GAs) and Flocking Strategies (FSs) is proposed. FSs \cite{Reynolds87} consist of simple rules for the interactions of the agents that determine the next move of a ghost according to its distance to the other agents in the game. Despite their simplicity, FSs often result in complex emerging behaviors which, in turn, might result in a performance improvement or lead to a more entertaining gameplay. GAs are used to design FSs for the Ghost Team that are effective at minimizing Ms. Pac-Man final score and that are also robust with respect to the stochastic elements of the game. To the best of the authors knowledge this is the first work to actually applying flocking algorithms to the game of Pac-Man. Our objective is to understand how the proposed methodology would perform in comparison to controllers that use different approaches.

%The rest of the paper is organized as follows. The next section presents the Ms. Pac-Man problem description, including the competition features and restrictions. Some preliminary concepts and previous work in the area are commented in Section \ref{sec:preliminaryconcepts_sota}. Subsequently, Section \ref{sec:ghosts_ai} introduces the ghosts' AI approach which will be analyzed in the paper. Then the set of experiments conducted to perform this analysis is presented in Section \ref{sec:experiments}. Finally, Section \ref{sec:conclusions} describes the reached conclusions and outlines some future lines of work.

%
%     %%%%%%%%%%   PROBLEM DESCRIPTION   %%%%%%%%%%
%

\section{Ms. Pac-Man. The Game and the Problem}
\label{sec:mspacman_problem}

The game of Pac-Man needs no presentation. Since its release in 1980 many variants have been proposed and Ms. Pac-Man was one of them. Released in 1981, Ms. Pac-Man presented several features that extended on the original game, such as a female character, new maze designs, and several game-play changes.

In this game, Ms. Pac-Man has to collect all the pills in the maze while avoiding the four ghosts chasing her. If Ms. Pac-Man is touched by a ghost the player loses one life, Ms. Pac-Man is relocated at the initial position, and the ghosts respawn from the center of the maze. The {\em powerpills} turn the ghosts vulnerable for a short time, allowing Ms. Pac-Man to ``eat'' them. When a ghost gets eaten, it disappears from the game and respawns at the center of the maze after a certain amount of time. As the levels are cleared, the game becomes more difficult by changing certain parameters such as respawn time, length of time the ghosts are vulnerable, and ghosts' speed. Differently from the game of Pac-Man, this game has elements of randomness, firstly included to make the game more engaging. In fact, occasionally there is a global reversal event when all the ghosts suddenly change direction.

Given its multiple challenges, the game has been chosen for the Ms. Pac-Man vs Ghosts competition, a game AI competition where participants can submit controllers for both Ms. Pac-Man and the Ghost Team. The aim of Ms. Pac-Man agents is to maximize the final score, while the aim of Ghost Team controllers is to minimize it. The version of the game implemented for the competitions differs slightly from the original one. A thorough description of the game rules can be found in \cite{MsPacManVSGhost2011}. For the purposes of this work, the relevant restrictions for the Ghost Team are briefly enlisted in the following:
\begin{itemize}
  \item A ghost can never stop and, when it is in a corridor, it can only move forward.
  \item A ghost can choose its direction only at a junction. Specifically, a ghost can only move into a corridor different from the one it is coming from. As a result, a ghost cannot turn back on itself.
  \item Every time a ghost is at a junction the controller has to provide a direction (i.e., UP, DOWN, LEFT, or RIGHT) from the set of feasible directions, i.e., those directions corresponding to corridors different from the one the ghost is coming from. If no direction or an unfeasible direction is returned by the controller, the game framework chooses a random direction from the set of feasible directions.
  \item At every tick of the game all the ghosts obligatorily reverse their direction according to a small random probability, set in the game implementation to 0.005.
  \item After 2000 game ticks, a level is considered completed: Ms. Pac-Man is rewarded with the points of the remaining pills and the game moves on to the next level.
\end{itemize}

%
%     %%%%%%%%%%   BACKGROUND   %%%%%%%%%%
%
\section{Background and State of the Art}
\label{sec:preliminaryconcepts_sota}

In this section, the main techniques applied in the development of this work are briefly described, along with some relevant bibliography.

SI \cite{BeniWang89} is the term used to describe the type of coordinated intelligence that arises from the collective behavior of decentralized, self-organized systems, either natural or artificial. SI techniques have been widely used in many fields including medicine, robotics, defense, astronomy, optimization, telecommunication, art, cinematography, and videogames. Flocking refers to a SI technique proposed by Reynolds \cite{Reynolds87} for the coordinated movement of multiple AI agents. Originally, flocking algorithms have been developed to mimic lifelike behaviors of groups of beings such as herds of animals and schools of fishes. A flocking system typically consists of a population of simple \textit{agents} (or \textit{boids}) interacting locally with one another depending on the distance between them. The agents follow very simple steering behaviors:

\begin{itemize}
	\item \textit{Separation} makes the agent steer away from close flock mates.
	\item \textit{Alignment} makes the agent steer toward the average heading of the flock.
	\item \textit{Cohesion} makes the agent steer toward the average position of distant flock mates.
\end{itemize} 

Despite the lack of a centralized control structure dictating how individual agents should behave, the interactions between such agents lead to the emergence of ``intelligent" global behavior, unknown to the individual agents \cite{SpectorEtAl03}. Due to this desirable property, the easiness of implementation, and the reduced computational cost, flocking algorithms have been extensively applied to many fields, such as cinematography, art, medicine, etcetera. A presentation of flocking algorithms applications in videogames can be found in \cite{Scutt02} and \cite{Rabin10}.

In the last years, a number of works regarding the Ghost Team have been proposed. Nguyen and Thawonmas \cite{Nguyen2011,Nguyen2013} presented a controller based on Monte Carlo Tree Search where the behavior of Ms. Pac-Man is simulated. Their controller won the Ms. Pac-Man Versus Ghost Team Competition held in 2011. Svensson and Johansson \cite{Svensson2012} exploited the behavior emerging capabilities of Influence Maps. A different line of research is pursued by Sombat \textit{et al.} \cite{Sombat2012} that analyzed Ms. Pac-Man matches to classify Ghost Team controllers according to their enjoyability and, therefore, understand the attribute that a NPC should posses for players to be engaged. 
In the last decade, a number of EAs have been proposed to address different aspects of the game of Pac-Man. One of the first works in the subject is the paper by Gallagher \cite{Gallagher03} that optimized rule-based fine state machines through population-based incremental learning to devise an adaptive Pac-Man agent. More recently, Galvan-Lopez \cite{Galvan-Lopez10} explored and compared the performance of two types of Grammatical Evolution (GE) mappings to generate controllers for Ms. Pac-Man. Alhejali and Lucas \cite{Alhejali10} applied Genetic Programming (GP) to evolve a diverse set of behaviors using different versions of the game. The resulting controller proved to be competitive with the best reactive controllers reported at the time. In a subsequent article \cite{AlhejaliLucas11}, the same authors extended their work by applying a ``training camp framework'' to GP, where a set of specialized behaviors is evolved according to specific training scenarios. A different approach is presented by Brandstetter and Ahmadi \cite{Brandstetter12} that proposed a GP-based controllers that relies exclusively on information retrieval terminals rather than action-issuing terminals.  Thawonmas \cite{Thawonmas10} applied a GA to optimize the parameters of the Ms. Pac-Man controller ICE Pambush 3, winner of the IEEE CEC 2009 Ms. Pac-Man competition. A number of authors made use of EAs to design neural network-based controllers, both for Ms. Pac-Man \cite{Lucas05,Burrow09} and the ghosts \cite{Jia-Yue11}. Gagne and Congdon \cite{Gagne2012} evolved rule-based intelligent agent for the ghost team. Finally, Cardona \textit{et al.} \cite{Cardona13} explored competitive co-evolution techniques to generate at the same time optimal Ms. Pac-Man and Ghost Team controllers.

In this work, an offline GA is applied to a flocking model for the Ghost Team, in order to improve its decision engine, which will be used later during game. To the best of the authors' knowledge, this is the first time FS have been applied to the game of Ms. Pac-Man.

%
%     %%%%%%%%%%   Ghost Team AI   %%%%%%%%%%
%

\section{Ghost Team AI: Evolutionary Flocking}
\label{sec:ghosts_ai}
In this section the evolutionary FS model developed for designing controllers for the Ghost Team is described. In the flocking system described, each one of the four ghosts is an independent agent. Nevertheless, all the ghosts determine their movement according to the same FS, as explained in the following.

\subsection{Generalized Flocking Strategies}
\label{subsec:flocking_strategies}
We define a Flocking Rule (FR) for boids (ghosts, in this case), $\phi$, as a set of two vectors, $\phi^d$ and $\phi^m$, that jointly describe the steering behavior of a ghost under certain conditions. Each FR considers a number $N$ of concentric ring-shaped neighborhoods centered on the ghost. The limits of each neighborhood are specified by vector $\phi^d \in \mathbb{N}^N$; please note that the elements of vector $\phi^d$ are always sorted in ascending order as they represent the radii of the concentric neighborhoods . The last element of this vector, $\phi_N^d$, is set to $\infty$ by default to cover the whole space. Vector $\phi^m \in [-1, 1]^N$ defines the magnitude of the steering force applied on the ghost when an agent falls into one of the neighborhoods.

To find the steering force on an agent A, located at position $v_a$, resulting from the interaction with agent B, located at position $v_b$, the difference vector $v_\delta$ and the Euclidean distance $\delta$ between the two agents are calculated:
\begin{small}
\begin{equation}
	\label{eq:v_delta}
	v_\delta=v_b-v_a
\end{equation}
\begin{equation}
	\label{eq:delta}
	\delta=\|v_\delta\|
\end{equation}
\end{small}
To identify the magnitude $\phi^m_N$ we need to determine the neighborhood $0<n\leq N$ where agent B belongs, by applying the following condition:
\begin{small}
\begin{equation}
	\label{eq:neighborhood}
	\phi^d_{n-1} < \delta \leq \phi^d_{n}
\end{equation}
\end{small}
where $\phi^d_0 = 0$. Finally, the steering force on agent A resulting from the interaction with agent B is given by:
\begin{small}
\begin{equation}
	\label{eq:force}
	f^{B} = \phi^m_n \cdot \frac{v_\delta}{\delta}
\end{equation}
\end{small}
%As an example, let us consider a scenario with two ghosts, Ghost A located at position $v_a=(1,0)$ and Ghost B located at position $v_b=(5,3)$, and the following flocking rule for Ghost A:

%$$\phi^d = \{10, 20, \infty\}$$
%$$\phi^m = \{-0.5, 0, 0.3\}$$

%This sample rule considers only three neighborhoods ($N=3$). 

% or neighborhood radius = 3 ? - JJ
%Federico : Los radius son los especificados en \phi^d. N dice cuantas zonas se consideran. Quizas esta parte necesite una imagen explicativa...

%We want to find the steering force on Ghost A resulting from the interaction with Ghost B. In this example, $v_\delta=(4,3)$ and $\delta = 5$. As $\delta$ is less than $10$, Ghost B falls into the first neighborhood ($n=1$) and the magnitude $\phi^m_n$ of the steering behavior is the first element of vector $\phi^m$, $\phi_1^m=-0.5$. As the magnitude is negative, the ghosts repel each other. If $delta\in (10,20]$ the magnitude would have been $0$ (no interaction), while if $\delta > 20$ the magnitude would have been $0.3$ (attraction). We can finally compute the steering force on Ghost A due to Ghost B: $f^{B} = (-0.4, -0.3)$.

%From the example it can be easily seen that a negative magnitude corresponds to the behavior of separation, while a positive magnitude corresponds to the behavior of cohesion. No alignment behavior is included in this strategy model for the Ghost Team as it would make the controllers very predictable.

A negative magnitude corresponds to the behavior of separation, while a positive magnitude corresponds to the behavior of cohesion. No alignment behavior is included in this strategy model as it would make the ghosts very predictable.

Differently from the basic flocking algorithm where only one type of agent is considered, in the game Ms. Pac-Man a variety of different actors are present. Also, the ghosts can be in different states (e.g., normal or edible). To be effective, a strategy for the Ghost Team should take into account at least all the elements presented to the player on the screen. Let $S$ be the set of possible ghost states:
\begin{small}
\begin{equation}
S=\{\mathrm{HUNTER}, \mathrm{HUNTED}, \mathrm{BLINKING}\}
\end{equation}
\end{small}
$HUNTER$ is the ``normal'' state of a ghost (i.e., kills Ms. Pac-Man if touched). When Ms. Pac-Man eats a powerpill all the ghosts become $HUNTED$ for a certain length of time (i.e., is killed by Ms. Pac-Man on contact). When this period is about to expire, every ghost blinks to warn the player; we call this state $BLINKING$. Let $A$ be the set of all the actors in the game:
\begin{small}
$$A=\{\mathrm{PACMAN}, \mathrm{POWERPILL}, \mathrm{HUNTER}, \mathrm{HUNTED}, \mathrm{BLINKING}\}$$
\end{small}
$HUNTER$, $HUNTED$, and $BLINKING$ refers to ghosts in that state. We can now define a Flocking Strategy (FS) for the Ghost Team, $\Phi$, as:
\begin{small}
$$\Phi : S \times A \to \phi$$
\end{small}
A FS is a function that, given a ghost state and the type of actor considered, returns the flocking rule that has to be applied to calculate the steering force on the ghost resulting from the interaction with the actor.

As explained in Section \ref{sec:mspacman_problem}, every time a ghost is at a junction the game needs to calculate its next move. The controller based on the FS provides the next move by following the steps illustrated in Algorithm \ref{alg:FS_Controller}.

\begin{algorithm}[ht]
\caption{Flocking Strategy-based Ghost Controller.}
\label{alg:FS_Controller} 
\begin{algorithmic}
\STATE $s\gets$ status of the current ghost $G$;
\STATE $v_a\gets$ position of the current ghost $G$;
\FORALL{actor $b$ in the game}
	\STATE $\phi\gets\Phi(s,b)$; \COMMENT{Determine the Flocking Rule to be applied.}
	\STATE $v_\delta\gets v_b-v_a$; \COMMENT{Calculate the difference vector (Equation \ref{eq:v_delta}).}
	\STATE $\delta\gets\|v_\delta\|$; \COMMENT{Calculate the Euclidean distance (Equation \ref{eq:delta}).}
	\STATE $n\gets n^\prime|\phi^d_{n^\prime-1} < \delta \leq phi^d_{n^\prime}$; \COMMENT{Identify the neighborhood (Equation \ref{eq:neighborhood}).}
	\STATE $f^{b}\gets\phi^m_n \cdot \frac{v_\delta}{\delta}$; \COMMENT{Compute the steering force (Equation \ref{eq:force}).}
\ENDFOR
\STATE $f\gets\sum_b f^b$; \COMMENT{Calculate the total steering force.}
\STATE \COMMENT{Translate the steering force in a ranking for the next ghost direction as follows:}
\STATE $\mathrm{UP}\gets- f_2$;
\STATE $\mathrm{DOWN}\gets f_2$;
\STATE $\mathrm{LEFT}\gets- f_1$;
\STATE $\mathrm{RIGHT}\gets f_1$;
\RETURN the feasible direction (see restrictions in Section \ref{sec:mspacman_problem}) having maximum rank;
\end{algorithmic}
\end{algorithm}


\subsection{Devising Optimized Flocking Strategies by Means of GAs}
\label{subsec:GAs_flocking}
In this work we are dealing with a two-player competitive game with stochastic elements. A FS could be manually designed by an expert with decent results. Nevertheless,given as the number of parameters and the inherent complexity of the game, it is desirable to automatize the definition of an effective strategy by means of an optimization algorithm. Given the characteristics of the problem and the reduced number of constraints involved, GAs appear to be a sensible choice. 

In the following, the elements comprising the GA implemented are described.

\subsubsection{Initial Population}
\label{par:GA_Initial}
Each individual is represented by a FS $\Phi$. The initial population is created as a random set of FSs defined as:
\begin{small}
\begin{equation}
	\label{eq:init_dist}
	\forall s \in S, a \in A,  i = 1, \ldots, N, \quad \phi^d_i \sim U(\phi^d_{i-1},\infty)
\end{equation}
\begin{equation}
	\label{eq:init_magn}
	\forall s \in S, a \in A,  i = 1, \ldots, N, \quad \phi^m_i \sim N(0,1/3), \phi^m_i \in [-1,1]
\end{equation}
\end{small}
The elements of $\phi^d$ have a uniform distribution, while the elements of $\phi^m$ have a truncated normal distribution. The parameters of the normal distribution have been set so as to generate most of the magnitudes close to zero and assign similar probabilities to the appearance of cohesion, separation, and no interaction behaviors.

\subsubsection{Fitness Function}
\label{subsubsec:GA_Fitness}
The definition of the fitness function is one of the most critical
aspects in a GA. The proposed optimization algorithm should generate
Ghost Team strategies that perform well against any possible
Ms. Pac-Man strategy and, at the same time, should be resilient to the
random ghosts reverse direction events (see Section
\ref{sec:mspacman_problem}). To achieve this result, each flocking
strategy is pitted against two different Ms. Pac-Man controllers
included in the Ms. Pac-Man vs Ghosts competition framework:
\textit{StarterPacMan} (SPM) and \textit{NearestPillPacMan}
(NPPM) (for a description of the controllers, please refer to the competition framework documentation\footnote{http://www.pacman-vs-ghosts.net/, last visited on {\today}}). The game is simulated 30 times for each Ms. Pac-Man
controller. Thanks to that we can take advantage of the central limit
theorem to compute a relatively precise 95\% confidence interval of
the final score obtained by the Ms. Pac-Man controllers. This is done to minimize the effect of noise present in this problem and in videogames in general \cite{Mora12}. In fact, due to the stochastic elements of the game, the same FS could perform very well sometimes and quite bad some others. Let $\mu_\mathrm{SPM}$, $\sigma_\mathrm{SPM}$, $\mu_\mathrm{NPPM}$, $\sigma_\mathrm{NPPM}$ be the average score obtained by controller SPM in the 30 runs, the standard deviation of the SPM's scores, the average score obtained by controller NPPM in the 30 runs, and the
standard deviation of the NPPM's scores, respectively. The upper limits of
the confidence intervals for the scores of the two controllers are:
% no son muy b�sicos estos dos controladores? �No deber�as usar alguno de los ganadores - JJ 
% Federico : si, son bastante basicos. Los controladores mas avanzados se utilizarán en una version futura del trabajo. (tristemente, ahora no tengo tiempo para hacerlo)
\begin{small}
\begin{equation}
\overline{\mathrm{CI}}_\mathrm{SPM} = \mu_\mathrm{SPM} + Z \cdot \frac{\sigma_\mathrm{SPM}}{\sqrt{30}}
\end{equation}
\begin{equation}
\overline{\mathrm{CI}}_\mathrm{NPPM} = \mu_\mathrm{NPPM} + Z \cdot \frac{\sigma_\mathrm{NPPM}}{\sqrt{30}}
\end{equation}
\end{small}
where $Z$ is the 95\% percentile of the standard normal distribution. Therefore, 95\% the Ms. Pac-Man controllers should get score below the upper limits of
the confidence intervals. Our objective is to obtain a Ghost Team controller that minimizes these two values. The fitness function is defined as the average of the confidence intervals' inverses:
\begin{small}
\begin{equation}
\mathrm{FITNESS} = \frac{1}{\overline{\mathrm{CI}}_\mathrm{SPM}} + \frac{1}{\overline{\mathrm{CI}}_\mathrm{NPPM}}
\end{equation}
\end{small}

\subsubsection{Selection, Recombination, and Mutation}
\label{subsubsec:GA_Selection_Recombination_Mutation}
After all the individuals (FSs) of the current generation have been evaluated, the offspring will be generated. For each $\Phi$ to be generated, two individuals $\Phi_1$ and $\Phi_2$ are chosen by \textit{roulette-wheel selection}  (i.e., every member of the population has a probability of being chosen proportional to its fitness). The children individual $\Phi$ is created by random recombination of the parameters of parents $\Phi_1$ and $\Phi_2$:
\begin{small}
\begin{equation}
\forall s \in S, a \in A,  i = 1, \ldots, N, \quad \phi^d_i  = \mathrm{rand}(\phi^d_i \in \Phi_1(s,a), \phi^d_i \in \Phi_2(s,a))
\end{equation}
\begin{equation}
\forall s \in S, a \in A,  i = 1, \ldots, N, \quad \phi^m_i = \mathrm{rand}(\phi^d_i \in \Phi_1(s,a), \phi^d_i \in \Phi_2(s,a))
\end{equation}
\end{small}
where \textit{rand} is a function that returns a random value chosen
among its arguments.

During the recombination, mutations can occur
with probability $p^\mathrm{mut}$. When a mutation happens, the
current parameter is re-initialized to a random value, according to
Equations \ref{eq:init_dist} and \ref{eq:init_magn}.
The mutation probability $p^\mathrm{mut}$ is determined
dynamically. Initially, its value is set to $p^\mathrm{mut} =
0.00125$. At each iteration $t$, its value changes depending on the
coefficient of variation of the current population fitness, $c_v$:
\begin{small}
\begin{equation}
c_v = \frac{\sigma^t_{FITNESS}}{|\mu^t_{FITNESS}|}
\end{equation}
\end{small}
% Por qu�? No funciona bien el GA est�ndar? Cualquier experimento de este estilo debe justificarse - JJ
% Federico : El GA estandar solia quedarse muy atascado en optimos locales. Utilizando esta estrategia de mutación se consigue salir de vez en cuando. Estoy de acuerdo que se deberia experimentar con ello.

% �esa estrategia no recuerda un poco a un simulated-annealing...?   --Pedro--
% �no se podr�a, como trabajo futuro aplicar un SA o algo as� para determinarlo?
% Federico: Sugiero utilizar Particle Swarm Optimization ya que no se queda atascado en optimos locales.
where $\mu^t_{FITNESS}$ and $\sigma^t_{FITNESS}$ are the current population fitness' average and standard deviation, respectively. $c_v$ measures the degree of variability of the population in terms of fitness. When the variability is low, we increment the mutation probability to introduce new chromosomes in the genetic pool of the population. When the variability is too big, the mutation probability is set to a low initial value:
\begin{small}
\begin{equation}
 p^\mathrm{mut}_t = \left\{
\begin{array}{l l}
	0.00125 & \quad \textrm{if $c_v > 0.6$}\\
	p^\mathrm{mut}_{t-1} & \quad \textrm {if $0.3 < c_v \leq 0.6$}\\
	2 \cdot p^\mathrm{mut}_{t-1} & \quad \textrm {if $0.2 < c_v \leq 0.3$}\\
	4 \cdot p^\mathrm{mut}_{t-1} & \quad \textrm {if $0.1 < c_v \leq 0.2$}\\
	8 \cdot p^\mathrm{mut}_{t-1} & \quad \textrm {if $c_v \leq 0.1$}\\
   \end{array} \right.
\end{equation}
\end{small}
Once the recombination is done, vectors $\phi^d_i$ and $\phi^m_i $ are sorted in ascending order with respect to the values of $\phi^m_i $ to preserve their feasibility. In fact, by definition vector $\phi^m_i $ are required to be always sorted in ascending order, and the recombination and the mutation operators might generate vectors that do not comply with this rule.

%
%     %%%%%%%%%%   EXPERIMENTS   %%%%%%%%%%
%

\section{Experiments and Results}
\label{sec:experiments}

In this section, it will be tested how well a GA evolved controller performs, compared to non-evolutionary strategies. The standard Ghost Team controllers included in the competition framework will be used as a comparative basis.  
In the experiments, the GA described in the previous chapter has been
run for 50 generations with a population of 50 candidate 
%why? Justificar siempre n�meros m�gicos o probar diferentes tama�os - JJ
%Federico : Efectivamente son numeros magicos. No los he justificado mucho porque pensaba que la parte mas importante del trabajo fuera relacionada con las flocking strategies y no el genetico, que es estandar. A ver si queda espacio para mas experimentos...
strategies. At each iteration, the next generation was constituted by
49 recombined individuals plus the best solution of the current
generation. 

All the algorithm have been implemented in Java within the framework provided for the Ms Pac-Man vs Ghosts competition. The final program run on a Intel(R) Core(TM) i5-2500K @ 3.3GHz with four cores and 4GB of RAM. Each experiment run on a single core and made use of less than 300MB of memory.

% No comparison with a standard controller? Or a baseline? - JJ
% Federico : Si, en la segunda parte de los experimentos.

The first experiment performed regards the comparison of the performance of the Ghost Team controllers obtained with different values of the parameter $N$ (i.e., the number of neighborhoods considered in the Flocking Rules). The best strategy found with a certain number of neighborhoods should be at least as good as those found with a lower number of neighborhoods, as the solution space of the former is bigger and contains the others'. Nevertheless, as the number of neighborhoods increases, the solution space increases substantially, that in turn could affect the actual performance of the GA. In this section we use the inverse fitness, $\mathrm{FITNESS}^-1$, as a measure of performance as it is easier to interpret than the fitness function that takes values close to zero. Being the inverse of the fitness value, a lower $\mathrm{FITNESS}^-1$ value corresponds to a better controller, and vice-versa. 

\begin{small}
\begin{table} [htbp]
\centering
{
\begin{tabular}{|c||c|c|c|c|c|}
\hline & $N = 1$ & $N = 2$ & $N = 3$ & $N = 4$ & $N = 5$ \\
\hline
Best $\mathrm{FITNESS}^-1$& 783.38 & 726.84 & 815.66 & 766.96 & 720.20 \\
\hline
Avg. $\mathrm{FITNESS}^-1$& 871.30$\pm$55.45 & 861.57$\pm$70.13 & 876.31$\pm$65.06 & 905.86$\pm$62.66 & 863.17$\pm$72.15 \\
\hline
Worst $\mathrm{FITNESS}^-1$ & 951.75 & 969.20 & 1,032.39 & 986.48 & 980.50 \\
\hline
Avg. CPU time (s)& 1373$\pm$150.66 & 1484.3$\pm$122.01 & 1561$\pm$193.94 & 1562.60$\pm$109.90 & 1473.00$\pm$74.02 \\
\hline
\end{tabular}}
\caption{Performance of the controllers for the Ghost Team obtained by the GA using different numbers of neighborhoods.
\label{tab:results_GA}}
\end{table}
\end{small}

Table \ref{tab:results_GA} shows the performance of the evolved controllers over 10 runs of the GA with $N=1,\ldots,5$. Each column is associated to a different number of neighborhoods. The first row displays the inverse fitness of the best individual found. The second row presents the average controllers fitness; the standard deviation is also reported after the plus-minus sign. Next, the third row illustrates the fitness value of the worst controller found. Finally, the last row reports the average optimization CPU time in seconds over the 10 runs and the corresponding standard deviation. By observing the table some conclusions can be drawn.

\begin{itemize}
	\item The best controller is obtained when $N=5$. This result suggests that increasing the complexity of the FS benefits the controller.
	\item No clear pattern can be identified in the values taken by Best $\mathrm{FITNESS}^-1$, nor Avg. $\mathrm{FITNESS}^-1$, with respect to variations in the value of $N$. This could be due to the increased complexity in the search space resulting from the high number of parameters to be optimized by the GA.
	\item The gap between the best and the worst fitness in each column and the fitness standard deviation suggest that the problem presents many local optima in the solution space. 
\end{itemize}

When playing against the best Ghost Team controller obtained by the GA it is possible to observe interesting behaviors. A sample video\footnote{https://www.youtube.com/watch?v=I9rL0jUwHhk, visited on {\today}} shows the best Ghost Team controller pitted against the Ms. Pac-Man controller \textit{StarterPacMan}, included in the competition framework. The video illustrates that, despite the lack of explicit coordination between them, the ghosts show complex strategic behaviors:

\begin{itemize}
	\item Initially, Blinky (red ghost) and Inky (blue ghost) entrap Ms. Pac-Man in a small corridor. It can be observed that Pinky (pink ghost) is not heading directly toward Ms. Pac-Man using the shortest route, therefore allowing room for alternative strategies.
	\item In the second round, Blinky chases Ms. Pac-Man and pushes her through the tunnel, at the end of which Pinky is waiting for her. It is interesting to notice that, would Ms. Pac-Man have chosen to move away from the tunnel, she would have moved toward a conveniently located Inky.
	\item In the last round, Ms. Pac-Man eats a powerpill. Immediately, the ghosts flee in the opposite direction. After getting caught, Blinky takes advantage of the situation to interpose between Ms. Pac-Man and the vulnerable ghosts. This forces Ms. Pac-Man to take a detour and waste precious time, which results in Ms. Pac-Man losing the game because of a recently invulnerable-turned Pinky.
\end{itemize} 

Without including complex rules, which is a desirable feature in this type of problems (i.e., AI in games), the proposed methodology generates emerging behaviors. This, in turn, results in the ghosts behaving in a ``intelligent'' fashion although they are not explicitly programmed with this objective in mind.

In the next experiment, we compare our controllers to the five Ghost Team controllers included in the competition framework. Their $\mathrm{FITNESS}^-1$ values, computed exactly as per the GA solutions, are illustrated in Table \ref{tab:results_controllers}.

\begin{small}
\begin{table} [htbp]
\centering
{
\begin{tabular}{|c||c|c|c|c|c|}
\hline Controller & \textit{AggressiveGhosts} & \textit{Legacy} & \textit{Legacy2TheReckoning} & \textit{RandomGhosts} & \textit{StarterGhosts} \\
\hline
$\mathrm{FITNESS}^-1$& 1893.13 & 2210.94 & 1429.20 & 4200.70 & 1603.49 \\
\hline
\end{tabular}}
\caption{Performance of the standard Ghost Team controllers included in the competition framework.
\label{tab:results_controllers}}
\end{table}
\end{small}

According to these results, the best controller is \textit{Legacy2TheReckoning}, followed by  \textit{StarterGhosts}. Nevertheless, their $\mathrm{FITNESS}^-1$ value is twice that of the best evolved FS found, approximately. These results support the claim that FSs are a viable option for the definition of intelligent controllers.
% Pero no por qu� no hab�is usado esos controladores en  el entrenamiento. �Por qu� no pones la puntuaci�n? - JJ 
% Federico : hay dos tipos de controladores: los de Ms.Pac-Man y los de los fantasmas. Los de Ms.Pac-Man se incluyen en la fitness function. Estos son controladores para los fantasmas y los usamos para verificar si los nuestros son buenos o malos.  

% Por otro lado, habr�a que describir el juego de los controladores que han evolucionado. - JJ
% Federico : Seria interesante hacerlo. A ver si al final de la revision queda espacio.

% Antonio: Otra cosa que suele gustar es comentar alguna experiencia/comparativa con jugadores humanos. Es decir, puedes contar que un 'experto' (que puedes ser tú) ha jugado con Ms Pac-Man contra tu controlador de fantasmas y que ha sudado tinta para ganarle, o que ha perdido siempre, no sé. :D

%
%     %%%%%%%%%%   CONCLUSIONS  %%%%%%%%%%
%

\section{Conclusions and Future Work}
\label{sec:conclusions}

In this paper, a new controller for the Ghost Team based on FSs is proposed. FSs are sets of behavior rules that determine the next move of an agent as a force resulting from the interaction of the agents in the game. A GA is presented to design optimized strategies offline. The fitness function evaluates each individual by pitting it against two Ms. Pac-Man controller 30 times, so as to avoid noise in the function. Parents are chosen by roulette-wheel selection and the children are generated by random recombination of the parents' chromosomes. The mutation probability is adaptive and increases when the population is homogeneous, while it decreases when the population is too heterogeneous. The methodology has been empirically tested: the fitness of the best individual found by the GA has been compared to the fitness of the five standard controller included in the competition framework. The results show that FSs model complex behaviors and that the GA successfully optimize the design of the ghosts controller, producing effective and challenging agents.

This work is just scratching the surface and there is still a lot to be investigated. Some possible future lines of research are highlighted in the following. The fitness function can be easily extended by including more Ms. Pac-Man controllers. This should result in a Ghost Team controller that performs better against a wider range of opponents. By considering in the GA fitness function the best Ms. Pac-Man controllers that took part to the competitions, it would be possible to generate Ghost Team controllers that are capable of tackling the best known Ms. Pac-Man strategies. 

Moreover, it would be interesting to compare the controllers obtained by applying the presented methodology with the best Ghost Team controllers that took part to the Ms. Pac-Man vs Ghosts competition. This would allow us to really understand the limits of FSs.

The GAs as a means to optimize FSs have proven to be satisfactory. Nevertheless, the recombination step causes abrupt changes in the solutions' parameters and might generate individuals that are very different from the initial ones. It would be interesting to investigate the effectiveness of optimization methods that allows for small changes in the solutions parameters. Particle Swarm Optimization (PSO) algorithms  might be a sensible choice. In fact, on top of making few or no assumptions about the problem, PSO algorithm are particularly effective with problems that are noisy and present many multiple optima, such as this one.

%
%     %%%%%%%%%%   ACKNOWLEDGMENTS  %%%%%%%%%%
%

\section*{Acknowledgments}
This work has been supported in part by CANUBE (CEI2013-P-14), ANYSELF (TIN2011-28627-C04-02), and the TIN2011-28627-C04-02 project, awarded by the Spanish Ministry of Science and Innovation. Liberatore's research was financed by the Government of Spain (TIN2012-32482). Also, Liberatore would like to thanks the GeNeura research group at University of Granada for their kind hospitality. All the supports are gratefully acknowledged.
%Federico : Add Acknowledgments for JJ and Pedro

\bibliographystyle{splncs03}

\bibliography{flocking_pacman}

\end{document}
