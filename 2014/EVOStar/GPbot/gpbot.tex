\documentclass[runningheads]{llncs}
\usepackage{amssymb}
\setcounter{tocdepth}{3}
\usepackage{graphicx,epsfig}
\usepackage{algorithmic}
\usepackage{listings}
\usepackage{rotating}
\usepackage{subfig}

%%%%

\usepackage{color}
\usepackage{alltt}
\usepackage{verbatim}
\usepackage{url}
\usepackage[latin1]{inputenc}
%\usepackage[spanish]{babel}

%%

\usepackage{url}
\urldef{\mailsa}\path|pgarcia@atc.ugr.es|



\newcommand{\keywords}[1]{\par\addvspace\baselineskip
\noindent\keywordname\enspace\ignorespaces#1}

\lstset{
basicstyle=\ttfamily \scriptsize,
language=c++,
frame=single,
stringstyle=\ttfamily,
showstringspaces=false
}

\begin{document}
 %\pagestyle{empty} %ESTO QUITA LOS NUMEROS DE PAGINA
\mainmatter  % start of an individual contribution



% first the title is needed
\title{Comparing different migration policies in a noisy fitness problem: an study in Evolutionary Robotics}

% a short form should be given in case it is too long for the running head
\titlerunning{MADE: A Massive BLABLABLA}
\author{R.H. Garc\'ia-Ortega\inst{1}, P. Garc\'ia-S\'anchez\inst{2} and J.J. Merelo\inst{2}}

%

\authorrunning{R.H. Garc\'ia-Ortega et al.}

% (feature abused for this document to repeat the title also on left hand pages)
% the affiliations are given next; don't give your e-mail address
% unless you accept that it will be published

\institute{Fundaci\'on I+D del Software Libre, Granada, Spain \and Dept. of Computer Architecture and Technology, University of Granada, Spain 
\mailsa}




\toctitle{BLABLABLA}
%ES ANONIMO????
\tocauthor{Authors' Instructions}
\maketitle


\begin{abstract}

This work presents the results obtained from comparing ... Results show that 

\end{abstract}


% En el GPBot tienes que hacer �nfasis en que las estrategias probadas
% hasta ahora en este juego son tan buenas como la estrategia original
% que parametrizan; GPBot te permite no s�lo conseguir estrategias
% mejores, sino descubrirlas. En los resultados tienes que hacer
% �nfasis no s�lo en lo obtenido, sino tambi�n en qu� significan las
% estrategias, c�mo juegan; tendr�s que ponerlas a jugar a ver qu� pasa.


\section{Introduction}
\noindent 

Blablalba




With the previous result we try to answer the next questions:
\begin{itemize}
\item ?
\item ?
\item ?
\end{itemize}

The rest of the work is structured as follows: after the state of the art, the developed system is presented in Section \ref{sec:made}. Then, the experiments conduced with the EA are showed (Section \ref{sec:results}). Finally, conclusions and future works are discussed.


%%%%%%%%%%%%%%%%%%%%%%%%SEC SOA
\section{State of the art}
\label{sec:soa}



\section{MADE}
\label{sec:made}

\section{Extraction of WHATEVER using Evolutionary Algorithms}

\section{Conclusions}
\label{sec:conclusion}

\bibliographystyle{splncs}
\bibliography{made}

\end{document}

