\documentclass[runningheads]{llncs}
\usepackage{amssymb}
\setcounter{tocdepth}{3}
\usepackage{graphicx,epsfig}
\usepackage{algorithmic}
\usepackage{listings}
\usepackage{rotating}
\usepackage{subfig}
\usepackage{algorithmic}

%%%%

\usepackage{color}
\usepackage{alltt}
\usepackage{verbatim}
\usepackage{url}
\usepackage[latin1]{inputenc}
%\usepackage[spanish]{babel}

%%

\usepackage{url}
\urldef{\mailsa}\path|pgarcia@atc.ugr.es|



\newcommand{\keywords}[1]{\par\addvspace\baselineskip
\noindent\keywordname\enspace\ignorespaces#1}

\lstset{
basicstyle=\ttfamily, %\scriptsize, QUITAR LA COMA DE TTFAMILY SI DESCOMENTAS
language=c++,
frame=single,
stringstyle=\ttfamily,
showstringspaces=false
}

\begin{document}
 %\pagestyle{empty} %ESTO QUITA LOS NUMEROS DE PAGINA
\mainmatter  % start of an individual contribution



% first the title is needed
\title{Difference of tree maximum depth in Genetic Programming to generate competitive agents for RTS games}

% a short form should be given in case it is too long for the running head
\titlerunning{Difference of tree maximum depth in GP to generate competitive agents for RTS games}
\author{P. Garc\'ia-S\'anchez, A. Fern\'andez-Ares, A. M. Mora, P. A. Castillo and J.J. Merelo}

%

\authorrunning{P. Garc\'ia-S\'anchez et al.}

% (feature abused for this document to repeat the title also on left hand pages)
% the affiliations are given next; don't give your e-mail address
% unless you accept that it will be published

\institute{Dept. of Computer Architecture and Technology and CITIC-UGR, University of Granada, Spain 
\mailsa}




%\toctitle{BLABLABLA}
%ES ANONIMO????
%\tocauthor{Authors' Instructions}
\maketitle


\begin{abstract}

This work presents the results obtained from comparing ... Results show that 

\end{abstract}


% En el GPBot tienes que hacer �nfasis en que las estrategias probadas
% hasta ahora en este juego son tan buenas como la estrategia original
% que parametrizan; GPBot te permite no s�lo conseguir estrategias
% mejores, sino descubrirlas. En los resultados tienes que hacer
% �nfasis no s�lo en lo obtenido, sino tambi�n en qu� significan las
% estrategias, c�mo juegan; tendr�s que ponerlas a jugar a ver qu� pasa.


\section{Introduction}
\noindent 

Blablalba




With the previous result we try to answer the next questions:
\begin{itemize}
\item ?
\item ?
\item ?
\end{itemize}

The rest of the work is structured as follows: after the state of the art, the developed system is presented in Section \ref{sec:made}. Then, the experiments conduced with the EA are showed (Section \ref{sec:results}). Finally, conclusions and future works are discussed.


%%%%%%%%%%%%%%%%%%%%%%%%SEC SOA
\section{State of the art}
\label{sec:soa}





\section{Proposed Agent}
\label{sec:agent}

The proposed agent receives a tree to be executed. The generated tree is a binary-tree of expressions formed by two different types of nodes:

\begin{itemize}
\item {\em Decission}: a logical expression formed by a variable, a less than operator ($<$), and a number between 0 and 1.
\item {\em Action}: the leaves of the the tree. Each decission calls an method that indicates to which planet send a percentage of available ships (from 0 to 1) from the planet that executes the tree.
\end{itemize}

The different variables for the decissions are:

\begin{itemize}
\item {\em myShipsEnemyRatio}: Ratio between the player's ships and enemy's ships.
\item {\em myShipsLandedFlyingRatio}: Ratio between the player's landed and flying ships.
\item {\em myPlanetEnemyRatio}: Ratio between the number of player's planets and the enemy's ones.
\item {\em myPlanetsTotalRatio}: Ration between the number of player's planet and total planets (neutrals and enemy included)-
\item {\em actualMyShipsRatio}: Ratio between the number of ships in the specific planet that evaluates the tree and player's total ships.
\item {\em actualLandedFlyingRatio}: Ratio between the number of ships landed and flying from the specific planet that evaluates the tree and player's total ships.
\end{itemize}

The decission list is:

\begin{itemize}
\item {\em Attack Nearest (Neutral|Enemy|NotMy) Planet}: The objective is the nearest planet.
\item {\em Attack Weakest (Neutral|Enemy|NotMy) Planet}: The objective is the planet with less ships.
\item {\em Attack Wealthest (Neutral|Enemy|NotMy) Planet}: The objective is the planet with higher lower rate.
\item {\em Attack Beneficious (Neutral|Enemy|NotMy) Planet}: The objective is the planet more beneficious, that is the one with growth rate divided by the number of ships.
\item {\em Attack Quickest (Neutral|Enemy|NotMy) Planet}: The objective is the planet with higher facility to conquest: the lowest product between the distance from the planet that executes the tree and the number of the ships in the objective planet.
\item {\em Attack (Neutral|Enemy|NotMy) Base}: The objective is the planet with more ships (that is, the base).
\item {\em  Attack Random Planet}.
\item {\em Reinforce Nearest Planet}: Reinforce the nearest player's planet to the planet that executes the tree.
\item {\em Reinforce Base}: Reinforce the player's planet with higher number of ships.
\item {\em Reinforce Wealthest Planet}: Reinforce the player's planet with higher grown rate.
\item {\em Do nothing}.


\end{itemize}

An example of a possible tree is shown in Figure \ref{fig:java}.

\begin{figure}[tb] 
\begin{center}
\begin{lstlisting}
if(myShipsLandedFlyingRatio<0.796)
	if(actualMyShipsRatio<0.201)
		attackWeakestNeutralPlanet(0.481);
	else
		attackNearestEnemyPlanet(0.913);
else
	attackNearestEnemyPlanet(0.819);
\end{lstlisting}
\end{center}
\caption{Example of a generated Java tree.}
\label{fig:java}
\end{figure}

In each turn, the loop presented in Figure \ref{fig:pseudo} is executed.
\begin{figure}[htb]
\begin{algorithmic}
\STATE tree $\gets$ ReadTree()

\COMMENT {In each turn}
\LOOP
	
	\STATE calculateGlobalPlanets()
	\COMMENT{for example Base, Enemy Base...}
	\STATE calculateGlobalRatios ()
	\COMMENT {for example myPlanetEnemyRatio, myShipsEnemyRatio...}
		\FOR{each Planet: p}
			\STATE calculateLocalPlanets (p)
			\COMMENT{for example NearestNeutralPlanet to planet p}
			\STATE calculateLocalRatios (p)
			\COMMENT{for example actualMyShipsRatio}
			\STATE executeTree(p,tree)
			\COMMENT{Send a percentage of the ships to another planet}
		\ENDFOR
\ENDLOOP
\end{algorithmic}
\caption{Pseudocode of the proposed agent. The tree is fixed during all the agent's execution}
\label{fig:pseudo}
\end{figure}




\section{Experimental Setup}

To generate the tree we have used a Genetic Programming Algorithm with the parameters.

The fitness used is 

The selected bot to confront is {\em GeneBot}, proposed in our previous work \cite{Mora2012Genebot}. This bot was trained using a GA to optimize the 8 parameters that conforms a set of hand-made rules.

\section{Results}

\section{Conclusions}
\label{sec:conclusion}

\bibliographystyle{splncs}
\bibliography{gpbot}

\end{document}

