\documentclass[runningheads]{llncs}
\usepackage{amssymb}
\setcounter{tocdepth}{3}
\usepackage{graphicx,epsfig}
\usepackage{algorithmic}
\usepackage{listings}
\usepackage{rotating}
\usepackage{subfig}

%%%%

\usepackage{color}
\usepackage{alltt}
\usepackage{verbatim}
\usepackage{url}
\usepackage[latin1]{inputenc}
%\usepackage[spanish]{babel}

%%

\usepackage{url}
\urldef{\mailsa}\path|pgarcia@atc.ugr.es|



\newcommand{\keywords}[1]{\par\addvspace\baselineskip
\noindent\keywordname\enspace\ignorespaces#1}

\lstset{
basicstyle=\ttfamily \scriptsize,
language=c++,
frame=single,
stringstyle=\ttfamily,
showstringspaces=false
}

\begin{document}
 %\pagestyle{empty} %ESTO QUITA LOS NUMEROS DE PAGINA
\mainmatter  % start of an individual contribution



% first the title is needed
\title{MADE: Massive Artificial Drama Engine for non player characters}

% a short form should be given in case it is too long for the running head
\titlerunning{MADE: A Massive BLABLABLA}
\author{R.H. Garc\'ia-Ortega\inst{1}, P. Garc\'ia-S\'anchez\inst{2} and J.J. Merelo\inst{2}}

%

\authorrunning{R.H. Garc\'ia-Ortega et al.} %FERGU: OJO! ES DOBLE CIEGO

% (feature abused for this document to repeat the title also on left hand pages)
% the affiliations are given next; don't give your e-mail address
% unless you accept that it will be published

\institute{Fundaci\'on I+D del Software Libre, Granada, Spain \and Dept. of Computer Architecture and Technology, University of Granada, Spain 
\mailsa}




\toctitle{BLABLABLA}

\tocauthor{Authors' Instructions}
\maketitle


\begin{abstract}


%Es mejor que empecéis por escribir el abstract, qué es lo que queréis
%hacer, porque todo el paper va a girar alrededor de eso. Siempre se
%puede modificar luego si no sale del todo bien, pero lo importante es
%tenerlo como referencia.

The creation of fictional stories is a very complex task that implies a creative process where the author has to mix characters, conflicts and plots. This work presents a simulated environment with hundreds of characters that allows the study of coherent and interesting literary archetypes (or behaviours), plots and subplots. We will use this environment to perform a study about the number of profiles (parameters that define the personality of a character) needed to create two emergent groups of archetypes: "natality control" and "revenge". A Genetic Algorithm will be used to find the fittest number of profiles and parameter configuration that enables the existence of the desired archetypes (played by the characters without their express knowledge). The results show that parameterizing this complex system is possible and that these kind of archetypes can emerge in the given environment.



\end{abstract}





\section{Introduction}
\noindent 

In videogames, NPCs (Non Player Characters) are a type of characters that live in the game world to provide a more inmersive experience. Modern RPGs (Role Playing Games), such as The Witcher\texttrademark or Skyrim\texttrademark count with hundreds of NPC characters. The effort to create a good interactive fiction script is directally proportional to the number of these characters. That is the reason this kind of agents usually counts with limited behaviours, such as wandering in the villages, selling groceries or guarding the cities. Also, they usually offer scripted conversations, for example, for buy and sell objects to the player. In other cases they interact with the player depending of the player's behaviour: for example, if the player steals something a city guard would attack him.  However, these characters do not interact among them, only with the player, and their activities are only guided with this purpose. In a world with such a number of characters, their collective interactions could improve the gaming experience, leading to a richier and more inmersive world. For example, hungry inhabitants could become thieves, guards could pursuit the thieves, villagers could fell in love with others or different war alliances could emerge.

These facts have motivated us to develop a multi-agent system called MADE (Massive Artificial Drama Engine) to model a self-organized virtual world where their elements influences each other, following a cause-effect behaviours in a coherent manner. This system needs to be a suitable environment for the plot of a specific literary work, being also interesting for the player/spectator. A set of probabilities and states are associated to agents' actions, and these probabilities are optimized by means of an Evolutionary Algorithm (EA) to match with a specific literary archetype, defined by the fiction creator. The archetypes are behaviours and patterns universally accepted and present in the collective imaginary \cite{ArchetypesGarry05}, that allows empathize with the characters and immerse yourself in the story (for example, the well-known ``hero'' archetype).

In this work, several experiments have been carried out to answer following questions: 
%No se da "insight" a una
%pregunta: lo dan las respuestas.
% (Rubén) -> cambiado a "answers"

\begin{itemize}
 \item Is it possible to model a virtual environment inhabited by hundred of characters with interesting auto-generated behaviour based on literary archetypes?
 \item Could the personality of the agents be parameterized to obtain different behaviours? 
 \item How many profiles (groups of parameters that define a personality) are necessary to generate emergent quality subplots?
 \item Could a Genetic Algorithm be used to find the fittest parameter values that allow the creation of this kind of subplots?

\end{itemize}

% si esto es todo, es muy pobre. En un paper tienes que hacerte
% preguntas y repsonderlas. Aquí tendría que venir "in this paper we
% prove that EAs, together with the proper design of a
% (Rubén) -> OK, añadido un párrafo donde señalo los resultados obtenidos

In this paper we prove that EAs, together with a proper design of literary patterns, can be used to find the parameters that promote the generation of drama plots and subplots in a multi agent based environment.\\

The rest of the work is structured as follows: after the state of the art, the developed system is presented in Section \ref{sec:made}. Then, the experiments conduced with the EA are showed (Section \ref{sec:results}). Finally, conclusions and future works are discussed.


%%%%%%%%%%%%%%%%%%%%%%%%SEC SOA
\section{State of the art}
\label{sec:soa}

%Por cierto, acabo de ver esto: \cite{StoryTecGobel2008}.

Auto-generated interactive fiction research is mainly focused in methods to create the process of a story generation \cite{nairat2011character}. Story generation can be divided in two areas: interactive and non-interactive. In the first area, and according to \cite{ReviewArinbjarnar09}, an Interactive Drama is defined in a virtual world where the user has freedom to interact with the NPCs and objects in a dramatically interesting experience, different in each execution, and adapted to the interactions of the user.

%Decir qué técnicas utiliza y qué se suele hacer en este caso

As opossed to this concept, MADE is focused in Artificial Non Interactive Drama, because its aim is the massive generation of plots for secondary characters, to provide a context for the writer and the player to perceive a virtual world as coherent, detailed and enriched. The story generation (that is, the narrative) is not adressed by MADE, but it has been studied in the systems presents in the survey by Arinbjarnar et al. in \cite{ReviewArinbjarnar09}.

%Decir qué te hace suponer que esto va a ser una buena solución.

The generation of interactive dramas can also be based in script
structure \cite{ArchitectureYoung04}, where each possibility in the
story must be previously defined, so there is a limited number of
possible plot combinations. There exist other techniques, not based in
plot structure, such as... % Pero esto debería ir antes de la
                           % descripción de MADE. Lo que digas de MADE
                           % al final de todo, para expresar el avance
                           % con respecto al estao del arte actual - JJ

On the other side, in non-interactive plot generation systems the user does not take control as the protagonist. For example, in the system presented by Pizzi et al. \cite{pizzi2007interactive} the user can interact with the characters, changing their emotions, but making the user an spectator, rather as an actor. %FERGU: ESTO NO ME QUEDA TAMPOCO CLARO, SI NO ES INTERACTIVA COMO MODIFICAMOS LOS PERSONAJES?

Previous works define the plot as an emergence for the behaviour of the agents that follow a set of rules. In MADE, the agents' behaviour is product of its personality and the environment. That is, the agents does not follow the plot, but they generate the plot itself. %%% FERGU: ESTO NO LO TENGO MUY CLARO

Futhermore, the previous works generate plots in worlds with a limited number of characters. This restriction does not exist in MADE, where the number of characters to create is unlimited.


\section{MADE}
\label{sec:made}

% Me parece mal comienzo... ¿Por qué sigues esas ideas, por ejemplo?
% Tendrías que haberlo dicho en la intro - JJ
Following the ideas of the work of Epstein and Axtell \cite{epstein1996growing} an environment based in {\em Sugarscape} has been developed, with concepts such as food, methabolism and vision. This enviroment, uses the elemenents  by Gershenson \cite{gershenson2005general}: a virtual world, agents who born, grow, interact, reproduce and dead; resources (food), mediators, and relations of rivalry (friction) and cooperation (sinergy). The actions of these agents are parameterized accordign the work of Nairat \cite{nairat2011character}, based in the existence of profiles, and will be maped into a chromosome to be used in a Genetic Algorithm (GA). This system allows the definition of behaviour patterns (or archetypes, and using a multi-objective fitness function to measure the presence of the desired archetypes. %OJO! ESTO QUE HE DICHO ES VERDAD? (Fergu)

%FERGU: Comprobar que hemos mencionado en el SOA los trabajos de Epstein y Gerhenson

\subsection{MADE Environment}
The MADE enviroment is the one where the agents are placed. Its functions are:

\begin{itemize}
\item \textbf{Create an initial set of agents:} MADE environment initializes a set of orphan agents, with age 0, and profile sequentially assigned. Thes agents are just born in the MADE environment and must compet or collaborate in order to survive.
\item \textbf{Position agents in the map:} the environment has a map (initially a squared one), formed by cells that can be occupied by an (and only one) agent. The environment allows to the agents to discover and interact other agents in the neighborhood.
\item \textbf{Start and control the time:} after create the initial set of agents, the MADE environment start the timer.
\item \textbf{Execute each agent during a time unit (a day):} A number of iterations is performed in each execution. In each iteration the list of agents is ordered, each agent perform an iteration of its life-cycle, and the dead agents are removed.
\item \textbf{Perform as an external agent that changes th environment:} In each iteration in the MADE environment food rations are placed in random cells. An agent only can eat if it is over a cell with a ration, so agents could move the other forcibly.
\item \textbf{Offer services to the agents:} MADE environment allow the agents to check which closer cells have food, are occupied, who agents are in a near position or which positions can they occupy.
\item \textbf{Decide the profile of the agents:} MADE allows the existence of different agent profiles. A profiles is a set of probabilities which governs the agent's characteristics and behaviour.
\end{itemize}

The parameters of the MADE environment are:
\begin{itemize}
\item Number of agents
\item Map dimension
\item Number of rations each day
\item Number of days
\item Average NO SE QUE ES ESTO xD
\end{itemize}
\subsection{MADE Agent}

\section{Extraction of WHATEVER using Evolutionary Algorithms}

\section{Conclusions}
\label{sec:conclusion}

\section*{Acknowledgements}
This work has been supported in part by %FPU research grant AP2009-2942 and projects EvOrq (TIC-3903), CANUBE (CEI2013-P-14) and ANYSELF (TIN2011-28627-C04-02).

\bibliographystyle{splncs}
\bibliography{made}

\end{document}

