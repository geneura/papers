\documentclass[smallextended]{svjour3}

\journalname{Evolutionary Intelligence}


\newcounter{linecounter}
\newcommand{\linenumbering}{\makebox[2em][r]{\arabic{linecounter}:}}
\renewcommand{\line}[1]{\refstepcounter{linecounter}
\linenumbering}
\newcommand{\resetline}{\setcounter{linecounter}{0}}


\begin{document}

\title{Introduction to the Special Issue on Evolutionary Intelligence
  in Games}

\author{ J. J. Merelo$^\dag$, Paolo Burelli$^\ddag$}
\institute{$^\dag$ Departamento de Arquitectura y Tecnolog\'{\i}a de Computadores,
Universidad de Granada,
Granada, Spain. \\
$^\ddag$ Aalborg University, Denmark}

%
\date{Received: date / Revised version: date}
% The correct dates will be entered by the editor
%
\authorrunning{Merelo, Burelli}
\titlerunning{Evolutionary Intelligence in Games.}
\maketitle
%

\section{Presentation}

This issue includes three papers selected from those presented to the
EvoGames track within the EvoStar 2013 conference \cite{DBLP:conf/evoW/2013a}. All authors with a
paper accepted in that conference were invited and then an open call
was made to anybody working in the topic of evolutionary intelligence
in games. After the authors submitted an extended version a selection
was made and these three papers were accepted after two rounds of
revision.

The three papers represent a wide range of games and puzzles: a
relatively recent
trading-card game, Dominion \cite{mahlmann2012evolving}, which was created by Donald Vacarino and
published by Rio Grande Games in 2008, a classic platform game, Super
Mario \cite{togelius2009super} and a board game, MasterMind, which is
actually a puzzle proposed initially in the 70s by a telecommunication
engineer and whose first solution was proposed by Donald Knuth a few
years later \cite{Knuth}. 

The first paper by Ransom Winder \cite{evin:dominion} is the one that
tries to find a winning deck of ten cards for the Dominion game; in
fact, the problem that is tackled is to try and predict the game
states to make decisions about the next move. Neural networks are
applied to the state of the game when the player has to make any
decision, and these neural networks have to be trained so that the
outcome of these decisions is as favorable to the player as
possible. Several methods have been tried, and an
evolutionary-algorithm based game bot is the only one that obtains an
edge over classical strategies. 

The second paper also combines evolutionary algorithms with other
algorith, in this case Finite State Automata, to evolve in this case
strategies for advancing a Super Mario avatar in platforms of
increasing difficulty \cite{evin:mario}. This game has been
traditionally used for computational intelligence competitions; FSM
are also traditionally used to drive game bots in strategy and first
person shooter games. But desining them can be tricky, and an
evolutionary algorithm that designs the transitions between states is
succesfully used in this paper for a Mario that is able to obtain good
results up to a high level; besides, a study is made on the population
needed to obtain good results in each level. 

The final paper is more purely evolutionary \cite{evin:mm} in the
sense that it does not combine evolutionary techniques with others in
order to solver the MasterMind game, a puzzle in which a secret
combination must be found with the help of hits given as response to
every move made by a player. The paper approaches this problem from
two different angles: first, evolving populations of combinations
scores according to the hints made and then using genetic programming
to evolve good scoring methods for these combinations that have,
traditionally, been chosen heuristically.

All in all, a diverse set of papers that use evolutionary algorithms
to find solutions to a wide range of games combining them with other
techniques such as neural nets or finite state automata. 


\bibliographystyle{spbasic} 
\bibliography{evin,geneura}

\end{document}
