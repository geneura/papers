\documentclass{sig-alternate}
\usepackage[latin1]{inputenc}
\usepackage{graphicx}        % standard LaTeX graphics tool
\usepackage{subfigure}
\usepackage{url}


\begin{document}
%
% --- Author Metadata here ---
\conferenceinfo{GECCO'14,} {July 12-16, 2014, Vancouver, BC, Canada.}
    \CopyrightYear{2014}
    \crdata{TBA}
    \clubpenalty=10000
    \widowpenalty = 10000

\title{Enforcing Corporate Security Policies via Computational Intelligence Techniques}

\numberofauthors{2}
 \author{
 \alignauthor
 Author1\\
        \affaddr{Maracena Institute of Technology (MIT)}\\
        \affaddr{Anonymous Street num. X}\\
        \affaddr{AnonymousLand}\\
        \email{aut1@mit.com}
 \alignauthor
 Author2\\
 \affaddr{Authors University (AU)}\\
 \affaddr{Authors Street num. AAA}\\
 \affaddr{Metropolis}\\
 \email{aut2@au.com}
 }

%\numberofauthors{2}
% \author{
% \alignauthor
% A.M. Mora, P. De las Cuevas, J.J. Merelo\\
%        \affaddr{University of Granada}\\
%        \affaddr{Department of Computer Architecture and Technology, ETSIIT/CITIC}\\
%        \affaddr{Granada, Spain}\\
%        \email{amorag,paloma,jmerelo@geneura.ugr.es}
% \alignauthor
% S. Zamarripa, Anna I. Esparcia-Alc�zar\\
%        \affaddr{S2 Grupo}\\
%        \affaddr{Valencia, Spain}\\
%        \email{szamarripa,aesparcia@s2grupo.es}
% }


\maketitle

\begin{abstract}
This paper presents an approach, based in a project in development, which combines Data Mining, Machine Learning and Computational Intelligence techniques, in order to create an user-centric and adaptable corporate security system. Thus, the system, named MUSES, will be able to analyse the user's behaviour (modelled as events) when interacting with the company's server, accessing to corporate assets, for instance. As a result of this analysis, and after the application of the commented techniques, the Corporate Security Policies, and specifically, the Corporate Security Rules will be adapted to deal with new anomalous situations, or to better manage user's behaviour.
The work reviews the current state of the art in security issues resolution by means of these kind of methods. Then it describes the MUSES features in this respect and compares them with the existing approaches.
\end{abstract}

% *** Esto hay que buscarlo y fijarlo para este art�culo ***
% A category with the (minimum) three required fields
%\category{I.2.1}{Computing Methodologies}{ARTIFICIAL INTELLIGENCE}[Applications and Expert Systems]\\
%A category including the fourth, optional field follows...
%\category{G.1.6}{Mathematics of Computing}{Numerical Analysis}[Optimization]
%
%D.2.8 [Software Engineering]: Metrics�complexity mea-
%sures, performance measures
%\terms{Algorithms}

\category{I.2.6}{Artificial Intelligence}{Learning - Induction} 
\category{D.4.6}{Operating Systems}{Security and Protection - Access controls} \category{I.2.1}{Artificial Intelligence}{Applications and Expert Systems}


\keywords{Computational Intelligence, Evolutionary Computation, Corporate Security Policies, Security Rules}


%
%%%%%%%%%%%%%%%%%%%%%%%%%%%%%%%   INTRODUCTION   %%%%%%%%%%%%%%%%%%%%%%%%%%%%%%%
%
\section{Introduction}
\label{sec:intro}

Security in distributed systems has been a very profiting research area from the arisen of the first client/server architectures \cite{computer_security_80}. Inside this, corporate security is one of the main topics. This environment has changed dramatically in the last years, starting with the distribution of the information (instead of being centralised in corporate servers, it has been spread among multiple machines such as portable devices, external servers, or cloud storage systems); and continuing with the so-called Bring Your Own Device (BYOD) philosophy, in which the devices that access to the system are owned by the users (company's employees), and could contain personal and professional information.

This scenario opens up new security issues \cite{Opp_Security11}, which should be dealt in a different way, having into account both (company's) data security and (user's) privacy. In order to protect them, there are defined \textit{Corporate Security Policies}.

To dealt with this new situation, a novel system is being developed (inside an European Project). It is named \textit{MUSES}, from \textit{Multiplatform Usable Endpoint Security System} \cite{MUSES_SAC_14_anonymous}, which is a device-independent end-to end user-centric tool. 
It considers a set of security rules defined as specifications of the Company Security Policies, and its main feature is the ability of `learning' from the user's past behaviour and adapt, even inferring new ones, the set of rules in order to deal with potential future security incidents due to the user's behaviour. Then, the system will react, in a non-intrusive way, to the potentially dangerous sequence of actions (events) that he or she is conducting at any time.

To this end MUSES will analyse the users' behaviour by means of Data Mining
(DM) techniques \cite{DataMining_Lee01} and Machine Learning (ML) methods \cite{MachineLearning_Bishop06}, extracting a set of patterns which will be later processed by means of Computational Intelligence (CI) algorithms, mainly Evolutionary Computation methods \cite{EAs_Back96,GAs_Goldberg89,GP_Koza92}.

This is a step beyond the current state of the art in two senses: first regarding the current security systems for dealing with the new scenario inside the enterprises, as it can be read in \cite{MUSES_SAC_14_anonymous}; and second concerning the application of Computational or Artificial Intelligence (AI) techniques to corporate security issues, focused on (and adapted to) the users' behaviour, as will be analysed in this work.

%Seguridad en la empresa, pol�ticas y reglas, computational intelligence en seguridad (breve).
%
%Cosas del DOW de por qu� es bueno aplicar cosas nuevas de CI a seguridad.
%Decir si hay algo hecho al respecto.

The paper is structured as follows. Next section gives a background in the current enterprise security issues. Section \ref{sec:stateofart} reviews related work regarding the application of DM, ML, AI and CI techniques to a wide range of security problems inside the enterprise, but mainly focused on the user's behaviour and the security policies adaptation, which are the main advantages of MUSES.
Then, the MUSES system's features in this respect (application of DM, ML, CI) are described in Section \ref{sec:muses_ci}, and compared with the existing works.
Finally, the reached conclusions are presented in Section \ref{sec:conclusions}.


%%%%%%%%%%%%%%%%%%%%%%%%%%%%%%%   BACKGROUND  %%%%%%%%%%%%%%%%%%%%%%%%%%%%%%%
%
\section{Enterprise security}
\label{sec:background}

Until these days, enterprises used to follow a static Security Policy devoted to control a certain structure \cite{BYOD13}, where the Information Assets and the devices were purchased and maintained by the company. Now that corporate networks are becoming dynamic for being adapted to the BYOD philosophy, there is an additional risk because the devices that the employees use are not always company-owned. A needed security policy, or in this case, an \textit{Information Security Policy} (ISP from now on) should deal with the way of protecting a certain organization's information against a security breach. Though there are standards, such as the ISO27002 or the Security Forum's Standard of Good Practice\footnote{https://www.securityforum.org}, an ISP is defined depending on the characteristics of the community/organisation that they are built for.

Normally, the enterprise network architecture was being adapted to cope with external attackers \cite{MIT05}. However, with the consideration of BYOD, the threat is about corporate assets being compromised due to employees' devices with vulnerabilities \cite{android11}, or leaked because they are being accessed from a device connected through an unsecured (public) network.

%On the other hand, employee-owned devices, like smartphones, have the possibility of maintaining a good balance between work and private life. For this reason, the risk of uncontrolled devices (with potentially dangerous applications) accessing to corporate assets in unsafe conditions is bigger.

%Thus now, more things should be considered than the usual ones when designing a company network architecture. 
In Figure \ref{fig:proposed_diagram} there is a proposal which can be used for the beginning of the study of solutions that may make secure such a dynamic environment. It includes the possibility of having employee-owned mobile (smartphones and tablets) and portable (laptops) devices, and also the opportunity that the employees have of connecting these devices either from inside or outside the company premises. Moreover, company information assets are constantly accessed under these conditions, considering that an information asset means every \textit{piece of information} that has a \textit{value} (cost depending on the risk of being lost or leaked) for the company. It can be referred to files with sensitive information, to certain mails, or even to company applications.

\begin{figure}[ht]
	\begin{center}
		\includegraphics[scale=0.36]{proposed_diagram.eps}
		\caption{Architecture approach of an Enterprise Network assuming that the Company has adopted the BYOD philosophy.}
	\label{fig:proposed_diagram}
	\end{center}
\end{figure}

The other issue to cope with is the elaboration of a good ISP, understandable for every user of the company, and more importantly, non-intrusive for him. A lot of researchers have studied the natural tendency of employees to comply with the ISP \cite{SecPolComp07,SecPolComp10,SecPolComp12}, reaching conclusions such as the employees compliance with the security policies increases educating/training them in information security awareness  \cite{SecPolComp09}, and decreases applying too much sanctions when a misuse or abuse occurs \cite{SecPolPenalty09}. 

This situation leads to a need of protecting the organisation side, but also the users side, making non-interfering easy-to-follow ISPs, and leaving them to use their devices for personal purposes while working, without putting organisation's information assets under risk. The compliance of these requirements would compose an End-to-End Security Solution (protecting both enterprise and employee), which is the aim of the MUSES project \cite{MUSES_SAC_14_anonymous} (see Section \ref{sec:muses_ci}).

%%%%%%%%%%%%%%%%%%%%%%%%%%%%%%  STATE OF THE ART  %%%%%%%%%%%%%%%%%%%%%%%%%%%%%
%
\section{State of the Art}
\label{sec:stateofart}
%

Security is a wide area of research since the very beginning of the eighties \cite{computer_security_80}. Thousands of works have been published in a number of different issues in this topic. 
One of the most profiting fields is the application of Artificial Intelligence techniques to different security-based problems. This research line was started more than twenty years ago \cite{ai_intrusion_detection_94}, and will be still open for several years more \cite{ai_cybersecurity_11}.

The topics addressed by the researchers are quite varied, going from the data mining \cite{botnet_detection_clustering_09,feature_selection_anomalies_08}, to the machine learning area \cite{learning_network_intrusion_09,user_classification_ml_13}, both applied to many different problems.

Computational intelligence techniques have been also widely used in this area, being the most profiting methods the Evolutionary Computation (EC) metaheuristics: Genetic Algorithms (GAs) and Genetic Programming (GP).

There are several works using GAs for solving security issues, such as the intrusion detection (see \cite{GA_intrusion_detection_survey_14} for a survey), the design and evaluation of security protocols \cite{detecting_intrusion_gp_03,eval_security_gas_07,GAs_security_protocols_10}, or the optimisation of different aspects related with security: IT security costs \cite{optimizing_IT_costs_ea_10} and cryptographic protocols \cite{cryptographic_gas_06}, to cite a few.

We will focus in this work on the application of different DM, ML and AI/CI techniques to new set of security issues, which has been open by the new interactions between systems, and by the user's habits and behaviour (including the BYOD scenario), as it is described in Section \ref{sec:background}. 
Then, the works that we are interested in are those related with the users' information and behaviour (in this scope), and the management (and adaptation) of Information or Corporate Security Policies (ISPs).

In this line, the paper by Greenstadt and Beal \cite{cognitive_security_08} combined biometrics signals with machine learning methods in order to get a reliable user authentication in a computer system.
P.G. Kelley et al. \cite{user-controllable_learning_08} presented a method named \textit{user-controllable policy learning} in which the user gives feedback to the system every time that a security policy is applied, so these policies can be refined according to that feedback to be more accurate with respect to the user's needs. This approach could be useful for a personal device, but our aim in MUSES is to have a global set of rules that could be adapted for all the users.
On the other hand, policies could be created for enhancing user's privacy, as proposed by Danezis in \cite{inferring_policies_socialnetworks_09}, who defined a system able to infer privacy-related restrictions by means of a machine learning method applied in a social network environment. The idea of inferring policies will be also considered in MUSES, but in the scope of the company, and focused on ISPs.

A closer work to our approach is the one presented by Samak et al. \cite{policy_generation_clustering_10}, in which the authors use a clustering-based approach to infer new traffic security policies in a network.
However our idea in MUSES, explained in Section \ref{sec:muses_ci}, is to infer new Security Rules (as a specialisation of the ISPs) by means of GP.

Some other authors have applied this rule-based method for evolving (improving) a set of policies, such as Lim et al. \cite{sec_policy_evolution_gp_08,pol_evol_gp_3_approaches_08}, who inferred new policies based in the decisions on a system, considering the user's feedback. A similar approach will be considered in MUSES, but there will be an automatic evaluation system for the new inferred rules, rather to the strict need of an user control. Moreover, these works have been tested on synthetic testbeds, but MUSES will work in real companies with real users and real data.

Finally, the work by Suarez-Tangil et al. \cite{rule_generation_gp_09}, combines GP with the event correlation process which is also applied in MUSES \cite{MUSES_SAC_14_anonymous}. However their approach is designed to create the rules/engine for that process, instead of the security rules to be considered as the output of the event correlation, i.e. the decisions to be made according to the events produced and to the enterprise's ISPs.



%%%%%%%%%%%%%%%%%%%%%%%%%%%%%%%   MUSES SYSTEM  %%%%%%%%%%%%%%%%%%%%%%%%%%%%%%%
%
\section{MUSES System}
\label{sec:muses_ci}


As previously stated, MUSES will be a whole corporate security system aimed to deal with the new BYOD philosophy, i.e. it will manage user's accesses to the company's servers from diverse own devices, which could be dangerous for several reasons, including the user's behaviour.

One of the main features of this system will be the self-adaptation (to the user and context) of the set of Corporate Security Rules (specification of the ISPs). 
To this end, there is a component in the designed architecture \cite{MUSES_SAC_14_anonymous} named \textit{MusKRS}, from \textit{Knowledge Refinement System}. This will be run asynchronously in the server and will be in charge of analysing all the gathered information (events, context, user-related data), and adapting/refining the security rules to better deal with these events, also trying to predict future threats due to the user's behaviour.

This process will be composed by two steps: first, a Data Mining procedure will be performed (in the \textit{Data Miner} sub-component);  second, a refinement and inference process will be done (in the \textit{Knowledge Compiler} sub-component) by means of Machine Learning and Computational Intelligence techniques.

% ------------------------------------------------------------------
%
\subsection{Data Mining/Machine Learning}
\label{subsec:dm_ml}

%Data miner
%From System_Architecture#Security_Server_Architecture:
%This component takes the �raw� data from the database and processes the information, using Data Mining (DM) techniques, in order to yield a set of relevant data for
%the Knowledge Compiler submodule. To this end, some techniques such as feature selection, clustering, or classification are performed. The process will be mainly
%non-supervised and eventually, the datasets can be huge (depending on the company�s data flows), so Big Data processing methods are applied.

This task will involve some techniques:

\begin{itemize}

\item \textit{Classification}: The aim is to use database information to look for patterns that had been allowed or denied, in order to build a classifier. When an incoming event (or set of events) has no rule to apply on, the classifier should tell the similarity with previous (and already labelled) patterns.

\item \textit{Clustering}: The patterns could be grouped considering some similarity criteria, in order to deal with them as a set. This could be used for providing data visualisation mechanisms. In order to make it easier to interpret the data interaction and the distribution in clusters with respect to the different properties/features of the patterns.

\item \textit{Data Analysis}: This will provide a controller (user) with mechanisms to visualise interesting facts about the data, such as more frequent events, dangerous or suspicious users (according to their behaviour), more triggered rules, etc.

\item \textit{Feature Selection}: It consists of extracting the most important features from the data. This could be done by means of one of the previous techniques, and could be useful if we want to discard non-key features. This is useful in order to reduce the database weight, or for improving the classification running time and even the performance of the system.
\end{itemize}


% ------------------------------------------------------------------
%
\subsection{Computational Intelligence}
\label{subsec:ci}

%- Clasificaci�n con GP
%- Inferencia de nuevas reglas con GP
%- Ajuste de valores de recursos con AGs???

These techniques are used in an inference or refinement process, which considers several factors:

\begin{itemize}

\item The information extracted from the data miner, mainly concerning the unclassified or misclassified patterns. These are those patterns which did not match with any of the existing classes (they are quite different from the patterns belonging to those classes), so they cannot be included in any of the classes and thus they should be taken into account for a potential inference or update in the set of security rules, in order to 'cover' them.

\item User related information corresponding to those unlabelled patterns (actions). Thus, the user's ID, location and role, for instance, will be considered in order to select the applicable set of rules for that conditions.

\item Context information for the same unclassified patterns, in order to also restrict the applicable set of rules considering this information.
The process will be performed over the previous sets.

\item Risk information extracted from the user profile (reputation), e.g. "Did the user received a lot of 'denies'/'allows' before?", or in terms of the RT2AE "Is he trustworthy?". In case the user is not, more restrictive rules would be created for him, otherwise the corresponding rules could be 'eased' for that user.

\item The information included in logs along the system, which can, for instance, tell about how the user responds to system messages (either an action or if he gives feedback). This could result in the inference of new rules or in adaptation, in order to deal with e.g. users that repeatedly ignore warning messages.

\end{itemize}


% ------------------------------------------------------------------
%
\subsection{Evolutionary Computation Approaches}
\label{subsec:ec}




%%%%%%%%%%%%%%%%%%%%%%%%%%%%%%%   RESULTS  %%%%%%%%%%%%%%%%%%%%%%%%%%%%%%%%
%
%\section{Expected results and comparison with other approaches}
%\label{sec:comparison}
%
%describir lo que se pretende obtener y lo que mejorar� a lo existente en el estado del arte.


%%%%%%%%%%%%%%%%%%%%%%%%%%%%%%  CONCLUSIONS  %%%%%%%%%%%%%%%%%%%%%%%%%%%%%%%
%
\section{Conclusions and Future Work}
\label{sec:conclusions}

qu� se puede concluir del trabajo y qu� haremos a continuaci�n


%ACKNOWLEDGMENTS are optional
\section{Acknowledgments}

Anonymous...

%This paper has been funded in part by European and National projects FP7-318508 (MUSES) and TIN2011-28627-C04-02 (ANYSELF) respectively, project P08-TIC-03903 (EV\-ORQ) awarded by the Andalusian Regional Government, and project 83 (CANUBE) awarded by the CEI-BioTIC UGR.

%
% The following two commands are all you need in the
% initial runs of your .tex file to
% produce the bibliography for the citations in your paper.
\bibliographystyle{abbrv}
\bibliography{CI_security_rules} 

\end{document}
