\documentclass{sig-alternate}
\usepackage[latin1]{inputenc}
\usepackage{graphicx}        % standard LaTeX graphics tool
\usepackage{subfigure}
\usepackage{url}


\begin{document}
%
% --- Author Metadata here ---
\conferenceinfo{GECCO'14,} {July 12-16, 2014, Vancouver, BC, Canada.}
    \CopyrightYear{2014}
    \crdata{TBA}
    \clubpenalty=10000
    \widowpenalty = 10000

\title{Data Mining Applied to Secure URL Accesses in the Company}

\numberofauthors{2}
 \author{
 \alignauthor
 Author1\\
        \affaddr{Maracena Institute of Technology (MIT)}\\
        \affaddr{Anonymous Street num. X}\\
        \affaddr{AnonymousLand}\\
        \email{aut1@mit.com}
 \alignauthor
 Author2\\
 \affaddr{Authors University (AU)}\\
 \affaddr{Authors Street num. AAA}\\
 \affaddr{Metropolis}\\
 \email{aut2@au.com}
 }

%\numberofauthors{2}
% \author{
% \alignauthor
% A.M. Mora, P. De las Cuevas, J.J. Merelo\\
%        \affaddr{University of Granada}\\
%        \affaddr{Department of Computer Architecture and Technology, ETSIIT/CITIC}\\
%        \affaddr{Granada, Spain}\\
%        \email{amorag,paloma,jmerelo@geneura.ugr.es}
% \alignauthor
% S. Zamarripa, Anna I. Esparcia-Alc�zar\\
%        \affaddr{S2 Grupo}\\
%        \affaddr{Valencia, Spain}\\
%        \email{szamarripa,aesparcia@s2grupo.es}
% }


\maketitle

\begin{abstract}
This paper presents an analysis of a dataset based in URL requests performed by the employees of a company. The aim is to get a classification according to their features which can assign a new request a decision, based in the enterprise security policies...
\end{abstract}

% *** Esto hay que buscarlo y fijarlo para este art�culo ***
% A category with the (minimum) three required fields
%\category{I.2.1}{Computing Methodologies}{ARTIFICIAL INTELLIGENCE}[Applications and Expert Systems]\\
%A category including the fourth, optional field follows...
%\category{G.1.6}{Mathematics of Computing}{Numerical Analysis}[Optimization]
%
%D.2.8 [Software Engineering]: Metrics�complexity mea-
%sures, performance measures
%\terms{Algorithms}


\keywords{Data Mining, Security Policies, URL request, Classification}


%
%%%%%%%%%%%%%%%%%%%%%%%%%%%%%%%   INTRODUCTION   %%%%%%%%%%%%%%%%%%%%%%%%%%%%%%%
%
\section{Introduction}
\label{sec:intro}

seguridad en la empresa, pol�ticas y reglas, data mining en seguridad (breve)

%The paper is structured as follows. Next section describes the problem enclosed in the Planet Wars game.
%Section \ref{sec:stateofart} reviews related work regarding the scope (videogames) and the implementation (EAs with no fitness computation or different selection mechanisms).
%Section \ref{sec:survival_bots} presents the proposed method, termed {Survival Bot}, and its different implementations. 
%The experiments and results are described and discussed in Section \ref{sec:experiments}. Finally, the conclusions and future lines of research are presented in Section \ref{sec:conclusions}.

%%%%%%%%%%%%%%%%%%%%%%%%%%%%%%  STATE OF THE ART  %%%%%%%%%%%%%%%%%%%%%%%%%%%%%
%
\section{State of the Art}
\label{sec:stateofart}
%

data mining/clasificaci�n aplicado a problemas de seguridad

%
%%%%%%%%%%%%%%%%%%%%%%%%%%%%% PROBLEM DESCRIPTION %%%%%%%%%%%%%%%%%%%%%%%%%%%%%
%

\section{Problem and Data Description} 
\label{sec:problemDescription}


problema de las URLs, reglas a aplicar, datos de los que disponemos (descripci�n de los mismos y an�lisis preliminar)


%%%%%%%%%%%%%%%%%%%%%%%%%%%%%%%   SURVIVAL BOT  %%%%%%%%%%%%%%%%%%%%%%%%%%%%%%%%
%
\section{Methodology}
\label{sec:methodology}

descripci�n de la soluci�n/metodolog�a a aplicar. 

% ------------------------------------------------------------------
%
\subsection{Security rules parsing}
\label{subsec:ruleparsing}


% ------------------------------------------------------------------
%
\subsection{URL data parsing}
\label{subsec:urlparsing}


% ------------------------------------------------------------------
%
\subsection{Data Mining Methods}
\label{subsec:methods}



%%%%%%%%%%%%%%%%%%%%%%%%%%%%%%%   RESULTS  %%%%%%%%%%%%%%%%%%%%%%%%%%%%%%%%
%
\section{Obtained Results}
\label{sec:results}

describir y analizar lo que se ha obtenido en la clasificaci�n (acierto, tasa de error, etc)

%%%%%%%%%%%%%%%%%%%%%%%%%%%%%%  CONCLUSIONS  %%%%%%%%%%%%%%%%%%%%%%%%%%%%%%%
%
\section{Conclusions and Future Work}
\label{sec:conclusions}

qu� se puede concluir del trabajo y qu� haremos a continuaci�n


%ACKNOWLEDGMENTS are optional
\section{Acknowledgments}
%This paper has been funded in part by European and National projects FP7-318508 (MUSES) and TIN2011-28627-C04-02 (ANYSELF) respectively, project P08-TIC-03903 (EV\-ORQ) awarded by the Andalusian Regional Government, and project 83 (CANUBE) awarded by the CEI-BioTIC UGR.

%
% The following two commands are all you need in the
% initial runs of your .tex file to
% produce the bibliography for the citations in your paper.
\bibliographystyle{abbrv}
\bibliography{datamining_urls} 

\end{document}
