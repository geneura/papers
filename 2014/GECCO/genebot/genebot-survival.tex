% This is "sig-alternate.tex" V2.0 May 2012
% This file should be compiled with V2.5 of "sig-alternate.cls" May 2012
%
% This example file demonstrates the use of the 'sig-alternate.cls'
% V2.5 LaTeX2e document class file. It is for those submitting
% articles to ACM Conference Proceedings WHO DO NOT WISH TO
% STRICTLY ADHERE TO THE SIGS (PUBS-BOARD-ENDORSED) STYLE.
% The 'sig-alternate.cls' file will produce a similar-looking,
% albeit, 'tighter' paper resulting in, invariably, fewer pages.
%
% ----------------------------------------------------------------------------------------------------------------
% This .tex file (and associated .cls V2.5) produces:
%       1) The Permission Statement
%       2) The Conference (location) Info information
%       3) The Copyright Line with ACM data
%       4) NO page numbers
%
% as against the acm_proc_article-sp.cls file which
% DOES NOT produce 1) thru' 3) above.
%
% Using 'sig-alternate.cls' you have control, however, from within
% the source .tex file, over both the CopyrightYear
% (defaulted to 200X) and the ACM Copyright Data
% (defaulted to X-XXXXX-XX-X/XX/XX).
% e.g.
% \CopyrightYear{2007} will cause 2007 to appear in the copyright line.
% \crdata{0-12345-67-8/90/12} will cause 0-12345-67-8/90/12 to appear in the copyright line.
%
% ---------------------------------------------------------------------------------------------------------------
% This .tex source is an example which *does* use
% the .bib file (from which the .bbl file % is produced).
% REMEMBER HOWEVER: After having produced the .bbl file,
% and prior to final submission, you *NEED* to 'insert'
% your .bbl file into your source .tex file so as to provide
% ONE 'self-contained' source file.
%
% ================= IF YOU HAVE QUESTIONS =======================
% Questions regarding the SIGS styles, SIGS policies and
% procedures, Conferences etc. should be sent to
% Adrienne Griscti (griscti@acm.org)
%
% Technical questions _only_ to
% Gerald Murray (murray@hq.acm.org)
% ===============================================================
%
% For tracking purposes - this is V2.0 - May 2012

\documentclass{sig-alternate}
\usepackage[latin1]{inputenc}
\usepackage{graphicx}        % standard LaTeX graphics tool
\usepackage{subfigure}
\usepackage{url}


\begin{document}
%
% --- Author Metadata here ---
\conferenceinfo{GECCO'14,} {July 12-16, 2014, Vancouver, BC, Canada.}
    \CopyrightYear{2014}
    \crdata{TBA}
    \clubpenalty=10000
    \widowpenalty = 10000

\title{Fighting to Survive: Evolving RTS Bots without Evaluation}

\subtitle{[Abstract]}
%\titlenote{A full version of this paper is available as
%\textit{Author's Guide to Preparing ACM SIG Proceedings Using
%\LaTeX$2_\epsilon$\ and BibTeX} at
%\texttt{www.acm.org/eaddress.htm}}}
%
% You need the command \numberofauthors to handle the 'placement
% and alignment' of the authors beneath the title.
%
% For aesthetic reasons, we recommend 'three authors at a time'
% i.e. three 'name/affiliation blocks' be placed beneath the title.
%
% NOTE: You are NOT restricted in how many 'rows' of
% "name/affiliations" may appear. We just ask that you restrict
% the number of 'columns' to three.
%
% Because of the available 'opening page real-estate'
% we ask you to refrain from putting more than six authors
% (two rows with three columns) beneath the article title.
% More than six makes the first-page appear very cluttered indeed.
%
% Use the \alignauthor commands to handle the names
% and affiliations for an 'aesthetic maximum' of six authors.
% Add names, affiliations, addresses for
% the seventh etc. author(s) as the argument for the
% \additionalauthors command.
% These 'additional authors' will be output/set for you
% without further effort on your part as the last section in
% the body of your article BEFORE References or any Appendices.


\numberofauthors{2}
 \author{
 \alignauthor
 Bot1\\
        \affaddr{BotLand Institute of Techonology (BIT)}\\
        \affaddr{CPU Street num. 1110010}\\
        \affaddr{BotLand}\\
        \email{bot1@bit.com}
 \alignauthor
 Bot2\\
 \affaddr{Bots University (BU)}\\
 \affaddr{ALU Street num. 101110}\\
 \affaddr{Metropolis}\\
 \email{bot2@bu.com}
 }

%\numberofauthors{2}
% \author{
% \alignauthor
% J.J. Merelo, A.M. Mora, C. M. Fernandes\\
%        \affaddr{University of Granada}\\
%        \affaddr{Department of Computer Architecture and Technology, ETSIIT}\\
%        \affaddr{18071 - Granada}\\
%        \email{jmerelo,amorag,cfernandes@geneura.ugr.es}
% \alignauthor
% Anna I. Esparcia-Alc�zar\\
% \affaddr{S2 Grupo}\\
% \email{aesparcia@s2grupo.es}
% }


%\numberofauthors{4} %  in this sample file, there are a *total*
% of EIGHT authors. SIX appear on the 'first-page' (for formatting
% reasons) and the remaining two appear in the \additionalauthors section.
%

%\author{
% You can go ahead and credit any number of authors here,
% e.g. one 'row of three' or two rows (consisting of one row of three
% and a second row of one, two or three).
%
% The command \alignauthor (no curly braces needed) should
% precede each author name, affiliation/snail-mail address and
% e-mail address. Additionally, tag each line of
% affiliation/address with \affaddr, and tag the
% e-mail address with \email.
%
% 1st. author
%\alignauthor
%Jack\\
%       \affaddr{Lost island}\\
%       \affaddr{unknow}\\
%       \affaddr{Pacific Ocean}\\
%       \email{jack_the_doctor@lost.com}
% 2nd. author
%\alignauthor
%Sawyer\\
%       \affaddr{Lost island}\\
%       \affaddr{unknow}\\
%       \affaddr{Pacific Ocean}\\
%       \email{sawyer_tom@lost.com}
% 3rd. author
%\alignauthor 
%Lock\\
%       \affaddr{Lost island}\\
%       \affaddr{unknow}\\
%       \affaddr{Pacific Ocean}\\
%       \email{lock@lost.com}
% 4rd. author
%\alignauthor 
%Hurley\\
%       \affaddr{Lost island}\\
%       \affaddr{unknow}\\
%       \affaddr{Pacific Ocean}\\
%       \email{hugo@lost.com}
%}

%\and  % use '\and' if you need 'another row' of author names
% 4th. author
%\alignauthor Lawrence P. Leipuner\\
%       \affaddr{Brookhaven Laboratories}\\
%       \affaddr{Brookhaven National Lab}\\
%       \affaddr{P.O. Box 5000}\\
%       \email{lleipuner@researchlabs.org}
% 5th. author
%\alignauthor Sean Fogarty\\
%       \affaddr{NASA Ames Research Center}\\
%       \affaddr{Moffett Field}\\
%       \affaddr{California 94035}\\
%       \email{fogartys@amesres.org}
% 6th. author
%\alignauthor Charles Palmer\\
%       \affaddr{Palmer Research Laboratories}\\
%       \affaddr{8600 Datapoint Drive}\\
%       \affaddr{San Antonio, Texas 78229}\\
%       \email{cpalmer@prl.com}
%}
% There's nothing stopping you putting the seventh, eighth, etc.
% author on the opening page (as the 'third row') but we ask,
% for aesthetic reasons that you place these 'additional authors'
% in the \additional authors block, viz.
%\additionalauthors{Additional authors: John Smith (The Th{\o}rv{\"a}ld Group,
%email: {\texttt{jsmith@affiliation.org}}) and Julius P.~Kumquat
%(The Kumquat Consortium, email: {\texttt{jpkumquat@consortium.net}}).}
%\date{30 July 1999}
% Just remember to make sure that the TOTAL number of authors
% is the number that will appear on the first page PLUS the
% number that will appear in the \additionalauthors section.

\maketitle

\begin{abstract}
This paper proposes an evolutionary algorithm based in the fight between individuals, implemented as a selection mechanism. Thus only the winners will survive (pass to the next generation), so the evaluation and fitness calculation needs are omitted. This algorithm has been designed as an optimization method for the parameters of the behavioural engine of bots for the RTS game Planet Wars. So, the fights consist in real battles (inside the game) between different rivals. There are two objectives: first, better deal with the noisy nature of the fitness function (the evaluation for the same individual may vary from one time to another, due to the stochastic component of the combats); and second, obtain more general bots than those evolved considering a specific opponent, which eventually are optimized to fight against it, and so, they are specialized bots. 
In addition, avoiding the evaluation step ideally will reduce the algorithm time consumption.
Four different approaches are proposed and compared, namely steady state and generational implementations, each of them considering 1 vs 1 and 1 vs 3 bots combats. They implement different exploration versus exploitation tradeoffs in order to decide the best balance between these factors. 
\end{abstract}

% A category with the (minimum) three required fields
\category{I.2.1}{Computing Methodologies}{ARTIFICIAL INTELLIGENCE}[Applications and Expert Systems]
%A category including the fourth, optional field follows...
\category{G.1.6}{Mathematics of Computing}{Numerical Analysis}[Optimization]

%D.2.8 [Software Engineering]: Metrics�complexity mea-
%sures, performance measures
\terms{Algorithms}


\keywords{Real-Time Strategy Games, Evolutionary Algorithms, Bots, Fighting-based Evolution}

\end{document}
