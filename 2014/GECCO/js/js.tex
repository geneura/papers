\documentclass{sig-alternate}
\usepackage[latin1]{inputenc}

\begin{document}
%
% --- Author Metadata here ---
\conferenceinfo{GECCO'14} {Vancouver, Canada}
\CopyrightYear{2014}
%\CopyrightYear{2007} % Allows default copyright year (20XX) to be over-ridden - IF NEED BE.
%\crdata{0-12345-67-8/90/01}  % Allows default copyright data (0-89791-88-6/97/05) to be over-ridden - IF NEED BE.
% --- End of Author Metadata ---

\title{Assessing different architectures for evolutionary algorithms in JavaScript}
\subtitle{Abstract}

\author{
\alignauthor
J.J. Merelo, Pedro Castillo, Antonio Mora\\
       \affaddr{University of Granada}\\
       \affaddr{GeNeura, Department of Computer Architecture and Technology, ETSIIT + CITIC}\\
       \affaddr{18071 - Granada}\\
       \email{jmerelo,pedro,amorag@geneura.ugr.es}
\alignauthor
Anna I. Esparcia-Alc�zar\\
\affaddr{S2 Grupo}\\
\email{aesparcia@s2grupo.es}
\alignauthor
V�ctor Rivas-Santos\\
\affaddr{Universidad de Ja�n}\\
\email{vrivas@ujaen.es}
}

\maketitle
\begin{abstract}
After almost fifteen years, JavaScript has finally risen as a popular
language for implementing all kind of applications, from server-based
to rich Internet Applications. Its features are also interesting for
implementing evolutionary algorithm frameworks that encompass both
tiers (client and server), but, besides, they allow a change in paradigm that goes beyond
the canonical evolutionary algorithm. In this paper we will experiment
with different architectures, implementations and evolutionary
algorithms to assess which ones offer the most advantages in terms of
performance, scalability and ease of use for the computer
scientist. All code used in this paper has been released as open source.
\end{abstract}

% A category with the (minimum) three required fields
\category{H.4}{Information Systems Applications}{Miscellaneous}
%A category including the fourth, optional field follows...
\category{D.2.8}{Software Engineering}{Metrics}[complexity measures, performance measures]

\keywords{ACM proceedings, \LaTeX, text tagging}

\section{Introduction}

JavaScript was introduced in 1998 as a browser-oriented language by
Netscape. 


%ACKNOWLEDGMENTS are optional
\section{Acknowledgments}

%Este trabajo se est� desarrollando gracias a la financiaci�n de  los
%proyectos CANUBE (CEI2013-P-14) y ANYSELF (TIN2011-28627-C04-02). 

%
% The following two commands are all you need in the
% initial runs of your .tex file to
% produce the bibliography for the citations in your paper.
\bibliographystyle{abbrv}
\bibliography{geneura}  % sigproc.bib is the name of the Bibliography in this case

\end{document}
