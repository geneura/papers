\documentclass{sig-alternate}
\usepackage[latin1]{inputenc}

\begin{document}
%
% --- Author Metadata here ---
\conferenceinfo{GECCO'14} {Vancouver, Canada}
\CopyrightYear{2014}
%\CopyrightYear{2007} % Allows default copyright year (20XX) to be over-ridden - IF NEED BE.
%\crdata{0-12345-67-8/90/01}  % Allows default copyright data (0-89791-88-6/97/05) to be over-ridden - IF NEED BE.
% --- End of Author Metadata ---

\title{Assessing different architectures for evolutionary algorithms in JavaScript}
\subtitle{}

\author{
% \alignauthor
% J.J. Merelo, Pedro Castillo, Antonio Mora\\
%        \affaddr{University of Granada}\\
%        \affaddr{GeNeura, Department of Computer Architecture and Technology, ETSIIT + CITIC}\\
%        \affaddr{18071 - Granada}\\
%        \email{jmerelo,pedro,amorag@geneura.ugr.es}
% \alignauthor
% Anna I. Esparcia-Alc�zar\\
% \affaddr{S2 Grupo}\\
% \email{aesparcia@s2grupo.es}
% \alignauthor
% V�ctor Rivas-Santos\\
% \affaddr{Universidad de Ja�n}\\
% \email{vrivas@ujaen.es}
\alignauthor
Three wise guys\\
       \affaddr{University of Miskatonic}\\
       \affaddr{Wise affairs dep}\\
       \affaddr{Metropolis}\\
       \email{some.good.e@mails.com}
\alignauthor
Another author\\
\affaddr{Adrress}\\
\email{e@ma.il}
\alignauthor
Last one\\
\affaddr{Science Institute}\\
\email{last@o.ne}
}

\maketitle
\begin{abstract}After almost fifteen years, JavaScript has finally risen as a popular
language for implementing all kind of applications, from server-based
to rich Internet Applications. Its features are interesting for
implementing evolutionary algorithm frameworks 
% Maribel, y por qué? 
that encompass both
tiers, but, besides, they allow a change in paradigm that goes beyond
the canonical evolutionary algorithm. 
%Maribel, muchos frameworks van más allá del algoritmo genético canónico, pero esto no pasa porque estén hechos en javascript
In this paper we will experiment
with different architectures
% Maribel, ¿Cuales?, una arquitectura diferente no es un sistema operativo diferente 
implementations and evolutionary
algorithms to assess which ones offers most advantages in terms of
performance, scalability and ease of use 
% Maribel, lo de la facilidad de uso no hay forma de medirlo a no ser que sea tu propia intuición o experiencia y no vale a no ser que hayas usado otros lenguajes y otras herramientas
for the computer scientist. All implementations  have been released as open source.
\end{abstract}

% A category with the (minimum) three required fields
\category{H.4}{Information Systems Applications}{Miscellaneous}
%A category including the fourth, optional field follows...
\category{D.2.8}{Software Engineering}{Metrics}[complexity measures, performance measures]

\keywords{ACM proceedings, \LaTeX, text tagging}

\section{Introduction}

JavaScript was introduced in 1998 as a browser-oriented language by
Netscape \cite{flanagan1998javascript}; it was quickly adopted, in
several versions, by the rest of the existing browsers (Internet
Explorer, Opera and the offspring of Netscape, Mozilla and then
Firefox). It became a standard by the European Association for
Standardizing Information and Communication Systems in 1999
\cite{ecma1999262}. However, it was for a long time considered just a
language for the browser, and in fact most books on JS
\cite{goodman2007javascript} start by
telling you how to embed your scripts in the browser HTML code.

As such as a embedded language, Javascript is an interpreted language
that uses classes and objects, uses functions as first-rate types and
is dynamic and weakly typed-checked (a variable can chenge its type during
its lifetime, but whatever type it has needs to be explicitly coerced
to other type in some contexts, but not in all). These features make
it an ideal language for quick prototyping and productive
programming. 

However, the graduation of JS to a full-fledged language did not arrive until the first years of
this century with the introduction of command-line interpreters such
as SpiderMonkey or Rhino \cite{mikkonen2007using}, but it was not
until Google's introduction of the V8 \cite{richards2010analysis}
interpreter and its adoption by  {\tt node.js} that it started to
become what it is now, one of the most popular development languages
\cite{ogrady14:ranking}.

What we propose in this paper is to measure different ways of
implementing distributed evolutionary algorithms using Javascript. It
has already been proved that implementation matters
\cite{DBLP:conf/iwann/MereloRACML11}, and it does so since the
evolutionary framework must be translated to a particular language in
a way that goes with its grain and not in the way that is more easily
translated from C or Java, but also because it offers new ways of
implementing evolutionary algorithms.

The experiments proposed in this paper have been released under an
open source license and are available right now (address withheld for
anonymous review). They offer a glimpse into the different
possibilities of programming using Javascript, but at the same time
show the kind of performace we should expect from them and what it
wins from parallel oepration. In doing so, our intention is to prove
that the translation of the usual distributed evolutionary paradigm
into this new language is valid, but also which option offers the best
performance in terms of scalability and number of evaluations needed
to reach the solution.

The rest of the paper is organized as follows: next we describe the
state of the art in new implementations of evolutonary algorithms and
past implementations using different JavaScript interpreters. The
algorithm that has been adapted to JS is described next along with the
experimental setup; finally results are presented and its implications
discussed in the last section of the paper. 

 


%ACKNOWLEDGMENTS are optional
\section{Acknowledgments}

%Este trabajo se est� desarrollando gracias a la financiaci�n de  los
%proyectos CANUBE (CEI2013-P-14) y ANYSELF (TIN2011-28627-C04-02). 

%
% The following two commands are all you need in the
% initial runs of your .tex file to
% produce the bibliography for the citations in your paper.
\bibliographystyle{abbrv}
\bibliography{geneura,javascript}  % sigproc.bib is the name of the Bibliography in this case

\end{document}
