%.%TODO GENERAL, hablar de arquetipos jungianos y de jung. No entremos en psicología pero lo usamos.
%TODO Hablar del paper que decía el señor con alto expertise
%TODO Revisar y cerrar todos los comentarios
%TODO Es importante remarcar que esto no es una receta, es una guía
%TODO podría ser interesante hacer un grafo que relacione todos los conceptos
%TODO terminos: setting, plot, archetype, behavior, backstory, 
%TODO Cada paso explicado en la metodología debe tener un ejemplo
%TODO no usar el término ciclo de vida (no se usa como lo estoy usando)

\documentclass{llncs}
\usepackage[latin1]{inputenc}
\usepackage{graphicx}        % standard LaTeX graphics tool
\usepackage{url} 
\usepackage{subfigure}
\usepackage[english]{babel}

\providecommand{\e}[1]{\ensuremath{\times 10^{#1}}}
\begin{document}
%


\title{Evolving literary backstories for non-player characters}

\author{Rub�n H. Garc�a \inst{1} \and  Pablo Garc�a-S�nchez \inst{2} J.J. Merelo\inst{2}  \and A.M. Mora\inst{2}  } 

\institute{Fidesol\\
\email{rhgarcia@fidesol.org}
\and
University of Granada\\
       Department of Computer Architecture and Technology, ETSIIT\\
       18071 - Granada\\
       \email{\{pgarcia,jmerelo,amorag\}@geneura.ugr.es}
}


\maketitle


\begin{abstract}

Stories are not only painfully weaved by crafty writers in the
solitude of their studios; they also have to be produced massively for
non-player characters in the video game industry or tailored to
particular tastes in personalized stories. However, the creation of
fictional stories is a very complex task that usually implies a
creative process where the author has to combine characters, 
conflicts and backstories to create an engaging narrative.
This work presents a general methodology to generate cohesive and coherent
backstories where desired archetypes (universally accepted literary symbols) can emerge in complex stochastic systems.
This methodology supports the modeling and parameterization of the agents, the environment where they will live and the desired literary setting. The use of a Genetic Algorithm (GA) is proposed to establish the parameter configuration that will lead to backstories that best fit the setting. Information extracted from a simulation can then be used to create the literary work.
To demonstrate the adequacy of the methodology, we perform an implementation using a specific multi-agent system and evaluate the results.

\end{abstract}


%
%%%%%%%%%%%%%%%%%%%%%%%%%%%%%%%   INTRODUCTION   %%%%%%%%%%%%%%%%%%%%%%%%%%%%%%%
%
\section{Introduction}
\label{sec:intro}

Non-Player Characters (NPCs) in video games are introduced  to provide a more inmersive
experience and, in some cases, present a challenge to the human
player. Modern  Role Playing Games (RPGs) include hundreds of
them and their behavior must be designed individually, which is why
they usually present limited range of them such as loitering in
the villages, selling groceries or guarding the cities. Their
conversations and behaviors are often scripted conversations  interacting
with the player only as a reaction to a specific action: for
example, if the player steals something then a city guard would attack
him; they usually do not interact with each other. % Todo esto
                                % deber�ais justificarlo - JJ
 In a world
with such a number of characters, their collective interactions could
improve the gaming experience, leading to a richer and more inmersive
world. For example, hungry inhabitants could become thieves, guards
could pursuit the thieves, villagers could fell in love with others,
or different war alliances could emerge. 
% Pero ten�is que aclarar qu� diablos es MADE. No es para ficci�n
% interactiva. Es para dise�ar historias de personajes que se incluyan
% en un juego, por lo que este �ltimo p�rrafo sobrar�a (y dejar�a
% sitio para otro experimento) - JJ 


%FERGU: HABLAR DE LA METODOLOGIA COMO UNA NARRACION Y DESCRIBIR LITERARY SETTING!
These facts have motivated us to develop a methodology to model the language, agents and literary setting to generate backstories. In this methodology, a set of probabilities and 
states are associated to agents' actions, and these probabilities are
optimized by means of a Genetic Algorithm to match with a
specific literary archetype, defined by the fiction creator. The {\em
archetypes} are behaviors and patterns universally accepted and
present in the collective imaginary
 \cite{ArchetypesGarry05}. They allow to empathize  % mal escrito, y  FERGU: arreglado. Sí, existe.¿existe empathize? - JJ
with the characters and aid to immerse the spectator in the story
(for example, the well-known {\em hero} archetype).




To validate our methodology we also present a multi-agent system called
MADE (Massive Artificial Drama Engine) to model a self-organized
virtual world where every element influences each other, following
cause-effect behaviors in a coherent manner. This system must be
a suitable environment for the plot of a specific literary work, also being
interesting for the player/spectator. 

In this work we investigate whether it is possible to model a virtual environment inhabited by hundred of characters with interesting auto-generated behaviors based on literary archetypes. Also, we test if the personality of the agents can be parametrized to obtain these different emergent behaviors.A Genetic Algorithm will be used to find these adequate parameter values that allow the creation of quality sub-plots.
%Could this methodology be exported to be used in a bigger task such as creating a literary piece or work?



%In this paper we show that GAs, together with a proper design of literary patterns, can be used to find the parameters that promote the generation of drama plots and sub-plots in a multi agent based environment.



The rest of the work is structured as follows: after the state of the art, the proposed methodology is presented in Section \ref{sec:methodology}. Then, some experiments, showing possible applications of this methodology (including the GA application) are presented (Section \ref{sec:applying}). Finally, conclusions and future work are discussed.

%
%%%%%%%%%%%%%%%%%%%%%%%%%%%%%%%   STATE OF THE ART  %%%%%%%%%%%%%%%%%%%%%%%%%%%%%%%
%

% -----------------------------------------------------------------------------
% SEC SOA

\section{State of the art}
\label{sec:soa}

% (JJ2013): Por cierto, acabo de ver esto: \cite{StoryTecGobel2008}.
%TODO

The presented methodology is focused on \textit{generating a setting}
because its aim is the massive background generation
for secondary characters, in order to provide a context for the writer and the
player to perceive a virtual world as coherent, detailed and
enriched.
%In this respect, story generation systems such as Tale-Spin \cite{talespin}, Universe \cite{universe}, Minstrel \cite{minstrel} or Virtual Story Teller\cite{virtualstoryteller} are more aligned with our goal, since they are not focused on the interaction with the user.

%The narrative is addressed by our methodology as the final step, giving freedom to creators. This issue has been studied in the systems presented in
%the survey by Arinbjarnar et al. in \cite{ReviewArinbjarnar09}. 

Previous works (see survey by  Arinbjarnar et al. in \cite{ReviewArinbjarnar09}) define the plot as as something that emerges from the behaviour of the agents that follow a set of rules. In the proposed methodology, the agents' behaviour is produced by their personality and the environment. That is, the agents do not follow a plot, but they generate the plot itself. Also, previous works generate plots in worlds with a limited number of characters, as opposed to our methodology. %In our methodology the number of characters to create is unlimited: they are created massively and their goal is to relate in a complex system and generate backgrounds that fit the setting of a plot, not the plot itself.

The present methodology follows the ideas of the work by Epstein and Axtell
\cite{epstein1996growing}, 
the first widely known multi-agent generative social model. As a step of the methodology, a self-organizing system is defined following the methodology introduced by Gershenson
\cite{gershenson2005general}: a virtual world, agents who are born, grow, 
interact, reproduce and dead; resources (food), mediators, and
relations of rivalry (friction) and cooperation (synergy). In other step of our methodology, the actions of the agents are parametrized according the work of Nairat
\cite{nairat2011character}, based on the use of GAs in order to
obtain a plot (solution) where two characters interact in a creative way.

The proposed methodology is innovative since the goal has not been addressed before. Previous researches are focused on the plot, the interaction, and the narrative, but our proposal is focused on backgrounds (not in plot), where different archetypes emerge involving a starting point for future plots and subplots.

%Antonio: Entiendo que esto es nuevo, no lo que hizo ya el tal Nairat,¿no? Queda un poco lioso y puede parecer que esto ya lo hizo ese menda, cuando yo creo que tú quieres decir que os basásteis en su trabajo para hacer esto, que él no hizo. ;)

According to the taxonomy described by Togelius et al. \cite{Togelius2011}, the present methodology can be classified as a \textit{procedural content generator} (\textit{PGC})
mainly related to \textit{optional content}, with \textit{stochastic generation}
and modelled as a \textit{generate-and-test} algorithm (search based), that
performs the optimizations of the process during the game development
(\textit{offline}). % este p�rrafo deber�a ir a la intro JJ PPSN

%ANTONIO: Echo de menos una mayor defensa de lo bueno que es MADE, que es algo nunca hecho, que avanza mucho el estado el arte (porque no hay mucho, aparentemenet), que abre nuevas líneas de investigación, nu se, dadle bombo. ;D


%
%%%%%%%%%%%%%%%%%%%%%%%%%%%%%%%   METHODOLOGY   %%%%%%%%%%%%%%%%%%%%%%%%%%%%%%%
%

\section{Methodology}
\label{sec:methodology}

The mood, or the atmosphere, of the literary setting is one element in the narrative structure of a piece of literature. It is established in order to affect the reader emotionally and psychologically and provide a feeling for the narrative.
%TODO añadir referencia a The Book of Literary Terms, University Press of New England, 1999


The present methodology defines different steps that will lead the user to obtain secondary characters for his/her desired setting, with coherent backstories consistent with its mood. The result is a system that can automatically generate a setting massively populated with characters and backstories, where different desired behaviours or archetypes emerge from their interactions.

In our approach, the mood of the setting will be modelled as a group of abstract archetypes and different conditions over them. These archetypes, conceptually modelled, would be designed and instantiated as regular expressions over a language used to describe the backstories along with the social relationships between them. Those instantiations will be used by the GA to obtain the fitness of each solution. %The process is iterative and presented as a waterfall model, where each step influences the following one.

The methodology includes the following steps:  modelling (the
language, the literary setting, the agent), definition of the GA
characteristics, instantiation and execution. % No forma
                                % parte del paper, as� que lo he
                                % eliminado JJ PPSN



\subsection{Modelling} % Las subsecciones habr�a que eliminarlas para
                      % conservar espacio. JJ PPSN

Each agent has to be modelled as a Finite State Machine (FSM), whose transitions are based on actions OJO!!!!!!! HAY QUE EXPLICAR LA FSM CON UN PARRAFO!!!!! % whose transitions generate symbols in a language based on the following one, expressed in Backus-Naur Formalism (BNF):

%\begin{verbatim}
%<backstory> ::= <action_line> [<backstory>]
%<action_line> := <date> <action>;
%<action> = <action_id> [<direct_object>]
%           [to <indirect_object>]
%\end{verbatim}

%Starting from this partial grammar, the user has to define the expressions:
%\begin{itemize}
%\item \textbf{<date>}: Should be expressed in a way that is meaningful
%  to the type of background that we are looking for. For example, if we are talking about persons, it could be the age of the agent expressed in days.

%\item \textbf{<action\_id>}: Different actions can be taken into
%  account, and will depend on the kind of archetypes we will look
%  for. For example, if we are trying to find archetypes about love
%  like the classical Romeo and Juliet one, where two characters are in
%  love but belongs to families in war and finally ends up with their
%  death, we could include actions and relations like ``loves'',
%  ``hates'', ``is son of'', ``is daughter of'', ``is parent of'',
%  ``die'', etc. % Tienes demasiado ejemplos. Y no se usa etc - JJ PPSN FERGUPPSN: quitado todo esto

%Obviously, the smaller the group of possible actions is, the easier would be to find out the archetypes. In order to minimize the number of available %actions and maximize the number of archetypes they allow we propose generic actions instead of specific ones.

%\item \textbf{<direct\_object>}: The possible values of the direct objects depend on the nature of the actions.

%\item \textbf{<indirect\_object>}: Should define the unique id of an agent or a group of agents.

%\end{itemize}

The backstory generated by each agent using the language modelled this way will be automatic and human readable. Eventually, the backstories will evoke archetypes and behaviours. As we will see in the next sections, we will try to promote the emergence of specific archetypes in order to retrieve the best backstories for our setting.


%In a top-down approach (the opposite to the one proposed in the current methodology), if the creative needs a second character or an extra for a story, he or she creates it out of the blue, with the background that better fits with his/her needs. This approach, valid for a small amount of characters and poor requisites of coherence between them, becomes more and more complex when we add more characters to the environment, because each addition implies new relations in their social network. FERGUPPSN: quitado esto
%TODO Alguna referencia a la complejidad exponencial de las redes sociales cuando se añade un indivíduo


%Otherwise, the bottom-up approach promotes the agent as one of the pillars of the system because it offers the coherence of the backstory regarding to the timeline (for example, an agent cannot die before he or she is born). Some actions need to be sequential and to the relations (many actions involve two different agents). An agent is just a simple tool to generate a background, a character. In an environment where many characters can ``live'', make decisions taking into account the other characters and affecting them, the generated background becomes complex, many links have been created and no creative process has taken part on it (the creative process is present in the definition and modeling of the agent itself).

%Given this premise, a multi-agent system seems promising for generating this kind of complex relations and backstories.
%FERGUPPSN he quitado un mont�n de parrafazos de c�mo deben ser las metodolog�as y todo ese rollo de esta secci�n, que no interesa.

%In this section we propose an agent's architecture and explain its characteristics.

%\textbf{Agent as a finite state machine}\\


An agent can be seen as a Finite State Machine (FSM), so the multi-agent system relies in the sequential executions of time portions. Every execution implies a review of the current state and (maybe) one or more actions done, depending on local and external properties.\\

Given a set of states, properties and actions modelled, the logic of
the main process should be:  % Esta frase no termina - JJ PPSN

%\begin{enumerate}
%\item Collect internal and external information.
%\item Depending on the state and the information collected, select the possible actions and probabilities to perform them.
%\item Use random numbers to decide the action to perform.
%\item Generate an action line using the language selected in the previous stage.
%\item Select the new state.
%\end{enumerate}
%FERGUPPSN esto tambi�n fuera

The agent should contemplate different states depending on the type of
information we are interested in (for instance, ``alive'', ``dead'', ``pregnant'') and should also have local properties that define the possibility to make certain decisions.
% Y esta frase sobra tambi�n, est� repitiendo lo de arriba - JJPPSN

%It is important to remark that an agent is a simplification of the character (in the way that it will perform only the actions we are interested in) and that the agent is inspired in conceptual behaviours, so some actions and transitions are mandatory and depend on the personality and internal states.\\ FERGUPPSN: esto sobra, como diria Ferran Adria.

%\textbf{The agents' execution environment}

%Agents live in an {\em environment}, understood as the virtual spatio-temporal frame where the agents play their lives. This environment is also responsible for:
%\begin{itemize}
%\item execute the main process of each agent (corresponding to a portion of time). Each iteration the agents should be chosen randomly or based on classical role-playing games features like ``initiative''.
%\item provide a set of functions to interact with other characters and the environment itself.
%\end{itemize}
%FERGUPPSN: Adria: to esto sobra!

%\textbf{Modeling the parameterization of the agent}

%The key in the use of agents to create background for characters relies in the following fact: Some actions are performed statically depending on the inputs and the current state but other (the majority) also depends on agent's local features and probabilities. Even if all the agents share these values, it is difficult to predict the result when the system is so complex. A minimal change in one probability to make one decision can imply completely different scenes.
%For this reason, some actions should be statically defined and others have to depend on initialization properties. We define two kind of properties:
%TODO revisar y añadir referencia a sistemas complejos, seguro que alguien más listo que yo ya ha dicho esto antes que yo


%\begin{itemize}
%\item \textbf{Base properties}: Those that are intrinsically defined by the nature of the conceptual agent. For example, if we are modeling a simplification of a person, the average life expectancy could take values from 70 to 85 years. An agent that lives 100 years would be unusual, and an agent that lives 200 years should not be possible.
%\item \textbf{Searchable properties}: Those used to increment or decrement the base values in the established limits or those that are used directly as probabilities to make decisions.
%\end{itemize}

%FERGUPPSN: Ferran Adria! TODO::::: buscar searchable properties y base properties!


%Due to the complexity of the system and the introduction of probabilities, even if all the agents have the same features and probabilities, the sequence of the actions for every agent becomes unpredictable. For example, if we have our agent modelled and we choose random values for the configuration, roughly, the characters' backstories will be ``normal'', showing more or less the stereotyped behaviors (that make the character a group representative rather than an individual), but maybe, some of them show higher level non-modelled behaviors. In the next subsection, we will provide a way to model those high-level behavior in the way that, given a executed environment, a computer is able to find them  and, in the Execution step, obtain a configuration that promote the apparition of these archetypes in the execution.

%TODO tener en cuenta a lo largo del texto que actions tb pueden ser predicates

%The setting of a story is defined as the historical moment in time and geographic location in which it takes place, and helps initiate the main backdrop and mood for a story.
%Conceptually, the setting of a story is, in this methodology, independent from the agent design, but has to agree with it in the selected language: 
%The agent generates actions in a previously defined language with a clear semantic interpretation. On the other hand, the setting is composed by criteria over patterns for this language and the social networks derived from its usage. In other words, the agents create backstories and the settings matches specifics patterns inside these backstories.

%For Example, if we have a medieval storytelling setting and we have defined a language where the actions ``love'', ``hate'', ``attack'', ``defend'', ``win'' y ``lose'' are used, we could try to find classic 
%TODO citar de donde he sacado que son clásicos
%archetypes like the ``villain'' and the ``hero''. A villain could be a character that hate many others, attacks them and win. A hero could be a character who is loved by many people, that eventually defends them and then attacks to the villain. Moreover, if the mood of the story is sentimental, we could be interested in a ``heartbreak and reencounter'' archetype, where two characters are in love, after that they hate each other and finally fall in love again.

%In our methodology, an archetype is designed as a function that receives the backstory of an agent, processes it, tries to find patterns and interpret the social network related to the agent, and returns a /true/ value if the agent matches the behavior. 
%TODO dibujo intuitivo: la ambientación te ofrece una función cuya entrada los los agentes
%The complexity of the agent is not evaluated, just the result of its execution in this environment. It is important to remark that the agent has to be modelled to play a normal life (stereotype), and that the archetypes work as high level behaviors not directly implemented.
%FERGUPPSN: Adria!

A setting can be seen as a function over the agents that matches an archetype. If the archetypes emerge in the desired way, the setting function would return high values. In the next sub-section, we will explain the mechanism that will optimize the agent's searchable features to obtain backstories coherent with the mood of the setting. %FERGUPPSN: extender un poco esto, si eso...

\subsection{Features of the Genetic Algorithm}

%The Genetic Algorithm needs:
%\begin{itemize}
%\item a chromosome that codes the solution.
%\item a fitness function to evaluate each chromosome.
%\end{itemize} %FERGU: quito esto, que ya se sabe!

In this work, in individual represents a whole environment where 
a number of different ``agents'' are living. So, an ``individual'' in
the GA is not equal to an ``agent''.  %FERGU: dejo esto claro para que
                                %no se lien los revisores
The properties of this environment are coded in a chromosome, and its
fitness will result from the running of the enviroment a
representative number of times. % se deber�a decir cu�l es ese
                                % n�mero. 

% The codification of the chromosome consists in mapping the
% searchable properties to an array of values. %This array of values
% can be the chromosome.\\ quitado por JJPPSN

%The fitness function of a chromosome is the result of calculating the
%average of the application of the setting function over \textit{n}
%executions of the environment with defined base properties (where
%\textit{n} is the minimum number that reduces the error and has to be
%calculated empirically). % P�rrafo redudico arriba. 



%TODO dibujo aquí
% No us�s "we remark" ni "we clarify". Di las cosas y punto. 

One of the properties of the enviroment is the number of {\em
  profiles} that will be used to represent an agent; different agent
profiles will yield different behaviors. The chromosome that
represents an environment will be a sequence of {\em genes}
representing different profiles. % Por favor, mirad esto a ver si est�
                                % bien - JJPPSN 
% In this context, a profile is a set of properties assigned to an agent. If two agents have different profiles, their behaviour in the face of the same inputs can be different. In the chromosome, the use of \textit{n} profiles means a chromosome size equal to the number of searchable properties multiplied by \textit{n} 

Intuitively, increasing the number of profiles will lead to richer
backgrounds, but \textit{a priori}, since this is a complex system, it is
very difficult to establish the number of profiles will lead to the
fittest solution without empirical tests. In some cases, a small
number of profiles can generate many different archetypes and
\textit{vice versa}. There is a tradeoff in using a bigger number of
profiles: a bigger search space, which will make convergence slower. 

% He revisado casi todo esto. En un congreso de algoritmos evolutivos,
% convendr�a meter aqu� un gr�fico de una evoluci�n t�pica del
% fitness. - JJ
%TODO revisar todos los pasos y poner explćitamente que si no funciona bien hay que volver patrás

\subsection{Instantiation and execution}

After the Genetic Algorithm is configured, some properties need to be established in order to fit the desired setting:
\begin{itemize}
\item Base properties: including the threshold for the agent's features, and environment parameters (i.e. size of the world, resources, etc).
\item Genetic Algorithm parameters: selection, genetics operators (crossover and mutation and termination condition)
\item Number of profiles
\end{itemize}

Once the parameters for the GA, the environment, the agent and the number of profiles are set, the GA can be executed.


In some cases, the fitness can be improved by modifying base parameters, the evaluation of the archetypes or the logic of the agent. The process is iterative and a fine tunning is essential to obtain the best fitness.

Once the best solution has been found, the environment can be executed and the backstories generated can be used.

% -----------------------------------------------------------------------------
% SEC APPLY


\section{Applying the methodology}
\label{sec:applying}

To validate our methodology we apply its steps using a specific scenario and environment. In this scenario we want to model the next story: 
 - ``A number of rats live, eat, reproduce, compete for food and death within the walls of the Invisible University of Ankh-Morpork\footnote{This is our tribute to writer Terry Pratchett, whose books have inspired us to create these agents.}. As the University professors, these rats are very vindictive and territorial.''

\subsection{Agents in the MADE environment}

The first step is to model the agents and their environment. In this work we propose the MADE (Massive Artificial Drama Engine for non-player characters) environment, a virtual place where different agents play their artificial lives. Its functions are to initialize agents in the map, control the time, execute the agents during a time unit (for example, a day in the story) and update the map. %FERGUPPSN: resumido to el chorro que viene ahora

%\begin{itemize}
%\item \textbf{Create an initial set of agents:} MADE environment
%  initializes a set of just born orphan agents, each of them with a profile
%  sequentially assigned, that must compete or collaborate in order to survive.
%\item \textbf{Place agents in the map:} the environment is a squared map, formed by cells that can be occupied by one (and only one) agent. The environment allows the agents to discover and interact with other agents in the neighborhood.
%\item \textbf{Start and control the time:} after the creation of the initial set of agents, the MADE environment starts the timer, day by day until a maximum date is reached.
%\item \textbf{Execute each agent during a time unit (a day):} In each iteration the list of agents is randomly reordered, and after that following the new order, each agent perform an iteration of its life-cycle. Then the dead agents are removed from the grid.
%\item \textbf{Perform as an external agent that changes the environment:} In each iteration in the MADE environment, food rations are placed in random cells. An agent only can eat if it is over a cell with a ration, so agents could move any others forcibly.
%\item \textbf{Offer services to the agents:} MADE environment allows the agents to check which neighbor cells have food, are occupied, who agents are in a near position or which positions can be occupied.
%\item \textbf{Decide the agents' profile:} MADE allows the existence of different agent profiles, as previously said. A {\em profile} is a set of characteristics which governs the agent's behavior.
%\end{itemize}

The MADE environment can be configured by using the following parameters (with the values that will be used in the experimental section): Number of agents initially placed (15), map square grid dimension (10), number of rations randomly placed in the grid each day (10) and duration in virtual days of the  execution of the environment (1000). These parameters can affect directly to the behaviour of the agents.


A MADE Agent lives in a MADE Environment, occupies a cell in the grid,
moves around looking for food or mates and interacts with other
agents; it is defined by a set of probabilities that define its {\em
  living thing} behaviour. In this paper we model a virtual rat with  four states (be alive, be hungry, look for
mate and be pregnant) and seven actions (move, eat, attack, defend, escape,
find mate and have offspring) that lead to a basic instinctive animal
behaviour, very useful for this work since it can be the canvas of 
complex \textit{humanized} behaviour patterns.
It is important to remark that no ``feelings'' and no ``memory''
have been modelled in the MADE agent for this study. % �Si no hay
                                % memoria, c�mo hay venganza? - JJ


Every
decision made by the agent is based on its state and its
characteristics (probabilities to perform different actions). 





The MADE Agent is created using twelve parameters, that define its
base features and probabilities. The execution of an
agent is dynamic, and depends on the internal probabilities and states
but also on the neighborhood, and the map configuration. Even so, we
can say that these initial parameters define in some way the possible
situations where the agent could be involved.


The source code of the MADE environment and the algorithms used in this experiment are publicly available in \url{https://github.com/raiben/made} under a LGPL license.

\subsection{Definition of the literary setting}

As previously said, agents are independent of the desired literary settings. To validate our approach two different literary settings are going to be tested.

The first literary scene is called ``revenge'' and its goal is to model an individual complex memory
based behaviour between two characters. This scene is performed to make more complex memory based behaviour emerging between two characters:  It tries to find the number of profiles and values which are optimal to make \textit{revenge} archetype emerge in as many agents as possible after 1000 days.  An agent (a) will be considered as a \textit{avenger} if it has been attacked by other agent (b) and after that, in a moment in its life, it has satisfactory attacked the agent b, in \textit{revenge}. The value of the days is set to 1000 because is a duration long enough to make the archetype emerge.

The secondary literary scene, ``territorial war'', aggregates different sample archetypes where many factors must be taken into account.  It tries to find the number of profiles and values that generate at the end of the run an equally distribution of the archetypes \textit{downtrodden} (an agent that has been attacked at least two times and has defended the position), \textit{warrior} (if it has satisfactory attacked at least five times), \textit{helpless} (if it has been attacked at least ten times and has not defended the position) and \textit{bad warrior} (if it has unsatisfactory attacked at least ten times). Also,  after 1000 virtual days, the alive population will be the 60\% of the total population (archetype \textit{survival population}). We have used the presented values to define this scene because, in our opinion, they model an interesting literary setting.
%FERGUPPSN: reducido lo de abajo arriba

%\begin{itemize}
%\item Ensure that, after 1000 virtual days, the alive population will be the 60\% of the total population. This archetype, called \textit{survival population}, affects all the population, so it is a \textit{global archetype}.
%\item Emerge the \textit{downtrodden} archetype in the 22\% of the
%  population. An agent will be considered as a \textit{downtrodden} or
%  \textit{defender} if it has been attacked at least two times and has
%  defended the position.
%\item Emerge the \textit{warrior} archetype in the 22\% of the population. An agent will be considered as a \textit{warrior} if it has satisfactory attacked at least five times.
%\item Emerge the \textit{helpless} archetype in the 22\% of the population. An agent will be considered as a \textit{helpless}  if it has been attacked at least ten times and has not defended the position.
%\item Emerge the \textit{bad warrior} archetype the in 22\% of the population. An agent will be considered as a \textit{bad warrior}  if it has unsatisfactory attacked at least ten times.
%\end{itemize}



The archetypes of both scenes can be searched in the set of generated logs from all agents at the end of a run.

\section{Experiments and results}

%In this work, we have implemented a method based on regular expressions with backreferences. The proposed technique puts annotations in every agent whose log matches a complex regular expression able to find emerging high level behaviours, not implemented in the life-cycle.
%FERGUPPSN: quitadas movidas destas

In this proposal, the parameters used to define an agent, mentioned in Section~\ref{sec:methodology}, are mapped into a chromosome, and a Genetic Algorithm is used to evolve the solution. %The fitness function is expressed in terms of:

%\begin{itemize}
%\item \textbf{Regular expressions applied to the log of each agent in the environment:} An agent is tagged when a regular expression matches its log.
%\item \textbf{A numeric function over the number of tagged agents for each archetype:} the fitness of the solution is incremented with the returning value.
%\end{itemize}

%FERGUPPSN: quitado

Thanks to the agents' logs, we can know every event (internal and
external) of their lives, so we can use it to evaluate the adequacy
of the generated parameters for the environment with our desired literary setting.

If only one profile is used
in a run, all the agents are created with the same parameters, evolved
by the Genetic Algorithm. If more profiles are used, they are assigned
to the agents in order of appearance in a loop. Our assumption is that
some archetypes could emerge using one profile and other will need
more (those that require two clearly differentiated roles). It is
important to remark that the number of alleles of the chromosome are
multiplied by the number of profiles, so the convergence of the
solution will be be affected by the number of profiles used.

%In both cases we do not exactly know how many roles are necessary to create a world of ``warrior'' or ``vindictive''
%  rats, so different number of profiles (from one to five) will be considered.

For the experiments performed in this work, we have used the parameters shown in Table~\ref{fig:ga_parameters}. These values have been chosen empirically after several test runs.

\begin{table}[htb]
\begin{center}
\caption{Parametrization of the Genetic Algorithm}
\label{fig:ga_parameters}
\begin{tabular}{cc}%{p{3cm}p{7cm}}
\hline\noalign{\smallskip}
\noalign{\smallskip}
Parameter & Value \\
\hline
\noalign{\smallskip}
Codification & 12 alleles per profile\\
Fitness function & Average of 10 executions.\\
Natural selector & Original Rate: 0.9 \\
Crossover operator & Rate: 35\% \\
Mutation operator & Desired Rate: 12 \\
Stop condition & 100 generations\\ % �A qu� se refiere? Esto confund�a
                                % tambi�n al revisor - JJ PPSN FERGU: generations. Culpa mia
Population size & 30 \\
\hline
\end{tabular}

\end{center}
\end{table}

%and evaluate their interest or
%adequacy to a specific literary setting.
                                % interest in
                                % what? Quieres decir si son
                                % interesantes? - JJ2013 FERGUPPSN: reescrito arriba

\subsection{Simple literary scene: revenge} 
For the first literary setting (``revenge''), every agent whose log matches the archetype adds one point to the fitness. Therefore, the goal is to recreate an environment where several ``avenger'' agents exist.

The results of the first setting are shown in
Table~\ref{fig:tabresults}. It shows the average of the best fitness
and the average population fitness at the end of 30 executions for
each configurations. Boxplots of the best fitness obtained are shown
in Figure \ref{fig:subfig2}. In this case, Kruskal-Wallis and Wilcoxon
pairwise comparison shows significant differences among all
configuration (p-value $<<$ 0.05) except between P2 and P3
(p-value=0.3). Therefore, we can conclude that in this kind of global
archetype only a profile must be used for obtaining the best
results. This makes sense, because we are looking for one type of
local archetypes ({\em avenger}), so adding extra profiles leads to
different behaviours of the agents. 
% si vas a analizar los resultados, deberías mostrar tambień cómo
% cambia el mejor fitness y el intermedio, para que se pueda ver si el
% AG se hace bien o mal o regular. - JJ



\subsection{Complex literary scene: territorial war}

To model the ``territorial war'' setting we have defined the fitness function as follows: if the exact percentage of agents are tagged with one archetype defined in previous subsection, 1 point is added to the fitness. The maximum is therefore, 5 points. However, all the fitnesses use a normal distribution over the percentage of appearance. For example, with the archetype ``survival population'', the maximum value (1) is obtained when the 60\% percent of the population is alive, and the normal distribution begins in the 30\% and ends in the 90\%. For the rest of the archetypes, 1 point is added if the 22'5\% of the population is tagged with each one of the archetypes, and each normal distribution begins in the 8\% and ends in the 30\%.

Table~\ref{fig:tabresults} shows the average of the best fitness and the average population fitness at the end of 30 executions for each configuration: number of profiles from 1 (P1) to 5 (P5).
 We have performed a Kruskal-Wallis test for the
best individuals fitness, obtaining differences among all the number
of profiles (p-value $<<$0.05). As we suspected, it is clear that
using one profile is not enough to emerge the desired
archetype. However, the pairwise comparison using Wilcoxon does not
find significant differences  between the results that use more than
two profiles. This can be explained because an agent could share more
than one archetype at the same time.  A promising number of profiles
could be 4, because their only lower outlier is not as distributed as
the others.  



% Our assumption is that
% some archetypes could emerge using one profile but others may need
% more (those that require two clearly differentiated roles).

\begin{table}
\begin{center}
\caption{Results for 30 executions of each configuration using 1 to 5 profiles (average best fitness $\pm$ std. dev).}
\label{fig:tabresults}
\begin{tabular}{|c|c|c|}
\hline
Number of profiles& Setting 1 & Setting 2 \\
\hline
1 & 495,513 $\pm$ 20,091 & 0,765 $\pm$ 0,037\\
2 & 471,206 $\pm$ 24,550 & 1,063 $\pm$ 0,115 \\
3 & 455,42 $\pm$ 28,240 & 1,093 $\pm$ 0,063 \\
4 & 431,926 $\pm$ 31,682 & 1,084 $\pm$ 0,048 \\
5 & 411,24 $\pm$ 25,023 & 1,045 $\pm$ 0,110 \\
\hline
\end{tabular}
\end{center}
\end{table}




\begin{figure}[htb]
\centering
\subfigure[Setting 1]{
   \includegraphics[scale=0.32] {img/exp2_v2.pdf}
   \label{fig:subfig2}
 }
\subfigure[Setting 2]{
   \includegraphics[scale=0.32] {img/exp1_v2.pdf}
   \label{fig:subfig1}
 }
\caption{Average fitness of the 30 best individuals of the GA for each
  configuration.}
% Los gr�ficos deben de tener siempre el origen en 0, si no se
% exageran las diferencias. En el primero, por ejemplo, parece que hay
% una diferencia bestial cuando en realidad no hay ninguna - JJ

\label{fig:graph}
\end{figure}


\section{Conclusions}
\label{sec:conclusion}


This work presents a general methodology to design emergent literary stories in massive virtual worlds. The described steps include the modelling of the agents and literary setting. Then a Genetic Algorithm is used to optimize the parameters of the agent's profiles (behaviours) using
as fitness a function that models the literary setting. 

This implies that each scene or combination of desired archetypes for a given
environment can be mapped to a number of profiles that maximize the appearance
of these archetypes, and that the fittest number of profiles and its values can
be obtained by using Genetic Algorithms.  
Given a literary setting, an author of a story or a video game could define 
different rates of archetypes or behaviour patterns, and use the techniques described in
the present work to obtain the optimal profiles. The execution of the MADE (Massive Drama Engine for non-player characters) Environment
using these profiles as input would produce a background (or set of characters' lives)
where the archetypes have emerged and have automatically created massive backstories coherent
with the settings of the artwork.

In future works, more complex agents will be used, with different rules to be modelled. For example, we plan to model more human behaviours such as love or envy, to generate interesting plots such as wars, weddings, or family crimes. Different fitness functions will be used, for example, taking into account human opinions to establish the interestingness of a generated plot. Also, this system will be tested into an existent and well-known game, such as Skyrim\texttrademark, whose AI engine is publicly available for players and researchers.

% -----------------------------------------------------------------------------
% SEC ACKNOWLEGGEMENTS


\section{Acknowledgments}
This work has been supported in part by FPU research grant AP2009-2942
and TIN2011-28627-C04-02. 


\bibliographystyle{splncs03}
\bibliography{made}  


\end{document}
