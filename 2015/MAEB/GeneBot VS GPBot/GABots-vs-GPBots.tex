\documentclass[twocolumn]{maeb2015}
\usepackage[spanish]{babel}
\usepackage[latin1]{inputenc}

\def\BibTeX{{\rm B\kern-.05em{\sc i\kern-.025em b}\kern-.08em
    T\kern-.1667em\lower.7ex\hbox{E}\kern-.125emX}}


\begin{document}

\title{Algoritmos Gen�ticos contra Programaci�n Gen�tica para la creaci�n de Bots de Planet Wars} %!PN

\author{Nombre del Autor\thanks{Direcci�n del autor.
E-mail: autor@xx.xx.xx .}}

\maketitle

\begin{abstract}
Este trabajo establece una comparativa entre dos m�todos basados en Computaci�n Evolutiva para el dise�o y mejora de agentes aut�nomos (Bots) para jugar al juego de estrategia en tiempo real (RTS) Planet Wars.
En concreto, se analiza un enfoque basado en la mejora de una M�quina de Estados Finitos (MEF), definida en base al conocimiento de un experto, mediante la aplicaci�n de un Algoritmo Gen�tico (AG) que evoluciona el valor de los par�metros de los que �sta depende en sus transiciones.
Por otra parte, se considera una aproximaci�n basada en la aplicaci�n de Programaci�n Gen�tica (PG), para la generaci�n del motor de comportamiento completo del Bot, componiendo con esta t�cnica un sistema de reglas o �rbol de decisi�n, en el que se apoyar� el Bot para comportarse en el juego.
El an�lisis se realiza a varios niveles, comparando en primer lugar la efectividad de cada propuesta al enfrentar a los mejores individuos contra rivales de la literatura. Adem�s, se estudia el tipo de motor obtenido mediante PG, y se comparan las decisiones que podr�a tomar un bot usando estas reglas, en relaci�n a la forma de actuar que tendr�a el mismo bot usando la MEF del primer caso.
\end{abstract}

\begin{keywords}
AG, Algoritmo Gen�tico, PG, Programaci�n Gen�tica, Bot, NPC, RTS, Inteligencia Computacional, Inteligencia Artificial, Reglas, M�quina de Estados Finitos
\end{keywords}

%******************************************************************************

\section{Introducci�n}


%******************************************************************************

\section{Estado del Arte}

\begin{figure}[hbt]
\begin{center}
\setlength{\unitlength}{0.0105in}%
\begin{picture}(242,156)(73,660)
\put( 75,660){\framebox(240,150){}}
\put(105,741){\vector( 0, 1){ 66}}
\put(105,675){\vector( 0, 1){ 57}}
\put( 96,759){\vector( 1, 0){204}}
\put(105,789){\line( 1, 0){ 90}}
\put(195,789){\line( 2,-1){ 90}}
\put(105,711){\line( 1, 0){ 60}}
\put(165,711){\line( 5,-3){ 60}}
\put(225,675){\line( 1, 0){ 72}}
\put( 96,714){\vector( 1, 0){204}}
\put( 99,720){\makebox(0,0)[rb]{\raisebox{0pt}[0pt][0pt]{\tenrm $\varphi$}}}
\put(291,747){\makebox(0,0)[lb]{\raisebox{0pt}[0pt][0pt]{\tenrm $\omega$}}}
\put(291,702){\makebox(0,0)[lb]{\raisebox{0pt}[0pt][0pt]{\tenrm $\omega$}}}
\put( 99,795){\makebox(0,0)[rb]{\raisebox{0pt}[0pt][0pt]{\tenrm $M$}}}
\end{picture}
\end{center}
\caption{El texto de las figuras va tras la figura.}
\end{figure}

\begin{table}[htb]
\caption{El texto de las tablas va antes de la tabla.}
\begin{center}
{\tt
\begin{tabular}{|c||c|c|c|}\hline
&title page&odd page&even page\\\hline\hline
onesided&leftTEXT&leftTEXT&leftTEXT\\\hline
twosided&leftTEXT&rightTEXT&leftTEXT\\\hline
\end{tabular}
}
\end{center}
\end{table}

%******************************************************************************

\section{Planet wars}


%******************************************************************************

\section{Enfoques evolutivos de generaci�n de Bots}


%---------------------------------------------------------

\subsection{GeneBot. Algoritmo Gen�tico sobre MEF}


%---------------------------------------------------------

\subsection{GPBot. Programaci�n Gen�tica}


%******************************************************************************

\section{Comparativa de resultados}

%---------------------------------------------------------

\subsection{Test contra rivales}

%---------------------------------------------------------

\subsection{Comparativa de Motores de comportamiento generados}

%******************************************************************************

\section{Conclusiones}


%******************************************************************************

\section*{Agradecimientos}

\nocite{*}
\bibliographystyle{maeb2015}

\begin{thebibliography}{1}
\bibitem{LaTeX}
Leslie Lamport,
\newblock {\em A Document Preparation System: \LaTeX, User's Guide and
  Reference Manual},
\newblock Addison Wesley Publishing Company, 1986.
\bibitem{LaTeXD}
Helmut Kopka,
\newblock {\em \LaTeX, eine Einf\"uhrung},
\newblock Addison-Wesley, 1989.
\bibitem{TeX}
D.K. Knuth,
\newblock {\em The {\rm T\kern-.1667em\lower.7ex\hbox{E}\kern-.125emX}book},
\newblock Addison-Wesley, 1989.
\bibitem{METAFONT}
D.E. Knuth,
\newblock {\em The {\rm METAFONT}book},
\newblock Addison Wesley Publishing Company, 1986.
\end{thebibliography}
\end{document}
