\documentclass[twocolumn]{maeb2015}
\usepackage[spanish]{babel}
\usepackage[latin1]{inputenc}

\def\BibTeX{{\rm B\kern-.05em{\sc i\kern-.025em b}\kern-.08em
    T\kern-.1667em\lower.7ex\hbox{E}\kern-.125emX}}


\newtheorem{theorem}{Teorema}

\begin{document}

\title{Ejemplo de trabajo para el congreso \\MAEB 2015} %!PN

\author{Nombre del Autor\thanks{Dirección del autor.
E-mail: autor@xx.xx.xx .}}

\maketitle

\begin{abstract}
Este es el resumen del trabajo.###Escribir
\end{abstract}

\begin{keywords}
primera palabra clave, segunda palabra clave, tercera palabra clave###Escribir
\end{keywords}

\section{Introducción}
El fichero de estilo {\tt maeb2015.cls} puede usarse en combinación con el fichero de estilo para bibliografía {\tt maeb2015.bst}. Recuerde incrustar las fuentes en el PDF. ###Eliminar


La predicción en tiempo real del tráfico de vehículos por carretera o ciudad se ha convertido en una necesidad cada vez más demandada tanto por las administraciones como por los propios usuarios~\cite{Min2011606}. Por este motivo, cada vez existen más mecanismos que permiten medir el volumen del tráfico, generando una vasta cantidad de información pendiente de ser debidamente tratada y analizada para obtener predicciones cada vez más fiables. Actualmente, la posibilidad de obtener datos en tiempo real sobre volumen de tráfico o velocidades medias en numerosos puntos de las redes viales es ya un hecho; sin embargo, la toma de decisiones acerca de las medidas a tomar y la predicción de en qué modo aumentarán o disminuirán ambas variables sigue siendo una tarea que llevan a cabo tanto los gestores de las salas de control de tráfico como los propios conductores.

Durante los tres últimos años, gracias al proyecto SIPESCA, se ha desarrollado y puesto en marcha una nueva técnica para estimar el número de vehículos que circulan por una vía, así como su velocidad. En concreto, se han ubicado en distintos puntos de Andalucía (tanto en ciudad como en autovía) unos sensores capaces de detectar, almacenar y transmitir el identificador único de los dispositivos Bluetooth de los vehículos que circulan próximos a cada sensor. La base de datos que se ha generado (y que sigue mejorándose y ampliándose) es ahora un recurso de enorme valor desde el cual estamos realizando distintos tipos de estudio, tanto por su capacidad de proporcionar información en tiempo real como por las posibilidades que ofrece para realizar predicciones.

Partiendo de los datos recopilados por el proyecto SIPESCA, este trabajo presenta los resultados obtenidos al aplicar distintos métodos de predicción de series temporales (entre ellos {L-Co-R} del cual somos autores \cite{LCOR}) para la predicción a muy corto plazo de volumen de tráfico. Concretamente, el trabajo se centra en la predicción a 15 minutos, 30 minutos, 45 minutos y 1 hora. La motivación para usar concretamente estos tiempos es la misma que la indicada en \cite{Min2011606}: por un lado, las oficinas de gestión de tráfico necesitan actualizar de forma dinámica la señalización y mensajes dirigidos a ordenar correctamente el tráfico, y para ello deben basarse en las condiciones previstas para el tráfico en un futuro cercano, no en predicciones hechas con mucha anterioridad y que pueden estar totalmente obsoletas. Por otro lado, los propios conductores solicitan cada vez más información y predicciones actualizadas en el mismo instante en que están desarrollando la actividad de conducir, mostrando poca confianza en las condiciones que existían cuando planificaron el trayecto. Este segundo aspecto se ha demostrado especialmente importante en trayectos de duración igual o superior a 10 minutos.

Uno de los aspectos más innovadores que introducimos en este trabajo es la selección del conjunto de valores pasados que utilizaremos para construir el modelo con el que realizar la predicción. En este sentido, hemos considerado solamente los datos recopilados en las últimas 24 horas. Los motivos son dos: en primer lugar, permite tratar el problema bajo el paradigma correcto de la predicción de series temporales, esto es, utilizar un conjunto de datos consecutivos para realizar una predicción más o menos diferida en el tiempo\footnote{Este importante aspecto fue puesto de manifiesto por los ganadores de la competición de predicción de series temporales realizada en el simposio SICO, integrado dentro del congreso CEDI 2010 realizado en Zaragoza)}. En segundo lugar, nos permite situar un punto de referencia para futuros trabajos ya planificados en los cuales la construcción del modelo predictor se realizará en dispositivos de poca capacidad de cálculo y almacenamiento, los cuales cooperarán en un entorno distribuido logrando mejorar las soluciones usando un enfoque colaborativo. Estos trabajos se enmarcan dentro de una línea de investigación ya iniciada \cite{ jsEO, nodeEO, etc.} y permitirán la descentralización del proceso de creación de los modelos predictores al mismo tiempo que la difusión de los mismos.

El resto del trabajo...###



\section{Estado del arte}
Existen numerosos ejemplos en la literatura en los que el problema de predicción de tráfico a 15 minutos ha sido tratado. En la mayor parte de ellos, el método utilizado ha sido ARIMA \cite{ARIMA}, como por ejemplo en Smith et al. (2002) donde se compara la técnica del vecino más cercano con los modelos autorregresivos. Un trabajo más reciente es el correspondiente a Chandra and Al-Deek (2009) que comparte con el anterior el estar centrado en los modelos ARIMA, y a su vez ambos son similares al ya citado \cite{Min2011606} en cuanto a la metodología utilizada, consistente en ###

Desde hace una década, es posible encontrar trabajos en los que distintos tipos de redes neuronales han sido empleados para predecir problemas similares. Así, los trabajos de van Lint et al., 2005, Vlahogianni et al., 2005 and Zheng et al., 2006). Este último,  Zheng et al. (2006) realiza un enfoque muy similar al nuestro, capturando los datos en zonas puntuales y tratando de predecir el volumen de tráfico de los 15 minutos que siguen al último dato registrado. No obstante, los autores aunque exclyen de la captura de datos las horas correspondientes a la noche, así como los fines de semana.

###Hablar de que no tratan los datos como ST y describir bien el de Min

%Kamarianakis and Prastacos (2003) estimate parameters in a model that takes into account both spatial and temporal correlations across the road network. While the basic form of their model has some similarity to ours, significant differences exist. Their model requires estimating a very large number of parameters, and yet does not take into account several important characteristics of a transportation network. In particular, they assume that the spatial correlations are represented by a fixed set of matrices, which depend upon the distances between links. However, on a transportation network, depending upon whether a link is congested or not, the other network links influencing its traffic flow will vary considerably. This is not captured by the approach of Kamarianakis and Prastacos (2003). Furthermore, the method proposed by the authors in Kamarianakis and Prastacos (2003) assumes stationarity of the system. While the authors note that the traffic flow parameters are clearly not stationary over the time period being modeled, they propose to perform differencing of the data points, with a differencing period of 1 day. This does not, however, deal with inter-day fluctuations, which should violate the stationarity assumption and introduce non-negligible bias into the estimated parameters. A more recent work by the authors makes use of GARCH models to handle the fact that variance in the data is different at peak and off-peak times; however, the accuracy achieved was quite poor. A line of references by Wang et al. makes use of macroscopic traffic flow modeling for real-time traffic prediction. That approach can handle only segments of expressways and while no numerical accuracy is provided by the authors, the graphics do not suggest a level of accuracy near that which is provided by our method (see Wang et al., 2007 and references therein).

\subsection{Esta es una subsección}
\begin{figure}[hbt]
\begin{center}
\setlength{\unitlength}{0.0105in}%
\begin{picture}(242,156)(73,660)
\put( 75,660){\framebox(240,150){}}
\put(105,741){\vector( 0, 1){ 66}}
\put(105,675){\vector( 0, 1){ 57}}
\put( 96,759){\vector( 1, 0){204}}
\put(105,789){\line( 1, 0){ 90}}
\put(195,789){\line( 2,-1){ 90}}
\put(105,711){\line( 1, 0){ 60}}
\put(165,711){\line( 5,-3){ 60}}
\put(225,675){\line( 1, 0){ 72}}
\put( 96,714){\vector( 1, 0){204}}
\put( 99,720){\makebox(0,0)[rb]{\raisebox{0pt}[0pt][0pt]{\tenrm $\varphi$}}}
\put(291,747){\makebox(0,0)[lb]{\raisebox{0pt}[0pt][0pt]{\tenrm $\omega$}}}
\put(291,702){\makebox(0,0)[lb]{\raisebox{0pt}[0pt][0pt]{\tenrm $\omega$}}}
\put( 99,795){\makebox(0,0)[rb]{\raisebox{0pt}[0pt][0pt]{\tenrm $M$}}}
\end{picture}
\end{center}
\caption{El texto de las figuras va tras la figura.}
\end{figure}

\begin{table}[htb]
\caption{El texto de las tablas va antes de la tabla.}
\begin{center}
{\tt
\begin{tabular}{|c||c|c|c|}\hline
&title page&odd page&even page\\\hline\hline
onesided&leftTEXT&leftTEXT&leftTEXT\\\hline
twosided&leftTEXT&rightTEXT&leftTEXT\\\hline
\end{tabular}
}
\end{center}
\end{table}

\subsubsection{Entornos}
\begin{theorem}[Nombre del teorema]
Considere el sistema
\begin{equation}
\begin{array}{rrr}
\dot x&=&A.x+B.u\\[2mm]
y&=& C.x+D.u
\end{array}
\end{equation}
Si $A$ es estable, entonces el par $\{A,B\}$ es estabilizable.
Esto es cierto para cualquier $B$.
\end{theorem}

\begin{proof}
Esta prueba es trivial
\end{proof}


\section{Metodología experimental}


\section{Resultados}

\section{Conclusiones}



\section{Conclusiones}
Estas son las conclusiones.
Estas son las conclusiones.
Estas son las conclusiones.
Estas son las conclusiones.
Estas son las conclusiones.
Estas son las conclusiones.
Estas son las conclusiones.
Estas son las conclusiones.
Estas son las conclusiones.
Estas son las conclusiones.
Estas son las conclusiones.
Estas son las conclusiones.
Estas son las conclusiones.
Estas son las conclusiones.
Estas son las conclusiones.
Estas son las conclusiones.
Estas son las conclusiones.
Estas son las conclusiones.
Estas son las conclusiones.
Estas son las conclusiones.

\section*{Agradecimientos}
Estos son los agradecimientos.
Estos son los agradecimientos.
Estos son los agradecimientos.
Estos son los agradecimientos.
Estos son los agradecimientos.
Estos son los agradecimientos.

\nocite{*}
\bibliographystyle{maeb2015}

\begin{thebibliography}{1}
\bibitem{LaTeX}
Leslie Lamport,
\newblock {\em A Document Preparation System: \LaTeX, User's Guide and
  Reference Manual},
\newblock Addison Wesley Publishing Company, 1986.
\bibitem{LaTeXD}
Helmut Kopka,
\newblock {\em \LaTeX, eine Einf\"uhrung},
\newblock Addison-Wesley, 1989.
\bibitem{TeX}
D.K. Knuth,
\newblock {\em The {\rm T\kern-.1667em\lower.7ex\hbox{E}\kern-.125emX}book},
\newblock Addison-Wesley, 1989.
\bibitem{METAFONT}
D.E. Knuth,
\newblock {\em The {\rm METAFONT}book},
\newblock Addison Wesley Publishing Company, 1986.
\end{thebibliography}
\end{document}




@article{Min2011606,
title = "Real-time road traffic prediction with spatio-temporal correlations ",
journal = "Transportation Research Part C: Emerging Technologies ",
volume = "19",
number = "4",
pages = "606 - 616",
year = "2011",
note = "",
issn = "0968-090X",
doi = "http://dx.doi.org/10.1016/j.trc.2010.10.002",
url = "http://www.sciencedirect.com/science/article/pii/S0968090X10001592",
author = "Wanli Min and Laura Wynter",
keywords = "Intelligent transport systems",
keywords = "Volume",
keywords = "Speed",
keywords = "Predictive modeling ",
abstract = "Real-time road traffic prediction is a fundamental capability needed to make use of advanced, smart transportation technologies. Both from the point of view of network operators as well as from the point of view of travelers wishing real-time route guidance, accurate short-term traffic prediction is a necessary first step. While techniques for short-term traffic prediction have existed for some time, emerging smart transportation technologies require the traffic prediction capability to be both fast and scalable to full urban networks. We present a method that has proven to be able to meet this challenge. The method presented provides predictions of speed and volume over 5-min intervals for up to 1&#xa0;h in advance. "
}
